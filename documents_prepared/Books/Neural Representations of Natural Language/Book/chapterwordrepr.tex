\documentclass[parskip]{komatufte}
\usepackage[subpreambles=false]{standalone}

%%%%%%%%%%%%%%%%%%%%%%%%%%%
% Silence warning messages
\usepackage{silence}
\WarningsOff[scrlayer-notecolumn]
\WarningsOff[biblatex]

%%%%%%%%%%%%%%%%%%%%
% Commenting

%\usepackage[author=Lyndon]{pdfcomment}
%\newcommand{\pdfcomment}[1]{} %ignore all comments

%\usepackage{todonotes}
%\newcommand{\pdfcomment}{\todo}


%%%%%%%%%%%%%%%%%%%%
% Tables
\usepackage{booktabs}

%%%%%%%%%%%%%%%%%%%
% Fonts
\usepackage{tgadventor} %sans
\usepackage{tgpagella}  %serif
\usepackage{inconsolata} %mono
\usepackage[T1]{fontenc}

\usepackage{microtype}
\usepackage[all]{nowidow}
%%%%%%%%%%%%%%%%%%%%%%%
% Styling
\setcounter{secnumdepth}{4}
\setcounter{tocdepth}{2}

\usepackage{placeins}



%%%%%%%%%%%%%%%%%%%
% Math
\usepackage{amsmath, amssymb, stmaryrd, mathtools}
\DeclareMathOperator*{\argmin}{argmin}
\DeclareMathOperator*{\argmax}{argmax}

\usepackage{xparse,xstring,etoolbox}
% crossref this against notation section
\newcommand{\vv}[1]{\tilde{#1}} % vector
\newcommand{\seq}[1]{\mathcal{#1}} % sequence
\newcommand{\set}[1]{\mathbb{#1}} % set

%%%%%%%%%
% Indexing/sequence indexing
\newcommand{\seqind}[2]{#1^{#2}} % seqence index
\newcommand{\ind}[2]{#1_{#2}} % indexed
\newcommand{\disamb}[2]{#1^{\mathrm{#2}}} %disambiguated

%% Smart indexing and naming
\newcommand{\ifupper}[3]{
    \normalexpandarg
	\exploregroups
	\StrCount{ABCDEFGHIJKLMNOPQRSTUVWXYZ}{#1}[\uppercount]
	\ifnumgreater{\uppercount}{0}{#2}{#3}
}

%smart index
\DeclareDocumentCommand{\ii}{u{_} m}{
	\ifupper{#1}%
	{% just a single uppercase character, i.e. a matrix
		  %make sure the index is the right length
		\StrCount{#2}{,}[\indcount]
		\ifnumgreater{\indcount}{0}
		{ % Got multiple indexes so all good
		 	\ind{#1}{#2}
		}
		{ % Only 1 index so grab the column
		 	\ind{#1}{{:,#2}}
		}
	}%
	{% Not just a single upper case character
		\ind{#1}{#2}
	}
}

\DeclareDocumentCommand{\nn}{u{_} m}{
	\seqind{#1}{#2}
}

\DeclareDocumentCommand{\dd}{u{_} m}{
	\disamb{#1}{#2}
}

% Index of a vector
\DeclareDocumentCommand{\iv}{u{_} m}{\ii{\vv #1}_{#2}}
\DeclareDocumentCommand{\dv}{u{_} m}{\dd{\vv #1}_{#2}}
\DeclareDocumentCommand{\nv}{u{_} m}{\nn{\vv #1}_{#2}}

%exp
\let\oldexp\exp
\renewcommand{\exp}[1]{\oldexp \left( #1 \right)}
\newcommand{\exptwo}[1]{\oldexp_2 \left( #1 \right)}

\newcommand{\softmax}{\mathrm{smax}}

\DeclareMathOperator*{\expectedop}{\mathbb{E}}
\DeclareDocumentCommand{\expected}{u{_} m}{
	\expectedop\limits_{\mathrlap{#2}}
}

%%%%%%%%%%%%%%%%
%Graphics
\usepackage{tikz}
\usetikzlibrary{positioning, fit,  shapes.geometric}
\usepackage{ifthen}
\usepackage{etoolbox}

\tikzset{
	backgroundcolor/.style ={fill=white},
	every node/.append style={
		minimum height=7mm,
	},
	labe/.append style={
		%Blue,
		align = center,
		backgroundcolor,
		fill opacity=0.6,
		text opacity=1,
		font={\footnotesize\itshape}	
	},
	layer/.append style={
		draw,
		align = center,
		minimum height=7mm,
	},
	tight/.append style={
		inner sep=0.2mm,
	},
	lookupbox/.append style={
		draw=none,
		append after command={
		       	[shorten <= -0.5\pgflinewidth]
		       	([shift={(-1.5\pgflinewidth,-0.5\pgflinewidth)}]\tikzlastnode.north east)
		       	edge([shift={( 0.5\pgflinewidth,-0.5\pgflinewidth)}]\tikzlastnode.north west) 
		       	([shift={( 0.5\pgflinewidth,-0.5\pgflinewidth)}]\tikzlastnode.north west)
		       	edge([shift={( 0.5\pgflinewidth,-1.5\pgflinewidth)}]\tikzlastnode.south west)            
		       	([shift={( -1.5\pgflinewidth,+0.5\pgflinewidth)}]\tikzlastnode.south east)
		       	edge([shift={(-1.5\pgflinewidth,-0.5\pgflinewidth)}]\tikzlastnode.north east)
		},
		inner sep=0.7mm,
		outer sep=0mm,
		minimum width=25mm
	}
}

\usepackage{pgfplots}
\pgfplotsset{compat=1.14}
\pgfplotsset{sideplot/.append style={
		width=\notescolwidth,
		domain=-10:10,
		samples=101,
		smooth,
		enlarge y limits={abs=2},
		axis lines=middle,
		xlabel  = $z$,
		ylabel  = $y$,
	},
	equ/.append style={
		color=blue,
		thick,
		mark=none
	}
}

% Function  For a plot 
% it  needs to be declared in preamble because of how \makenote* interacts with multiple files
\def\errorsurface(#1,#2){(0.5*#1 + 0.7*#2 + sin(deg(1.5*#1 + #2^2)))^2}


\usepackage{graphicx}
\graphicspath{{./figs/}, {./}, {./figs/chaptersentencerrepr/}, {./figs/chapterintromachinelearning/}, {./figs/chapterwordrepr/}}
\usepackage{adjustbox}


%%%%%%%%%%%%%%%%%%%
% Refs
\usepackage{cleveref}

\addbibresource{master.bib}

%%%%%%%%%%%%%%%%%%%%
% Formatting

% for examples from natural language space.
\newcommand{\natlang}[1]{\ifmmode \text{``\texttt{#1}''} \else {``\texttt{#1}''}\fi}
% \ifmmode ``trick'' from https://tex.stackexchange.com/a/15194/5834

%%%%%%%%%%%%%%%%%%%%%




\begin{document}
\chapter{Word Representations}\label{sec:word-representations}
%\begin{addmargin}[0pt]{.3\textwidth}
%•	Chapter 4: Word Representations (10 pages)
%o	The core work that began to draw a lot of attention to this area.
%o	This will cover crucial works such as Skip-Grams, CBOW, GLoVe, and of course the original neural probabilistic language model.
%o	As well as more recent techniques based on matrix factorisations.


\aside[Word Vector or Word Embedding?]{Note that older literature in the area will use the term \emph{word~vector} to refer to representations from LDA and LSA etc. Whereas in a lot of newer work the term is used synonymously with \emph{word~embedding}.}
We begin the consideration of the representation of words using neural networks with work on language modelling
This is not the only place one could begin the consideration: the information retrieval (IR) models such as LDA and LSA based on word co-location would be the other obvious starting point.
However, as will be discussed there models are closer to the end point of of the chapter than the begin.
For the language modelling works, comes the contextual language model works such as skip-gram, which in turn lead to the post-neural network co-occurrence based works.
These model fundamentally are in fact more similar to the information retrieval co-location works than the language modelling from which we began.


\section{Representations for Language Modelling}

\aside[Random Variable Naming Convention]{We follow convention that capitalised $W_i$ is a random variable, and $w_i$ is a particular value which $W_i$ may take. Writing $P(\ldots W_i{=}w_i)$.
	Often though we will make uses of the abridged (abuse-of)notation and simply write $P(\ldots w_i)$, allowing the random variable to be implicitly given by the name of its value.
}

The language modelling task it to predict the next word given the words that came prior \pcite{rosenfeld2000two}.
For example, if a sentence begins \natlang{For lunch I will have a hot}, then there is a high probability that the next word will be \natlang{dog} or \natlang{meal},
and lower probabilities of words such as \natlang{day} or \natlang{are}.
Mathematically it is formulated: 
\begin{equation}\label{equ:langmodel}
	P(W_n{=}w_n \mid W_{n-1}{=}w_{n-1}, \ldots, W_1=w_1)
\end{equation}
or to use the compact notation 
\begin{equation}\label{equ:langmodelcompact}
P(w_n \mid w_{n-1}, \ldots, w_1)
\end{equation}
$P(w_n \mid \natlang{hot}, \natlang{a}, \natlang{want}, \natlang{I}, \natlang{lunch}, \natlang{For})$.
The task is to find the probabilities for the various words that $w_n$ could represent.

\aside[Google n-gram corpora]{Google has created several very large scale corpora of 1,2,3,4, and 5-grams from over $10^{11}$ words.
Including the 2006 Web corpus from the public internet (Available for fees at \url{https://catalog.ldc.upenn.edu/LDC2006T13},
and the 2010-2012 Books corpus made words from the Google Books project (available freely from \url{https://books.google.com/ngrams/datasets}), as described in \textcite{lin2012syntactic}.
These very large corpora also used outside of statistical language modelling by corpus linguists investigating the use of language.
}

  
The classical approach is trigram statistical language modelling.
In this, the number of word triples in a corpus is counted,
then given the previous two words, the next is predicted based on those counts.
This is making the Markov assumption that the next state depends only on the current state,
and that that state can be described by previous two words.
Under this assumption \Cref{equ:langmodelcompact} becomes:
\begin{equation}\label{equ:trigramlangmodel}
P(w_n \mid w_{n-1}, \ldots, w_1) = P(w_n \mid w_{n-1}, w_{n-2})
\end{equation}
More generally, one can use $n$-gram language model where for any value if $n$,
this is simply a matter of defining the Markov state to contain fewer or greater numbers of previous words than in the trigram case.

This Markov assumption is, of-course, an approximation.
In the previous example, a trigram language model is finding $P(w_n \mid \natlang{hot}, \natlang{a})$.
It can be seen that the approximation has lost key information.
Based only on the previous 2 words the next word $w_i$ could now reasonably be \natlang{day}, but the sentence: \natlang{For lunch I will have a hot day} makes no sense.
However, the Markov assumption in using $n$-grams is required in order to make the problem tractable -- otherwise an unbound amount of information would need to be stored. 

They key issue with n-gram language models in that in training them there is a data-sparsity problem.
Most combinations of words occur very rarely \pcite{ha2009extending}.
It is thus hard to estimate their occurrence probability.
Combinations of words that do not occur in the model is derived from as naturally given a probability of zero.
This is unlikely to be true though -- it is simply a matter of rare words never occurring together in a finite corpus.
Several approaches have been taken to handle this.
Simplest is add-one smoothing which adds an extra pseudo-observation of every combination of terms.
In common use is are various back-off methods \pcite{katz1987estimation,kneser1995improved} which use the bigram probabilities to estimate the probabilities of unseen trigrams.
However, these methods are merely clever statistical tricks -- ways to reassign probability mass after the fact to leave some left-over for unseen cases.
Back-off is smarter than add-one smoothing, as it portions the probability fairly based of the $n{-}1$-gram probability.
Better still would be a method which can learn to see the common-role of words.
By looking at the fragment: \natlang{For lunch I want a hot}, any reader knows that the next word is most likely going to be a food.
We know this for the same reason we know the next word in \natlang{For elevenses  I had a cold ...} is also going to be a food.
Even though \natlang{elevenses} is a vary rare word, we know from the context that it is a meal (more on this later), and we know it shares other traits with meals, and similarly \natlang{have} / \natlang{had}, and \natlang{hot} / \natlang{cold}.
These traits influence the words that can occur after them.
Thus the motivation is for a language modelling method that use these common properties of the words in the language modelling.
We need representation that holds this information.

\tcite{NPLM} presents a method that uses neural network representations in a language model, where this representations implicitly learn the crucial traits of words.
Capturing the trait information in the representation of a word for use in language modelling is not an idea exclusive to neural network approaches.
Almost a decade earlier was the work of  \tcite{brown1992class}.
The method of \tcite{NPLM} implictly learns this information in doing the language modelling task.



\tcite{NPLM} 


\tcite{collobert2008unified}

\tcite{turian2010word}


\subsection{RNN Language Models}
\tcite{mikolov2011RnnLM}
\tcite{mikolov2010recurrent}
\tcite{6163930}

\section{Contextual Language Modeling}

\tcite{mikolov2012contextRNNLM} a

\subsection{CBOW}
\tcite{mikolov2013linguisticsubstructures}
\tcite{mikolovSkip}
 
\subsection{SkipGram}
\tcite{mikolov2013linguisticsubstructures}
\tcite{mikolovSkip} a

\aside[Pretrained Word-Embeddings]{
Pretrained Word Embeddings are available for most models discussed here.
There trained on a lot more data than most people have access too.
It can be useful to subsitute word embeddings are a repressentation in most systems.}

\subsection{Glove}

General concessus has emerged that Glove performs marginally worse than skip-grams

\subsection{What ever it is that people do instead now that is basically SVD on colloations}



%\end{addmagin}

\end{document}
