\documentclass[12pt,parskip]{komatufte}
\usepackage[subpreambles=false]{standalone}

%%%%%%%%%%%%%%%%%%%%%%%%%%%
% Silence warning messages
\usepackage{silence}
\WarningsOff[scrlayer-notecolumn]
\WarningsOff[biblatex]

%%%%%%%%%%%%%%%%%%%%
% Commenting

%\usepackage[author=Lyndon]{pdfcomment}
%\newcommand{\pdfcomment}[1]{} %ignore all comments

%\usepackage{todonotes}
%\newcommand{\pdfcomment}{\todo}


%%%%%%%%%%%%%%%%%%%%
% Tables
\usepackage{booktabs}

%%%%%%%%%%%%%%%%%%%
% Fonts
\usepackage{tgadventor} %sans
\usepackage{tgpagella}  %serif
\usepackage{inconsolata} %mono
\usepackage[T1]{fontenc}

\usepackage{microtype}
\usepackage[all]{nowidow}
%%%%%%%%%%%%%%%%%%%%%%%
% Styling
\setcounter{secnumdepth}{4}
\setcounter{tocdepth}{2}

\usepackage{placeins}



%%%%%%%%%%%%%%%%%%%
% Math
\usepackage{amsmath, amssymb, stmaryrd, mathtools}
\DeclareMathOperator*{\argmin}{argmin}
\DeclareMathOperator*{\argmax}{argmax}

\usepackage{xparse,xstring,etoolbox}
% crossref this against notation section
\newcommand{\vv}[1]{\tilde{#1}} % vector
\newcommand{\seq}[1]{\mathcal{#1}} % sequence
\newcommand{\set}[1]{\mathbb{#1}} % set

%%%%%%%%%
% Indexing/sequence indexing
\newcommand{\seqind}[2]{#1^{#2}} % seqence index
\newcommand{\ind}[2]{#1_{#2}} % indexed
\newcommand{\disamb}[2]{#1^{\mathrm{#2}}} %disambiguated

%% Smart indexing and naming
\newcommand{\ifupper}[3]{
    \normalexpandarg
	\exploregroups
	\StrCount{ABCDEFGHIJKLMNOPQRSTUVWXYZ}{#1}[\uppercount]
	\ifnumgreater{\uppercount}{0}{#2}{#3}
}

%smart index
\DeclareDocumentCommand{\ii}{u{_} m}{
	\ifupper{#1}%
	{% just a single uppercase character, i.e. a matrix
		  %make sure the index is the right length
		\StrCount{#2}{,}[\indcount]
		\ifnumgreater{\indcount}{0}
		{ % Got multiple indexes so all good
		 	\ind{#1}{#2}
		}
		{ % Only 1 index so grab the column
		 	\ind{#1}{{:,#2}}
		}
	}%
	{% Not just a single upper case character
		\ind{#1}{#2}
	}
}

\DeclareDocumentCommand{\nn}{u{_} m}{
	\seqind{#1}{#2}
}

\DeclareDocumentCommand{\dd}{u{_} m}{
	\disamb{#1}{#2}
}

% Index of a vector
\DeclareDocumentCommand{\iv}{u{_} m}{\ii{\vv #1}_{#2}}
\DeclareDocumentCommand{\dv}{u{_} m}{\dd{\vv #1}_{#2}}
\DeclareDocumentCommand{\nv}{u{_} m}{\nn{\vv #1}_{#2}}

%exp
\let\oldexp\exp
\renewcommand{\exp}[1]{\oldexp \left( #1 \right)}
\newcommand{\exptwo}[1]{\oldexp_2 \left( #1 \right)}

\newcommand{\softmax}{\mathrm{smax}}

\DeclareMathOperator*{\expectedop}{\mathbb{E}}
\DeclareDocumentCommand{\expected}{u{_} m}{
	\expectedop\limits_{\mathrlap{#2}}
}

%%%%%%%%%%%%%%%%
%Graphics
\usepackage{tikz}
\usetikzlibrary{positioning, fit,  shapes.geometric}
\usepackage{ifthen}
\usepackage{etoolbox}

\tikzset{
	backgroundcolor/.style ={fill=white},
	every node/.append style={
		minimum height=7mm,
	},
	labe/.append style={
		%Blue,
		align = center,
		backgroundcolor,
		fill opacity=0.6,
		text opacity=1,
		font={\footnotesize\itshape}	
	},
	layer/.append style={
		draw,
		align = center,
		minimum height=7mm,
	},
	tight/.append style={
		inner sep=0.2mm,
	},
	lookupbox/.append style={
		draw=none,
		append after command={
		       	[shorten <= -0.5\pgflinewidth]
		       	([shift={(-1.5\pgflinewidth,-0.5\pgflinewidth)}]\tikzlastnode.north east)
		       	edge([shift={( 0.5\pgflinewidth,-0.5\pgflinewidth)}]\tikzlastnode.north west) 
		       	([shift={( 0.5\pgflinewidth,-0.5\pgflinewidth)}]\tikzlastnode.north west)
		       	edge([shift={( 0.5\pgflinewidth,-1.5\pgflinewidth)}]\tikzlastnode.south west)            
		       	([shift={( -1.5\pgflinewidth,+0.5\pgflinewidth)}]\tikzlastnode.south east)
		       	edge([shift={(-1.5\pgflinewidth,-0.5\pgflinewidth)}]\tikzlastnode.north east)
		},
		inner sep=0.7mm,
		outer sep=0mm,
		minimum width=25mm
	}
}

\usepackage{pgfplots}
\pgfplotsset{compat=1.14}
\pgfplotsset{sideplot/.append style={
		width=\notescolwidth,
		domain=-10:10,
		samples=101,
		smooth,
		enlarge y limits={abs=2},
		axis lines=middle,
		xlabel  = $z$,
		ylabel  = $y$,
	},
	equ/.append style={
		color=blue,
		thick,
		mark=none
	}
}

% Function  For a plot 
% it  needs to be declared in preamble because of how \makenote* interacts with multiple files
\def\errorsurface(#1,#2){(0.5*#1 + 0.7*#2 + sin(deg(1.5*#1 + #2^2)))^2}


\usepackage{graphicx}
\graphicspath{{./figs/}, {./}, {./figs/chaptersentencerrepr/}, {./figs/chapterintromachinelearning/}, {./figs/chapterwordrepr/}}
\usepackage{adjustbox}


%%%%%%%%%%%%%%%%%%%
% Refs
\usepackage{cleveref}

\addbibresource{master.bib}

%%%%%%%%%%%%%%%%%%%%
% Formatting

% for examples from natural language space.
\newcommand{\natlang}[1]{\ifmmode \text{``\texttt{#1}''} \else {``\texttt{#1}''}\fi}
% \ifmmode ``trick'' from https://tex.stackexchange.com/a/15194/5834

%%%%%%%%%%%%%%%%%%%%%

\usepackage{tabulary}
\newcolumntype{K}[1]{>{\centering\arraybackslash}p{#1}}
\begin{document}

\setchapterpreamble{%
	\dictum[the seven senses of \natlang{literally},
	 \textit{Oxford English Dictionary}, 3rd ed., 2011]
	{
		\begin{description}
			\setlength\itemsep{0em}
			\item[1a.] In a literal, exact, or actual sense; not figuratively, allegorically, etc.
			\item[1b.] Used to indicate that the following word or phrase must be taken in its literal sense, usually to add emphasis.
			\item[1c.] colloq. Used to indicate that some (frequently conventional) metaphorical or hyperbolical expression is to be taken in the strongest admissible sense: `virtually, as good as'; (also) `completely, utterly, absolutely' \ldots
			\item[2a ]With reference to a version of something, as a transcription, translation, etc.: in the very words, word for word.
			\item[2b.] In extended use. With exact fidelity of representation; faithfully.
			\item[3a.] With or by the letters (of a word). Obs. rare.
			\item[3b.] In or with regard to letters or literature. Obs. rare.
		\end{description}		
}}

\chapter{Word Sense Representations}\label{sec:word-sense-representations}
\begin{abstract}
	Chapter 5: Word Sense Representations (5-10 pages)
	In this chapter, technologies for representing the multiple meanings of a single word can have will be discussed.
	This is a growing area, and is particularly important in languages (including English, but other languages even more so), where polysemous and homonymous words are common.
	It leads naturally to the next section on phrase representation. Rather than a single word having many meanings, the next chapter will discuss how a single meaning may take multiple words to express.
\end{abstract}
	
\section{Word Senses}

Words have multiple meanings.
When assigning a single representation to a word, it is impossible for that representation to truly describe the meaning in all contexts.
It may have some features applicable to some uses but not to others,
it may be an average of all features for particular contexts,
or it may only represent the 


\aside[lemma/lexeme]{}
\aside[tense]{}
\aside[Part of Speech/POS]{}
\aside[synset]{}
\aside[stem]{}
\aside[synset]{A synset is a set of synonymous words: that is words that have the same meaning}
\aside[gloss]{A gloss is the dictionary entry for a word-sense, it normally includes both the definition and an example of use.
In wordnet each synset shares a common gloss.}



The standard way to assign word senses is via some lexicographical resource.
i.e. a dictionary, or a thesaurus.
There is not a canonical list of word senses that are truly real and consistently defined in English.
Every dictionary is unique, with different definitions and numbers of word senses.
The most commonly used lexicographical resource is WordNet \pcite{miller1995wordnet},
there are several WordNets equivalents in other languages,
as well as the multi-lingual  BabelNet \pcite{navigli2010babelnet}.

\pdfcomment{Rewrite this stuff about wordnet being a poor moral baseline to fit the books intended audience better}
\aside[WordNet is not a strong moral baseline]{

	WordNet is the standard "dictionary/thesusaus/source of lexicographical information" for algorithms. (ML and otherwise)
	(It is also the seed for some online dictionaries)
	
	The definitions in wordnet were written in a large part by undergrads at Princeton in 1990. That is why it has things like "S: (v) nag, peck, hen-peck (bother persistently with trivial complaints) "She nags her husband all day long"".
	
	Further, the dataset used to determine frequency of Synsets (that is the total count of all words) comes from a dataset now known as SemCore.
	Which is a sense annotated subset of the Brown Corpus.
	
	The Brown Corpus is a sampling of American texts from 1961.
	The norms of 1961 were not the norms of today.
	(Remember the US didn't pass the Civil Rights act to end segregation until 1964).
	
	Now here is a fact I will return to later. The word Wheelchair is only mentioned once in the Brown corpus.
	
	Moving on:
	ImageNet is a common resource for image recognition.
	It labels images with their WordNet synset.
	More generally, using a high performing ImageNet network minus its final classification layer, is commonly used as a feature extractor on any computer vision machine learning problem.
	
	ImageNet only uses the most common synsets.
	(in 2009 it has 8,000. It is now up to  22,000)
	Where most common is determined using the counts reported by WordNet (at least originally it was).
	Which were extracted from SemCore, i.e. The Brown Corpus.
	Which means that it did not include wheelchairs for example (it does now).
	
	And this is why automatic captioning systems recognize wheelchairs as skateboards.
	
	Because of the fact that they are based on what people were writing about in 1961.}


\subsection{Word Sense Disambiguation}


\aside[Semantic Syllepsis]{
	Also known as pathological sentences that kill almost all WSD systems.
	Consider the sentence: \natlang{John used to work for the \emph{newspaper} that you are carrying.},
	In this sentence the word newspaper simultaneously have two different meanings: it is both the company, and the object
	As word sense disambiguation systems normally attempt to assign a single sense  to an each word they are unable to handle these sentences.
	Most word sense induction systems can not much better: at best a new sense could be allocated for the join use, which does not correspond to the human notion of the word having two senses for different parts of the sentence.
	Most works on word sense disambiguation outright ignore these sentences, or consider them to be ungrammatical, or incorrect.
	However, they are readily understood and used without thought by most native speakers.
	These constructions are also known as zeugma, although zeugma is itself as highly polysemous word, so its usage varies. 
}



Word sense disambiguation is one of the hardest problems in NLP.
Very few systems significantly out perform the baseline most common sense (MCS).

Progress on the problem is made difficult by several factors.

There are problems with the data available.
The lack of very large scale training corpora rendering fully supervised methods difficult.
The limited testing corpora can result in systems that allow 
Lack of human agreement on the correct sense, resulting in weak ground truth.

There are also issues inherent in the task.
Determining the sense may require very long range information:
for example the information on context may not even be in the same sentence.
It may require knowing the domain of the text, where word sense uses vary between domain.
It may in-fact be intentionally unclear, with multiple correct interpretations, as in a pun.
Or be unintentionally unknowable, due to poor writing style, such that it would confuse any human reader also.


It can also be said that word senses are highly artificial and do not adequately represent meaning.
However, WSD is required to access lexicographical resources,
such as translation dictionaries, ontologies (e.g. OpenCyc), and other datasets (e.g. ImageNet).

\pdfcomment{Talk abow MFS baseline and why it always wins}

\section{Word Sense Representation}

\subsection{Supervised Methods}
The simple and direct method, is to take a dataset that is annotated with word-senses,
and then treat each tagged word as if it were a word, then apply any of the methods for word representation discussed in \Cref{sec:word-representations}.
\tcite{iacobacci2015sensembed} use a Continuous Bag of Word language model \parencite{mikolov2013efficient}, doing exactly this.
This does however run into the aforementioned problem, that there is relatively little training data that has been manually sense annotated.
Iacobacci et al. use a 3rd party WSD tool, BabelFly \cite{Moroetal:14tacl}, to annotate the corpus with senses.
This allows for existing word representation techniques to be applied, but being limited to the use of an existing WSD tool makes it unsuitable for use in WSD tasks.


\subsection{Pseudo-semi-supervised Method}
\tcite{Chen2014} use an almost semi-supervised approach to train sense vectors.
They partially disambiguate their training corpus, using initial word sense vectors and WordNet.
They then discard these original sense-vectors, and use the partially disambiguated corpus to train sense-vectors via a skip-gram variant.
\pdfcomment{Don't call things Pseudo-semi-supervised Method}

%
\aside[Cosine distance]{
As a refresher:
here we talk of cosine distance, as it is more reasonable as smaller implies closer.

(Though it is still not a true metric as $d_{cos}(v,kv)=0$ for all $k\in \mathbb{R}_+$).
Other times you may see cosine similarity, it ranged between -1 (most different) and 1 (most similar).
Cosine similarity is given by $sim(a,b)=\frac{a\cdot b}{\left\Vert a\right\Vert _{2}\left\Vert b\right\Vert _{2}}=\cos(\angle a b )$
i.e. the unit-length normalised dot product of the vectors.
Cosine distance is usually defined at $d_{cos}(a,b)=\frac{1-sim(a,b)}{2}$.
It ranged between 0 (most similar) and 1 (most different).
}

The first phase of there method is in essence a word-embedding based WSD system.
When assessed as such a WSD result they report that it only marginally exceeds the MFS baseline,
though that is not at all unusual for WSD algorithms as discussed above.

They initially assign every word-sense in WordNet, a sense vector.
This sense vector is the average of word-embeddings of a subset if words in the gloss,
as determined using pretrained skip-grams \parencite{mikolov2013efficient}.
For the word $w$ with word-sense $w_{s_i}$,
they select a set of candidate words, $cand(W_{s_i})$, from the gloss to be included in the average
based on the follow set of requirements.
First the word must be a content word: that is a verb, noun, adverb or adjective.
Secondly, its cosine distance to $w$ must be below some threshold $\delta$.
Finally, it must not be the word itself (One can assume this is in terms of identical lexemes).
Or written mathematically, where $C_v$ is the skip-gram word vector for $v$
\begin{equation}
cand(w_{s_{i}})=\left\{ u\left|\begin{array}{c}
u\in gloss(w_{s_{i}})\\
\wedge d_{cos}(C_w,C_u)<\delta\\
\wedge pos(u)\in\left\{ verb,adv,noun,adj\right\} \\
\wedge u\ne w
\end{array}\right.\right\} 
\end{equation}
The initial sense vector for $w_{s_i}$ is the mean of the word vectors for all the words in $cand(w_{s_{i}})$.
The effect of this is that most words in the same synset will have similar but not necessarily identical initial vectors.
\pdfcomment{Consider what this does to rare senses of a word, i.e. the base wordsense will be distant}.



They then use these initial sense vectors to disambiguate the words in their unlabelled training corpus.
For each sentence in the corpus, they define an initial content vector.
This is done by taking the mean of the skip-gram word embedding for all content words in the sentence.
For each word in the sentence, they then compare each sense vector to the context vector.
If the closes sense vector is below a given threshold,
then it that word is tagged with that word-sense, and the context vector is updated to use the sense-vector instead of the word vector.
Words that do not come within the threshold are not tagged.
This is an important features, as it means that words without clear senses do not get a sense given to them.
Thus avoiding any dubious sense tags for the next training step.


Once the corpus is (partially) tagged Chen et. al. employ the skip-gram word-embedding method with a variation to predict the word-senses.
The original sense vectors are discarded.
Rather than the model being tasked only to predict the surrounding words,
it is tasked to predict surrounding words and their sense-tags (where present).
For purposes of the loss function predicting tags and words is weighted equally.
Note that the input the the skip-gram is the central word, not the central word-tag.
In this method the word-sense embeddings are output embeddings.




\section{Word Sense Induction}
In this section we will discuss methods for finding a word-sense, without reference to a standard set of senses.
Such systems must discover the word senses at the same time as they find their representations.
One strong advantage of these methods is that they do not require a labelled dataset.
As discussed there are relatively few high-quality word-sense labelled datasets.
The other key advantage is not relying on fixed senses determined by a lexicographer.
This is particular useful if the words senses are highly domain specific;
or in a language without strong lexicographical resources.
This allows the data to inform on what word-senses exist.




\section{Word Sense Alignment}
As there are a great many word sense induction methods


\subsection{Directly Learning Lexical Sense Embeddings}
In this area of research, the induction of word sense embeddings is treated as a supervised, or semi-supervised task, that requires sense labelled corpora for training.

 using word senses as the labels rather than words.
This is a direct application of word embedding techniques.
To overcome the lack of a large sense labelled corpus, , to add sense annotations to a previously unlabelled corpus. 




\subsection{Mapping induced senses to lexical senses}\label{mapping}
By defining a stochastic map between the induced and lexical senses, \tcite{agirre2006}, propose a general method for allowing WSI systems to be used for WSD.
Their work was used in SemEval-2007 Task 02 \parencite{SemEval2007WSIandWSD} to evaluate all entries. 
Agirre et al. use a mapping corpus to find the probability of a lexical sense, given the induced sense according to the WSI system.

By exploiting the particular properties of sense embedding based WSI systems we propose a system that can better facilitate the use of this subset of WSI systems for WSD.

\section{Word sense representation (WSR)}
Word sense representation (WSR) is the process by which a variety of representations, for different work senses is generated.
Here we use the term to refer to the supervised case -- that the word senses are labelled in what ever training data is used.
We are concerned with vector representations.

\tcite{iacobacci2015sensembed}

\section{Word sense induction (WSI)}
Word sense induction (WSI) is (for purposes of this discussion) the process of using unsupervised data to discover, (and implect in that task represent) word senses.
We can see this as similar to an unsupervised analogue to WSR.

Most vector WSI and WSR approaches are evaluated on similarity tests.
Like WordSim-353 \cite{WordSim353}, for contextless, or Stanford's Contextual Word Similarities (SCWS) \cite{Huang2012}. This is also how normal Word2Vec variants are often evaluated.


\tcite{pantel2002WSI}

\tcite{Reisinger2010}

\tcite{Huang2012}

\tcite{Chen2014}

\tcite{AdaGrams}



\section{Word sense disambiguation (WSD)}
Word sense disambiguation (WSD) is the process to assign a word sense, to an instance of a word. We use this term primarily in the context of WSD to a sense from a standard sense inventory. Those it is also used 

\tcite{veronis2004hyperlex}
\tcite{pinto2007upv}

\tcite{basile-caputo-semeraro:2014:Coling}

\section{Sense alignment -- going from WSI to WSD}
Here we look at methods that allow induced word senses (from WSI), to be used on WSD tasks.
A WSD task, requires determining which of the standard word-senses a particular example of a word in its context belongs to.
These standard word senses, are from some dictionary (or Sense Inventory) created by 
lexicographers, such as WordNet, BabelNet.

A related task to this word sense alignment is the composite word embedding creation used in similarity Called LocalSim in \textcite{Huang2012} (see above).


\tcite{agirre2006}

\subsubsection{\textcite {pantel2002WSI}}
The method used for evaluation by Pantel et. al. is a form of alignment. I think.
I really haven't manage to wrap my head around it.

\end{document}