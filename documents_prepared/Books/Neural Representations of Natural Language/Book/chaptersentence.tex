\documentclass[12pt,parskip]{komatufte}
\usepackage[subpreambles=false]{standalone}

%%%%%%%%%%%%%%%%%%%%%%%%%%%
% Silence warning messages
\usepackage{silence}
\WarningsOff[scrlayer-notecolumn]
\WarningsOff[biblatex]

%%%%%%%%%%%%%%%%%%%%
% Commenting

%\usepackage[author=Lyndon]{pdfcomment}
%\newcommand{\pdfcomment}[1]{} %ignore all comments

%\usepackage{todonotes}
%\newcommand{\pdfcomment}{\todo}


%%%%%%%%%%%%%%%%%%%%
% Tables
\usepackage{booktabs}

%%%%%%%%%%%%%%%%%%%
% Fonts
\usepackage{tgadventor} %sans
\usepackage{tgpagella}  %serif
\usepackage{inconsolata} %mono
\usepackage[T1]{fontenc}

\usepackage{microtype}
\usepackage[all]{nowidow}
%%%%%%%%%%%%%%%%%%%%%%%
% Styling
\setcounter{secnumdepth}{4}
\setcounter{tocdepth}{2}

\usepackage{placeins}



%%%%%%%%%%%%%%%%%%%
% Math
\usepackage{amsmath, amssymb, stmaryrd, mathtools}
\DeclareMathOperator*{\argmin}{argmin}
\DeclareMathOperator*{\argmax}{argmax}

\usepackage{xparse,xstring,etoolbox}
% crossref this against notation section
\newcommand{\vv}[1]{\tilde{#1}} % vector
\newcommand{\seq}[1]{\mathcal{#1}} % sequence
\newcommand{\set}[1]{\mathbb{#1}} % set

%%%%%%%%%
% Indexing/sequence indexing
\newcommand{\seqind}[2]{#1^{#2}} % seqence index
\newcommand{\ind}[2]{#1_{#2}} % indexed
\newcommand{\disamb}[2]{#1^{\mathrm{#2}}} %disambiguated

%% Smart indexing and naming
\newcommand{\ifupper}[3]{
    \normalexpandarg
	\exploregroups
	\StrCount{ABCDEFGHIJKLMNOPQRSTUVWXYZ}{#1}[\uppercount]
	\ifnumgreater{\uppercount}{0}{#2}{#3}
}

%smart index
\DeclareDocumentCommand{\ii}{u{_} m}{
	\ifupper{#1}%
	{% just a single uppercase character, i.e. a matrix
		  %make sure the index is the right length
		\StrCount{#2}{,}[\indcount]
		\ifnumgreater{\indcount}{0}
		{ % Got multiple indexes so all good
		 	\ind{#1}{#2}
		}
		{ % Only 1 index so grab the column
		 	\ind{#1}{{:,#2}}
		}
	}%
	{% Not just a single upper case character
		\ind{#1}{#2}
	}
}

\DeclareDocumentCommand{\nn}{u{_} m}{
	\seqind{#1}{#2}
}

\DeclareDocumentCommand{\dd}{u{_} m}{
	\disamb{#1}{#2}
}

% Index of a vector
\DeclareDocumentCommand{\iv}{u{_} m}{\ii{\vv #1}_{#2}}
\DeclareDocumentCommand{\dv}{u{_} m}{\dd{\vv #1}_{#2}}
\DeclareDocumentCommand{\nv}{u{_} m}{\nn{\vv #1}_{#2}}

%exp
\let\oldexp\exp
\renewcommand{\exp}[1]{\oldexp \left( #1 \right)}
\newcommand{\exptwo}[1]{\oldexp_2 \left( #1 \right)}

\newcommand{\softmax}{\mathrm{smax}}

\DeclareMathOperator*{\expectedop}{\mathbb{E}}
\DeclareDocumentCommand{\expected}{u{_} m}{
	\expectedop\limits_{\mathrlap{#2}}
}

%%%%%%%%%%%%%%%%
%Graphics
\usepackage{tikz}
\usetikzlibrary{positioning, fit,  shapes.geometric}
\usepackage{ifthen}
\usepackage{etoolbox}

\tikzset{
	backgroundcolor/.style ={fill=white},
	every node/.append style={
		minimum height=7mm,
	},
	labe/.append style={
		%Blue,
		align = center,
		backgroundcolor,
		fill opacity=0.6,
		text opacity=1,
		font={\footnotesize\itshape}	
	},
	layer/.append style={
		draw,
		align = center,
		minimum height=7mm,
	},
	tight/.append style={
		inner sep=0.2mm,
	},
	lookupbox/.append style={
		draw=none,
		append after command={
		       	[shorten <= -0.5\pgflinewidth]
		       	([shift={(-1.5\pgflinewidth,-0.5\pgflinewidth)}]\tikzlastnode.north east)
		       	edge([shift={( 0.5\pgflinewidth,-0.5\pgflinewidth)}]\tikzlastnode.north west) 
		       	([shift={( 0.5\pgflinewidth,-0.5\pgflinewidth)}]\tikzlastnode.north west)
		       	edge([shift={( 0.5\pgflinewidth,-1.5\pgflinewidth)}]\tikzlastnode.south west)            
		       	([shift={( -1.5\pgflinewidth,+0.5\pgflinewidth)}]\tikzlastnode.south east)
		       	edge([shift={(-1.5\pgflinewidth,-0.5\pgflinewidth)}]\tikzlastnode.north east)
		},
		inner sep=0.7mm,
		outer sep=0mm,
		minimum width=25mm
	}
}

\usepackage{pgfplots}
\pgfplotsset{compat=1.14}
\pgfplotsset{sideplot/.append style={
		width=\notescolwidth,
		domain=-10:10,
		samples=101,
		smooth,
		enlarge y limits={abs=2},
		axis lines=middle,
		xlabel  = $z$,
		ylabel  = $y$,
	},
	equ/.append style={
		color=blue,
		thick,
		mark=none
	}
}

% Function  For a plot 
% it  needs to be declared in preamble because of how \makenote* interacts with multiple files
\def\errorsurface(#1,#2){(0.5*#1 + 0.7*#2 + sin(deg(1.5*#1 + #2^2)))^2}


\usepackage{graphicx}
\graphicspath{{./figs/}, {./}, {./figs/chaptersentencerrepr/}, {./figs/chapterintromachinelearning/}, {./figs/chapterwordrepr/}}
\usepackage{adjustbox}


%%%%%%%%%%%%%%%%%%%
% Refs
\usepackage{cleveref}

\addbibresource{master.bib}

%%%%%%%%%%%%%%%%%%%%
% Formatting

% for examples from natural language space.
\newcommand{\natlang}[1]{\ifmmode \text{``\texttt{#1}''} \else {``\texttt{#1}''}\fi}
% \ifmmode ``trick'' from https://tex.stackexchange.com/a/15194/5834

%%%%%%%%%%%%%%%%%%%%%

\begin{document}

\setchapterpreamble{%
	\dictum[\textit{English Composition and Literature}, Webster, 1923]
	{
		A sentence is a group of words expressing a complete thought.
	}
}
\chapter{Sentence Representations and Beyond}\label{sec:sentence-representations-and-beyond}
\begin{abstract}
	Chapter 8: Sentence representations and beyond (5-10 pages)
	This chapter takes the previous discussion of phrases to the next level: sentences.
	This will include discussions of works on recursive structure
	As well work leveraging recurrent neural networks.
	Methods that do not strongly consider order (including Sum of Word Embeddings; paragraph vectors) will also be discussed here.
	Many of these techniques extent to arbitrary length sequences of words.
\end{abstract}

\aside[Documents vs Sentences]{All techniques which can represent documents (or paragraphs) can by necessity represent sentences as well. A single sentence can be a whole document. Similarly for paragraphs. The reverse is not necessarily true.}


It can be argued that the core of true AI,
is the capture a machine manipulatable representation of an idea.
In natural language a sentence (as defined by Webster in the quote above),
is such a representation of an idea, but it is not machine manipulatable.
As such the conversion of sentences to a machine manipulatable representation is an important task in AI research.

\aside[Word Sense Embeddings as a by-product]{
Almost all sentence representation methods product word embeddings as a by-product
Either output embeddings, from softmax,
or input embeddings that are used as the networks inputs.}
\aside[Initialising input embeddings]{
In the case of systems with input embeddings it is common (but not ubiquitous) to initialised them using a pretrained embeddings from one of the methods discussed in the \Cref{sec:word-representations},
then allow them to be fine-tuned while training the sentence embeddings.
}

\section{Unordered and Weakly Ordered Representations}

\subsection{Sums of Word Embeddings}
\aside[SOWE is a BOW product]{
	The reader may recall from \Cref{sec:word-representations},
	that a word-embedding lookup is the same as a one-hot vector product:
	$C_{w_i} = C\,\hat{e}_{w_i}$.
	Similar can be said for sum of word embeddings (SOWE) and bag of words (BOW).
	For some set of words $\mathcal{W} = \lbrace{w_1,\ldots, w_n}$:
	the BOW representation is $B_\mathcal{W} = \sum_{w_i\in \mathcal{W}} \hat{e}_{w_i}$;
	the SOWE representation is $\sum_{w_i\in \mathcal{W}} C_{w_i} = CB_\mathcal{W}$.
	As with word-embeddings it is immensely cheaper to calculate this via lookup, than via matrix product;
	except on systems with suitable sparse matrix product tricks.	
}

Classically in information retrieval documents have been represented by bags of words (BOW).
That is to say a vector with length equal to the size of the vocabulary, with each position representing the count of the number of occurrences of a single word.
This is much the same as a one-hot vector representing a word, but with every word in the sentence/document counted.
The word embedding equivalent is sums of word embeddings (SOWE), and means of word embeddings (MOWE).
These methods, like BOW, lose all order information in the representation.
In many cases it is possible to recover a BOW from a much lower dimensional SOWE 

Surprisingly, they have been found on many tasks to be extremely well performing, better than several of the more advanced techniques discussed later in this chapter  \pcite{White2015SentVecMeaning}, \pcite{RitterPosition}.
It has been suggested that this is because in English there are only a few likely ways to order any given bag of words and thus original sentences can often be recovered from BOW \pcite{Horvat2014} and thus SOWE, \pcite{White2016a}, given a simple ngram language model.
The step beyond this is to encode the ngrams into a bag of words like structure.
The bag of ngrams (BON), e.g. bag of trigrams, has been classically used as an alternative to BOW.
Each index in the vector thus represents the occurrence of a ngram in the text.
So \natlang{It is a good day today}, has the 3grams: \natlang{(It is a)},\natlang{(is a good)},\natlang{(a good day)}, \natlang{(good day today)}.
As is obvious for all but the most pathological sentences, recovering the full sentence order from a bag of ngrams is possible even without a language model.

The natural analogy to this with word embeddings might seem to be to find ngram embeddings by the concatenation of $n$ word embeddings; and then to sum these.
However, such a sum is less informative than it might seem.
As the sum in each concatenated section is equal to the others, minus the edge words. To illustrate the case of trigram sums of word embeddings, for \natlang{It is a good day today},
the sums in each position would be:
\begin{align}
\left[\begin{array}{c|c|c}
	\sum & \sum & \sum\\
	C_{it} & C_{was} & C_{a}\\
	C_{was} & C_{a} & C_{good}\\
	C_{a} & C_{good} & C_{day}\\
	C_{good} & C_{day} & C_{today}
\end{array}\right]=\left[\begin{array}{c|c|c}
	C_{M}+C_{it}-C_{day} & C_{M} & C_{M}-C_{was}+C_{today}\end{array}\right]
\end{align}
\begin{align}
\left[\begin{array}{c|c|c}
	\quad C_{w_{1}} & \quad C_{w_{2}} & \quad C_{w_{3}}\\
	+C_{w_{2}} & +C_{w_{3}} & +C_{w_{4}}\\
	+C_{w_{3}} & +C_{w_{4}} & +C_{w_{5}}\\
	+C_{w_{4}} & +C_{w_{5}} & +C_{w_{6}}
\end{array}\right]=\left[\begin{array}{c|c|c}
	\sum C_{w_{i}}-C_{w_{n-1}}-C_{w_{n-2}} & \sum C_{w_{i}}-C_{w_{1}}-C_{w_{n-1}} & \sum C_{w_{i}}-C_{w_{1}}-C_{w_{2}}\end{array}\right]
\end{align}


\section{Sequential Models}

\begin{figure}
	\caption{The unrolled structure of an RNN for us in Encoding, Decoding and Encoding-Decoding (sequence-to-sequence) problems. RU is the recurrent unit -- the neural network which reoccurs at each time step. (Repeated from \Cref{fig-rnns})
	}
	
	\label{fig-rnns-sq}
	
	\resizebox{\textwidth}{!}{\documentclass[landscape]{article}
\usepackage[a3paper]{geometry}

\usepackage{tikz}
\usetikzlibrary{positioning, fit,  shapes.geometric}
\usepackage{ifthen}
\usepackage{etoolbox}

\tikzset{
	backgroundcolor/.style ={fill=white},
	every node/.append style={
		minimum height=7mm,
	},
	labe/.append style={
		%Blue,
		align = center,
		backgroundcolor,
		fill opacity=0.6,
		text opacity=1,
		font={\footnotesize\itshape}	
	},
	layer/.append style={
		draw,
		align = center,
		minimum height=7mm,
	},
	tight/.append style={
		inner sep=0.2mm,
	},
	lookupbox/.append style={
		draw=none,
		append after command={
		       	[shorten <= -0.5\pgflinewidth]
		       	([shift={(-1.5\pgflinewidth,-0.5\pgflinewidth)}]\tikzlastnode.north east)
		       	edge([shift={( 0.5\pgflinewidth,-0.5\pgflinewidth)}]\tikzlastnode.north west) 
		       	([shift={( 0.5\pgflinewidth,-0.5\pgflinewidth)}]\tikzlastnode.north west)
		       	edge([shift={( 0.5\pgflinewidth,-1.5\pgflinewidth)}]\tikzlastnode.south west)            
		       	([shift={( -1.5\pgflinewidth,+0.5\pgflinewidth)}]\tikzlastnode.south east)
		       	edge([shift={(-1.5\pgflinewidth,-0.5\pgflinewidth)}]\tikzlastnode.north east)
		},
		inner sep=0.7mm,
		outer sep=0mm,
		minimum width=25mm
	}
}
\usepackage{amsmath, amssymb, stmaryrd, mathtools}
\DeclareMathOperator*{\argmin}{argmin}
\DeclareMathOperator*{\argmax}{argmax}

\usepackage{xparse,xstring,etoolbox}
% crossref this against notation section
\newcommand{\vv}[1]{\tilde{#1}} % vector
\newcommand{\seq}[1]{\mathcal{#1}} % sequence
\newcommand{\set}[1]{\mathbb{#1}} % set

%%%%%%%%%
% Indexing/sequence indexing
\newcommand{\seqind}[2]{#1^{#2}} % seqence index
\newcommand{\ind}[2]{#1_{#2}} % indexed
\newcommand{\disamb}[2]{#1^{\mathrm{#2}}} %disambiguated

%% Smart indexing and naming
\newcommand{\ifupper}[3]{
    \normalexpandarg
	\exploregroups
	\StrCount{ABCDEFGHIJKLMNOPQRSTUVWXYZ}{#1}[\uppercount]
	\ifnumgreater{\uppercount}{0}{#2}{#3}
}

%smart index
\DeclareDocumentCommand{\ii}{u{_} m}{
	\ifupper{#1}%
	{% just a single uppercase character, i.e. a matrix
		  %make sure the index is the right length
		\StrCount{#2}{,}[\indcount]
		\ifnumgreater{\indcount}{0}
		{ % Got multiple indexes so all good
		 	\ind{#1}{#2}
		}
		{ % Only 1 index so grab the column
		 	\ind{#1}{{:,#2}}
		}
	}%
	{% Not just a single upper case character
		\ind{#1}{#2}
	}
}

\DeclareDocumentCommand{\nn}{u{_} m}{
	\seqind{#1}{#2}
}

\DeclareDocumentCommand{\dd}{u{_} m}{
	\disamb{#1}{#2}
}

% Index of a vector
\DeclareDocumentCommand{\iv}{u{_} m}{\ii{\vv #1}_{#2}}
\DeclareDocumentCommand{\dv}{u{_} m}{\dd{\vv #1}_{#2}}
\DeclareDocumentCommand{\nv}{u{_} m}{\nn{\vv #1}_{#2}}

%exp
\let\oldexp\exp
\renewcommand{\exp}[1]{\oldexp \left( #1 \right)}
\newcommand{\exptwo}[1]{\oldexp_2 \left( #1 \right)}

\newcommand{\softmax}{\mathrm{smax}}

\DeclareMathOperator*{\expectedop}{\mathbb{E}}
\DeclareDocumentCommand{\expected}{u{_} m}{
	\expectedop\limits_{\mathrlap{#2}}
}

\begin{document}

\numdef{\N}{8}
\numdef{\labelwidth}{5.5cm}
%%%%%%%%%%%%%%%%%%%%%%%%%
% Encoder

\begin{tikzpicture}[]

\begin{scope}
	\node(lblEncoder)[text width= \labelwidth] {\textbf{RNN Encoder:}\\%
	Variable $n$ inputs: $\nv x_t$\\%
	1 output: $\hat{y}$ \\%
	};
	
	\coordinate (L0) at (lblEncoder.east);
	
	\foreach \I[count=\j from 0] in {1,...,\N}{
		\ifnumequal{\I}{\N - 1}{%
			\node(L\I)[dashed, layer, right = of L\j] {...};
			\node(w\I)[below = of L\I]{...};
		}%
		{	
			\node(L\I)[layer, right = of L\j] {RU};
			\node(w\I)[below = of L\I]{\ifnumequal{\I}{\N}{$\nv x_n$}{$\nv x_\I$}};
			\draw[->](w\I) -- (L\I);
		}
	}
	\foreach \I[count=\j from 1] in {2,...,\N} {
		\draw[->] (L\j) edge node[labe]{state} (L\I);
	}
	
	\node(out) [above = of L\N]{$\hat{y}$};
	\draw[->] (L\N) -- (out);
\end{scope}

%%%%%%%%%%%%%%%%
% Decoder
\begin{scope}[yshift=-5cm] 
\node(lbl)[text width= \labelwidth] {\textbf{RNN Decoder:}\\%
1 input: $x$\\%
variable $m$ outputs: $\n \hat{y}_t$ \\%
with prompts: $\nv r_t$ (often $\nv y_{t-1}$)
};

\coordinate (L0) at (lbl.east);
\coordinate (L1c)[right = of L0];
\node(x)[below right = 4 of L1c]{$x$};

\foreach \I[count=\j from 0] in {1,...,\N}{
	\ifnumequal{\I}{\N - 1}{%
		\node(L\I)[dashed, layer, right = of L\j] {...};
		\node(w\I)[above = of L\I]{...};
		\node(y\I)[below = of L\I]{...};
	}%
	{	
		\node(L\I)[layer, right = of L\j] {RU};
		\node(v\I)[above = of L\I]{\ifnumequal{\I}{\N}{$\n \hat{y}_m$}{$\n \hat{y}_\I$}};
		\node(w\I)[below = of L\I]{$[\nv r_\I; \v x]$};
		\draw[->](w\I) -- (L\I);
		\draw[->](L\I) -- (v\I);
		\draw[->](x) to[bend right = 5] (w\I.300);
	}
}
\foreach \I[count=\j from 1] in {2,...,\N} {
	\draw[->] (L\j) edge node[labe]{state} (L\I);
}

\end{scope}
%
%%%%%%%%%%%%%%%%%%
% Encoder Decoder
%
\begin{scope}[yshift=-14cm]
\node(lbl)[text width= \labelwidth] {\textbf{RNN Encoder-Decoder:}\\%
Variable $n$ inputs: $\nv x_t$\\%
Variable $m$ outputs $\n \hat{y}_t$\\%
Prompts: $\nv r_t$ (often $y_{t-1}$)
};

\coordinate (L0) at (lbl.east);
\numdef{\NN}{4}
\foreach \I[count=\j from 0] in {1,...,\NN}{
	\ifnumequal{\I}{\NN - 1}{%
		\node(L\I)[dashed, layer, right = of L\j] {...};
		\node(w\I)[below = of L\I]{...};
	}%
	{
		\node(L\I)[layer, right = of L\j] {$\mathrm{RU_E}$};
		\node(w\I)[below = of L\I]{\ifnumequal{\I}{\NN}{$\nv x_n$}{$\nv x_\I$}};
		\draw[->] (w\I) -- (L\I);
	}
}
\foreach \I[count=\j from 1] in {2,...,\NN} {
	\draw[->] (L\j) edge node[labe] {state} (L\I);
}




\coordinate[above = 3 of L\NN] (Lp\NN);
\numdef{\NP}{\N - 1}
\foreach \j in {\NN,...,\NP}{
	\numdef{\I}{\j+1}
	\numdef{\y}{\I - \NN}
	\ifnumequal{\I}{\N-1}{%
		\node(Lp\I)[dashed, layer, right = of Lp\j] {...};
		\node(w\I)[below = of Lp\I]{...};
		\node(y\I)[above = of Lp\I]{...};
	}%
	{
		\node(Lp\I)[layer, right = of Lp\j] {$\mathrm{RU_D}$};
		\ifnumequal{\I}{\N}{
			\node(w\I)[below = of Lp\I]{$[\v z; \nv r_m]$};
			\node(y\I)[above = of Lp\I]{$\n \hat{y}_m$};
		}
		{
			\node(w\I)[below = of Lp\I]{$[\v z; \nv r_\y]$};
			\node(y\I)[above = of Lp\I]{$\n \hat{y}_\y$};
		}

		\draw[->] (w\I) -- (Lp\I);
		\draw[->] (Lp\I) -- (y\I);
		\path[->] (L\NN.north) edge node[labe]{$\v z$} (w\I.south west);
	}
}


\numdef{\NNp1}{\NN + 1}
\foreach \I in {\NNp1,...,\NP} {
	\numdef{\j}{\I+1}
	\draw[->] (Lp\I) edge node[labe] {state} (Lp\j);
}
 

\end{scope}


\end{tikzpicture}




\end{document}}
\end{figure}

The majority of this section draw on the recurrent neural networks (RNN) as discussed in \Cref{sec:rnn}.
We can used RNN encoders and decoders (as shown in \Cref{fig-rnns-sq}) to generate representations of sequences by extracting a coding layer.
In one can take any RNN encoder,
and select the one of the hidden layers after the final recurrent unit (RU) that has processed the last word in the sentence.
Similarly for any RNN decoder, one can select any hidden layer before the first recurrent unit that begins to produce words.
For an RNN encoder-decoder, this means selecting the hidden layer from between.
As for example in \tcite{cho-EtAl:2014:EMNLP2014}.


\tcite{Bowman2015SmoothGeneration} presents an extension on this notion,
where in-between the encoder and the decode stages there is a variational autoencoder (VAE).
The variational autoencoder \pcite{2014VAE} an autoencoder that has been demonstrated to have very good properties in a number of machine learning applications.
Using the VAE it is hoped that a better representation can be found for the sequence of words in the input and output.


\begin{figure}
	
	\numdef{\n}{6}
	\numdef{\labelwidth}{5.5cm}
	%%%%%%%%%%%%%%%%%%%%%%%%%
	% Encoder
	\resizebox{\textwidth}{!}{
	\begin{tikzpicture}[]
	\coordinate (L0) at (lbl.east);
	\numdef{\nn}{\n/2}
	\foreach \i[count=\j from 0] in {1,...,\nn}{
		\ifnumequal{\i}{\nn - 1}{%
			\node(L\i)[dashed, layer, right = of L\j] {...};
			\node(w\i)[below = of L\i]{...};
		}%
		{
			\node(L\i)[layer, right = of L\j] {$\mathrm{RU_E}$};
			\node(w\i)[below = of L\i]{\ifnumequal{\i}{\nn}{$w_n$}{$w_\i$}};
			\draw[->] (w\i) -- (L\i);
		}
	}
	
	\numdef{\nvae}{\nn+1}
	\node(L_\nvae)[right=of L\nn] {VAE};
	\numdef{\nd}{\nvae+1}
	
	\numdef{\np}{\nd - 1}
	\foreach \j in {\nvae,...,\np}{
		\numdef{\i}{\j+1}
		\numdef{\y}{\i - \nn}
		\ifnumequal{\i}{\n-1}{%
			\node(L\i)[dashed, layer, right = of L\j] {...};
			\node(w\i)[below = of L\i]{...};
			\node(y\i)[above = of L\i]{...};
		}%
		{
			\node(L\i)[layer, right = of L\j] {$\mathrm{RU_D}$};
			\ifnumequal{\i}{\n}{
				\node(w\i)[below = of L\i]{$r_n$};
				\node(y\i)[above = of L\i]{$\hat{y}_m$};
			}
			{
				\node(w\i)[below = of L\i]{$r_\y$};
				\node(y\i)[above = of L\i]{$\hat{y}_\y$};
			}
			
			\draw[->] (w\i) -- (L\i);
			\draw[->] (L\i) -- (y\i);		
		}
	}
	
	\foreach \i[count=\j from 1] in {2,...,\n} {
		\draw[->] (L\j) edge node[labe] {\ifnumequal{\j}{\nn}{\underline{output}}{state}} (L\i);
	}
	% 
	\end{tikzpicture}}
	
\end{figure}


\section{Structured Models}
A limitation of sequential models


\end{document}