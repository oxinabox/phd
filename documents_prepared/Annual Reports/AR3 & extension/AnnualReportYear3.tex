%% LyX 2.1.3 created this file.  For more info, see http://www.lyx.org/.
%% Do not edit unless you really know what you are doing.
\documentclass[english]{article}
\usepackage{standalone}
\usepackage{lmodern}
%\usepackage{microtype}
\usepackage{geometry}
\geometry{verbose,tmargin=2.5cm,bmargin=2cm,lmargin=1.5cm,rmargin=1.5cm}
\usepackage[english]{babel}
\usepackage{verbatim}

\usepackage{tikz}
\usetikzlibrary{positioning, fit,arrows,shapes.geometric}
\usetikzlibrary{backgrounds}
\usetikzlibrary{shadows}
\usepgflibrary{shadings}

\usepackage{xcolor}
\usepackage{booktabs}
\usepackage{enumitem}
\usepackage{pifont}


\usepackage[backend=bibtex,
style=verbose,
bibencoding=ascii,
maxcitenames=99,
url=true,
hyperref=true
]{biblatex}
\bibliography{master}
\DeclareFieldFormat{abstract}{\par Abstract: \emph{#1}}
\renewbibmacro*{finentry}{\printfield{abstract}\finentry}

\usepackage{cleveref}

\begin{document}

\title{Annual Report 2017-2018}
\author{Lyndon White}

\maketitle

\section{Summary of Research Progress to Date  (including any change in focus, and list of publications)} \label{sec:past}

The primary deviation from planning was the request to produce a book for Springer Briefs.
This did however overlap significantly with the literature review refining that was already planned as a major task.

\subsection{Publications Arising} \label{sec:pubs}

\subsubsection{Accepted}
\begin{description}
	\item[(Conference Paper) Sentences Meaning Capture] published ADCS 2015. Will be in  Part C (advantages of one or the other) of thesis.
	Bibliographic details\\ \fullcite{White2015SentVecMeaning}
	\item[(Conference Paper) SOWE2BOW] published CICLing 2016. Will be part of Part B (bridging representations)  of thesis
	Bibliographic details\\ \fullcite{White2015BOWgen}
	\item[(Conference Paper) SOWE2Sent] published ICDM HDD Workshop 2016. Will be part of Part B (bridging representations) of thesis.
	Bibliographic details\\ \fullcite{White2016a}
	\item[(Conference Paper) Word-Sense Alignment] CICLing 2018. Will be in  Part B (bridging representations)  of thesis.
	Bibliographic details\\ \fullcite{WhiteRefittingSenses}
\end{description}

\subsubsection{Under Review}
\begin{description}
	\item[(Conference Paper) ColorDist]  under-review for ACL 2018. Will be part of Part C (advantages of one or the other) of thesis. Preprint bibliographic details\\ \fullcite{2017arXiv170909360W}
\end{description}

\subsubsection{Pending}
\begin{description}
	\item[(Book): Neural Representations of Natural Language (NRoNL)] In final stage of editing. To be published as a SpringerBrief. 3 chapters of it will form Part A (literature review) of thesis.
	Bibliographic details\\ \fullcite{NRoNL}
	
	\item[(Conference Paper) NovelPersective]. First Draft Written. May tweak model slightly again. Some revision to the software required. To be submitted to ACL 2018 Software and Demonstration Track. Will be part of Part C (advantages of one or the other) of thesis. Will also be presented to users at the Western Australian Regional Science Fiction Convention (Swancon), as part of there non-archival academic stream.
	Bibliographic details\\ \fullcite{novelperspective}
	
	\item[(Short Journal Paper) DataDeps.jl] 2nd Draft Written. To be submitted to JMLR Software Track will be part of Part D (tooling), or appendix of thesis. This will also be the basic of a presentation ``Foundational Tools for Data Driven Research, with Applications in NLP and ML'' to be submitted to the (non-archival) Julia Language Conference (JuliaCon).
	Bibliographic details\\ \fullcite{DataDeps}
	
	\item[(Unknown) Nonnegatively weighted neural networks]{I had begun to prepare a theory paper on this, however I discovered it was very similar to an existing work from serveral years early. While there remain some novelties, it may not be worth continuing at this stage as it will not be a nice fit for the thesis topic. Though it does allow for a nice extension to the the ColorDist paper, it would be nearly completely ML focused rather than NLP focused.}
\end{description}

\clearpage

\subsection{Completion Plan}
The table below 
shows how the publications from \Cref{sec:pubs},
and the key dates  from \Cref{sec:timeline},
together make a plan to result in a completed dissertation.
Note that it is non-chronological, see the chronological time-line in  \Cref{sec:timeline}.


\vspace{1cm}
\begin{tabular}{l c c r}
	\toprule[1pt]
	Part & Publication & Publication Status & Finalisation Date\\
	\midrule[1pt]
	Introduction / Conclusion etc. & --- & --- & 30/7/18\\
	\midrule
	Part A: Literature Review & Chapters from NRoNL Book & Submit 9/3/18 & 17/6/18\\
	\midrule
	Part B: Bridging Embeddings to  Classical & SOWE2BOW & Pub. 2016 & \\
											  & SOWE2Sent & Pub. 2016 & \\
											  & Word-Sense Alignment & Pub. 2018 & 14/5/18\\
											  
	\midrule
	Part C: Embeddings vs Classical & Sentences Meaning Capture & Pub. 2015  & \\
									& ColorDist Estimation & In Review & \\
									& NovelPerspective & Submit 23/3/18 & 28/5/18\\
	\midrule
	Part D/Appendix & DataDeps.jl & Submit 15/4/18 & 24/8/18 \\
	\midrule
	\textbf{Overall}  & --- & --- & \textbf{8/9/18} \\

	\bottomrule[1pt]

\end{tabular}
\vspace{1cm}

The majority of the publications for the thesis are now written.
Some still need some editing before they are submitted.
They will be grouped into the 4 parts (A-D),
which will have introductions and conclusions written for them.
The finalisations of these parts will be the predominant task for the remainder of the candidature.
The dates given for finalisation here, and in the detailed time-line include the time taken to receive and incorporate supervisor feedback.



\clearpage
\subsubsection{Detailed Time-line} \label{sec:timeline}
\newcommand{\travel}{\textcolor{orange!70!black}{\ding{40}}}
\newcommand{\hard}{\textcolor{purple!70!black}{\ding{72}}}
\newcommand{\soft}{\textcolor{blue!70!black}{\ding{74}}}
\newcommand{\outside}{\textcolor{green!70!black}{\ding{73}}}

Many of the events shown in this timeline are for my own planning purposes.
Beyond the events marked as tasks (\hard{} and \soft{}), 
The remainder are for informational purposes, and my own planning,
e.g. as they impact upon my availability for the points that are core to the completion plan.

\paragraph{Key} For the symbols in the Timeline
\begin{description}
	\item[\hard{}] Task with externally mandated deadline
	\item[\soft{}] Task with self-imposed deadline
	%\item[\outside{}] Externally controlled event
	\item[\travel{}] Conference or similar restriction on availability.	
\end{description}


\paragraph{Timeline}
for the remainder of candidature.
\newcommand{\timepoint}[3]{#1 & #3 & #2 \\} 
\newcommand{\timepointm}[3]{\timepoint{#1}{#2}{#3} \hline}


\vspace{1cm}
\begin{tabular}{|l c | r|}
	
	%\toprule[1pt]
	\hline
	\timepoint{}{Deadline}{Task/Event}
	\hline
	%\midrule[1pt]
	\timepointm{\hard{}}{9/3/18} {Neural Representations of Natural Language Book, submitted to SpringerBriefs.}
	\timepointm{\hard{}}{23/3/18} {NovelPersective Conference Paper submitted to ACL 2018 Demo. Track.}
	\timepointm{\travel{}}{18-24/3/18} {CICLing Conference}
	\timepointm{\travel{}}{29-2/3/18} {Swancon Convention/Easter Break}
	\timepointm{\soft{}}{15/4/18} {DataDeps.jl submitted to JMLR Software Track}
	\timepointm{\soft{}}{29/4/18} {Dissertation Document Created, with papers all in place, but blank introductions. }
	\timepointm{\hard{}}{30/4/18} {JuliaCon Conference abstract due}
	\timepointm{\soft{}}{14/5/18} {Dissertation Part B (bridging representations) preparation/finalisation complete.}
	\timepointm{\soft{}}{28/5/18} {Dissertation Part C (advantages of one or the other) preparation/finalisation complete.}
	\timepointm{\soft{}}{17/6/18} {Dissertation Part A (literature review) written based on book}
	\timepointm{\travel{}}{15-19/7/18} {ACL Conference (if accepted)}
	\timepointm{\soft{}}{30/7/18} {Dissertation Overall Introduction/Conclusion written}
	\timepointm{\travel{}}{7-11/8/18} {JuliaCon Conference (if accepted)}
	\timepointm{\hard{}}{18/8/18} {Nomination of Examiners submitted to GRS}
	\timepointm{\soft{}}{24/8/18} {Dissertation Part D (tooling)/Appendix preparation/finalisation complete.}
	\timepoint{\hard{}}{8/9/18} {Thesis submitted}
%	\timepointm{\outside{}}{17-30/11/18} {Potential dates for Oral Defence (optional)}
%	\timepoint{\outside{}}{1-14/12/18} {Rough time that Examiner's Response is expected}
	
	\hline
	
\end{tabular}
\vspace{2cm}

\hfill \large{Signed} \hfill ........................ (student) \hfill\hfill ........................ (supervisor) \hfill\null

%\begin{description}[leftmargin=8em,style=nextline, align=right]
	
%\end{description}
\clearpage

\subsection{Problems Encountered So Far}

\begin{itemize}
	\item Supervisor  Dr Wei Liu left on maternity leave at the end of 2016 until start of 2018. This meant complete loss of her important input on research for
	3 months,  and only limited ability to get her input for the remainder of the year. This resulted in one project needing to be abandoned after ~1 month of
	work as it required that input; and more generally, Dr Liu's feedback is a very valuable part of the research training program for my studies.
	
	\item The NeCTAR allocation which I had been using as the primary compute for my research was due to expire on the 13th of September 2017. However, due to the Pawsey centre's unexpected de-federation,
	it was instead terminated on the 5th of July 2017, which severely interrupted the middle of a larger series of work. (I was further expecting that I would be able to renew it, as I had previously for 12 more months after September) 
	It was not possible to move to 	the new Nimbus cloud (should I have succeeded in an application to that) as Nimbus does not support the Swift Object Storage which I was
	making heavy use of. The computations had to be closed off, and I still do not have access to compatible Swift Object Storage.
	As such, when it becomes necessary to to extend upon that work, time will have to be spent rewriting a nontrivial portion of the research software.
	
	\item  A  repeated series of interruptions to computation has occurred due to the multiple power failures in the EE building in the last 12
	Months. These have been due to the storm in August, the constructions works from October onwards, and more recently unexplained power glitches.
	Due to failures of communication, even the "planned" power outages were unexpected and resulted in the termination of long running computations.
	
	%\item During the color investigation, I noticed some very interesting properties of non-negatively weighted neural networks. I spent several weeks investigating those mathematically. It was only at the end of this period that I properly understood what these property were enough to discover other literature on it. It turns out several of my results are not novel, in particular the ones I had written the most proofs for. While no doubt several of the other results are are novel and the research could be salvaged, and would allow for interesting extensions of the ColorDist paper (and more generally of density estimation.), at present given the time remaining, and their limited relevance to the core of the thesis, it has been determined not to continue with that line of research.
	
	\item The Neural Representations of Natural Language Book was not a planned publication. It was an unexpected opportunity to create a book (on the request of the publisher), at what is an excellent time for such a book to come into existence; while also creating the literature review required for the thesis. The more extensive time required for the more extensive work, however, did delay other works.
	
	
\end{itemize}



\newpage


\end{document}
