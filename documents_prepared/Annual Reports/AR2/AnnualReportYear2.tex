%% LyX 2.1.3 created this file.  For more info, see http://www.lyx.org/.
%% Do not edit unless you really know what you are doing.
\documentclass[english]{article}
\usepackage{standalone}
\usepackage{lmodern}
\usepackage{microtype}
\usepackage{geometry}
\geometry{verbose,tmargin=2.5cm,bmargin=2cm,lmargin=1.5cm,rmargin=1.5cm}
\usepackage[english]{babel}
\usepackage{verbatim}

\usepackage{tikz}
\usetikzlibrary{positioning, fit,arrows,shapes.geometric}
\usetikzlibrary{backgrounds}
\usetikzlibrary{shadows}
\usepgflibrary{shadings}

\usepackage{xcolor}
\usepackage{pgfgantt}


\usepackage[backend=bibtex,
style=verbose,
bibencoding=ascii,
maxcitenames=99,
url=true,
hyperref=true
]{biblatex}
\bibliography{master}
\DeclareFieldFormat{abstract}{\par Abstract: \emph{#1}}
\renewbibmacro*{finentry}{\printfield{abstract}\finentry}

\usepackage{cleveref}

\begin{document}

\title{Annual Report 2016-2017}
\author{Lyndon White}

\maketitle

\section{Summary of Research Progress to Date  (including any change in focus, and list of publications)} \label{sec:past}

Work has proceeded, with some deviation from the Project Proposal. 
In some areas of work poor preliminary findings encouraged efforts to be redirected to more promising areas.
In other cases circumstances outside our control rendered continuation of a project impractical.

\subsection{Sentence Generation from Sum of Word Embeddings}
This work was largely carried out in late 2015/ early 2016.
The publications were finalised and presented in 2016.

\subsubsection*{Publications Arising}
This covers the first half of going from Sum of word embedding, to a bag of words as a conference paper. This received the Best Student Paper award at CICLing 2016. 

\vspace{1cm}
\fullcite{White2015BOWgen}
\vspace{1cm}

The second publication, covers the remaining step to go from the bag of words to an ordered sentence; and the evaluation of the overall systems.
This was presented as a workshop paper at ICDM.

\vspace{1cm}
\fullcite{White2016SOWE2Sent}
\vspace{1cm}

\subsection{Characteristic Vector Autoregression}

It was intended that in this time, the next project to be worked on was Characteristic Vector Autoregression.
Early investigations were not promising.
At around this time I had the opportunity to speak to the guest presenter Dr Hien Nguyen, at the Big Data Master Class.
He has had some experience using these models, and confirmed that there was little chance of them proving useful for anywhere near the number of dimensions commonly used for word embeddings.
On this basis, that line of inquiry was terminated.

\subsection{Word Sense Embeddings}
It had become clear that word embeddings were the key building block of the larger expression of which the core inquiry focused upon.
A key limitation of them, for this use, was using a single representation for a word which can (and normally will) have multiple meanings.
Without considering this, it seems that any larger work would be bounded in it upper limits.
Towards improving understanding of this area, and building the tools required for further work multiple sense word embeddings.

Initial efforts towards defining a new method for word sense induction were met with frustration, due to numerical instability.
A side avenue of how to leverage pretrained embeddings  toward new uses proved more fruitful.
This has been prepared for publication, and is currently under review.

\vspace{1cm}
\fullcite{WhiteRefittingSenses}
\vspace{1cm}

\subsubsection{Meaning of Water Usage}
It was intended to extend this work on determining the multiple meanings of words to the domain of time series utilities usage.
The reason for a person using a particular amount of water at a particular time can be at least partially inferred from the context.
By using similar contextual features as was used to discover word senses, so too could the meanings of water uses be isolated.

This work heavily relied on Dr Liu's expertise with this kind of data;
as this expertise was understandably rendered in-accessible when Dr Liu left on maternity leave, the decision was made to place this work on hold.

\subsection{Colour Description}\label{sub:colour}
To extend the work on words with multiple meanings into longer phrases,
colour descriptions have been chosen as a constrained domain for that investigation.
The description of colours, for example ``bluish green'' vs ``dark teal``, is a subset of a language with many of the interesting features of a whole language.
Color description it is a microcosm of language; with a ground truth on meaning (albeit a weak one).
In particular it has large amounts of multiple sense words such as ``tan'' which can be any number of shades from where it overlaps with the senses of ``terracotta``, to where it overlaps with ``beige''.
It also features significant structural patterns, such as the use of ``ish'' as a term to combine two colours.
The intent is to investigate both the generation of colour from its description,
and of the description from the colour.

As well as having theoretic interest as a demonstration of the concepts;
such research has application as a component of in user input systems (such as intent recognition), and in natural language generation systems (such as captioning). It also has direct applications as a teaching tool for English language learners, and as an accessibility aid for the visually impaired.


\subsection{Problems}


There were several difficulties in configuring and setting up the new GPU work station.
The initial setup took longer than expected (INC0867929);
due to incompatibilities with the UWA RedHat SOE.
Until this issue could be resolved by the faculty IT;
work continued by using the earlier used CPU based workstations/NeCTAR, for the projects that had anticipated using the new hardware.
This did have some effect of increasing time taken in investigation.

When moving over to the new set up once it was complete, there were further issues with having the correct versions of software (INC0966100).
This took much longer than expected to be resolved, as its timing coincided with the new student intake; which faculty IT had to prioritise.
It has now been largely overcome, as I have now built and installed some of the software myself.

Dr Liu went on maternity leave in December 2016 and loss of ready access to Dr Liu's advice and expertise has been an issue since then. Currently it is planned to have meetings with Dr Liu in the coming weeks but her availability will be necessarily restricted.




\newpage

\section{Completion Plan}
The gantt chart shown in \Cref{gantt} is updated from that submitted with the first annual report. It shows the tasks remaining to be completed; and the tasks completed. 


\subsection{Tasks Completed (2016-2017)}
As the thesis is by publication, the key progress indicator towards textual completion, in the publications.
At this point in total 3 publications have been published, with a 4th currently in review.
These publications will form key chapters in the thesis.

For detail on the tasks completed see \Cref{sec:past}.


\subsection{Tasks Remaining (2017-2018)}
\subsubsection*{Colour Description (Jan 2017--Jun 2016)}
See \Cref{sub:colour}.


\subsubsection*{Structured Models (Jul 2017--Dec 2017)}
This work combined the previously planned Structural Search, and Higher Order models; based on greater understanding.
It will consider the possible analysis words in a non-left-to-right structure, for the better capturing of meaning, by considering how each word shifts the possible meanings of the other. This will be considered for a generative application.

The investigation will initially focus on  color descriptions, and then will be generalised to apply to arbitrary paraphrases.
Both forms of data have ground truth on the similarity of their meanings.
In the case of colour descriptions this is in the form of the HSV, HSL, or XY representations.
In the case of paraphrases this is only a binary relationship, either a sentence is a paraphrase or it is not.

This work will thus complete the circle of research by linking back to the original work on Semantic Evaluation which focused on determining the similarity of paraphrases.

\subsubsection*{Dissertation Compilation (Jan 2018 -- Mar 2018)}
In this time the papers produced during the candidature will be compiled into the thesis.
This will involve large scale updating and refining the literature review produced for the research proposal.
For this, a lot of information will be extracted from the literature review sections of the papers making up the dissertation, and from the reviews prepared during their investigations.

Dissertation preparation will also involve enhancing the supplementary materials from the prior papers.
Due to the tight page limits of the publication venues in this area,
most of the papers have associated supplementary material which has been posted online.
For example, the paper ``Generating Bags of Words from the Sum of their Word Embeddings'' has a mathematical proof of the NP-Hardness of the problem, which is as almost long as the paper itself.
These materials will either be expanded into chapters, 
or used to enhance the papers they were supplementing.



\begin{figure}[h!]
	\centering
	\resizebox{0.9\textwidth}{!}{\centering
		\documentclass{standalone}
\usepackage{tikz}
\usetikzlibrary{positioning, fit,arrows,shapes.geometric}
\usetikzlibrary{backgrounds}
\usetikzlibrary{shadows}
\usepgflibrary{shadings}

\usepackage{xcolor}
\usepackage{pgfgantt}

\begin{document}

\ganttset{
group/.append style={fill=none},
milestone/.append style={},
%
ms/.style={milestone left shift=0.8,
	milestone right shift=0.2,
	milestone/.append style={fill=purple}},
%
paper/.style={milestone left shift=1.9,
	milestone right shift=-0.9,
	milestone/.append style={violet, circle}},
%
halfslot/.style={milestone left shift=2.9},
}

\pgfdeclarelayer{background}
\pgfdeclarelayer{foreground}
\pgfsetlayers{background,main,foreground}

\tikzset{node distance=1.5ex}
\tikzset{ms/.style={circle, minimum height=11pt}}
\begin{ganttchart}[%Specs
     y unit title=0.5cm,
     y unit chart=1.1cm,
     x unit = 1.5cm,
     vgrid={
     	*2{draw=none}, *1{dotted}, *1{red, dashed}
     	},
     hgrid,
     title height=1,
%     title/.style={fill=none},
     title label font=\bfseries\footnotesize,
     bar/.style={fill=blue},
     bar height=0.7,
%   progress label text={},
%
     group right shift=0,
     group top shift=0.7,
     group height=.3,
     group peaks width={0.2},
     newline shortcut=true,
     bar label node/.append style={align=right},
     canvas/.append style={fill=none},
   	 today = 9,
   	 today rule/.append style={red},
   	 today label=Work Complete,
]
{2}{14}

\gantttitle{2015}{3} 
\gantttitle{2016}{4} 
\gantttitle{2017}{4}
\gantttitle{2018}{2}
\\
\foreach \ii in {1, ..., 3}{
	\gantttitle{Apr--Jun}{1}
	\gantttitle{Jul--Sep}{1}
	\gantttitle{Oct--Dec}{1}
	\gantttitle{Jan--Mar}{1}
}
\gantttitle{Apr--Jun}{1}



\ganttgroup{Preparation}{2}{3} \\
\ganttbar{Review of\\ Literature}{2}{3}
\ganttbar[bar top shift=0.6pt, bar height=0.1pt, bar/.append style={fill=blue!20,}]{}{4}{12}\\
\ganttbar{Determine Focus\\ of Research}{2}{2} \\
\ganttbar{Investigate/Implement\\ Current Methods}{2}{2} \\


\ganttgroup{Research Tasks}{0}{0} \\
\ganttset{bar/.append style={fill=green!90!black}}
\ganttbar{Semantic\\ Evaluation}{3}{3} 
\ganttmilestone[paper, inline]{}{3}
\\
\ganttbar{Sum of Word\\ Embeddings}{4}{5}
\ganttmilestone[paper, halfslot,inline]{}{4}
\ganttmilestone[paper,inline]{}{5}
 \\
 \ganttbar{Word-Sense \\ Embeddings}{6}{7} 
\ganttmilestone[paper]{}{7}
\\
\ganttbar{Meaning\\ of Water}{8}{8} 
\\

\ganttbar{Color\\ Description}{9}{10} 
\ganttmilestone[paper]{}{10}

\\


\ganttbar{Structured\\ Models}{11}{12}
\ganttmilestone[paper]{}{12}
\ganttmilestone[paper]{}{11}
 \\


\ganttgroup{Milestones}{0}{0} \\
\ganttset{bar/.append style={fill=purple}}
\ganttmilestone[ms]{Research Proposal}{3}\\
\ganttmilestone[ms]{Annual Reports}{5}
\ganttmilestone[ms]{}{9}\\
\ganttbar{Dissertation \\Compilation}{13}{14}\\
\ganttmilestone[ms]{Submission}{14}

\begin{pgfonlayer}{foreground}
	\node(legend) at ([yshift=-100pt, xshift=-120pt]current bounding box.north east){Key:};
	\node(milestone)[below = 2ex of legend.west,ms, fill=purple, shape=diamond]{};
	\node[right = of milestone]{Milestone};
	\node(paper)[below = of milestone, ms, fill=violet] {};

	\node(lbl_paper)[right = of paper]{Paper Draft};
 \end{pgfonlayer}

\begin{pgfonlayer}{main}
	\node[drop shadow, fill=white, rounded corners=1pt, draw, fit = (legend) (lbl_paper)(paper)] {};
 \end{pgfonlayer}
\end{ganttchart}


\end{document}

	}
	\caption{\label{gantt} an Updated Gantt Chart}
\end{figure}




\end{document}
