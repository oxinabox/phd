\documentclass[12pt,landscape,english]{beamer}


\usetheme[crossbullets]{uwa}
%You can remove crossbullets option to have normal beamer bullets etc
\usepackage{etoolbox, verbatim}

\graphicspath{{../figs/}}

%%%%%%%%%%%%%%%TIKZ
\usepackage{tikz}
\usetikzlibrary{positioning, fit,arrows,chains,shapes.geometric}
\usetikzlibrary{graphs,graphdrawing}
\usetikzlibrary{petri, shapes}
\usetikzlibrary{calc}
\usegdlibrary{force, layered, trees}
%%%%%%%%%%%%%%%%%%


%%%%%%%%%%%%%Bibliography
\usepackage[backend=bibtex, url=false,
bibstyle=ieee,firstinits=true]{biblatex}
\bibliography{master.bib}
\renewcommand*{\thefootnote}{} %No symbol or marker
\renewcommand{\footnotesize}{\scriptsize}
%%%%%%%%%%%%%%%%%



\mode<presentation>
%\setbeameroption{show notes}


\begin{document}

% Titlepage info

\title[White et al.]{How Well Sentence Embeddings Capture Meaning}
\author[White et al.]{Lyndon White \and Roberto Togneri \and Wei Liu \and Mohammed Bennamoun}
\institute[UWA]{The University of Western Australia}
\date{\includegraphics[scale=0.4]{ss}}

\begin{frame}[plain]
	\titlepage
\end{frame}



\section{Introduction}


\begin{frame}{Vector space should be partioned the same as sentence space}
 \includegraphics[scale=0.6]{equiv}
\end{frame}

\begin{frame}{When are sentences equivalent?}
	$A$ and $B$ are semantically equivalent iff $A\models B\:\wedge\:B\models A$
	\begin{columns}
		\begin{column}{0.5\textwidth}
			\begin{itemize}
				\note[item]{ $A$ cannot be true without $B$ also being true}
				\note[item]{and $B$ cannot be true without $A$ also being true}
								
				\item<1-> a bidirectional entailment relationship
				\note[item]{As bidirectional they must have the same truth value}
				\item<2> Semantically equivalent sentences are called \alert{paraphrases}
				\item<3> This defines equivalence classes over all sentences
			\end{itemize}
		\end{column}
		\begin{column}{0.5\textwidth}
			
		\end{column}
	\end{columns}
\end{frame}

\begin{frame} {PV DM} 
	\hskip-1cm
	\resizebox{!}{0.8\textheight}{\input{../figs/pvdm/PVDM1.pgf}}
	\footfullcite{le2014distributed}
\end{frame}


\begin{frame} {PV DM} 
	\hskip-1cm
	\resizebox{!}{0.8\textheight}{\input{../figs/pvdm/PVDM2.pgf}}
	\footfullcite{le2014distributed}
\end{frame}


\begin{frame} {PV DM}
	\hskip-1cm
	\resizebox{!}{0.8\textheight}{\input{../figs/pvdm/PVDM3.pgf}}
	\footfullcite{le2014distributed}
	
	\note[item]{This is a nice method. Now we have vectors for sentences, or paragraphs. As well as for words.}
\end{frame}

\begin{frame} {PV DM}
	\centering\resizebox{!}{0.4\textheight}{\input{../figs/pvdm/PVDM3_no_label.pgf}}
	\vskip3ex
	\begin{itemize}
		\item Uses Paragraph Vector as ``Memory'' Slot
		\item Useful Embedding for Word Sequence
		\item No Linguistic Structures Use
	\end{itemize}	
	
\end{frame}






\section{Conclusion}
\begin{frame}{Proposal}
	\begin{itemize}
		\item A Method is Proposed
		\begin{itemize}
			\item Described In Detail
			\item as a plan to be carried out
		\end{itemize}
		\item The Method is good
		\begin{itemize}
			\item Pleasing
			\item Affordable
			\begin{itemize}
				\item in time
				\item in budget 
			\end{itemize}
		\end{itemize}
		\item There is no other
	\end{itemize}

\end{frame}








\end{document}
