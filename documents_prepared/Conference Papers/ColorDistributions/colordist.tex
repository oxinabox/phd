\documentclass[11pt,a4paper]{article}
\usepackage{acl2018}
\usepackage{times}
%\usepackage[author={Lyndon}]{pdfcomment}

\usepackage{booktabs}
\usepackage{pgfplotstable}
\pgfplotsset{compat=1.14}

\usepackage{url}


\usepackage{tikz}
\usetikzlibrary{positioning, fit,  shapes.geometric}
\usepackage{graphicx}

\graphicspath{{./figs/}, {./}}

\makeatletter
\setlength{\@fptop}{0pt}
\setlength{\@fpbot}{0pt plus 1fil}
\makeatother


\usepackage[subpreambles=false]{standalone}

\usepackage{amssymb}
\usepackage{amsmath}
\usepackage{mathtools}
\DeclareMathOperator*{\argmin}{arg\,min}
\DeclareMathOperator*{\argmax}{arg\,max}

\usepackage[subtle]{savetrees}
\setlength{\abovedisplayskip}{0pt}
\setlength{\belowdisplayskip}{0pt}
\newcommand{\compactmath}[1]{\noindent\resizebox{\columnwidth}{!}{$#1$}}

\usepackage{cleveref}

%%%%%%%%%%%%%%%%%%%%%%%%%%%%%%%%%%%%
\usepackage{url}
\usepackage{natbib}
\bibliographystyle{acl_natbib}

\newcommand{\parencite}{\citep}
\newcommand{\textcite}{\citet}


%opening
\title{Learning Distributions of Meant Color}
%\author{Lyndon White \\ lyndon.white@research.uwa.edu.au %
%	\and Roberto Togneri \\ roberto.togneri@uwa.edu.au%
%	\and Wei Liu \\ wei.liu@uwa.edu.au %
%	\and Mohammed Bennamoun \\ mohammed.bennamoun@uwa.edu.au %
%}


\begin{document}

\maketitle

\begin{abstract}
When a speaker says the name of a color, the color that they picture is not necessarily the same as the listener imagines.
Color is a grounded semantic task, but that grounding is not a mapping of a single word (or phrase) to a single point in color-space.
Proper understanding of color language requires the capacity to map a sequence of words to a probability distribution in color-space.
A distribution is required as there is no clear agreement between people as to what a particular color describes -- different people have a different idea of what it means to be ``very dark orange''.
Further, such a model must learn an understanding of how each word in the sequence contributes to the color being described.
It is impossible to collect data for every possible sequence of color words.
A system is required that solves both these issues.
To achieve this, we propose a novel GRU-based model, which when given such a color description produces an estimated probability distribution across HSV color-space.
This model learns how each word in the color name comes together to determine the shape of the distribution.
This allows the prediction of distributions for color description phrases never seen in training.
\end{abstract}

\section{Introduction}\label{sec:intro}

Color understanding is an important subtask in natural language understanding.
It is a challenging domain, due to ambiguity, multiple roles taken by the same words, the many modifiers, and shades of meaning.
Due to its difficulty, texts containing color descriptions such as \texttt{the flower has petals that are bright pinkish purple with white stigma} are used as demonstrations for state of the art image generation systems \parencite{reed2016generative, 2015arXiv151102793M}.
The core focus of the work we present here is addressing these linguistic phenomena around the descriptions of the color in a single patch as represented in a color-space such as HSV \parencite{smith1978color}.
Issues of illumination and perceived color based on visual context are considered out of the scope.


Consider that the word \texttt{tan} may mean one of many colors for different people in different circumstances ranging from the bronze of a tanned sunbather, to the brown of tanned leather;
\texttt{green} may mean anything from \texttt{aquamarine} to \texttt{forest green};
 and even \texttt{forest green} may mean the rich shades of a rain-forest, or the near grey of the Australian bush.
Thus the color intended cannot be uniquely inferred from the color name.
Without further context, it does remain possible  to estimate likelihoods of which colors might be intended based on the knowledge of the population's use of the words.
The primary aim of this work is to map a sequence of color description words to a probability distribution in a color-space such as HSV.
This is required for a proper understanding of color language.
This is in contrast to other recent work where terms are mapped to a single point in color-space \parencite{DBLP:journals/corr/KawakamiDRS16}.


Estimating color distributions has a clear use as a subsystem in many systems.
For example, in a human-interfacing image-processing system, when asked to select the \texttt{dark bluish green} object, each object can be ranked based on how likely its color is according to the distribution.
This way if extra information eliminates the most-likely object, the second most likely object can immediately be determined.
Further, if the probability of the color of the object being described by the user input is known, a threshold can be given to report that no object is found, or to ask for additional information.


Proper understanding requires considering \emph{the color intended} as a random variable.
In other words, a color name should mapt to a distribution, not just a single point or region.
For a given color name, any number of points in the color-space could be intended, with some being more or less likely than others.
Or equivalently, up to interpretation, it may intend a region but the likelihood of what points are covered is variable and uncertain.
This distribution is often multimodal and non-symmetric with wide variance, which further renders regression to a single point unsuitable.
We estimate a probability distribution over the color-space.
To qualify our estimate of the distribution we discretize the color-space into a large number of patches, and produce an output much like a histogram.
This allows us to take advantage of the well-known methods for estimating a probability mass distribution using a neural network with a softmax output layer.


This understanding of color language also requires a model capable of understanding linguistic compositionality.
It must understand how modifiers such as \texttt{dark} modify basic colors; and  how other modifiers such as \texttt{very} would interact as a modifier to modifiers.
It also must understand the functioning of affixes such as \texttt{-ish} in \texttt{greenish}.
This compositional understanding is needed both as a point of theory and practically.
Practically, the generalisation ability from a compositional model allows it to handle color descriptions not seen in training.
Due to the combinatorial nature of language a data-sparsity problem exists that
for a large number of word combinations there are few examples in any given corpus.
This is a well known problem in n-gram language modelling \parencite{kneser1995improved,chen1996empirical,rosenfeld2000two}.
To handle this we take inspiration from a solution used in that area: the use of a recurrent neural network \parencite{mikolov2011RnnLM} to process each color description as a compositional sequence of terms.
This allows the compositional understanding that is required for color understanding.


The core contribution of this work is a novel method for estimating probability distributions in color-space, given a color name,
 which is able to generalise to estimate distributions for color descriptions which are never seen during training.
To handle distribution estimation we employ a discretization and blurring procedure.
To allow for the capacity to predict distributions for colors never seen in training we define a GRU-based neural network to learn the compositional relationship from the term sequences describing the colors.
We call this model the Color Distribution Estimation from Sequences of Terms (CDEST) model.
As, to our knowledge, there is no existing work on estimating distributions from color-names, in order to evaluate the CDEST model we also define a histogram-based baseline method, which while lacking the generalisation capacity, more directly extracts the information from the training data.


\section{Related Work}\label{sec:related-work}
The understanding of color names has long been a concern of psycholinguistics and anthropology  \parencite{berlin1969basic,heider1972universals,HEIDER1972337,mylonas2015use}.
It is thus no surprise that there should be a corresponds field of research in natural language processing.

The earliest works revolve around explicit color dictionaries.
This includes the ISCC-NBS color system \parencite{kelly1955iscc} of 26 words, including modifiers, that are composed according to a context free grammar such that phrases are mapped to single points in the color-space;
and the simpler, non-compositional, 11 basic colors of \textcite{berlin1969basic}.
Works including \textcite{Berk:1982:HFS:358589.358606,conway1992experimental,ele1994computational, mojsilovic2005computational, menegaz2007discrete,van2009learning} which propose methods for the automatic mapping of colors to and from these small manually defined sets of colors.
We note that \textcite{menegaz2007discrete,van2009learning} both propose systems that discretize the color-space, though to a much courser level than we consider in this work.


More recent works, including the work presented here, function with much larger number of colors, larger vocabularies, and larger pools of respondents.
In particular making uses of the large Munroe dataset \textcite{Munroe2010XKCDdataset}, as we do here.
This allows a data driven approach towards the modelling.

\textcite{mcmahan2015bayesian} and \textcite{meomcmahanstone:color} present color naming methods, mapping from colors to to their names, the reverse of our task.
These works are based on defining fuzzy rectangular distributions in the color-space to cover the distribution estimated from the data, which are used in a Bayesian system to non-compositionally determine the color name.
%
%\pdfcomment{During the similar time-period another online color naming survey was conducted.
%\textcite{mylonas2010online,mylonas2012colour} collected a total of 50,000 observations from 2500 respondents in 7 languages.
%In this work we use the larger, more publically available, Munroe dataset.}
%
\textcite{2016arXiv160603821M} maps a point in the color-space, to a sequence of distributions over color terms.
They extends beyond, all prior color naming systems to produce a compositional color namer based on the Munroe dataset.
Their method uses a recurrent neural network (RNN), which takes as input a color-space point, and the previous output word, and gives a probability of the next word to be output -- this is a conditional language model.
Our proposed CDEST model is the direct inverse of their conditional language model,
CDEST use a RNN to map a sequence of color terms to a distribution over colors.


\textcite{DBLP:journals/corr/KawakamiDRS16} also propose a compositional color naming model.
They use a per-character RNN and a variational autoencoder approach.
It is in principle very similar to \textcite{2016arXiv160603821M}, but functioning on a character, rather than a word level.
The work by Kawakami et al. also includes a method for generating colors.
However it generates just single points, rather than distributions.
This has significant limitations as discussed in \Cref{sec:intro}, which our work attempts to overcome by modeling the distributions.

\textcite{DBLP:journals/corr/MonroeHGP17} presents a neural network solution to a communication game, where a speaker is presented with three colors and asked to describe one of them, and the listener is to work out which is being described.
Speaker and listener models are trained, using LSTM-based decoders and encoders respectively.
The final time-step of their model produces a 100 dimensional representation of the description provided.
From this, a Gaussian distributed score function is calculated, over a high dimensional color-space from \textcite{2016arXiv160603821M}, which is then used to score each of the three options.
While this method does work with a probability distribution, as a step in its goal,
this distribution is always both symmetric and unimodal -- albeit in a high-dimensional color-space.
To the best of our knowledge no current work proposes as a distribution estimation system such as we describe in this paper.

\section{Color~Distribution~Estimation Framework}\label{sec:method}
We define two models for the estimation of colors from textual descriptions.
A baseline histogram-based model and the GRU-based CDEST model.
The baseline model estimates the distribution based on averaging the discretized observations of colors in the training set for each input color description.
It cannot handle combinations of terms not seen during training as there is no data to average.
The CDEST model relies on using machine learning to learn the relationship between words and the color distribution; and is trained on the same observations used in the baseline model.
As it is learning a relationship between words and the color-space probability output, it can handle inputs made up of any words that were seen during training, even if the whole color description has never been used before.
Both models rely on the same assumption of conditional independence, and the same method for discretization.

\subsection{Conditional Independence Assumption}\label{sec:conditional-independence-assumption}
We make the assumption that given the name of the color, the distribution of the H, S and V channels are independent.
That is to say, it is assumed if the color name is known, then  knowing the value of one channel would not provide any additional information as to the value of the other two channels.
The same assumption is made, though not remarked upon, in \textcite{meomcmahanstone:color} and \textcite{mcmahan2015bayesian}.
This assumption of conditional independence allows considerable saving in computational resources.
Approximating the 3D joint distribution as the product of three 1D distributions decreases the space complexity from $O(n^3)$ to $O(n)$ in the discretized step that follows.

Superficial checks were carried out on the accuracy of this assumption.
Spearman's correlation on the training data suggests that for over three quarters of all color names, there is only weak correlation between the channels (\mbox{Q3 = 0.187}).
However, this measure underestimates correlation for values that have circular relative value, such as hue.
HSV had the lowest correlation by a large margin of the 16 color-spaces evaluated.
Full details, including the table of correlations, are available in supplementary materials.
These results are suggestive, rather than solidly indicative, on the degree of correctness of the conditional independence assumption.
We consider the assumption sufficient for this investigation.



\subsection{Discretization and Blurring}\label{sec:discretization-and-blurring}
The core problem is to estimate a continuous  probability distributions, conditional on the color name.
Estimating a discrete conditional distributions is a significantly more studied application of neural networks
-- this is the basic function of any softmax classifier.
To simplify the problem, we therefore transform it to be a discrete distribution estimation task, by discretizing the color-space.
Discretization to a resolution of 64 and 256 bins per channel is considered.


Discretization to resolution $n$ is the process by which a scalar observation%
\footnote{In the Munroe dataset, the provided HSV values are scaled to between 0 and 1 in all channels. We make use of this convention throughout this paper, and in our implementation.}
 $x$
from one of the continuous color channels (hue, saturation or value) is converted into an $n$-vector with the properties expected of a probability mass function.
A na{\"i}ve approach is one-hot binning:
\[\Omega_{n}^{1hot}(x)=\left(\begin{cases}
1 & \mathrm{if}\:\:\frac{i-1}{n}<x\le\frac{i}{n}\\
0 & \:\:\:\:\:\:\mathrm{otherwise}
\end{cases}\right)_{i=1}^{i=n}
\]
This gives an $n$-vector that is zero everywhere, except for the element corresponding to the patch of color-space that the value $x$ lies within.
Discretization in this way loses all notion of continuousness of the color-space.
%
In truth the distribution in color-space is intrinsically continuous -- this comes as a logical consequence of human color sensitivity being continuous \parencite{STOCKMAN1999perception}.
Points near each other in the color-space should have similar probabilities of being the intended color for a color name.
While discretization inevitably renders the space discrete, it is desirable to bring back this notion of smoothness as prior knowledge.

We enhance the training data by adding a blur during discretization.
Consider $\mathcal{D}(\mu,\sigma^{2})$ some unimodal distribution, characterised by having an expected value $\mu$ and a variance parameter $\sigma^{2}$.
For saturation and value this is a truncated Gaussian.
Hue can elegantly be handled using a wrap-around Gaussian.
We write $P_{\mathcal{D}}(y_{1}{<}Y{\le} y_{2}\mid M{=}\mu,\,\Sigma=\sigma)$ to mean the probability of a value distributed according to $\mathcal{D}(\mu,\sigma^{2})$ being in the patch bordered by $y_1$ and $y_2$.
Using this, the blurred-binning function is defined: 

\compactmath{\Omega_{n}^{blur}(x,\mathcal{D},\sigma)=\left(P_{\mathcal{D}}\left(\dfrac{i-1}{n}<Y\le\dfrac{i}{n}\mid M=x,\,\Sigma=\sigma\right)\right)_{i=1}^{i=n}}
This function maps points $x$ in the continuous color-space, to probability mass vectors of length $n$.
The majority of the mass will be in the bin that the value $x$ would be in,
but some will be shared with the bins either side, and further.

By applying more or less blurring to the training data, the priority of smoothness v.s. exact matching is controlled.
Considering the limits:
for all $\mathcal{D}$ and values $x$: 
\mbox{$\lim_{\sigma \to 0}\, \Omega_n^{blur}(x, \mathcal{D}, \sigma) = \Omega_n^{1hot}(x)$},
and \mbox{$\lim_{\sigma \to \infty}\, \Omega_n^{blur}(x, \mathcal{D}, \sigma) =  \left(\frac{1}{n}\right)_{i=1}^{i=n}$} (uniform).
A coarse parameter sweep on the value of $\sigma$ was carried out using the development portion of the dataset (see \Cref{sec:data-preparation}).
Best results were found for $\sigma = \frac{1}{2n}$.
For a training point that would be at the center of a bin, this roughly corresponds to 68.3\% of the probably mass assigned to the central bin, 15.7\% assigned to adjacent bins, and the remaining 0.3\% distributed to the remaining bins.
All results presented here are for this level of blurring.
%Further tuning of this parameter might enable better performance -- such-as using different blurring levels for each channel.

Discretizing the data is is a useful solution used in several other machine learning systems.
\textcite{oord2016pixel, DBLP:journals/corr/OordDZSVGKSK16} apply a similar discretization step and found their method to outperforming the more complex continuous distribution outputs.
These works did not employ a blurring-step.
We found the blurring step to consistently improve results for all models during preliminary investigation using the development dataset.
This is expected as a blurred discrete distribution captures some of the notions of continuity that a truly continuous output distribution would intrinsically feature.

We note that a truly continuous output is pragmatically unnecessary as 24-bit color (as was used in the survey) can have all information captured by a 256 bin quantization per channel.
24 bit color allows for a total of $2^{24}$ colors to be represented, and even 1 hot encoding for each of the 256 bin quantized channels allows for the same.


\subsection{Baseline Model}\label{sec:baseline-model}
While the main interest in this work is in compositionally modelling the color language,
we also define a non-compositional baseline model to allow for comparison.
This model loosely resembles the histogram model discussed in \textcite{meomcmahanstone:color} and \textcite{mcmahan2015bayesian}.
Existing works do not aim to estimate a general distribution, and they are therefore unsuitable for comparison.
Our baseline must be able to estimate multimodal and asymmetric color distributions.

The baseline is defined using the the element-wise mean of discretized  training observations, with add-one smoothing.
During our investigations we found that without the add-one smoothing the baseline would predict a probability of zero for some observations in the development dataset.
Applying add-one smoothing to each output distribution solves this.

For the training data $V \subset \left[ 0,1 \right] ^{3}\times T$, where $\left[ 0,1 \right] ^{3} \subset \mathbb{R}^{3}$ is the scaled HSV color-space, and $T$ is the natural language space.
The subset of the training data for the description $t \in T$ is given by
$V_{|t}=\{\tilde{v}_i \: \mid \: (\tilde{v}_i,\,t_i) \in V \wedge t_{i}=t\}$.
Per channel $c\in \lbrace H,S,V\rbrace$ the baseline model is defined by: 

\compactmath{
	q_{c}(x_{c}\mid t)=\frac{\displaystyle
		\sum_{\mathrlap{\!\!\!\!\!\!
				\forall(v_{H},v_{S},v_{V})\in V_{|t}}}
			\Omega_n^{blur}(v_{c},\mathcal{D}_c,\sigma)
			\cdot
			\Omega_n^{1hot}(x_{c})		
			+1}
		{\displaystyle \left|V_{|t}\right|+n}
}
%
In this equation taking the dot-product with $\Omega_n^{1hot}(x_{c})$ is selecting the bin containing $x_c$.
Note the distinction between $x_c$ and $v_c$: $x_c$ is the point being queried, and $v_c$ is a point from the training set.
By the conditional independence assumption the overall baseline model is given by: $q(x_H,x_S,x_V\mid t) = \prod_{c\in {\{H,S,V\}}} q_c(x_c\mid t)$


The baseline model can be used to predict distributions for all color descriptions in the training set.
This is inferior in generalisability to the CDEST model, which can handle any combination of tokens from the training set.
We suggest that the baseline model is strong and reasonable.
It is a much simpler modelling problem as it does not have a requirement to learn the how the compositional structure of the terms in the color name function.
It directly captures the information from the training set.
If the CDEST model can match its performance, that would at least show that it was capturing the information from the training data.
If it can also have similar performance for cases that do require compositional understanding (see \Cref{sec:extrapodataset}), that would show that it is indeed achieving the goal of properly modelling the language use.

\subsection{CDEST Model}

\begin{figure}
	\resizebox{\columnwidth}{!}{\documentclass{standalone}
\graphicspath{figs, .}
\usepackage{verbatim}
\usepackage{tikz}
\usetikzlibrary{positioning, fit,  shapes.geometric}
\usepackage{graphicx}

\begin{document}

\newcommand{\picwidth}{60pt}
\begin{tikzpicture}[%
	node distance=1cm and 1.5cm,
	every text node part/.style= {
		align=center
	},
	word/.style= {
		blue,
    	font=\itshape,
    },
	layer/.style= {rectangle, black,
		draw
    },
]
	
	\node (hiddenoutput)[layer] at (0,0) {ReLU};
	\node (GRU1)[layer, below = of hiddenoutput]{GRU};
	
	\foreach[count=\i from 1] \j in {2,...,4}
	{
		\node (GRU\j)[layer, left = of GRU\i]{GRU};
		\draw[->] (GRU\j) to (GRU\i);
	}
	
	\foreach[count=\i from 1] \word in {green, ish, blue, light}
	{
		\node (emb\i)[layer, below = of GRU\i]  {Embedding};
		\node (word\i)[word, below = of emb\i]{\word};
	    \draw[->] (word\i) to  (emb\i);
    	\draw[->] (emb\i) to (GRU\i);
		\node[draw,dotted,fit= (emb\i)  (word\i)  (GRU\i)] {};
	}
	
    
	\draw[->] (GRU1) to (hiddenoutput);

	\node(outHue)[layer, above left = of hiddenoutput] {Softmax};
	\node(outSaturation)[layer, above = of hiddenoutput] {Softmax};
	\node(outValue)[layer, above right = of hiddenoutput] {Softmax};

	\foreach \p in {Hue, Saturation, Value} 
	{
		\draw[->] (hiddenoutput) to (out\p);
		
		\node(plot\p)[above = of out\p, text width=\picwidth]{
			\includegraphics[width=\picwidth]{netdia/\p}
			\\
			\p
			};
		\draw[->] (out\p) to (plot\p);
	}

%

\end{tikzpicture}


\end{document}}
	
	\caption{\label{network}
		The CDEST model for predicting the color-space probability distributions of color.
		The section in the dotted-boxes is repeated for each time step.
	}
\end{figure}

The CDEST model is an RNN which learns the compositional interactions of the terms making up a color description, to output a distribution estimate in color-space.
The general structure of this network, shown in \Cref{network} is similar to \textcite{2016arXiv160603821M}, or indeed to most other word sequence learning models.
Each word is first transformed to an embedding representation.
This representation is trained with the rest of the network allowing per word information to be efficiently learned.
The embedding is used as the input for a Gated Recurrent Unit (GRU)  \parencite{cho2014properties}.
The output of the last time-step is fed to a Rectified Linear Unit (ReLU)  \parencite{dahl2013reludropout}.
Finally, the output of the ReLU is the shared input for three distinct softmax output layers -- one for each of hue, saturation and value.
These outputs are vectors $\hat{y}_{H}(t)$, $\hat{y}_{s}(t)$, and $\hat{y}_{V}(t)$.
Using the conditional independence assumption the probability estimate is given by:
\[
	\hat{p}(x_H,x_S,x_V\mid t) = \displaystyle\prod_{
		\mathclap{c\in {\{H,S,V\}}}}
	 \hat{y}_{c}(t)\cdot \Omega_{n}^{1hot}(x_c))
\]
As in the baseline model, the dot-product with $\Omega_{n}^{1hot}(x_c)$ serves to select the bin containing $x_c$.

The distinguishing features of this model compared to other word sequence learning models, is the use of GRU, rather than Long Short Term Memory (LSTM), and the three output layers.

We chose GRU as the basis of our reused structure in the recurrent network
because it has fewer parameters to learn than the more established LSTM.
It has generally been found to preform similarly well to LSTM \parencite{chung2014empirical};
including on the color naming problem \parencite{2016arXiv160603821M}.
A component for processing per-term such as the GRU, is essential in allowing the model to learn the compositional function of each term,  and thus to learn to handle color descriptions from outside the training set.

The three output layers are used to predict the discretized distributions for the three channels.
Separating them like this requires a conditional independence assumption (see \Cref{sec:conditional-independence-assumption}).
The network is trained to minimize the sum of the three cross-entropy losses for these output layers.
Similar multiple output layers as used in multitask learning \parencite{caruana1997multitask,collobert2008unified}.
The layers prior to the output are shared, allowing common knowledge to be shared.


\section{Experimental Setup}\label{sec:experimental-setup}
\subsection{Data Preparation and Tokenization}\label{sec:data-preparation}
We make use of the  Munroe dataset as prepared by \textcite{mcmahan2015bayesian} from the results of the XKCD color survey.
The XKCD color survey \parencite{Munroe2010XKCDdataset}, collected over 3.4 million observations from over 222,500 respondents.
McMahan and Stone take a subset from Munroe's full survey, by restricting it to the responses from native English speakers, 
and removing rare color names with less than 100 uses.
This gives a total of 2,176,417 observations and 829 color names. 
They also define a standard test, development and train split.


In the dataset each observation is a textual color description, paired with a point in HSV color-space.
We tokenized the textual color descriptions into separate words and affixes, using a short list of word replacement rules.
Beyond simply breaking up a description \texttt{greenish blue} into words: \texttt{greenish} and \texttt{blue}, the suffixes \texttt{-ish} and \texttt{-y} are also separated into their own tokens: \texttt{green}, \texttt{ish}, \texttt{blue}.
Hyphens are also treated as their own tokens: \texttt{blue-green} becomes \texttt{blue}, \texttt{-}, \texttt{green}.
The beginning and end of the color description is not demarcated with any form of marker token.
Using this tokenization, each description is split into up to four tokens.
This results in a total of 311 unique tokens used by the CDEST model.
The baseline model does not function per token, and so uses the original 829 descriptions directly.

\subsection{Extrapolation Sub-Dataset} \label{sec:extrapodataset}
The primary goal in constructing the CDEST model was for it to be able to to predict the distribution for never before seen descriptions of colors.
For example, based on the learned understanding of \texttt{salmon} and of \texttt{bright}, from examples like \texttt{bright green} and \texttt{bright red}, our system can suggest the distribution in the color-space of \texttt{bright salmon}, even though that description never occurs in the training data.
This would demonstrating proper compositional learning.
%
To evaluate this generalisation capacity, we define an extrapolation sub-dataset.
This is defined by selecting the rarest 100 color descriptions from the dataset,
with the restriction that every token in a selected description must still have at least 8 uses in other descriptions.
The selected examples include multi-token descriptions such as: \texttt{bright yellow green} and also single tokens that occur more commonly as modifiers than as stand-alone descriptions such as \texttt{pale}.
The test and development datasets are restricted to contain only observations of these selected color descriptions.
Conversely, the training set has no observations of these color descriptions.
This produces a dataset suitable for evaluating the capacity of our model to estimate the distributions for color descriptions not seen in training.
A similar approach was used in \textcite{DBLP:journals/corr/AtzmonBKGC16}.

\subsection{CDEST Model Parameters}
All hidden layers have width 128, except the embedding layer which has width 16.
These values were found by a coarse search of the hyper-parameters using the development dataset with the output resolution being 64 bins.
These parameters were also used for the 256 bin output resolution, though we suggest increasing the hidden layer size would give additional benefit for the higher output resolution case.
During the hyper-parameter search, it was noted that the accuracy continued to improve as the hidden layer width was increased.
However significantly diminishing returns in terms of training time v.s. accuracy lead us to limit the hidden layer sizes.
Dropout \parencite{srivastava2014dropout} with a probability of 0.5 was used during training, on all hidden layers, except the embedding layer.

\subsection{Perplexity in Color-Space}
The perplexity allows us to evaluate how well our estimated distribution matches the distribution of the observations in the test set.
Perplexity is commonly used for evaluating language models. However here it is being used to evaluate the discretized distribution estimate.
It can loosely be thought of as to how well the model's distribution does in terms of the size of an equivalent uniform distribution.
Note that this metrics does not assume conditional independence of the color channels.

Here $\tau$ is the test-set made up of pairs consisting of a color name $t$, and color-space point $\tilde{x}$;
and  $p(\tilde{x}\mid t)$  the output of the evaluated model.
Perplexity is defined:
\[
 PP(\tau) = \exp_2{\left(
 	\frac{-1}{|\tau|} 
 	\sum_{
	 		\forall(t,(\tilde{x})) \in \tau}
 	 \log_2 p(\tilde{x}\mid t)\right)}
\]
As this varies depending on the output resolution,
we define a standardized perplexity $\frac{PP(\tau)}{n^3}$, where $n$ is the per channel output resolution of the model.
The standardised perplexity allows us to compare models of different output resolutions.
It is equivalent to comparing the relative performance of the model to that of a uniform distribution $PP_{uniform}=n^3$.
Perplexity is a measure of how well the distribution, estimated by the model, matches reality according to the observations in the test set.

\subsection{Implementation}
The implementation of the CDEST and baseline models was in the Julia programming language \parencite{Julia}.
The full implementation is included in the supplementary materials.
%can be downloaded from the GitHub repository\footnote{Implementation source is at \url{https://github.com/oxinabox/ColoringNames.jl}
%TO DO Unblind this
%It makes heavy use of the MLDataUtils.jl\footnote{MLDataUtils.jl is available from \url{https://github.com/JuliaML/MLDataUtils.jl}} and TensorFlow.jl\footnote{TensorFlow.jl is available from \url{https://github.com/malmaud/TensorFlow.jl}} packages.
%, the latter of which we enhanced significantly to allow for this work to be carried out.


\section{Results and Discussion}\label{sec:results-and-discussion}
\subsection{Qualitative Comparison of the Distribution}\label{resultsdistributions}

\newcommand{\multimodalfig}[2]{
	\begin{figure}
		\includegraphics[width=\columnwidth]{multimodal/empiri256#1}
		
		\vspace{3mm}
		
		\includegraphics[width=\columnwidth]{multimodal/gru256#1}	
		\caption{\label{fig#1} Distribution estimate for \\ \mbox{\texttt{#2}}}
	\end{figure}
}

\multimodalfig{indigo}{indigo}
\multimodalfig{greenish}{greenish}
\multimodalfig{purplishgrey}{purplish grey}


Shown in \Cref{figindigo,figgreenish,figpurplishgrey} are side-by-side comparisons of the output of the CDEST and the baseline models.
Overall, it can be seen that the baseline model is has a lot more spikes,
whereas the CDEST model tends to be much smoother, even though both use the same blurring during discretization.
This smoothness is in line with the desired results from the model.
As the bin boundaries are artificial and the bins are very narrow, it is not reasonable to expect that in reality viewers have such bands of colors that they think of as more connected to the color name than their neighbours.
We expect  (as discussed in \Cref{sec:discretization-and-blurring}) this continuity, where adjacent points in color-space have similar probability values.

This smoothness can however be taken too far, when it results in the filling-in of the area between peaks for multimodal colors.
It can be seen that the CDEST model fails for some multimodal colors -- such as the hue \texttt{greenish} (\Cref{figgreenish} Hue) where the concave section is filled in;
but succeeds for others such as \texttt{purplish grey} (\Cref{figpurplishgrey}).
We attribute this to the particular difficulty of \texttt{greenish} which functions very differently as a modifier vs as a standalone color, and suggest future models may benefit from disambiguating modifiers from the head-terms of the color name, during preprocessing.


The horizontal bands in the baseline model outputs are the result of the add-one smoothing process.
Notice that they are larger for colors with fewer examples -- such as \texttt{purplish grey}.
In the seminal work of \textcite{NPLM} one of the motivations for employing neural networks in natural language processing was to better handle cases that do not occur in the training data, by sharing information between terms.
CDEST efficiently applies the same core idea here for distribution estimation.
The neural model of CDEST can, by knowledge sharing, better estimate the values for the unseen points in color-space, as compared to using smoothing.
This is distinct from, but related to, its key capacity as a compositional model to handle unseen cases in the natural language space. 


\subsection{Direct Distribution Estimation}
\pgfkeys{/pgf/number format/.cd, fixed relative,precision=3, fixed zerofill=true}
\pgfplotstableset{resultsstyle/.append style={%
	col sep=tab,
	header=has colnames,
	ignore chars={"},
	every head row/.style={before row=\toprule,	after row=\midrule},
	columns={model,resolution,perp,perpstd},
	columns/model/.style={string type},
	columns/resolution/.style={column name=$n$},
	columns/perp/.style={column name=$PP$, column type={r}},
	%columns/perphue/.style={column name=$PP_{H}$},
	%columns/perpsat/.style={column name=$PP_{S}$},
	%columns/perpval/.style={column name=$PP_{V}$},
	columns/perpstd/.style={column name=$\frac{PP}{n^3}$, column type={l}, /pgf/number format/fixed},
	%columns/msetopeak/.style={column name=$MSE$}
	}%
}

\begin{table}
	\centering
	\resizebox{\columnwidth}{!}{
		\tiny
		\pgfplotstabletypeset[resultsstyle,
			every row 2 column perpstd/.style={
				postproc cell content/.style={@cell content/.add={$\bf}{$}}
			},
			every nth row={2}{before row=\cmidrule{2-4}},
			]{results/full.tsv}
	}
	\caption{\label{tblresfull} The results of evaluation on the full Munroe  dataset. Smaller $\frac{PP}{n^3}$ is better.}
\end{table}

We first test the capacity of the model to estimate the distributions on the standard test dataset, using the standard training dataset.
We perform this evaluation before the more difficult (and important) evaluation on the extrapolation task, to confirm that the models are capable of estimating distributions.
The results are shown in \Cref{tblresfull}.
It can be seen that all models perform similarly.
The CDEST model based on sequence of color tokens, reflects the real use of the color descriptions in the test set 
just well as the non-compositional baseline, which counts the exact uses of whole descriptions.
This confirms that the CDEST model is able to learn to estimate a color distribution, and that the tokenization and sequential processing did not reduce the mapping ability of the model.

The CDEST model matches baseline performance, when trained on a full set of color terms with all combinations of terms present in the training data.
It seems there is little reason to use the CDEST model in this case, since the baseline model is simpler.
However, the key advantage of the CDEST model is its ability to predict a distribution for an unseen combination of colors. This is evaluated using the extrapolation task.

\subsection{Extrapolation to Unseen Color Names}

\begin{table}
	\centering
	\resizebox{\columnwidth}{!}{
		\pgfplotstabletypeset[resultsstyle,
			every row 0 column 0/.style={
				postproc cell content/.style={@cell content/.add={\it}{}}},
			every row 3 column 0/.style={
				postproc cell content/.style={@cell content/.add={\it}{}}},
			every nth row={3}{before row=\cmidrule{2-4}}
			]{results/extrapo.tsv}
	}
	\caption{\label{tblresextrapo} The results of evaluation on the extrapolation sub-dataset. Smaller $\frac{PP}{n^3}$ is better.}
\end{table}


A core motivation of using the CDEST model, is its ability to learn to combine tokens in a description in ways not seen in training.
This demonstrates that the model is capable of learning the compositional effects of the tokens in the color name.
That is to say learning how each token influences the final distribution -- rather than simply memorising the training data, as is done in the case of the baseline.

When it comes to the extrapolation task, the best the baseline model can do is an uniform distribution as the color descriptions in the test set do not occur in the training set.
This is an uninteresting comparison as it is always $\frac{PP}{n^3}=1.0$ (and as such is not included in \Cref{tblresextrapo}).
Thus we look to comparing the results for extrapolation to the models when they are trained without the need for extrapolation.

We compare a CDEST model trained on the extrapolation sub-dataset, to the models trained on the full dataset.
Both the non-extrapolating, and extrapolating models are evaluated on the same test set of rare color descriptions,
but the non-extrapolating models are also shown these rare descriptions during training.
The non-extrapolating models are expected to perform better given they have direct information on the rare color descriptions' distributions.
The extrapolating model must use the knowledge of how those color terms influence the color distribution without direct training.

The results for this evaluation are shown in \Cref{tblresextrapo}.
As expected, the non-extrapolating CDEST outperforms the extrapolating CDEST.
However, the decrease in performance when forced to extrapolate is relatively small.
The extrapolation results are similar to the overall results from \Cref{tblresfull}.
These are good results, indicative that the model has learnt how the terms interact to define the color distribution.
By training on uses of color terms in other descriptions the model learns these useful relationships and encodes them into the networks weights, such that when the terms are used new descriptions, the network can still estimate the distribution.
This kind of learning allows knowledge sharing between color descriptions.


The non-extrapolating CDEST also benefits from the same knowledge sharing that enables the extrapolating CDEST model to function.
This knowledge sharing allows it to outperform the baseline model, as the relationship between terms provides extra-data to better estimate the shape of the low-data curves.
The baseline model does not have such knowledge sharing, thus has difficulties in estimating  the curve of these rare descriptions.
This is notable in the high resolution case (256 bin),
where the sparsity of the training data is high enough to demonstrate the benefits of the know-sharing as shown by the extrapolating CDEST model outperforming the non-extrapolating baseline.




\section{Conclusion}\label{sec:conclusion}
We have presented the CDEST model for estimating the probably distribution of colors that may be ascribed to an input name.
For each input color name our model outputs a probability distribution over discrete regions of the color-space.
Outputting a probability distribution, rather than a single point, allows for better handling of colors with observed distributions that are asymmetric, with high variance or which are multimodal in the color-space -- which is the case for most colors.

The CDEST model learns the compositional structure of a color name, which allows it to predict distributions for color names which are not seen during training.
As the it learns how each term influences the shape of the distribution, it can thus estimate a distribution for
arbitrary compound color names, based on the learnt understanding of the individual terms.
This allows it to excel when the sparsity of training data is high.

We find that the discretization process for representing the continuous probability distribution is pragmatically effective, but unsatisfying.
While it is possible to simply fit a GMM or other continuous model to the final discretized output;
in future work would investigate the extensions of works such as \textcite{1998NNpdfDiffCdf, likas2001probability, 2017arXivKernalMixtureNetworks}..


%\subsection{Acknowledgements}
%The computational resources required for this work were generously provided by the Australian National eResearch Collaboration Tools and Resources project (Nectar),
%as well as a GPU grant from NVIDIA.
%The first author would also like to thank Ari Herman (Portland State University) with whom long-ago discussion of a related problem lead to our initial interest in this area.

\clearpage
\bibliography{master}

%\clearpage
%\appendix
%\documentclass[11pt,a4paper]{article}
\usepackage{times}
%\usepackage[author={Lyndon}]{pdfcomment}

\usepackage{booktabs}
\usepackage{pgfplotstable}
\pgfplotsset{compat=1.14}


\usepackage{amssymb}
\usepackage{amsmath}
\usepackage{mathtools}
\DeclareMathOperator*{\argmin}{arg\,min}
\DeclareMathOperator*{\argmax}{arg\,max}


\usepackage{cleveref}


\title{Learning of Colors from Color Names: \\ Distribution and Point Estimation\\Supplementary Materials}
\author{}
\date{}

\begin{document}
\maketitle

\section{On the Conditional Independence of Color Channels given a Color Name}\label{sec:corrind}

As discussed in the main text, we conducted a superficial investigation into the truth of our assumption that given a color name, the distributions of the hue, value and saturation are statistically independent.

We note that this investigation is, by no means, conclusive though it is suggestive.
The investigation focusses around the use of Spearman's rank correlation.
This correlation measures the monotonicity of the relationship between the random variables.
A key limitation is that the relationship may exist but be non-monotonic.
This is almost certainly true for any relationship involving channels, such as hue, which wrap around.
In the case of such relationships Spearman's correlation will underestimate the true strength of the relationship.
Thus, this test is of limited use in proving the conditional independence.
However, it is a quick test to perform and does suggest that the conditional independence assumption may not be so incorrect as one might assume.


For the Monroe Color Dataset training data  given by $V \subset \mathbb{R}^{3}\times T$, where $\mathbb{R}^{3}$ is the value in the color-space under consideration, and $T$ is the natural language space.
The subset of the training data for the description $t \in T$ is given by
$V_{|t}=\{(\tilde{v}_i,\,t_i) \in V \: \mid \: t_{i}=t\}$.
Further let $T_V = \{t_i \: \mid \: (\tilde{v},t_i)\in V$ be the set of color names used in the training set.
Let $V_{\alpha|t}$ be the $\alpha$ channel component of $V_{|t}$, i.e. $V_{\alpha|t} = \left\lbrace v_\alpha \mid ((v_1,v_2,v_3), t) \in V_{|t} \right\rbrace$.

The set of absolute Spearman's rank correlations between channels $a$ and $b$ for each color name is given by
$S_{ab}=\left\lbrace \left|\rho(V_{a|t},V_{b|t})\,\right|\,t\in T_{V}\right\rbrace$.
\newpage

We consider the third quartile of that correlation as the indicative statistic in \Cref{tbl:colorcor}.
That is to say for 75\% of all color names, for the given color-space, the correlation is less than this value.



Of the 16 color-spaces considered, it can be seen that the HSV exhibits the strongest signs of conditional independence -- under this (mildly flawed) metric.
More properly put, it exhibits the weakest signs of non-independence.
This includes being significantly less correlated than other spaces featuring circular channels such as HSL and HSI.

Our overall work makes the conditional independence assumption, much like n-gram language models making Markov assumption.
The success of the main work indicates that the assumption does not cause substantial issues.


\pgfkeys{/pgf/number format/.cd, fixed relative, precision=4}
\pgfplotstableset{corstyle/.append style={%
		col sep=tab,
		header=has colnames,
		columns/Color Space/.style={reset styles, string type, column name=Color-Space},
		ignore chars={"},
		every head row/.style={before row=\toprule,	after row=\midrule}
	}%
}

\begin{table}
	\centering
%	\resizebox{\columnwidth}{!}{
		\pgfplotstabletypeset[corstyle]{results/colorcor.tsv}
%	}
	\caption{\label{tbl:colorcor} The third quartile for the pairwise Spearman's correlation of the color channels given the color name.}
	
\end{table}

\section{KDE based smoothing of Training Data}\label{sec:smoothed-results}

It can be seen that smoothing has very little effect on the performance of any of the neural network based distribution estimation models.
All 3 term based models (SOWE, CNN, RNN) all perform very similarly whether or not the training data is smoothed.
This is seen consistently in all the distribution estimation tasks.
Contrast \Cref{tbl:distfull-smoothed,tbl:distord-smoothed,tbl:distextrapo-smoothed}
to the tables for the unsmoothed results
\Cref{tbl:distfull,tbl:distord,tbl:distextrapo}.

If however, smoothing is not applied to the Operational Upper Bound, it works far worse.
In  \Cref{tbl:distfull-smoothed,tbl:distord-smoothed,tbl:distextrapo-smoothed} the Direct result refers to using the training histograms almost directly, without any smoothing or term-based input processing.
This is the same as the Operation Upper Bound, minus the KDE.
It works very poorly (by comparison).
This is because the bins values are largely independent: a very high probability in one bin does not affect the probability of the adjacent bin -- which by chance of sampling my be lower than the trust distribution would have it.

This is particularly notable in the case of the direct, full training set result on the unseen combinations task reported  in \Cref{tbl:distextrapo-smoothed}. As these were some of the rarest terms in the training set, several did not coincide with any bins for observations in testing set.
This is because without smoothing it results in estimating the probability based on bins unfilled by any observation.
We do cap that empty bin probability at $\epsilon_{64} \approxeq 2\times 10^{-16}$ to prevent undefined perplexity.
(We found capping the lower probability for bins like this to be far more effective than add-on smoothing).

Conversely, on this dataset the neural network models do quiet well, with or without smoothing.
As the network can effectively learn the smoothness, not just from the observations of one color but from all of the observations.
It learns that increasing the value of one bin should increase adjacent ones.
As such it does not need the smoothing applied to the training data.

\pgfkeys{
	/pgf/number format/.cd, fixed, precision=3, fixed zerofill=true
}

\pgfplotstableset{
	col sep=comma,
	header=has colnames,
	column type={c},
	ignore chars={"},
	string replace={Direct}{\emph{Direct}},
	clear infinite,
	empty cells with={--},
	every head row/.style={before row=\toprule,	after row=\midrule},
	columns/method/.style={reset styles, string type, column name=Method},
	%
	boldcell/.style = {
		postproc cell content/.append style={
			@cell content/.add={\mathversion{tabularbold}}{}
		}
	},
	%
	greycell/.style = {
		postproc cell content/.append style={
			@cell content/.add={\color{gray}}{}
		}		
	}
}
\pgfplotstableset{distresults/.append style={%
		columns={method, perpstd},
		columns/perp/.style={column name=$PP$},
		create on use/perpstd/.style={
			create col/expr={\thisrow{perp}/(256*256*256)},
		},
		columns/perpstd/.style={column name=$\frac{PP}{256^3}$},
		%
		every row 0 column 1/.style={greycell}
	},%
	%%%%%%%%%%%%%%%%%%%%%%%%%%%%%%%%%%%%%%%%%%%%%%%%
	extrapodistresults/.append style={
		every row 0 column 1/.style={greycell},
		every row 0 column 2/.style={greycell},
		%
		columns={method, nxperpstd, xperpstd},
		columns/nxperpstd/.style={column name={\shortstack{\small Full\\Training Set\\$\frac{PP}{256^3}$}}},
		columns/xperpstd/.style={column name={\shortstack{\small Restricted\\Training Set\\$\frac{PP}{256^3}$}}},
		%		
		create on use/xperpstd/.style={
			create col/expr={\thisrow{extrapolatingperp}/(256*256*256)},
		},
		create on use/nxperpstd/.style={
			create col/expr={\thisrow{nonextrapolatingperp}/(256*256*256)},
		},
	}
}

\begin{table}
	%	\resizebox{\columnwidth}{!}{
	\pgfplotstabletypeset[distresults,
	every row 1 column 1/.style={greycell},
	]{results/smoothed/res_dist_full.csv}
	%	}
	\caption{\label{tbl:distfull-smoothed}  The results for the \textbf{full distribution estimation task} using smoothed training data. Lower perplexity (PP) is better. This corresponds to the main results \Cref{tbl:distfull}.}
\end{table}




\begin{table}
	%	\resizebox{\columnwidth}{!}{
	\pgfplotstabletypeset[distresults,
	every row 1 column 1/.style={greycell},
	]{results/smoothed/res_dist_ord.csv}
	%	}
	\caption{\label{tbl:distord-smoothed}  The results for the \textbf{order distribution estimation task} using smoothed training data. Lower perplexity (PP) is better. This is a subset of the full test set containing only tests where the order of the words matters. This corresponds to the main results \Cref{tbl:distord}.}
\end{table}


\begin{table}
	%	\resizebox{\columnwidth}{!}{
	\pgfplotstabletypeset[extrapodistresults,
	every row 1 column 1/.style={greycell},
	every row 1 column 2/.style={greycell},
	]{results/smoothed/res_dist_extrapo.csv}
	%	}
	\caption{\label{tbl:distextrapo-smoothed}  The results for the \textbf{unseen combinations distribution estimation task} using smoothed training data. Lower perplexity (PP) is better. This uses the extrapolation subset of the full test set. In the extrapolating results certain rare word combinations were removed from the training and development sets. In the non-extrapolating results the full training and development stet was used. This corresponds to the main results \Cref{tbl:distextrapo}.}
\end{table}
%%%%%%%%%%%%%%%%%%%%%%%%%%%%%%%%%%%%%%%%%%%%%%%%%%%%%%%%%%%%%%%%%%%%%%%%%%%%%%%%%%%%%%%%%%%%%%%%%%%%%%%%%%%
%
%\pgfplotstableset{pointresults/.append style={%
%		columns={method,mse},
%		columns/mse/.style={column name={$MSE$}},
%		%
%		every row 0 column 1/.style={greycell},
%		every row 4 column 1/.style={greycell},
%	},%
%	extrapopointresults/.append style={
%		columns={method, nonextrapolatingmse, extrapolatingmse},
%		columns/nonextrapolatingmse/.style={column name=\shortstack{\small Full\\Training Set\\$MSE$}},
%		columns/extrapolatingmse/.style={column name=\shortstack{\small Restricted\\Training Set\\$MSE$}},
%		%		
%		every row 0 column 1/.style={greycell},
%		every row 4 column 1/.style={greycell},
%		every row 0 column 2/.style={greycell},
%		every row 4 column 2/.style={greycell},
%	},
%}
%
%
%\begin{table}
%	%	\resizebox{\columnwidth}{!}{
%	\pgfplotstabletypeset[pointresults,
%	every row 2 column 1/.style={boldcell},
%	every row 1 column 1/.style={boldcell}
%	]{results/smoothed/res_point_comb_full.csv}
%	%	}
%	\caption{\label{tbl:pointfull-smoothed}  The results for the \textbf{full point estimation task} using smoothed training data. Lower mean squared error (MSE) is better. This corresponds to the main results \Cref{tbl:pointfull}.}
%\end{table}
%
%
%
%\begin{table}
%	%	\resizebox{\columnwidth}{!}{
%	\pgfplotstabletypeset[pointresults,
%	every row 1 column 1/.style={boldcell},
%	every row 2 column 1/.style={boldcell},
%	%3
%	%4
%	%
%	every row 5 column 1/.style={boldcell},
%	every row 6 column 1/.style={boldcell}
%	]{results/smoothed/res_point_comb_ord.csv}
%	%	}
%	\caption{\label{tbl:pointord-smoothed}  The results for the \textbf{order point estimation task} using smoothed training data. Lower mean squared error (MSE) is better. This is a subset of the full test set containing only tests where the order of the words matters. This corresponds to the main results \Cref{tbl:pointord}.}
%\end{table}
%
%
%\begin{table}
%	%	\resizebox{\columnwidth}{!}{
%	\pgfplotstabletypeset[extrapopointresults,
%	every row 1 column 1/.style={boldcell},
%	every row 2 column 2/.style={boldcell},
%	]{results/smoothed/res_point_comb_extrapo.csv}
%	%	}
%	\caption{\label{tbl:pointextrapo-smoothed}  The results for the \textbf{unseen combinations point estimation task} using smoothed training data. Lower mean squared error (MSE) is better. This uses the extrapolation subset of the full test set. In the extrapolating results certain rare word combinations were removed from the training and development sets. In the non-extrapolating results the full training and development stet was used. This corresponds to the main results \Cref{tbl:pointextrapo}.}
%\end{table}


\end{document}

\end{document}
