\documentclass[11pt,a4paper]{article}
\usepackage[hyperref]{acl2018}

\usepackage{newtxtext}
%\usepackage[author={Lyndon}]{pdfcomment}

\usepackage{booktabs}
\usepackage{pgfplotstable}
\pgfplotsset{compat=1.14}

\usepackage{url}
\usepackage[subtle]{savetrees}


\usepackage{tikz}
\usetikzlibrary{positioning, fit,  shapes.geometric}
\usepackage{graphicx}

\graphicspath{{./figs/}, {./}}


\usepackage[subpreambles=false]{standalone}

\usepackage{amssymb}
\usepackage{amsmath}
\usepackage{mathtools}
\DeclareMathOperator*{\argmin}{arg\,min}
\DeclareMathOperator*{\argmax}{arg\,max}

\newcommand{\compactmath}[1]{\noindent\resizebox{\columnwidth}{!}{$#1$}}

\usepackage{cleveref}

%%%%%%%%%%%%%%%%%%%%%%%%%%%%%%%%%%%%
\usepackage{natbib}
\bibliographystyle{apalike}

\newcommand{\parencite}{\citep}
\newcommand{\textcite}{\cite}


%opening
\title{NovelPerspective}
%\author{Lyndon White \\ lyndon.white@research.uwa.edu.au %
%	\and Roberto Togneri \\ roberto.togneri@uwa.edu.au%
%	\and Wei Liu \\ wei.liu@uwa.edu.au %
%	\and Mohammed Bennamoun \\ mohammed.bennamoun@uwa.edu.au %
%}


\begin{document}

\maketitle

\begin{abstract}
We present a proof of concept for a tool to allow consumers to subset ebooks, based on the chapters main character.
Many novels have multiple main characters, and vary with each chapter which character the story is focused on.
A well known example is George R. R. Martin's ``Game of Thrones'' novel, and others from that series.
The NovelPerspective tool detects which character the chapter is about,
and allowed the user to generate a new ebook with only those chapters.
The detection of main character can be done by many means.
We present two simple baselines, and several machine learning based methods.
\end{abstract}

\section{Introduction}

Many books have multiple main characters, often each character is written from the perspective of a different main character.
Different sections are written from the perspective of different characters.
Generally, these books are written in limited third-person point of view (POV);
that is to say the reader has little or or more knowledge of the situation described than the main character does.
Having a large cast of character, in particular POV characters, is a hallmark of the epic fantasy genre.

We propose a method here to detect the main/POV character for each section of the book.
Detecting the main character is not a difficult task, as the strong baseline result shows.
However to our knowledge there does not exist any current software to do this.
We attribute this lack to it being impractical to physically implement until recent times.
The surge in popularity of ebooks has opened a new niche for consumer discourse processing.
Tools such as the one present here, give the reader new freedoms in controlling how they consume their media.

We focus here on novels written in the limited third-person point of view.
In these stories, the main character is the point of view (POV) characters.
Some examples include: 
Across its 15 books, Robert Jordan's "Wheel of Time" series which has 146 POV characters\footnote{\url{http://wot.wikia.com/w3iki/Statistical_analysis}}). Only about one fifth of the total word count was from the POV of the ``main character''.
George R.R. Martin's "A Song of Ice and Fire", have over 30 POV characters in the books published so far \footnote{\url{http://awoiaf.westeros.org/index.php/POV_character}}.
Other well-known books meeting this requirement include:
Robert Jordan's ``Wheel of Time'' series, all the novels from Brandon Sanderson's ``Cosmere'' universe, Brent Week's ``Nightangel'' and ``Lightbringer'' series,
Steven Erikson's "Malazan Book of the Fallen" series, and thousands of others.
This is also of interest for works written in omniscient third person point of view, such as J. R. R. Tolkien's ``Lord of the Rings'',
which also may feature a focus on different main characters however the correct split is much less clear in these cases.

The utility of dividing a book in this way varies with the book in question.
Some books will cease to make sense when the core storyline crosses over different character.
Other novels, particularly the large epic fantasy stories we are primarily considering,
have many parallel story lines focused on the different characters that only rarely intersect.
While we are unable to find formal study on this, 
many readers speak of ``skipping the chapters about the boring characters'',
or ``Only reading the real main character's sections''.
Particularly on a re-read, or after already having consumed the media in some other form such as watching a movie adaptation, or reading a summary.
We note that sub-setting the novel once does not prevent the reader going back and reading the intervening chapters if it ceases to make sense, or from sub-setting again to get the chapters for another character who's path the one they are reading intersects.
We (the first author) can personally attest for some books reading the chapters one character at a time is indeed possible, and indeed pleasant: having read George R.R. Martin's "A Song of Ice and Fire" series in exactly this fashion.






\section{Character Detection Systems}

\subsection{Baseline systems: First and Most Common}
The obvious way to determine the main character of the chapter is to select the first named entity. This works well for many examples: ``It was a dark and stormy night. Bill heard a knock at the door.''; however it fails for many others ``'Is that Tom knocking on my door' thought Bill, one storm night.''.
Sometimes a chapter may go several paragraphs describing events before it even mentions the character who is perceiving them.
This is a varying element of style.

A more robust method is to use the most commonly named entity.
This works well, as once can assume the most commonly named entity is the main character.
However, it is fooled by book chapters that are, for example, about the main character's relationship with a secondary character.
In such cases the secondary character may be mentioned more often.

A better system would combine both the information about when a named entity appeared,
with a how often it occurs, and other information about how that named entity token is being used.
It is not obvious as to how these should be combined to determine which named entity chapter is about.
We thus attempt to solve it using machine learning, to combine these features to make a classifier.

\subsection{Machine learning systems}








\section{Experimental Setup}\label{sec:experimental-setup}
\subsection{Datasets}
We make uses of two series of books selected from our own personal collections.
The first four books of George R. R. Martin's ``A Song of Ice and Fire'' series (hereafter referred to as ASOIAF);
and the two books of  Leigh Bardugo's ``Six of Crows'' duology (hereafter referred to as SOC).
ASOIAF has a total of 256 chapters and 15 point of view characters.
SOC has a a total of 92 chapters and 7  point of view characters.

The requirements of the books to use in the training and evaluation of the NovelPerspective system is that they provide ground truth for the chapter's main characters.
These books do so in the chapter names -- each matching to a character name.

\subsection{Evaluation Details}
In the evaluation, the books chapters are pre-separated into body text and chapter name (character name).
The detection systems are given the body text and asked to predict the character names.
To mimic the human users ability to select multiple aliases of a character, before final classification the scores of character's nicknames are merged.
For example merging \texttt{Ned} into \texttt{Eddard}.

\subsection{Implementation}
The text is preprocessed using NLTK \parencite{bird2009natural} to added features.
The text is first tokenised, part of speech (POS) tagged , and finally named entity chunked (binary), using NLTK's default methods\footnote{(Punkt sentence tokenizer, improved TreeBank tokenizer (regex), Greedy Averaged Perceptron POS tagger, Max Entropy Named Entity Chunker)}.
The use of a binary, rather than a multi-class named entity chunker is significant.
Because fantasy novels often use ``exotic'' names for characters, we found that it often  fooled the multi-class named entity recogniser, into thinking characters were organisations or places rather than people.

XGBoost tree ensemble's are used for the machine learning methods \parencite{chen2016xgboost}.
The full implementation is available at \url{https://github.com/oxinabox/NovelPerspective/}


\section{Results and Discussion}\label{sec:results-and-discussion}


\section{Demonstration}
An online demonstration is available at \url{http://white.ucc.asn.au/tools/np}.
This is a web-app, made using the CherryPy framework\footnote{\url{http://cherrypy.org/}}.
The users uploads an ebook, and selects one of the character detection systems we have discussed above.

The users is then presented with a page displaying a list of chapters,
with the predicted main character in the left, and an excerpt from the beginning of the chapter on the right.
To avoid the user having to wait while the whole book is processed this list is dynamically loaded as it is computed.
We find that the majority of the time is spend on running the preprocessing to annotate the data before the classification.

When the whole book has been loaded and classified, the user can select chapters to download as a new ebook.
This is done via checkboxes to the right of each author
The user can input a regular expression for a character name to have the corresponding check-boxes marked.
This uses the character name as predicted by the model.
As none of the models is perfect, some mistakes are likely to be mode.
The user can then manually correct the selection using the check-boxes before downloading the book.




\section{Conclusion}\label{sec:conclusion}

\clearpage
\bibliography{master}

\end{document}
