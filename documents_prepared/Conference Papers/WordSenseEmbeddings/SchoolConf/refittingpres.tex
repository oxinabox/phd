\documentclass[dvipsnames]{beamer}
\usepackage{verbatim}

\usepackage{microtype}
\usepackage{adjustbox}
\usepackage{amsmath}

\usepackage[subpreambles=false]{standalone}
\usepackage{tikz}

\newlength\xunit


%%%%%%%%%%%%%Bibliography
\usepackage[backend=bibtex, url=false,
bibstyle=ieee,firstinits=true]{biblatex}
\renewcommand*{\thefootnote}{} %No symbol or marker
\renewcommand{\footnotesize}{\scriptsize}
%%%%%%%%%%%%%%%%%


\usepackage{xcolor}
\definecolor{chamois}{RGB}{255,255,240}
\definecolor{darkbrown}{RGB}{124,79,0}
\definecolor{UniBlue}{RGB}{83,101,130}

\definecolor{hellgelb}{rgb}{1,1,0.8}
\definecolor{colKeys}{rgb}{0,0,1}
\definecolor{colIdentifier}{rgb}{0,0,0}
\definecolor{colComments}{rgb}{1,0,0}
\definecolor{colString}{rgb}{0,0.5,0}

\usefonttheme{professionalfonts}
\usepackage{tgadventor}
\renewcommand*\familydefault{\sfdefault}
\usepackage[T1]{fontenc}
\usepackage{microtype}


\newcommand{\topline}{%
  \tikz[remember picture,overlay] {%
    \draw[brown,ultra thick] ([yshift=-1.8cm]current page.north west)-- ([yshift=-1.8cm,xshift=\paperwidth]current page.north west);} }

\renewcommand{\topline}{}

\setbeamertemplate{frametitle}{\begin{minipage}[b][1.8cm][c]{\textwidth}%
	\centering%
	\insertframetitle\\\insertframesubtitle
	\end{minipage}}
	

\addtobeamertemplate{frametitle}{}{\topline%
}

\setbeamertemplate{navigation symbols}{}
\setbeamercolor{background canvas}{bg=chamois}
\setbeamercolor{itemize item}{fg=brown}
%\setbeamertemplate{itemize item}{\maltese}
\setbeamercolor{itemize subitem}{fg=brown}
%\setbeamertemplate{itemize subitem}{\begin{rotate}{90}$\diamondsuit$\end{rotate}}

\setbeamercolor{title}{fg=UniBlue}
\setbeamercolor{frametitle}{fg=UniBlue}    
\setbeamerfont{frametitle}{size=\Large}

\setbeamercolor{author}{fg=darkbrown}
\setbeamercolor{institute}{fg=darkbrown}
\setbeamercolor{date}{fg=darkbrown}


\setbeamercolor{structure}{fg=UniBlue}
\setbeamercolor{alerted text}{fg=UniBlue}
\setbeamercolor{alerted text}{fg=UniBlue}
\setbeamercolor{normal text}{fg=darkbrown!50!black}
\setbeamercolor{math text}{fg=darkbrown}
\setbeamercolor{math text displayed}{fg=darkbrown}



\addtobeamertemplate{block begin}{%
	\setlength{\textwidth}{0.8\textwidth}%
}{}
\setbeamercolor{block title}{bg=darkbrown!40,fg=darkbrown!90}
\setbeamercolor{block body}{bg=darkbrown!20,fg=UniBlue}
\setbeamercolor{block title alerted}{bg=yellow!60,fg=red}
\setbeamercolor{block body alerted}{bg=hellgelb!80,fg=UniBlue}

\AtBeginSection[]{
	\begin{frame}
		\vfill
		\centering
		\begin{beamercolorbox}[sep=8pt,center,shadow=true,rounded=true]{title}
			\usebeamerfont{title}\insertsectionhead\par%
		\end{beamercolorbox}
		\vfill
	\end{frame}
}

\usepackage[color]{../figs/blockdiagrambits}
\tikzset{
	backgroundcolor/.style ={fill=chamois}	
}



\bibliography{master.bib}



\usepackage{booktabs, array} % Generates table from .csv
\usepackage{pgfplots, pgfplotstable}
\pgfplotsset{
	compat=1.12,
	/pgfplots/table/search path={.,..,../data}
}

\makeatletter
\pgfplotsset{
	/pgfplots/flexible xticklabels from table/.code n args={3}{%
		\pgfplotstableread[#3]{#1}\coordinate@table
		\pgfplotstablegetcolumn{#2}\of{\coordinate@table}\to\pgfplots@xticklabels
		\let\pgfplots@xticklabel=\pgfplots@user@ticklabel@list@x
	}
}
\makeatother


%%%%%%%%%%%%%%%%%%%%%%%%%%%%%%%%
\newcommand{\inputcolumn}[1]{%
	\begin{column}{0.5\textwidth}
		\begin{adjustbox}{max width=\columnwidth}
			\input{#1}
		\end{adjustbox}
	\end{column}%
}


\newcommand{\fitmath}[1]{
	\begin{adjustbox}{max width = \textwidth}	
		$\displaystyle
		#1
		$
	\end{adjustbox}
}

%%%%%%%%%%%%%%%%%%%%%%%%%%%%%%%%%%%
\newcommand{\W}{\mathcal{W}}
\renewcommand{\c}{\mathbf{c}}
\newcommand{\s}{\mathbf{s}}
\renewcommand{\l}{\mathbf{l}}
\renewcommand{\u}{\mathbf{u}}
\newcommand{\ci}{\perp\!\!\!\perp} % from Wikipedia
\DeclareMathOperator*{\argmin}{arg\,min}
\DeclareMathOperator*{\argmax}{arg\,max}

\newcommand{\ubraceword}[2]{\underbrace{\mathtt{\alert{#1}}}_{#2}\:}
\newcommand{\obraceword}[2]{\overbrace{\mathtt{\alert{#1}}}^{#2}\:}

%%%%%%%%%%%%%%%%%%%%%


\author{\textbf{Lyndon White},\\ Roberto Togneri, Wei Liu, Mohammed Bennamoun}
\institute{School of Electical, Electronic and Computer Engineering\\The University of Western Australia}
\title{Finding Word Sense Embeddings of Known Meaning}

\subtitle{A method for refitting word sense embeddings,  using a single example, by application of Bayes' theorem to the language model}
\date{}
%\logo{\hfill\includegraphics[scale=0.12]{uwa}\hfill\hspace{1.5cm}\vspace{0.5cm}}
\begin{document}
\newcommand{\sentexample}{\fitmath{
		\overbrace{
			\obraceword{wow}{w_1}
			\obraceword{the} {w_2}
			\obraceword{wool} {w_3}
			\obraceword{from} {w_4}
			\obraceword{the} {w_5}
			\ubraceword{kid} {target\: word}
			\obraceword{is} {w_6}
			\obraceword{so}{w_7}
			\obraceword{soft}{w_8}
			\obraceword{and} {w_9}
			\obraceword{fluffy}{w_{10}}
		}^{\c}
	}}
	

%\centering %Center everywhere
\frame{\maketitle}

\newcommand{\glosses}{
	\alert{Kid \emph{(Noun)}}
	\begin{enumerate}
		\item (a young person of either sex) "she writes books for children"; "they're just kids"; "`tiddler' is a British term for youngster"
		\item (English dramatist (1558-1594)) 
		\item (a human offspring (son or daughter) of any age) "they had three children"; "they were able to send their kids to college"
		\item (young goat) 
	\end{enumerate}
}


\begin{frame}{Words don't only have one meaning}
	\vspace{1em}
	\glosses
	 \footfullcite{miller1995wordnet}
\end{frame}

\begin{frame}{Word embeddings represent each word as a single vector}
	\begin{block}<1->{SkipGram Language Model:}
		\begin{itemize}
			\item \alert{Input} a word $w_T$
			\item \alert{Output} probabilities of words appearing in its context \alert{$P(w_i \mid w_T)$}
		\end{itemize}
	\end{block}
	
	\begin{block}<2->{Word Embeddings Implementation}
		\begin{itemize}
			\item Represent each input word as a \alert{vector}
			\item Train a neural network to estimate  \alert{$P(w_i \mid w_T)$}
			\item Back-prop finds values for the input vector -- i.e. good representation for the word
		\end{itemize}
	\end{block}	
	\sentexample
\end{frame}



\begin{frame}{Word sense embeddings represent each word as a multiple vectors}
	\begin{itemize}
		\item Each word has multiple senses $\{u_1, u_2, ..., u_n\}$
		\item SkipGram Language Model becomes \begin{itemize}
			\item \alert{Input} a word sense $u_i$
			\item \alert{Output} probabilities of words appearing in its context \alert{$P(w_i \mid u_j$)}
		\end{itemize}
	\end{itemize}
	\footfullcite{AdaGrams}
	\sentexample
\end{frame}

\begin{frame}{Many sense embeddings don't produce human recognisable senses}
	 \begin{itemize}
	 	\item<1-> Embeddings are learnt by modelling what words occur near the sense
	 	\item<2-> No control over the meanings of the senses \begin{itemize}
	 		\item Cover overlapping definitions
	 		\item Find overly narrow meanings
	 		\item Capture rare jargon uses
	 	\end{itemize}
		\item<3-> Useful, but not interoperable with lexical knowledge bases.	
	 \end{itemize}
\end{frame}



\begin{frame}[label=refittingslide]{We will solve this by \emph{refitting} them to be for the sense we mean}
	\begin{columns}[T]
		\begin{column}{0.5\textwidth}
			\begin{itemize}
				\item \alert{Refitting} constructs new sense embeddings out of the old.
				
				\item It uses the \alert{probabilities} of example sentence occuring.
				
				\item The new embedding aligns to the meaning in \alert{that sentence}.
			\end{itemize}			
		\end{column}
		
		\inputcolumn{../figs/refitting.tex}
		
	\end{columns}
\end{frame}




\begin{frame}{Refitting uses a probability weighted sum}
	\vspace{-1em}
	\begin{columns}[T]
	\begin{column}{0.5\textwidth}
		
		\vspace{3em}
		New sense embedding:
		\begin{equation*} \label{eq:synth}
			l(\u \mid \c ) = \sum_{\forall u_i \in \u} u_i P(u_i \mid \c)
		\end{equation*}
		
	\end{column}
	
	\inputcolumn{../figs/refitting.tex}

	\end{columns}
	\vspace{0em}
	\sentexample
\end{frame}

\begin{frame}{The probabilities are found using Bayes' theorem}
	\vspace{-1em}
	\begin{columns}[T]
		\begin{column}{0.5\textwidth}
			
			\onslide<1->{Language model: $P(w_i \mid u_{i})$}
			\vspace{1em}
			\onslide<2->{
				Conditional Independence:
				$\displaystyle%
				P(\c \mid u_{i})=\prod_{\forall w_{j}\in\c}P(w_{j} \mid u_{i})
				$
			}
			\vspace{1em}
			\onslide<3->{
				Bayes Theorem: 
				$\displaystyle%
				P(u_{i} \mid \c) =
				\dfrac{P(\c \mid u_{i})P(u_{i})}
				{\sum_{u_{j}\in\s} P(\c \mid u_{j})P(u_{j})}
				$
			}

		\end{column}

		\inputcolumn{../figs/refitting.tex}
	\end{columns}
	\vspace{0em}
	\sentexample
\end{frame}


\begin{frame}{The probabilities are found using Bayes' theorem}
	\vspace{-1em}
	\begin{columns}[T]
		\begin{column}{0.5\textwidth}
			
			\vspace{2em}
			Bayes Theorem: 
				$\displaystyle%
				P(u_{i} \mid \c) =
				\dfrac{P(\c \mid u_{i})P(u_{i})}
				{\sum_{u_{j}\in\s} P(\c \mid u_{j})P(u_{j})}
				$
				
			\vspace{2em}
			Refitted Sense Embedding: 
				$\displaystyle%
				l(\u \mid \c ) = \sum_{\forall u_i \in \u} u_i P(u_i \mid \c)
				$

			\vfill
		\end{column}
		
		\inputcolumn{../figs/refitting.tex}
	\end{columns}
	\vspace{0em}
	\sentexample
\end{frame}


\begin{frame}{The posterior distribution (over senses) is too sharp, so we smooth it}
	\vspace{-1em}
	\begin{columns}[T]
		\begin{column}{0.5\textwidth}
			\begin{block}<1->{Original:}
				\vspace{1em}
				\alert{Context Likelihood:}
				
				\vspace{1mm}
				$\displaystyle%
				P(\c \mid u_{i})=\prod_{\forall w_{j}\in\c}P(w_{j} \mid u_{i})
				$
							
				\vspace{1em}
				\alert{Sense Likelihood:}
				
				\vspace{1mm}
				$\displaystyle%
				P(u_{i} \mid \c) =
				\dfrac{P(\c \mid u_{i})P(u_{i})}
				{\sum_{u_{j}\in\u} P(\c \mid u_{j})P(u_{j})}
				$
						
			\end{block}
		\end{column}
		
		
		\begin{column}{0.5\textwidth}
			\begin{block}<2->{Smoothed:}
				\vspace{1em}
				\alert{Context Likelihood:}
				
				\vspace{1mm}
				$\displaystyle%
				P_S(\c \mid u_{i})=\prod_{\forall w_{j}\in\c} \sqrt[|\c|]{P(w_{j} \mid u_{i})}
				$
				
				\only<3->{
					\vspace{1em}
					\alert{Sense Likelihood:}
					
					\vspace{1mm}
					\fitmath{
					P_S(s_{i} \mid \c) =
					\dfrac{\sqrt[|\c|]{P(\c \mid u_{i})P(u_{i})}}
					{\sum_{u_{j}\in\u} \sqrt[|\c|]{P(\c \mid u_{j})P(u_{j})}}
					}
				}
			\end{block}			
		\end{column}		
	\end{columns}
	\vspace{0em}
	%\sentexample
\end{frame}



\newcommand{\simsentone}{%
	\fitmath{
		\overbrace{
			\obraceword{stone}{w_1}
			\obraceword{axes}{w_w}
			\obraceword{was}{w_3}
			\obraceword{a}{w_4}
			\obraceword{major}{w_5}
			\ubraceword{advance}{word\: 1}
			\obraceword{because}{w_6}
			\obraceword{it}{w_7}
			\obraceword{allowed}{w_8}
			\obraceword{forest}{w_9}
			\obraceword{clearance}{w_{10}}
		}^{\c}%
	}
}

\newcommand{\simsenttwo}{%
	\fitmath{
			\overbrace{
			\obraceword{forces}{w_1^\prime}
			\obraceword{could}{w_2^\prime}
			\obraceword{make}{w_3^\prime}
			\obraceword{no}{w_4^\prime}
			\obraceword{further}{w_5^\prime}
			\ubraceword{headway}{word\: 2}
			\obraceword{without}{w_6^\prime}
			\obraceword{resting}{w_7^\prime}
			\obraceword{and}{w_8^\prime}
			\obraceword{regrouping}{w_9^\prime}
			\obraceword{.}{w_{10}^\prime}
		}^{\c^\prime}%
	}}

\begin{frame}{Similarity with context, is the task of ranking how similar a word is, given its usage}
	
	\vfill
	\simsentone
	\vfill
	\simsenttwo
	\vfill
	
	%We use the Stanford Word Similarity in Context (SCWS) dataset to assess this 
	\footcite{Huang2012}

\end{frame}


\begin{frame}{Use for word similarity with context}
	\vspace{-1.5em}
	\begin{adjustbox}{max height=0.5\textheight}
		\documentclass{standalone}

\usepackage{tikz}
\usetikzlibrary{positioning}
\usetikzlibrary{graphs} 
%\usegdlibrary{layered}
\usetikzlibrary{graphs,graphdrawing}
\usegdlibrary{force, layered, trees}
\usetikzlibrary{decorations.pathmorphing}
\begin{document}

\begin{tikzpicture}[align=center, 	decoration={bent,aspect=.4, amplitude=5},
	note/.style= {blue,
				  font=\tiny\itshape
	},
	notepoint/.style= {->,
						note,
						dashed,
						decorate, shorten <= -32pt
	},
	section/.style = {draw,
						dashed,
						fill opacity=0.2,
						font=\itshape,
						rounded corners,
						inner sep=3mm}
]

%AlignText/"" [draw];

\graph[ layered layout, layer sep = 5mm, sibling distance=23mm,
]{
 %grow down, branch right sep]{ %

	Word1/"Word 1";
	Word2/"Word 2";
	Context1/"Context 1 \\\small (context for word 1)";
	Context2/"Context 2 \\\small (context for word 2)";


	US1/"Unsupervised \\Senses\\For Word 1";
	US2/"Unsupervised \\Senses\\For Word 2";

	Refitting1/Refitting;
	Refitting2/Refitting;

	
	Distance Measure <-  {Refitting1, Refitting2};

	Refitting2 <- Context2;
	Refitting2 <- {US2};
	Refitting1 <- {US1};
	Refitting1 <- Context1;
	US1 <- Word1;
	US2 <- Word2;
	
%	{[same layer] Word1, Word2, Context1, Context2};


};	

	
	
\end{tikzpicture}


\end{document}
	\end{adjustbox}
	\vfill
	\fitmath{
	 \begin{aligned}
	 	\mathrm{RefittedSim}((\u,\c),(\u^{\prime},\c^{\prime})) 
	 	&= d(l(\u \mid \c), l(\u^\prime \mid \c^\prime)\\
	 	\mathrm{RefittedSim}((\u,\c),(\u^{\prime},\c^{\prime}))
	 	&= d\left(
	 	\sum_{u_{i}\in\u}u_{i}P(u_{i}\mid\c),\:
	 	\sum_{u_{j}^{\prime}\in\u^{\prime}}u_{i}P(u_{j}^{\prime}\mid\c^{\prime})\right)
	 \end{aligned}
	}
	\vfill
	

\end{frame}



\pgfplotstableread[header=has colnames, string type,
	columns/AvgSimC/.style={numeric type},
	columns/RefittedSim/.style={numeric type},
]{Method	AvgSimC	RefittedSim
{Greedy \\ Unsmoothed} 0.222228482	0.406756426
{AdaGram\\Unsmoothed} 0.437692659	0.478328794
{Greedy \\ Smoothed} 0.235813998	0.497038141
{AdaGram Smoothed}	0.538424026	0.647537113
}{\swsc}


\begin{frame}[fragile]{Results on word similarity with context}
	\begin{tikzpicture} 
		\begin{axis}[
			ybar,
			width=\textwidth,height=0.8\textheight,
			ymin=0,
			ymax=1.0,
			xmin=-0.2,
			xmax=3.2,
			ylabel=$\rho$,
			xticklabels from table={\swsc}{Method},
			legend style={fill=none},
			xticklabel style={text width=8em, align=center},
			xtick = data%
			] 
%		
			\only<1->{\addplot[fill=darkbrown] table[y=AvgSimC, x expr=\coordindex]{\swsc}; }
			\only<2->{\addplot[fill=UniBlue] table[y=RefittedSim, x expr=\coordindex]{\swsc}; }
			\legend{AvgSimC, RefittedSim};
		\end{axis} 
	\end{tikzpicture}
\end{frame}



\begin{frame}{WSD is the task of determining which sense is being used}
	\glosses
	\sentexample
\end{frame}

\begin{frame}{Use of refitted senses for word sense disambiguation }
	\begin{adjustbox}{max height=0.5\textheight}
		\xunit=2.5cm
		\documentclass{standalone}
\usepackage[dvipsnames]{xcolor}
\usepackage{tikz}

\usepackage[color]{blockdiagrambits}

\renewcommand{\c}{\mathbf{c}}
\renewcommand{\l}{\mathbf{l}}
\renewcommand{\u}{\mathbf{u}}

\newlength\xunit
\xunit=1cm


\begin{document}
		

\begin{tikzpicture}[align=center]

\node[input](lemma){Target\\Lemma};
\node[input, right = 1\xunit of lemma](POS){Target\\POS Tag};
\node[input, right = 1\xunit of POS](word){Target\\ Word};
\node[datastore, below=0.75 of lemma](wordnet){WordNet \\Sense Inventory};

\node[datastore, below=0.75 of word](embeddings) {Pretrained Unsupervised \\ Sense Embeddings};
\node[proc, below=3 of POS] (refitting) {Refitting};

\node[proc, below = 2 of refitting] (WSD) {Lexical WSD};
\node[input, right = 2\xunit of word] (sentence) {Sentence};

\node[input, right = 2\xunit of WSD] (output){Disambiguated\\ Sense\\ $l^\star$};


\draw[link] (lemma) edge (wordnet);
\draw[link] (POS) edge (wordnet);
\draw[link] (word) edge (embeddings);

\draw[multi] (wordnet) -> (refitting) node[midway, labe]{Synset Glosses \\ $\{\c_1,\c_2,...\}$};
\draw[multi] (embeddings) -> (refitting) node[midway, labe]{Induced \\ Sense Embeddings \\$\u=\{u_1,u_2,...\}$};;

\draw[multi] (refitting) -> (WSD) node[midway, labe]{Lexical\\ Sense Embeddings\\ $\l=\{l_1,l_2,..\}$};
\draw[link] (sentence) edge[bend left] (WSD);
%\draw[multi, bend right] (wordnet) -> (WSD);
\triplearrow{arrows={-Implies}}{(wordnet) to[bend right] (WSD)};
\node[labe, left = 1\xunit of refitting] {Lexical\\Sense Priors\\$\{P(l_1), P(l_2),...\}$}; %hack cos benlines don't label right
\node[labe, right = 3.5\xunit of refitting] {$\c_T$}; %hack cos benlines don't label right

\draw[link] (WSD) -> (output);

\end{tikzpicture}
\end{document}
	\end{adjustbox}
	\begin{equation*}
	\begin{aligned}\label{eq:lexicalwsd}
	l^\star (\l, \c_T) &= \argmax_{\forall l_i \in \l} P(l_i|\c_T) \\
	l^\star (\l, \c_T) &= \argmax_{\forall l_i \in \l} \frac{P(\c_T \mid l_i)P(l_i)}{\sum_{\forall l_j \in \l} P(\c_T \mid l_j)P(l_j)}
	\end{aligned}
	\end{equation*}

\end{frame}




% XXXXXXXXXXXXXXXXXXXXXXXXXXXXXXXXXXXXXXXX




\pgfplotstableread[header=has colnames, string type,
columns/F1/.style={numeric type},
]{Method	F1
{Mapped\footnote{\cite{agirre2006}}\\ AdaGram} 	0.736007052
{MFS\\ Baseline}	0.7889
{Refitted\\ Greedy}	0.792860291
{Refitted\\ AdaGram}	0.798941799
}{\wsd}

\begin{frame}[fragile]{Results for word sense disambiguation}
	\begin{center}
		\vspace{-1.5em}
		\structure{SemEval 2007 Task 7}
	\end{center} 
	\begin{tikzpicture} 
	\begin{axis}[
	ybar,
	width=0.9\textwidth,height=0.8\textheight,
	%xmin=-0.2,
	%xmax=3.2,
	ylabel=F1,
	xticklabels from table={\wsd}{Method},
	legend style={fill=none},
	xticklabel style={text width=8em, align=center},
	xtick = data%
	] 
	%		
	\addplot[fill=UniBlue] table[y=F1, x expr=\coordindex]{\wsd}; 

	\end{axis} 
	\end{tikzpicture}	

    
\end{frame}

\againframe{refittingslide}




\section{Appendix}

\begin{frame}{RefittedSim vs AvgSimC}
	\vspace{-1em}
	\vfill
	\begin{block}{RefittedSim}
		\fitmath{
			\mathrm{RefittedSim}((\u,\c),(\u^{\prime},\c^{\prime}))
			= d\left(
			\sum_{u_{i}\in\u}u_{i}P(u_{i}\mid\c),\:
			\sum_{u_{j}^{\prime}\in\u^{\prime}}u_{i}P(u_{j}^{\prime}\mid\c^{\prime})\right)
		}
		
		\vspace{0.2em}
		Time Complexity: 
		$O(n\left\Vert \c\right\Vert +n^{\prime}\left\Vert \c^{\prime}\right\Vert)$
	\end{block}
	\vfill
	
	\begin{block}{AvgSimC}
		\fitmath{
			\mathrm{AvgSimC}((\u,\c),(\u^{\prime},\c^{\prime})) 
			=  \frac{1}{n \times n^{\prime}}
			\sum_{u_{i}\in\u}
			\sum_{u_{j}^{\prime}\in\u^{\prime}}
			P(u_{i}\mid\c)\,P(u_{j}^{\prime}\mid\c^{\prime})\,d(u_{i},u_{j}^{\prime})
		}
		
		\vspace{0.2em}
		Time Complexity: $O(n\left\Vert \c\right\Vert +n^{\prime}\left\Vert \c^{\prime}\right\Vert +n\times n^{\prime})$
	\end{block}
\end{frame}


\begin{frame}[fragile]{Results on word similarity with context}
	\pgfplotstabletypeset[col sep=comma, header=has colnames, string type,
	columns/Smoothing/.style={column name={$\substack{\mathrm{Geometric}\\\mathrm{Smoothing}}$}},
	columns/Use Prior/.style={column name={$\substack{\mathrm{Use}\\\mathrm{Prior}}$}},
	%		columns/Use Prior/.style={column name={\small{Use Prior}}},
	columns/AvgSimC/.style={
		%column name={\small{AvgSimC}},
		numeric type,
		precision=1,
		fixed zerofill=true,
		preproc/expr={100*##1},
		column type=c
	},
	columns/RefittedSim/.style={
		%column name={\small{RefittedSim}},
		numeric type,
		precision=1,
		fixed zerofill=true,
		preproc/expr={100*##1},
		column type=c
	},
	every row 1 column RefittedSim/.style={
		postproc cell content/.style={
			@cell content/.add={$\bf}{$}
		}
	},
	every row 0 column AvgSimC/.style={
		postproc cell content/.style={
			@cell content/.add={$\bf}{$}
		}
	},
	every head row/.style={after row = {\toprule}}
	]{swsc-grid.csv}
	
\end{frame}


\begin{frame}{Refitting sense-embeedings allows us to know the sense}
	\vfill
	\begin{itemize}
		\item New embeddings are defined as a as a \alert{weighted sum} of unsupervised embeddings.
		\item The \alert{weights} are determined using the \alert{langauge model}, with a  \alert{example sentence}.
		\item This lets us find embedding for the sense of the word in \alert{that sentence}.
		
		\item<2-> Applications for \alert{similarity with context}, and lexical tasks, such as \alert{Word Sense Disambiguation}.
	\end{itemize}
	\vfill
	\sentexample
\end{frame}

\renewcommand*{\bibfont}{\scriptsize}
\begin{frame}{References}
	\tiny
	\printbibliography
\end{frame}
	
\end{document}
