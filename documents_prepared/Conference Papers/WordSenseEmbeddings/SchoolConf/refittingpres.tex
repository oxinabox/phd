\documentclass[dvipsnames]{beamer}
%\usepackage[dvipsnames]{xcolor}
\usepackage{verbatim}
\usepackage[subpreambles=false]{standalone}
\usepackage{microtype}
\usepackage{adjustbox}


\usepackage{tikz}

\usetikzlibrary{positioning}
\usetikzlibrary{shapes} 
\usetikzlibrary{arrows.meta}
\usetikzlibrary{decorations.pathmorphing}


\tikzset{
	tripleinner/.style args={[#1] in [#2] in [#3]}{
		#1,preaction={preaction={draw,#3},draw,#2}
	},
	triple/.style={%
		tripleinner={[line width=0.5pt,black] in
			[line width=3pt,white] in
			[line width=4pt,black]}	 
	}
}

\tikzset{
	input/.style={
		%circle,
		%draw,
		%Blue,
		minimum width = 2
	},
	datastore/.style={cylinder,
		draw,
		%Green,
		minimum width = 2
	},
	proc/.style={rectangle,
		draw,
		%Red,
		minimum size = 50
	},
	link/.style={->,
		black},
	multi/.style={-Implies,
		triple,
	},
	labe/.style={
		%Blue,
		fill=white,
		fill opacity=0.6,
		text opacity=1,
		font={\footnotesize\itshape}	
	}
}

%%%%%%%%%%%%%Bibliography
\usepackage[backend=bibtex, url=false,
bibstyle=ieee,firstinits=true]{biblatex}
\renewcommand*{\thefootnote}{} %No symbol or marker
\renewcommand{\footnotesize}{\scriptsize}
%%%%%%%%%%%%%%%%%


\usepackage{xcolor}
\definecolor{chamois}{RGB}{255,255,240}
\definecolor{darkbrown}{RGB}{124,79,0}
\definecolor{UniBlue}{RGB}{83,101,130}

\definecolor{hellgelb}{rgb}{1,1,0.8}
\definecolor{colKeys}{rgb}{0,0,1}
\definecolor{colIdentifier}{rgb}{0,0,0}
\definecolor{colComments}{rgb}{1,0,0}
\definecolor{colString}{rgb}{0,0.5,0}

\usefonttheme{professionalfonts}
\usepackage{tgadventor}
\renewcommand*\familydefault{\sfdefault}
\usepackage[T1]{fontenc}
\usepackage{microtype}


\newcommand{\topline}{%
  \tikz[remember picture,overlay] {%
    \draw[brown,ultra thick] ([yshift=-1.8cm]current page.north west)-- ([yshift=-1.8cm,xshift=\paperwidth]current page.north west);} }

\renewcommand{\topline}{}

\setbeamertemplate{frametitle}{\begin{minipage}[b][1.8cm][c]{\textwidth}%
	\centering%
	\insertframetitle\\\insertframesubtitle
	\end{minipage}}
	

\addtobeamertemplate{frametitle}{}{\topline%
}

\setbeamertemplate{navigation symbols}{}
\setbeamercolor{background canvas}{bg=chamois}
\setbeamercolor{itemize item}{fg=brown}
%\setbeamertemplate{itemize item}{\maltese}
\setbeamercolor{itemize subitem}{fg=brown}
%\setbeamertemplate{itemize subitem}{\begin{rotate}{90}$\diamondsuit$\end{rotate}}

\setbeamercolor{title}{fg=UniBlue}
\setbeamercolor{frametitle}{fg=UniBlue}    
\setbeamerfont{frametitle}{size=\Large}

\setbeamercolor{author}{fg=darkbrown}
\setbeamercolor{institute}{fg=darkbrown}
\setbeamercolor{date}{fg=darkbrown}


\setbeamercolor{structure}{fg=UniBlue}
\setbeamercolor{alerted text}{fg=UniBlue}
\setbeamercolor{alerted text}{fg=UniBlue}
\setbeamercolor{normal text}{fg=darkbrown!50!black}
\setbeamercolor{math text}{fg=darkbrown}
\setbeamercolor{math text displayed}{fg=darkbrown}



\addtobeamertemplate{block begin}{%
	\setlength{\textwidth}{0.8\textwidth}%
}{}
\setbeamercolor{block title}{bg=darkbrown!40,fg=darkbrown!90}
\setbeamercolor{block body}{bg=darkbrown!20,fg=UniBlue}
\setbeamercolor{block title alerted}{bg=yellow!60,fg=red}
\setbeamercolor{block body alerted}{bg=hellgelb!80,fg=UniBlue}

\AtBeginSection[]{
	\begin{frame}
		\vfill
		\centering
		\begin{beamercolorbox}[sep=8pt,center,shadow=true,rounded=true]{title}
			\usebeamerfont{title}\insertsectionhead\par%
		\end{beamercolorbox}
		\vfill
	\end{frame}
}

\bibliography{master.bib}




%%%%%%%%%%%%%%%%%%%%%%%%%%%%%%%%
\newcommand{\inputcolumn}[1]{%
	\begin{column}{0.5\textwidth}
		\begin{adjustbox}{max width=\columnwidth}
			\input{#1}
		\end{adjustbox}
	\end{column}%
}




%%%%%%%%%%%%%%%%%%%%%%%%%%%%%%%%%%%
\newcommand{\W}{\mathcal{W}}
\renewcommand{\c}{\mathbf{c}}
\newcommand{\s}{\mathbf{s}}
\renewcommand{\l}{\mathbf{l}}
\renewcommand{\u}{\mathbf{u}}
\newcommand{\ci}{\perp\!\!\!\perp} % from Wikipedia
\DeclareMathOperator*{\argmin}{arg\,min}
\DeclareMathOperator*{\argmax}{arg\,max}

\newcommand{\ubraceword}[2]{\underbrace{\mathtt{\alert{#1}}}_{#2}\:}
\newcommand{\obraceword}[2]{\overbrace{\mathtt{\alert{#1}}}^{#2}\:}

%%%%%%%%%%%%%%%%%%%%%


\author{\textbf{Lyndon White},\\ Roberto Togneri, Wei Liu, Mohammed Bennamoun}
\institute{School of Electical, Electronic and Computer Engineering\\The University of Western Australia}
\title{Finding Word Sense Embeddings of Known Meaning}

\subtitle{A method for refitting word sense embeddings,  using a single example, by application of Bayes' theorem to the language model}
\date{}
%\logo{\hfill\includegraphics[scale=0.12]{uwa}\hfill\hspace{1.5cm}\vspace{0.5cm}}
\begin{document}
\centering %Center everywhere
\frame{\maketitle}

\begin{frame}{Word embeddings represent each word as a single vector}

\end{frame}

\begin{frame}{Word sense embeddings represent each word as a multiple vectors}
	
\end{frame}

\begin{frame}{Many sense embeddings don't know what the senses mean}
	
\end{frame}

\begin{frame}{We will solve this by \emph{refitting} them to be for the sense we mean}
	\begin{columns}[T]
		\begin{column}{0.5\textwidth}
			\begin{itemize}
				\item<1-> Take in the \alert{word}, the \alert{context}, and the \alert{pretrained senses}
				\item<2-> Output a \alert{new sense embedding} that is for the meaning present in \alert{that sentence}
			\end{itemize} 
		\end{column}
		\inputcolumn{../figs/refitting.tex}

	\end{columns}
\end{frame}


\newcommand{\sentexample}{
	\[
	\overbrace{
		\obraceword{wow}{w_1}
		\obraceword{the} {w_2}
		\obraceword{wool} {w_3}
		\obraceword{from} {w_4}
		\obraceword{the} {w_5}
		\ubraceword{kid} {target\: word}
		\obraceword{is} {w_6}
		\obraceword{so}{w_7}
		\obraceword{soft}{w_8}
		\obraceword{and} {w_9}
		\obraceword{fluffy}{w_{10}}
	}^{\c}
	\]
}

\begin{frame}{Refitting uses a probability weighted sum}
	\vspace{-1em}
	\begin{columns}[T]
	\begin{column}{0.5\textwidth}

		\begin{equation*} \label{eq:synth}
			l(\u \mid \c ) = \sum_{\forall u_i \in \u} u_i P(u_i \mid \c)
		\end{equation*}
		

		Unsupervised Senses Vectors: $\u = \{u_1,...u_{n_u}\}$ 
		
		\vspace{1em}
		
		Example Sentence: $\c = \{w_1,...w_{n_c}\}$
		 
	\end{column}
	
	\inputcolumn{../figs/refitting.tex}

\end{columns}
\vspace{0em}
\sentexample
\end{frame}

\begin{frame}{The probabilities are found using Bayes' theorem}
	\vspace{-1em}
	\begin{columns}[T]
		\begin{column}{0.5\textwidth}
			
			\onslide<1->{Language model: $P(w_i \mid s_{i})$}
			\vspace{1em}
			\onslide<2->{
				Conditional Independence:
				\begin{equation*} \label{eq:contextprobtrue}
				P(\c \mid s_{i})=\prod_{\forall w_{j}\in\c}P(w_{j} \mid s_{i})
				\end{equation*}
			}
			\vspace{1em}
			\onslide<3->{
				Bayes Theorem:
				\[ \label{eq:generalwsd}
				P(s_{i} \mid \c) =
				\dfrac{P_S(\c \mid s_{i})P(s_{i})}
				{\sum_{s_{j}\in\s} P_S(s_{j} \mid \c)P(s_{j})}
				\] 
			}
			
		\end{column}
		\inputcolumn{../figs/refitting.tex}
	\end{columns}
	\vspace{0em}
	\sentexample
\end{frame}

\begin{frame}{Use for word similarity with context}
	
\end{frame}

\begin{frame}{Results on word similarity with context}
	
\end{frame}

\begin{frame}{Use for word sense disambiguation }
	
\end{frame}

\begin{frame}{Results for word sense disambiguation}
	
\end{frame}

\begin{frame}{What went wrong? The posterior distribution was too sharp}
	
\end{frame}

\begin{frame}{Geometric Smoothing -- Replace products of likelihoods, with the geometric mean of likelihoods}
	
\end{frame}


\begin{frame}{Improved results on word similarity with context}
	
\end{frame}


\begin{frame}{Improved results  for word sense disambiguation}
	
\end{frame}


\begin{frame}{A method for refitting sense embeddings}
	
\end{frame}

	
\end{document}
