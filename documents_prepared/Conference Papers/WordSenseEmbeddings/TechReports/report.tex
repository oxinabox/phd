%% LyX 2.1.3 created this file.  For more info, see http://www.lyx.org/.
%% Do not edit unless you really know what you are doing.
\documentclass[english]{article}
\usepackage[T1]{fontenc}
\usepackage[default]{raleway}
\usepackage[utf8]{luainputenc}
\usepackage{amsmath}
\usepackage{graphicx}


\usepackage{tikz}
\usetikzlibrary{positioning}
\usetikzlibrary{graphs} 
\usetikzlibrary{graphs,graphdrawing}
\usegdlibrary{force, layered, trees}
\usetikzlibrary{decorations.pathmorphing}


\makeatletter
%%%%%%%%%%%%%%%%%%%%%%%%%%%%%% User specified LaTeX commands.
\usepackage{standalone}
\DeclareMathOperator*{\argmin}{arg\,min}
\DeclareMathOperator*{\argmax}{arg\,max}

\makeatother

\usepackage{babel}
\begin{document}
%\global\long\def\argmin{\argmin}
%\global\long\def\argmax{\argmax}


Here we look at methods for producing word sense vectors which are
aligned to a dictionary -- lexical word sense vectors, by using induced
word sense vectors; which are then used for lexical word sense disambiguation
(lexical WSD).

\begin{figure}
\documentclass{standalone}

\usepackage{tikz}
\usetikzlibrary{positioning}
\usetikzlibrary{graphs} 
%\usegdlibrary{layered}
\usetikzlibrary{graphs,graphdrawing}
\usegdlibrary{force, layered, trees}
\usetikzlibrary{decorations.pathmorphing}
\begin{document}

\begin{tikzpicture}[align=center, 	decoration={bent,aspect=.4, amplitude=5},
	note/.style= {blue,
				  font=\tiny\itshape
	},
	notepoint/.style= {->,
						note,
						dashed,
						decorate, shorten <= -32pt
	},
	section/.style = {draw,
						dashed,
						fill opacity=0.2,
						font=\itshape,
						rounded corners,
						inner sep=3mm}
]

%AlignText/"" [draw];

\graph[ layered layout, layer sep = 5mm, sibling distance=23mm,
level 1/.style={sibling distance=15mm},
%level 2/.style={sibling distance=40mm},
%level 3/.style={sibling distance=10mm},
%level 4/.style={sibling distance=10mm},
%level 5/.style={sibling distance=10mm},
%level 6/.style={sibling distance=10mm},
%level 7/.style={sibling distance=10mm}
]{
 %grow down, branch right sep]{ %
	
	Context/"Context \\\small (sentence to be\\ disambiguated)";
	WordSense/"Word Sense\\\small (The synset with \\the meaning of\\ this word in this context)";
	UWSE/"Unsupervised \\Word Sense\\ Embeddings";
	LWSE/"Lexically Aligned\\ Word Sense\\ Embeddings";

%

	{ [same layer] Lemma, POS, Word};
	
	
	Context -> WSD -> WordSense;
	
	
	""//{
		"Offline: Lexical Refitting"[section, fill=blue] // {
			{Lemma, POS} -> Synsets -> Glosses -> Refit,
			Word -> UWSE -> Refit
		},
		Refit-> LWSE -> WSD	
	};
	
	
};


%\node[note, left = 0.8 of WordSense, text width=2cm] (WSDText) {Use language model to find the probability of the context given the input of each	 lexically aligned word sense embedding, teturn the sense that gives highest probability of context};
% \node[note, left = 0.3 of LWSE] (AlignText) {Use language model to find \\ the probability of each Gloss \\ given each embeddings. \\ Use this to find a new embeddings, \\ as a weighted sum of each unsupervised embedding \\ proportional how likely the gloss was to occur \\ when it is used input to the \\language model probability};

%\draw[notepoint] (AlignText) -> (Align);
%\draw[notepoint] (WSDText) -> (WSD);



\end{tikzpicture}


\end{document}

\protect\caption{The Process for going from induced word senses to lexical labeled
word sense uses}
\end{figure}


The process broadly has two parts. 

Alignment, followed by lexical WSD.


\section{Method}


\subsection{General WSD\label{sub:General-WSD}}

We define a general method for disambiguation, suitable for use on
induced, or lexical word sense vectors.

This method is used both as a step in the alignment, and as a step
in the final lexical WSD


\subsubsection{The method is so:}
\begin{itemize}
\item for $\mathbf{c}=(w_{1},...,w_{n})$ a set of words, that is around
the word to be disambiguate (but does not include it in the central
position)
\item for $\mathbf{s}=(s_{1},...,s_{m})$ the set of induced word senses.
\item The language model provides: $P(w_{i}\mid s_{i})$ (It actually gives
a log probability, but we will ignore that)
\item we can calculate $P_{\mathbf{c}}(\mathbf{c}|s_{i})=\prod_{i=1}^{i=n}P(w_{i}|s_{i})$, 

\begin{itemize}
\item we also find a normalized probability $P_{\mathbf{c}}^{*}(\mathbf{c}|s_{i},k)=\sqrt[{k\cdot n}]{\prod_{i=1}^{i=n}P(w_{i}|s_{i})}$, 

\begin{itemize}
\item For $k$ a normalizing factor -- normally $k=1$
\end{itemize}
\item which is equal to the geometric mean of per context word probability
\end{itemize}
\item We the estimate the probability of each sense by: $P_{s}(s_{i}\mid\mathbf{c})=\dfrac{P_{\mathbf{c}}(\mathbf{c}|s_{i})}{\sum_{j=0}^{j=m}P_{\mathbf{c}}(\mathbf{c}|s_{j})}=\dfrac{\prod_{i=1}^{i=n}P(w_{i}|s_{i})}{\sum_{j=0}^{j=m}\left(\prod_{i=1}^{i=n}P(w_{i}|s_{j})\right)}$

\begin{itemize}
\item We also define a normalized probability: $P_{s}^{\ast}(s_{i}\mid\mathbf{c},\,k)=\dfrac{\sqrt[{k\cdot n}]{P_{\mathbf{c}}(\mathbf{c}|s_{i})}}{\sum_{j=0}^{j=m}\sqrt[{k\cdot n}]{P_{\mathbf{c}}(\mathbf{c}|s_{j})}}=\dfrac{\sqrt[{k\cdot n}]{\prod_{i=1}^{i=n}P(w_{i}|s_{i})}}{\sum_{j=0}^{j=m}\sqrt[{k\cdot n}]{\left(\prod_{i=1}^{i=n}P(w_{i}|s_{j})\right)}}$
\end{itemize}
\end{itemize}

\subsection{Alignment}

The alignment process takes in induced and examples of use (Eg Glosses),
and outputs lexical word sense vectors.

It can be done offline and prior to WSD.


\subsubsection{Method}

We use the General WSD defined in \ref{sub:General-WSD} to find the
probability of each word sense and then sum them.
\begin{itemize}
\item Given a example of use of a particular word sense $l$, which we will
call $\mathbf{c}_{l}$

\begin{itemize}
\item If given many examples of use we can union them into a single example.
\end{itemize}
\item and given a collection $\mathbf{s}=(s_{1},s_{2},...,s_{m})$ of induced
sense vectors 

\begin{itemize}
\item For the special case of multi-word tokens, for example \texttt{british\_airways},
for which we do not have in our vocabulary, we take the sensed for
each component word (eg \texttt{british }and\texttt{ airways}), and
union them into a single larger set of sense vectors, which we still
call $\mathbf{s}$
\end{itemize}
\item produce the new sense vector for lexical sense $l$: $s_{l}^{\ast}=\sum_{i=0}^{i=m}P_{s}(s_{i}\mid\mathbf{c})\cdot s_{i}$ 
\end{itemize}

\subsection{WSD}

WSD takes lexical word sense vectors and sentence (the context to
be disambiguate), and outputs the correct word senses for each of
the words.
\begin{itemize}
\item Given a sentence containing the word to be which we will call $\mathbf{c}_{t}$
\item and given a collection $\mathbf{s^{\ast}}=(s_{1}^{\ast},s_{2}^{\ast},...,s_{m}^{\ast})$
of lexical sense vectors for the particular word we wish to disambiguate.
\item We again apply the general WSD method, this time in a winner-takes-all
approach. The output word sense $l^{\star}=\argmax_{i\in[0,m]}P_{s}(s_{i}^{\ast}\mid\mathbf{c})\cdot s_{i}$ 
\end{itemize}

\subsection{Similarity In Context}

Another task that goes very well with the previous two is to find
the similarity of two words, when given their contexts.
\begin{itemize}
\item Given a sentence containing the word $a$ to be which we will call
$\mathbf{c}_{a}$, and for word $b$ which we call $\mathbf{c}_{b}$ 
\item and given a collection $\mathbf{s}_{a}=(s_{a1},s_{a2},...,s_{am_{a}})$
of induced sense vectors for $a$, and $\mathbf{s}_{b}=(s_{b1},s_{b2},...,s_{bm_{b}})$
for word $b$
\item We again apply the general WSD method but this this normalizing, $s_{a}^{\ast}=\sum_{i=0}^{i=m_{a}}P_{s}^{*}(s_{ai}\mid\mathbf{c},1)\cdot s_{ai}$
, and $s_{b}^{\ast}=\sum_{i=0}^{i=m_{a}}P_{s}^{*}(s_{bi}\mid\mathbf{c},1)\cdot s_{bi}$
\item We define the similarity of $a$ and $b$ in there respective contexts
as the cosine similarity of $s_{a}^{\ast}$ and $s_{b}^{\ast}$
\end{itemize}
Here we used the normalized general WSD probabilities. The motivation
of this is empirical. It roughly doubles the performance at the Similarity
In Context task over the normalized version (making scores comparable
to state of the art methods using distance to context (instead of
our probability of context)).

It might be suggested that this is because the length of the context
sentences might differ ($m_{a}$ and $m_{b}$). However the contexts
sentence in the evaluation corpus are all 100 words long (mostly with
the target word right in the center).

Relating to that: windowing the sentences to 5 words either side (like
was used for training the embeddings), does not change things; though
does improve overall similarity correctness slightly.

For different models, different values of the normalizing parameter
($k$ ) produce different results.

See below

\begin{figure}
\includegraphics[width=0.5\textwidth]{twoplots}

\protect\caption{Performance of similarity task as normalising constant ($k$) is varied.
Results for two models are shown. The horizontal axis is $k$, and
the vertical is spearman rank correlation to human similarity rankings.
Not the vertical axis is very clipped. The peak for the first is at
$k\approx0.25$ and for the second $k\approx0.75$. And I don't really
know why they are different.}
\end{figure}

\end{document}
