\documentclass{sig-alternate}
%\documentclass{article}
%\usepackage[subpreambles=true]{standalone}


\usepackage{csquotes}
\usepackage{amsmath}

\usepackage{adjustbox}

\usepackage{booktabs, array} % Generates table from .csv
\usepackage{pgfplots, pgfplotstable}

\pgfplotsset{
compat=1.12,
/pgfplots/table/search path={.,..,../data}
}


\usepackage[backend=bibtex,
style=trad-abbrv,url=false, doi=false,
sorting=none, sortcites=true]{biblatex}
\bibliography{master}

\usepackage[author={Lyndon White}]{pdfcomment}
\usepackage{cleveref}

\newcommand{\W}{\mathcal{W}}
\renewcommand{\c}{\mathbf{c}}
\newcommand{\s}{\mathbf{s}}
\renewcommand{\l}{\mathbf{l}}
\renewcommand{\u}{\mathbf{u}}
\newcommand{\ci}{\perp\!\!\!\perp} % from Wikipedia
\DeclareMathOperator*{\argmin}{arg\,min}
\DeclareMathOperator*{\argmax}{arg\,max}


\begin{document}

\title{A Method for Refitting Word Sense Vectors Using a Single Example}
\subtitle{Based on Bayes Theorem Applied to the Language Model with a Novel Smoothing Technique}
\maketitle

\begin{abstract}
Word sense embeddings are a notable extension of word embedding methods.
Sense embeddings learnt though unsupervised word sense induction, do not correspond to any dictionary sense of the word.
This limits the ability to integrate these systems with existing knowledge bases.
As an approach to overcome these issues we propose a method to find new word sense vectors from a single example.
We term this method refitting, as the new embedding is fitted to model the meaning in the example sentence, using the existing unlabelled word sense vectors.

Our contributions are threefold:
The refitting method to find new sense vectors,
a novel smoothing technique, for use with the refitting method,
and a new similarity measure for words in context, using these refitted vectors.

By way of demonstrating the utility of aligning to the lexical senses we also show how the vectors can be used to perform word sense disambiguation.
We demonstrate the effectiveness of the aforementioned techniques, by applying them to Adaptive Skip-Grams;
an embedding method that has not, until now, been shown to be useful for word similarity, or  word sense disambiguation tasks. 
\end{abstract}

\section{Introduction}


Word embeddings represent a word's semantic meaning and syntactic role in a vector space \parencite{NPLM, collobert2008unified, mikolov2013efficient}. However, each word is only given one embeddings -- which restricts it to representing only one (combined) meaning. Word sense embeddings are the generalisation of this to handle polysemous and homonymous  words. Often these word sense embeddings are learn through unsupervised word sense induction \parencite{Reisinger2010,Huang2012,tian2014probabilistic, AdaGrams}. In these cases it is not possible to directly determine which meaning belongs two which embedding. Furthermore the induced senses are unlikely to directly match to any one human defined meaning at all; but rather be more fine, or more broad, or capture the meaning of particular jargon not in common use.


It can be argued that many word sense induction (WSI) systems may capture better senses than a human lexicographers, however this does not mean induce senses can replace standard lexical senses. While the induces senses may cover the space of meanings more comprehensively, or with better granularity than standard senses there is a vast wealth of existing knowledge build around various standard lexical senses. Methods to links induced senses to lexical senses, allow this information to be unlocked.



We propose a \emph{refitting} method to allow induced word sense vectors to be converted to labelled word sense vectors, allowing them to be used with lexical knowledge bases.
We show that technique can be used to allow for word sense disambiguation to a lexical sense inventory -- something that can not be done with the original induced sense vectors.
We further demonstrate the usefulness of refitting, and this correctness of the results, by using it to define a new similarity measure for words with context which we call \emph{RefittedSim}, this similarity measure is significantly faster than the commonly used \emph{AvgSimC}.
We noted that when refitting, often one induced sense overly dominated in finding the refitted sense. We developed a new smoothing method, \emph{geometric smoothing}, suitable for smoothing these. We demonstrated its effectiveness at improving out earlier results.

We thus propose a method for using a single example of a word in context, to synthesis a new embedding for that particular sense. We term this \emph{refitting} the induced senses, as it combines then to fit to the meaning in the example sentence. Our method allows us to uses a single labelled example to produce a labelled sense embedding. This allows a shift from an representation that can only work within the system, to one that uses a standard sense, or a user defined sense, which can be used as part of a more complex knowledge engineering system. This refitting method  has several applications.


One such use is to refit word sense vectors to a lexicographical sense inventory, such as WordNet, or from a translator's dictionary -- so long as the source features at least one example of use, or a definition. The new lexically refitted word sense can then be used for Word Sense Disambiguation (WSD). Applying WSD to a adds useful information to a unstructured document, allowing further processing methods to take advantage of this information. One particular application of this would be as part of a machine translation system. To properly translate a word, the correct word sense should be determined, as different word senses in the source language, often translate to entirely different words in the target language. Using WSD to add sense annotation to a document bridges bridges between unstructured data and further more structured processing.
Details on how our method is used for WSD are in \Cref{lexicalWSD}.

Beyond refitting to a standard lexical sense, refitting to a user provided example has applications in information retrieval. Consider the natural language query \enquote{Find me all webpages about banks as in \enquote{the river banks were very muddy.}}. By generating a embedding vector for that specific sense of ``banks'' from the example sentence, and generating one from use of the word in each retrieved document, we can approximate measuring distance in semantic space. This allows for similarity ranking -- discarding irrelevant uses of the word. The method we propose, using our refitted embeddings, has lower time complexity than the current state of the art alternative. This is detailed in \Cref{RefittedSimVsAvgSimC}

Our refitting method synthesises a new word sense vector using the existing the induced word sense vectors and the language model.
The new vector is formed as a appropriately weighted sum of the original embeddings. A weighting of the induced sense vectors is determined using the language model and the example sentence. The new vector approximately corresponds to the meaning given by the example sentence. The method is detailed in \Cref{refitting}.

We noted that while our refitted word sense vectors did capture the meaning of the example sentence, often a single induced sense would dominate the refitted sense representation. We note that rarely in natural language is the meaning so clear cut. There is generally a  significant overlap between the meaning of senses, and often a high level of disagreement when humans are asked to annotate a corpus \parencite{veronis1998study}. We thus would expect this also to be true when automatically determining the appropriateness of the induced senses, during refitting. Towards this end, we develop a smoothing method, which we call \emph{geometric smoothing} that emphasises the sharp decisions made by the (unsmoothed) refitting method. We found that this significantly improves results. This suggests that the sharpness of decisions from the language model may be an artefact of the training method and the sparseness of the training data, which smoothing can correct. The geometric smoothing method is presented in \Cref{smoothing}.



We demonstrate the refitting method on Adaptive Skip-Grams (AdaGram) \parencite{AdaGrams}, and on our own simple greedy multiple word-sense embeddings. The method is generally applicable to any skip-gram-like language model that can take multiple vectors as input, and can output the probability of a word appearing in their context.



The rest of the paper is organised as follows: \Cref{relatedwords} discusses related works on learning standard lexical sense repressions directly, and on associating the induced senses with standard lexical senses. \Cref{Framework} presents our refitting methods, and the derived methods for WSD and word similarity measurement that come from it. \Cref{method} describes our models and the setup used for evaluation. \Cref{results} discusses the results of this evaluation. Finally, the paper presents it's conclusions in \Cref{conclusion}.

\section{Related Works} \label{relatedwords}

\subsection{Directly Learning Lexical Sense Embeddings}
Our refitting method can be considered as learning lexical sense embeddings, as a one-shot learning problem. The alternative is to transform the problem into a supervised or semi-supervised learning task. The difficulty lies in how to get a sufficient number of labelled senses. One option is to use a seperate WSD method to artificially label an unlabelled corpus.

Iacobacci et al. \parencite{iacobacci2015sensembed} use a Continuous Bag of Word (CBOW) \parencite{mikolov2013efficient} language model, but use word senses as the labels rather than words. This is a direct application of word embedding techniques. To overcome the lack of a large sense labelled corpus, Iacobacci et al. use the 3rd party WSD tool BabelFly, to add sense annotations to a previously unlabelled corpus. An alternative approach would be to use a method that disambiguates the senses using a partially trained model.

Chen et al. use a semi-supervised approach to train word sense vectors \parencite{Chen2014}. They partially disambiguate their training corpus, using initial word sense vectors and WordNet; and use these labels to fine-turn their embeddings. 
Initial the word sense vectors are set as the average of the single sense word embeddings\parencite{mikolov2013efficient} for the words in the WordNet gloss.
Similarly, they define a context vector, as the average of all words in a sentence.
They then progressively label the words with the senses
that for the sense vector that is closest to the context vector, if it is closer than a fixed threshold.
The selected sense vector is used to update the context vector and then the next word is relabelled. They give two methods for selecting the order of relabelling.
They then fine turn the sense embeddings by defining a new skip-gram method with the objective of predicting both the words and word-senses in the context, when given a input word\footnote{Note that using the input word to predict the sense of words from the context is the opposite of the approach used by AdaGram. AdaGram uses the word-sense to predict the word of the context \parencite{AdaGrams}.}.The requirement for meeting a threshold in order to add the sense label,  decreases the likelihood of training on an incorrect sense label. The overall approach is the process the data add labels where confidant, based on initial sense vectors estimated from the glosses, and then use these as the training target.

The key practical difference between the one-shot approach used by refitting, compared to the supervised, and semi-supervised approach uses in existing works it the time to retrain to add a new sense. With our refitting fitting approach added a new sense is practically instantiations, and replacing the entire sense inventory only a matter of hours. Where-as for the existing approaches adding senses would require retraining nearly from scratch. This is because refitting is a process done to sense word embeddings, rather a method for finding word-sense embeddings itself. It can be applied to any trained word sense embeddings, so long as it meeting the requirement for a suitable language model.
\pdfcomment{Should I mention that it could be adapted to apply to both the embeddings above, with some adaption?}

One of the key reasons, that it is useful to have senses that aligned with standard lexical senses, is that it allows word sense embeddings to be used for word sense disambiguation. This is demonstrated by Chen et al.  \parencite{Chen2014}, and we discuss how refitted word sense vectors can do the same in \Cref{eq:lexicalwsd}. If the word senses are do not correspond to the lexical senses, then to use them for WSD requires forming an a mapping from induced senses, to the lexical senses.

\subsection{Mapping induces senses to lexical senses}
For senses that do not correspond to the senses defined by a lexicography,  Agirre et al. gave a general method to use them for lexical WSD \parencite{agirre2006}.
Their method maps induces sense disambiguation scores, to scores for disambiguating standard lexical senses. This method was used for Semeval-2007 Task 02 \parencite{SemEval2007WSIandWSD} to evaluate all entries.
They use a annotated \emph{mapping corpus} to construct a mapping matrix between induced senses and the lexical senses.
This mapping is given by for $\l=\{l_1,..., l_{n_l}\}$ the set of lexically defined senses, and for $\u={u_1,...,u_{n_u}}$ the set of unsupervised induced senses:

\begin{equation} \label{eq:agirremap}
M_{i,j} = P(l_i | u_j) = \frac{count(\mathrm{method\: gives\: u_i \wedge l_i\: is\: annotated})}{count(\mathrm{method\: gives\: u_i})}
\end{equation}

An issue with estimating $P(l_i \mid u_j)$ this way is that to get a accurate estimate the law of large samples must apply to the mapping corpus, which given the issues discussed in \Cref{corpussize} may not. Given such a mapping corpus with sufficient overlap to the test corpus Agirre's method works well.
From this  when presented with a sentence containing the target word to disambiguate ($\c$) the unsupervised sense score ($P(u_j \mid \c)$ are converted to supervised scores ($P(l_j \mid \c)$) by
\begin{equation} \label{eq:agireewsd}
P(l_i \mid \c) = P(l_i | u_j) P(u_j \mid c)
\end{equation}
This is a practical method that works well, assuming the corpus is large enough to reasonably converge the estimated distribution to the true distribution, by the law of large numbers. This was the case for the small Senseval 3 English Lexical Sample \parencite{mihalcea2004senseval} initially evaluated on by Agirre et al. -- it is less clear how well it will work on more complete corpora featuring rarer words and senses.
We evaluate this method in \Cref{results}.
Their method applies to all word sense induction systems, not just word embedding systems, so is more general than the refitting approach we present.

\pdfcomment{I generalised this to make uses of what the probabilities where, solving for M using LeastSquaresReg. Performance is basically unchanged over hard decisions.}


\section{Framework} \label{Frameword}

\subsection{Refitting} \label{refitting}

The key contribution of this work is to provide a way to synthesis a word sense embedding given only a single example. For lack of a better term we call this \emph{refitting} the word sense vectors. By refitting the unsupervised vectors we define a new vector, that lines up with the specific meaning of the word from the example sentence.

This can be looked at as a one-shot learning problem.
The training of the induced sense, and of the language model, can be considered the a unsupervised pre-training step. The new word sense embedding should give a high value for the likelihood of the example sentence, under the language model. Further more though, it should generalise to also give high likelihood of other contexts that were never seen, but which also occur near the word of this particular meaning.

We attempted directly optimising the sense vector as to maximise the probability of the example when input into the language model as part of preliminary investigations. This was found to generalise poorly, due to over-fitting. It also took a significant amount of time. Rather than a direct approach, we instead take inspiration from the famous locally linear relationship between meaning and vector position demonstrated with single sense word embeddings \parencite{mikolov2013efficient,mikolovSkip,mikolov2013linguisticsubstructures}.

We express the new word sense vector as a weighted sum of the existing vectors that were already trained. Where the weight is determined by the probability of each induced sense given the context.


Given a collection of induced (unlabelled) embeddings $\u={u_1,...,u_{n_u}}$, and example sentence which we will again call $\c={w_1,...,w_{n_c}}$. We define a function $l(\u \mid \c )$ which that determines the refitted sense vector, from the unsupervised vectors and the context.

\begin{equation} \label{eq:synth}
l(\u \mid \c ) = \sum_{\forall u_i \in \u} u_i P(u_i \mid \c)
\end{equation}

To do this, we need to estimate the posterior predictive distribution $P(u_i \mid \c)$. 
This can be done simply by using Baye's Theorem, as shown in \Cref{generalwsd}. Further to that though, we present an alternative, for estimating a smoothed version of the result in \Cref{eq:generalwsdsmoothed}. We find in general that it is beneficial to smooth the distribution, to prevent a single sense dominating the sum.


In the very first neural network language model paper, Bengio et al. describe a similar method to \Cref{eq:synth} for finding the word embeddings for words not found in their vocabulary \parencite{NPLM}. They suggest that if a word was not in the training data, \enquote{an initial feature vector for such a word, by taking a weighted convex combination of the feature vectors of other words that could have occurred in the same context, with weights proportional to their conditional probability}. The formula they give is as per \Cref{eq:synth}, but summing over the entire vocabulary of words (rather than just $\u$). To the best of our knowledge, this method has not been used for handling out-of-vocabulary words in any more recent word embedding architectures, nor has it ever been used for word sense vectors.


\subsubsection {Fallback for dictionary phrases (collocations)}
Unless a specialised tokenizing method is used, a word embedding method will not learn embedding for collocations such as ``civil war'' or ``martial arts''. Normal tokenizers will split them at the word level, learning embeddings for ``civil'', and ``war'', and for ``martial'' and ``arts''. This issue is often considered minimal for word embeddings, as a approximate embedding can be constructed by summing embeddings for each word  in the phrase.

It has been constantly noted that for single-sense word embeddings the summing the vectors for each word in the phrase results in reasonable representation \parencite{mikolovSkip, White2015SentVecMeaning} \pdfcomment{Doesn't Rui Wang have a paper than goes into some detail on this? I can't find it. I tried emailing him, but no response}. The intuition from this is that for multiple sense-word embeddings, there is a correct selection of senses for each word such that the sum of these senses will be a representation for the phrase.

Simply evaluating the each combination of word-sense sums for multiple word collocations is computationally expensive. For a $m$ length collocation, where each word has $n$ senses, this requires $O(m^n)$ evaluations of probability of the context ($P(\mathbb{c}\mid u_i)$). When inducing unsupervised word embeddings it is not unreasonable to have say $n=30$ senses (or even $50$), so for a 2 word phrase (eg ``civil war'') this is 900 evaluations.
Instead, we we just add the additional sense embeddings for each word to the total pool of sense embeddings to be combined ($\u$ above), requiring instead $m \times n$ evaluations. So for our 2 word example, this is 60 evaluations, instead of the 30 evaluations that is require for a single word entry. This also has other advantages.


In the case that no word-sense induced for one word does highlight the specialised collocation information, but the other word does have a particular induced

For example, ``civil war''. The context that ``war'' in ``civil war'' occurs, is very similar to the context that ``war'' appears in general. With both ``war'' and ``civil war'' we expect the context to include words like ``casualties'', ``militia'' etc, thus we would not expect a our word sense induction method to produce a specific sense of ``war'' for this context.
Compare ``civil'': ``civil'' as in ``civil war'', has very different expected contexts to ``civil'' as in ``civil servant'' or ``civil behaviour''. In this case we would expect 3 different senses of ``civil'' to be induced -- one of which does contain context information for ``civil war'' -- information not captured by the single representation of ``war''.

The extreme version of this is if one or more words in a multiword expression have no embedding defined at all. In this case we fall back to only using senses from words which do have embeddings. An example of this would be ``Fulton County Court'', while ``County'' and ``Court'' are common words, that are certain to occur in any reasonable training corpus; ``Fulton'' is a rare proper noun -- if it occurs at all it likely does not occur enough to train the representation. We use the remaining words: ``County'' and ``Court'' to determine the meaning of the whole. This is not always going to be suitable, but often is.



\subsection{A General WSD method} \label{generalwsd}

\subsubsection{Sense Language Model} \label{senselanguagemodel}
A traditional language model defined the probability of a word, given its context -- such as the two words preceding it for a trigram language model.
A skip-gram language model gives the probability of a word appearing in a context window around a input word.
This input word is associated with a vector called the word embedding. The word embedding which is trained, with the language model, to so that it captures the information about what contexts of a word.  For a word sense language model, this is generalise so that rather than a input word, there is a input word sense.
In the cases we consider that word sense representation is a vector, and the language model such as skip-gram\parencite{mikolov2013efficient} \footnote{Simple modification of our method would work for related language models such as continuous bag of words (CBOW) \parencite{mikolov2013efficient}}, predicting the words that occur near the word sense. Do note the asymmetry, in the input and output space: while the input is a vector representing a \emph{word sense}, the output is the probability of just a \emph{word}-- no difference is made between different senses occurring in the context. Stated formally, for a word $w$, and for a sense representation $s$ of another word (potentially the same word), the language model gives $$P(w \mid s)$$. Using this information we may apply Bayes Theorem to perform word sense disambiguation across the senses provided.

Using the language model, and simple application of Bayes' theorem, we define a general word-sense disambiguation method, that will be used for several parts of our process. This disambiguation method is used on both the unsupervised induced sense vectors, as well as the lexically refitted sense vectors once we have created them.
The general procedure of how to use Bayes' theorem has appeared may times before -- including in the work of Tian et al. applied to multi-sense language models to determine most-likely sense using multiple word sense vectors \parencite{tian2014probabilistic}. We present it here to give explanation to the new work that proceeds in the following sections.

Taking some collection of word sense representations, we aim to use the context to determine which is the most suitable for this use.
We will call the word we are trying to disambiguate the \emph{target word}.
Let $\s=(s_{1},...,s_{n})$, be the collection of possible word sense representations for target word, they may be induced senses, or lexical senses.
Let $\c=(w_{1},...,w_{m})$ be a sequence of word from around the target word -- that is to say it's context window.
For example for the target word \emph{kid}, the context could be {$\c=(wow,\; the,\; wool,\; from,\; the,\; is,\; so,\; soft,\; and,\; fluffy)$}, where \emph{kid} is the central word taken from between \emph{the} and \emph{fluffy}.
Ideally our contexts would be symmetric with similar window size to that used for training the language model, though this is not always possible.

For any particular sense, $s_i$, one can calculate the probability of the context with the language model:
\begin{equation} \label{eq:contextprobtrue}
 P(\c \mid s_{i})=\prod_{\forall w_{j}\in\c}P(w_{j} \mid s_{i})
\end{equation}

Here, the assumption is made that given any sense representation $s_i$, the probability of each word in its context is conditionally independent. Stated formally it is assumed $\forall a,b \in [1,m],\; a \ne b\; \wedge \forall s_i \in \s,\:w_a \perp w_b \mid s_j$.
The correctness of this assumption depends on the quality of the representation -- the ideal sense representation would fully capture all information about the contexts it can appear in -- thus making the other elements of those contexts not present any additional information, thus making $P(w_a \mid w_b,s_i)=P(w_a \mid s_i)$ i.e. conditionally independent.


Bayes' Theorem can be applied to this likelihood function and a prior for $P(s_i)$, to find the posterior probability.

\begin{equation} \label{eq:generalwsd}
P(s_{i} \mid \c) = \dfrac{P_S(\c \mid s_{j})P(s_{j})}{\sum_{s_{j}\in\s P_S(s_{j} \mid \c)P(s_{j})}}
\end{equation}

Note that in concrete implementation of this process, it is important to work with the logarithm of the probabilities, given that for any sense $P(\c \mid s_j)$ is extremely small --given huge range of values for each $w$ -- the entire size of the vocabulary.
\pdfcomment{Do I need to go into more detail here?}


\subsubsection{Geometric Smoothing} \label{smoothing}
Geometric smoothing is suitable for smoothing posterior probability estimates of sequences of conditionally independent likelyhoods.

Covered in the previous section was typical application of Bayes' theorem to the problem of finding the posterior distribution, given the likelihood of observations.
Our contribution beyond this is what we will call, for lack of a better term, geometric smoothing.
Rather than use \Cref{eq:generalwsd} directly, we consider instead replacing the normalised posterior estimate (i.e. the numerator), with it's $|\c|$-th root.

\begin{equation} \label{eq:contrextprobsmooth}
P_S(\c \mid s_{i})=\prod_{\forall w_{j}\in\c}\sqrt[|\c|]{P(w_{j} \mid s_{i})}
\end{equation}

This does not smooth, $P_S(\c \mid s_{i})$, rather when this is substituted into \Cref{eq:generalwsd}, it smooths $P(s_{i} \mid )$.

\begin{equation} \label{eq:generalwsdsmoothed}
P(s_{i}\mid\c)=\dfrac{P_{S}(\c\mid s_{j})P(s_{i})}{\sum_{s_{i}\in\s P_{S}(s_{j}\mid\c)P(s_{j})}}=\dfrac{\prod_{\forall w_{j}\in\c}\sqrt[|\c|]{P(w_{j}\mid s_{i})P(s_{j})}}{\sum_{s_{j}\in\s\prod_{\forall w_{k}\in\c}\sqrt[|\c|]{P(w_{k}\mid s_{j})P(s_{j})}}}
\end{equation}

The motivation for this comes from considering the case of the uniform prior.
In this case, it is the same as replacing $P_S(\c \mid s_{i})$ with the geometric mean of the individual word probabilities $P_S(w_j \mid s_{i})$.
If one has two sentences, $\c=\{w_1,...w_{|\c|}\}$ and $\c^\prime=\{w_1^\prime,...w^\prime_{|\c^\prime|}\}$, such that $|\c|^\prime > |\c|$:
then using \Cref{eq:contextprobtrue} to calculate $P(\c \mid s_{i})$ and $P(\c^\prime \mid s_{i})$ will generally result in incomparable results as addition probability terms will dominate -- often significantly more than then the relative values of the probabilities themselves. This is because for almost any word $w$ and any target word sense $P(w \mid \s_i)$ the number of words that could be in the targets context is a significant portion of the entire vocabulary.
This becomes clear when one considers the expected values. For $V$ the vocabulary size, we have the expected value:
\begin{equation} \label{eq:expectcontexprob}
\mathbb{E}_\c(P(\c \mid s_{i}))
=(\mathbb{E}_w(P(w \mid s_i)))^{|\c|}
= \frac{1}{V^{|\c|}}
\end{equation}
Taking the $|\c|$-th and $|\c^\prime|$-th roots of $P(\c \mid s_{i})$ and $P(\c \mid s_{i})$ normalises these probabilities so that they have the same expected value; thus making a fair comparison possible, where the context length does not dominate.
When this normalisation is applied to \Cref{eq:generalwsd}, we get a smoothing effect.


Geometric smoothing has, as we suggest with the name, a smoothing effect on the $$P(s_{i}\mid\c)$$. It effectively increases small values, and decrease large values, in way that monotonically to their magnitude. The smoothing makes the posterior distribution more similar to a uniform distribution. The reasoning behind applying smoothing is that very high, and very low probabilities of a context are not expected to truely occur in natural languages. In general one should almost never be completely certain that one sense, or another applies (or does not apply) to a given sentence. The context is not truely so informative as to allow hard decisions. However, due to data sparsity, only tiny small fraction of the possible contexts a word sense can appear in, will actually occur, even in a very large training corpus.

We theorise that the particularly problematic likelihoods $P(\c \mid s_{i})$ that are soften by geometric smoothing come from rare words in the training corpus -- a data sparsity problem. If a word only occurs a few times in the corpus, then it will only be considered a reasonable member of the contexts for a few words, and within those only for a few senses -- probably one. This result in extremely small $P(w \mid s_i)$ which in turn results in very small $P(\c \mid s_i)$, which in turn when finding $P(s_{i} \mid \c)$ eliminates all but the word sense which it has occurred with from the running. However, this would generally be a mistake -- it is unlikely that there is a particularly special link between this rare word and one particular word sense to the exclusion of all others. Thus this is a re-emergence of the data sparsity problem which plagues n-gram language models. In general this sparsity problem is considered largely solved by word embedding methods due to weight sharing \textcite{NPLM}; but it reoccurs here for word-sense embeddings due to the much larger size of the vocabulary of senses compared to the size of the vocabulary of words. Thus we apply this smoothing.

In general smoothing is desirable, for consider: \enquote{The CEO of the bank, went for a picnic by the river.} While \enquote{CEO} is closely linked to a financial bank, and \enquote{river} is strongly linked to a river bank, we do not wish for the occurrence of either word in the context to hard negate the possibility of either sense -- this use of bank does refer to a financial institution, but there are other sentences with very similar words in the context window that refer to a river bank.




\subsection{Word Sense Disambiguation} \label{lexicalWSD}
Given a example sentence ($\{\c_1,...,\c_{n_l}\}$), such as a gloss, for each lexical word sense we can refitted word sense vectors for each of $\l=\{l_1,..., l_{n_l}\}$ by using the method described in \cref{eq:synth}.

Then when given target word $w_{target}$ in a a sentence $\c_{T}$ to be disambiguated, we can apply \Cref{eq:generalwsd} (or the smoothed version \Cref{eq:generalwsdsmoothed}), again to find $P(l_i \mid c_{T})$ for each lexical word sense of $w$ using the lexically refitted sense vectors we found earlier.

\[
\begin{aligned}\label{eq:lexicalwsd}
l^\star &= \argmax_{\forall l_i \in \l} P(l_i|\c_T)
&= \argmax_{\forall l_i \in \l} \frac{P_s(\c_T) \mid l_i)P(l_i)}{\sum_{\forall l_j \in \l} P_s(\c_T \mid l_j)P(l_j)}
\end{aligned}
\]

As each lexical word-sense shares it's gloss with the rest of its set of synonymous senses, this means that the lexically aligned word senses for synonym are all fitted to the same example sentence. However this does not mean they are equal, as they use different word sense vectors to find the weighted sums. Similarly, as the glosses are defined per lemma (base form), the different tenses (and other variants) of a word also share the same example sentence, but once again, the refitted word sense vector will still be different, so we learn different word sense vectors for each form of the word.



Note that in this case, we do have a prior for $P(l_i)$.
WordNet includes frequency counts for each word sense based on Semcor \textcite{tengi1998design}.
However Semcor is not a immense corpus, being only a subset of the Brown corpus.

The comparatively small size of Semcor means that many word senses do not occur at all. As counted by WordNet 2.1 (used in the WSD task in \Cref{WSDtask}), there are counts for just  36973 word senses, out of the total 207,016 senses in WordNet; i.e. 82\% of all word senses have no count information.

There is an additional issue that the Semcor's underling texts from Brown are now significantly ageing being all from 1961 \cite{francis1979brown} -- it is not unreasonable to suggest that frequency of word-sense use has changed significantly in the last half century.

Never-the-less, the word count is the best prior readily available. Given the the highly unbalanced distribution of sense occurrence (as discussed in \Cref{corpussize}),
a uniform prior would not be a reasonable approximation.


\subsection{RefittedSim}\label{RefittedSimVsAvgSimC}

Using the refitting process we define a new similarity measure: RefittedSim, for determining the similarity of words with context.
It is simply defined by using \Cref{eq:synth} to refit the unsupervised word senses to get a new word sense vector for the example context provided, for each word (with context) to be compared, then finding the distance between those vectors.

\begin{multline} \label{eq:refittedsim}
\mathrm{RefittedSim}((\s,\c),(s^{\prime},\c^{\prime}))\\
\begin{aligned}
&= d(l(\s \mid \c), l(\s^\prime \mid \c^\prime)\\
&= d
\sum_{s_{i}\in\s}s_{i}P(s_{i}\mid\c),\:
\sum_{s_{j}^{\prime}\in\s^{\prime}}s_{i}P(s_{j}^{\prime}\mid\c^{\prime}))
\end{aligned}
\end{multline}

For $\c$ and $\c^\prime$ the contexts of target words $w$ and $w^\prime$ for which we want to measure the similarity of. Where $w$ has senses $\s=\{s_1,...,s_n\}$, and $w^\prime$ has senses $\s=\{s^\prime_1,...,s\prime_{n^\prime}\}$, and for $d$ the distance function -- normally cosine distance.

Reisinger and Mooney \parencite{Reisinger2010} propose several methods for finding similarity utilising multiple word sense vectors -- the best performing of which was AvgSimC.
The key difference between their method and ours is that AvgSimC, is a probability weighted average of of pairwise computed distances for each word senses vector,
where as with out Refitted Word sense vectors, it is a single distance computed over probability weighed word sense vector.

the formula for RefittedSim \Cref{eq:refittedsim} can be contrasted with that for AvgSimC:


\begin{multline}
	\mathrm{AvgSimC}((\s,\c),(s^{\prime},\c^{\prime})) \\
	=  \frac{1}{n \times n^{\prime}}
	\sum_{s_{i}\in\s}
	\sum_{s_{j}^{\prime}\in\s^{\prime}}
	P(s_{i}\mid\c)\,P(s_{j}^{\prime}\mid\c^{\prime})\,d(s_{i},s_{j}^{\prime})
\end{multline}


Note that several existing works, evaluated using AvgSimC, do not find $P(s_{i}\mid\c)$ and $P(s_{j}^{\prime}\mid\c^\prime)$ using a probability based method, but rather define them using the a distance function \parencite{Reisinger2010, Huang2012}. Context vector cluster based vectors can use distance from their sense cluster centroid, and then interpret that distance after with normalisation, the probably that context belonging that cluster. There is a clear rational for this interpretation, as we can see it as a soft membership function. \pdfcomment{Such a method could with \Cref{eq:synth}, and thus with the RefittedSim similarity function and other techniques discussed in this paper.}


There is a notable time complexity difference between AvgSimC and RefittedSim.
AvgSimC has time complexity $O(n\left\Vert \c\right\Vert +n^{\prime}\left\Vert \c^{\prime}\right\Vert +n\times n^{\prime})$
RefittedSim is $O(n\left\Vert \c\right\Vert +n^{\prime}\left\Vert \c^{\prime}\right\Vert)$.
The product of the number of senses of each word $n \times n^\prime$, may small for dictionary senses, but it is often large for induced senses. Dictionaries tend to define only a few sense per word -- the average\footnote{It should be noted, though, that the number of meanings is not normally distributed \parencite{zipf1945meaning}} number of senses in WordNet is less than three for all parts of speech \parencite{miller1995wordnet}. For induced sense, however, it is often desirable to train many more senses, to get better results using the more fine-grained information. In several evaluations performed by Reisinger and Mooney they found optimal results at near to 50 senses \parencite{Reisinger2010}; this aligned with our own preliminary experiments also.

In the information retrieval context, the probabilities of the word sense for the context of the document can be done off-line, during indexing. With this assumption, the query-time time complexity becomes: for AvgSimC becomes $O(n\times n^{\prime})$, and for RefittedSim is now $O(1)$.
We do note however that pre-computing the word sense probabilities for each word in the document during indexing remains expensive (though no more so than for AvgSimC) -- it is however trivially parallelisable. We suggest that when this indexing time is considered worthwhile, then RefittedSim is significantly more viable for use in information retrieve tasks, where the user provides a example of the words they query so that a search for \enquote{\enquote{Apple} as in \enquote{the fruit I might like to eat}}, can return different results from \enquote{\enquote{Apple} as in \enquote{the company that makes the iPod, ad the Macbook}}.


\section{Method} \label{method}


\subsection{A Baseline Greedy Word Sense Embedding Method}

To confirm that our refitting method, and associated techniques are not simply a quirk of the AdaGram method or implementation, we defined a new simple baseline word sense embedding method.
This method starts with a fixed number of randomly initialised word-sense embeddings, then at each training step greedily aligns each training cases to the sense for which predicts that context with highest probability (using \Cref{eq:generalwsd}). The task remains the same: using skip-grams with hierarchical softmax to predict context words for the word sense.
Our implementation is based on a heavily modified version of the Word2Vec.jl package by Tanmay Mohapatra, and Zhixuan Yang \footnote{\url{https://github.com/tanmaykm/Word2Vec.jl/}} for word embeddings.

Due to the greedy nature of this baseline method, it is a intrinsically worse than AdaGram. A particular embedding may get an initial lead at predicting a context, simply based on predicting high probability for all words. It then gets trained more, resulting in it generally predicting high probability of many words, while other embeddings remain untrained. There is no force in the model to encourage diversification and specialisation of the embeddings. As a greedy method it readily falls into traps where the most used embedding is the most trained embedding, and thus is likely to receive more training. The random initialisation does help with this. Manual inspection reveals that it does capture a variety of senses, though with significant repetition of common senses, and with rare senses being largely missed. It is however a fully independent method from AdaGram, and so is suitable for use in checking the generalisation of our method.


\subsection{Experimental Setup and Model Parameters}
During training we use the Wikipedia dataset as used by Huang et al. \parencite{Huang2012}.
We did not perform the extensive preprocessing used in that work,
We preprocessed the data merely by converting it to lower case, tokenizing it and removing punctuation tokens.
For both models, were trained with a single iteration over the whole data set.
Also in both cases sub-sampling of $10^-5$ was used, and decreasing learning rate starting at 0.25 was used.


\subsubsection{AdaGram}
The AdaGram model was configured to have up to 30 word senses, where each sense represented by a 100 dimension vector. The sense threshold was set to $10^-10$ to encourage many senses.
Only words with at least 20 occurrences were kept, this gave a total vocabulary size of 497,537 words.

We use the AdaGram \parencite{AdaGrams} implementation\footnote{\url{https://github.com/sbos/AdaGram.jl}} provided by Bartunov et al. with minor adjustments for Julia \parencite{Julia} v0.5 compatibility.


\pdfcomment{
	Dict\{String,Any\} with 18 entries:
	"prototypes" => 30
	"nprocessors" => 13
	"output\_fn" => "../models/adagram/more\_senses.adagram\_model"
	"sense\_treshold" => 1.0e-10
	"remove\_top\_k" => 0
	"context\_cut" => true
	"initcount" => 1.0
	"train\_fn" => "../data/corpora/WikiCorp/tokenised\_lowercase\_WestburyLab.wikicorp.201004.txt"
	"d" => 0.0
	"alpha" => 0.25
	"subsample" => 1.0e-5
	"epochs" => 1
	"window" => 10
	"min\_freq" => 20
	"save\_treshold" => 0.0
	"dim" => 100
	"stopwords" => Set\{AbstractString\}()
	"dict\_fn" => "../data/corpora/WikiCorp/tokenised\_lowercase\_WestburyLab.wikicorp.201004.1gram
	}

\subsection{Greedy Baseline Model}
The greedy word sense embedding model, was configured without significant thought for it's performance -- indeed it was selected to have significantly different parameters to the AdaGram model that we trained. So that impact of using our methods could be checked to see they performed consistently on different word sense embedding models.

Each sense embedding had 300 dimensions.
The vocabulary was restricted to only words with at least 250 occurrences, which resulted in a total vocabulary of 88,262 words. Words with at least 20,000 occurrences, were giving 20 senses, and the remainder just a single sense. This resulted in the most common 2,796 words having multiple senses. This is not very high coverage, however it is more substantial than may be expected as the most common words having the most word senses \parencite{zipf1945meaning}, and being vastly more common than the less common words \parencite{zipf1949human,gilmour2005understanding}.

With these greedy embeddings, we always use the uniform prior, as the assignment of contexts embeddings can change significantly during training, and to determinate an accurate estimation of the prior would take similar time to performing another full iterator of training.

\pdfcomment{
Vocab size: 88,262
n\_senses = 20
min\_count = 250
min\_count for multiple senses = 20\_000
multisense word count = 2796
dimensions=300
}


\section{Similarity in context}




\subsection{Results}
We evaluate our refitting method, with and without, the geometric smoothing mentioned in \Cref{eq:contrextprobsmooth}, using the Stanford's Contextual Word Similarities (SCWS) corpus \parencite{Huang2012}.
As per in the training, each challenge context was windowed to 5 word either side of the target word, where possible.

\begin{table*}
\pgfplotstabletypeset[col sep=comma, header=has colnames, string type,
columns/rho/.style={
	column name={$\rho \times 100$},
	numeric type,
	precision=1,
	fixed zerofill=true,
	preproc/expr={100*##1},
	column type=l
}]{swsc.csv}

\caption{Results on SCWS. $\rho$ is Spearman rank correlation between the output similarities from each method and the ground truth of the average rating of from the human annotators. For comparison we include subset of the results from the other indicated papers. \pdfcomment{TODO: This table needs trimming down, and relabelling.}} \label{swscres}
\end{table*}

In \Cref{swscres}, are shown the results on the SCWS similarity task. Shown are several variations of our method, as well as several results from other works. We note that it is not entirely reasonable to directly compare the different works, and they are each trained on a different dataset (Interestedly, all various plain text extractions of Wikipedia), with significantly different preprocessing steps, some considerably more complex than our own removal of punctuation and change of case. Never the less, all the results successful methods are fairly similar.


It can be seen that using the RefittedSim measure, our results perform similarly to those of Huang et al. and to those of Tian et al, using AvgSimC; but as discussed in \Cref{RefittedSimVsAvgSimC}, have a lower time-complexity to calculate.

We suggest that no results have been presented prior to our work using AdaGram evaluated on SCWS, due to the comparative failure of AvgSimC to produce suitable results when applied to AdaGram that task. As can be seen in \Cref{swscres} Adagram with AvgSimC produces lack-luster results.

Using RefittedSim without using geometric smoothing and without the prior from, does it not perform much better than just using AvgSimC. Adding the geometric smoothing to AvgSimC did improve that result. However it much more significantly improved the results with RefittedSim. We see similar pattern with the greedy multiple word sense embeddings -- though unsurprisingly these results are significantly worse. It seems that geometric smoothing also improved the greedy multiple sense embeddings in the same away.

The performance shift from AdaGram RefittedSim with the uniform prior by adding geometric smoothing $\rho=24.1$ to $\rho=65.0$ is much larger than for using adding geometric smoothing to AdaGram with the estimated prior, which is only $\rho=47.8$ to  $\rho=64.8$. \pdfcomment{Why?}



The results of Chen et al using AvgSimC, do significantly out perform both our method and that of Huang et al. This supports their approach of utilising WordNet sense data, to bootstrap their problem into a semi-supervised learning problem. This in and of itself highlights the importance and utility of bridging the gap between structures data such as WordNet, and unstructured data such as induced word sense vectors.


We do note that our method does have lower training time, as it only requires using the training data once, where as relabelling approaches, as uses by Chen et al, require a second training pass, using the boot-strapped labels. We suggest though that this extra time is well worth it, as the results from using it are notably better. It would be possible to apply a similar relabelling method to AdaGram, using our approach to solve the WSD problem to bootstrap the training data.

\subsection{Word Sense Disambiguation}

As discussed in \Cref{lexicalWSD} the language model, can be applied to a WSD Task by fitting to the glosses. We do not window the glosses as they are already short, and the actual word we are fitting may not occur in the gloss itself, being that it is a definition shared by mean words with the same meaning. We do convert the gloss to lower case, and strip any out of vocabulary words, including punctuation.

\begin{table*}
	\begin{adjustbox}{max width=\textwidth}
		\pgfplotstabletypeset[col sep=comma, header=has colnames, string type,
		columns/F1/.style={ numeric type,precision=3, fixed zerofill=true},
		columns/Precision/.style={ numeric type,precision=3, fixed zerofill=true},
		columns/Recall/.style={ numeric type,precision=3, fixed zerofill=true}
		]{semeval2007t7.csv}
	\end{adjustbox}

	\caption{Results on SemEval 2007 Task 7 -- course-all-words disambiguation.
		For comparison we include subset of the results from the other indicated papers.
	} \label{samevalres}
\end{table*}

The results of employing our method, are shown in \Cref{samevalres}, with several others for comparison. Our results marginally outperforms the baseline most frequents sense -- as often observed, a surprisingly difficult baseline to beat.
We note, that it is particularly important to include the prior based on sense counts -- to do otherwise results in incorrect formulation of Bayes theorem in \Cref{eq:lexicalwsd}. In this case, the uniform prior is snot a good approximation to the true prior distribution at all -- given the highly unbalanced frequencies of different word senses. \pdfcomment{Should I include some results for what happens if you try and use a uniform prior?}

Our method, like that of Chen et al only uses information from WordNet, and a unlabelled corpus \parencite{Chen2014}. It does not use hand-engineered features, or labelled training data -- beyond the glosses from WordNet.

Our method is beaten by the S2C method, when used with MFS backoff.
We do not employ any back-off strategy with AdaGram, as almost all words are found in the trained vocabulary. We suggest that threshold approach used with S2C is particularly effective at determining whether there is enough useful context information to make it clear as to if the word can be disambiguated, using the model.


When we compare our method, Refitted AdaGram, to Mapped Adagram which uses the method of Agirre et al \parencite{agirre2006}; which we implemented. Using both the Brown1 and Brown2 subcorpora of SemCor as the mapping corpus.
It can be seen that the mapping method does not perform well largely due to the number words not found in the mapped vocabulary -- but even when a backoff method is added to handle those cases, we see the results are still worse than MFS alone, and worse than the Refitted AdaGram. This indicates that the lack of data prevented the good estimation of the Mapping probabilities. Refitted AdaGram did not have this problem due to only needed a single example of each lexical wordsense.


Even the very best methods for WSD do not perform substantially better than the baseline. It has been suggested that a ensemble method as the way forward to improve WSD beyond this level \cite{saarikoski2006building,saarikoski2006defining}.
We suggest that a method such as ours, based on unsupervised word embeddings, would be well suited to use in an ensemble, as it operates using very different features, to more complex WSD systems employing semantic relations.


\section{Conclusion}\label{conclusion}

We have presented a new method for taking unsupervised word embeddings, and adapting them to align to particular given uses. This method can be seen as a one shot learning task, where only a single labelled example of each class is available for training.


We showed how our method could be used create embeddings to evaluate the similarity of words, given there contexts. This itself is a new method, which has time complexity of $O(1)$ vs the time complexity of commonly used AvgSimC which is $O(n \times n^\prime)$.
The performance of our method is comparable to AvgSimC.

We also demonstrated how similar principles, could be used to create a set of vectors that are aligned to the meanings in a sense inventory, such as WordNet. We showed how this could be used for word sense disambiguation. On this difficult task, it performed marginally better than the MFS baseline, and better than some implementations comparable systems. However, on it's own it we do not believe the method is strong enough for commercial use. Future similar systems, building upon our work would have strong uses in machine translation.

As part of our method for refitting the sense embeddings to there new senses, we presented a novel smoothing algorithm -- geometric smoothing.
This smoothing method, is required to compensate for the data sparsity problem that is intrinsic in learning senses from usage data. Even more so than the data sparsity problem for words which is normally solved by word embeddings, due to their being many more meanings for words than there are words, particularly when fine grained senses are employed. We show as part of the discussed methods that our smoothing gives substantial improvement over the results with unsmoothed language model.

\newpage
\printbibliography
\end{document}
