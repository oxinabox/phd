\documentclass{sig-alternate}

\newcommand{\W}{\mathcal{W}}
\newcommand{\ci}{\perp\!\!\!\perp} % from Wikipedia

\begin{document}
\title{Aligning Word Sense Embeddings, Using a Single Example, with Applications in Word Sense Disambiguation}
\maketitle


There have been many recent works focused on inducing representations for different word senses based on context; particularly using neural network vector embeddings.
However these induced representations do not normally line up with the standard human created definitions of any given word sense.
It can certainly be argued that the induced senses may capture better senses, more comprehensively, or with better granularity than human lexicographers do.
However, without links from the induced senses to lexical sense, many applications are limited; and such systems which induce word senses can not be integrated with systems that utilize the wealth of existing lexical knowledge.

We thus propose a method for quickly aligning the indicted senses, to correspond with the sense in a single example, so as to allow the shift from an representation that makes sense only in the context of other representations of that form, to a representation corresponding to some gold standard label.

The method we propose is applicable for the synthesising of new lexically aligned word sense vectors from unsupervised induced word sense vectors, which come from a language model such a skip-gram.



We demonstrate this method on AdaGrams\cite{AdaGrams}, and on our own simpler greedy multiple word-sense embeddings. The method is generally applicable to any skip-gram-like language model that can take multiple vectors as input, and can output the probability of a word appearing in their context.

\section{Framework}


\subsection{Sense Language Model}
A traditional language model defined the probability of a word, given its context -- such as the words preceding it.
With skip-gram style language models, the context is abstracted into the vector input -- representing the information from all contexts of a word. This is generalised to have multiple different senses per word, for a sense language model.
It takes a representation of a word sense as an in input, and outputs the probability of any word occurring in the context of the that word sense. In the cases we consider that word sense representation is a vector, and the language model is something similar to a skip-gram. Do note the asymmetry, in the input and output space: while the input is a vector representing a word-sense, the output is the probability of just a word -- no difference is made between different senses occurring in the context. Given a particular word-sense the language model gives us the probability of the word occurring in the context of that word sense. Stated formally, for a word from the space of words $w\in \W$, and for a sense representation $s$, the language model gives $$P(w | s)$$.

\subsection{A General WSD method}
Using the language model, we define a general word-sense disambiguation method, that will be used for serval parts of our process.
Taking some collection of word sense representations, we aim to use the context to determine which is the most suitable for this use.
We will call the word we are trying to disambiguate the \emph{target word}.
Let $\mathbf{s}=(s_{1},...,s_{m})$, be the collection of possible word sense representations for target word, they may be induced senses, or lexical senses.
Let $\mathbf{c}=(w_{1},...,w_{n})$ be a sequence of word from around the target word -- that is to say it's context.
For example for the target word \emph{kid}, the context could be \mbox{$\mathbf{c}=(wow,\; the,\; wool,\; from,\; the,\; is,\; so,\; soft,\; and,\; fluffy)$}, where \emph{kid} is the centeral word taken from between \emph{the} and \emph{fluffy}.
Ideally our contexts would be symmetric with similar window size to that used for training the language model, though as shall be discussed below this is not always possible.
 
For any particular sense, $s_i$, we calculate the probability of the context with the language model: \[P(\mathbf{c}|s_{i})=\prod_{\forall w_{j}\in\mathbf{c}}P(w_{j}|s_{i})\]
Here, we make the assumption that given any sense representation $s_j$, the probability of each word in it's context is conditionally independent. Stated formally we assume $\forall a,b \in [1,n],\; a \ne b\; \wedge \forall s_i \in \mathbf{s},\:w_a \perp w_b \mid s_i$.
The correctness of this assumption depends on the quality of the representation -- the ideal sense representation would fully capture all information about the contexts it can appear in -- that making elements of those contexts not present any additional information, thus making $P(w_a \mid w_b,s_i)=P(w_a \mid s_i)$ i.e. conditionally independent.

From this, by applying Baye's Theorem, we can calculate the likelyhood function, given a prior for $P(s_i)$.

\[P(s_{i}|\mathbf{c}) = \dfrac{P(\mathbf{c}|s_{j})P(s_{j})}{\sum_{s_{j}\in\mathbf{s}}P(s_{j}|\mathbf{c})P(s_{j})}\].








\end{document}