\documentclass{sig-alternate}
%\documentclass{article}


\usepackage{csquotes}
\usepackage{amsmath}

\usepackage{booktabs, array} % Generates table from .csv
\usepackage{pgfplotstable}

\usepackage[backend=bibtex,
style=trad-abbrv,url=false, doi=false, sorting=none]{biblatex}
\bibliography{master}

\usepackage[author={Lyndon White}]{pdfcomment}
\usepackage{cleveref}

\newcommand{\W}{\mathcal{W}}
\renewcommand{\c}{\mathbf{c}}
\newcommand{\s}{\mathbf{s}}
\renewcommand{\l}{\mathbf{l}}
\renewcommand{\u}{\mathbf{u}}
\newcommand{\ci}{\perp\!\!\!\perp} % from Wikipedia
\DeclareMathOperator*{\argmin}{arg\,min}
\DeclareMathOperator*{\argmax}{arg\,max}


\begin{document}
\title{A Bayes' Theorem Based, One-Shot Learning, Method for Refitting Word Sense Vectors}
\maketitle

\begin{abstract}

Our contributions are 3:

A method for refitting word sense vectors to using a single example;
a novel smoothing technique, for use in the afore mentioned refitting;
and a new similar measure for words in context, using these refitted vectors, of time complexity lower than for the current standard ArvSimC used by competing systems.

The vector refitting technique can be used to label unsupervised word senses with labels from standard lexical resources, such as WordNet: which as applications in word sense induction, which in tern has applications in machine translation.
The faster similarity measure has applications in information retrieval, where faster methods are required to allow comparisons to webscale 



We Demonstrate the effectiveness of the aforementioned techniques, using AdaGrams\cite{AdaGrams}, on word similarity in context, and word sense disambiguation tasks. These are tasks on which the AdaGram embeddings have not previously been successfully applied.

\end{abstract}


\pdfcomment{
One can break up word sense induction methods into two category, for how they determine word senses.
Clustering based methods, which cluster word instances, and contexts: such as the works of \citeauthor{Reisinger2010}}{}



---------

Many word embedding techniques train their word vectors, in a unsupervised fashion, using the task of training a language model \parencite{NPLM, collobert2008unified, mikolov2013efficient}.
For most applications, this language model is discarded, and only the trained word embeddings are kept and utilised.
This seems wasteful, as the language model itself contains useful information. In particular we consider word sense embeddings, where rather than training word representations, unsupervised representations are induced for the different meanings of the words.
In this application, there is no direct way to know which word sense a particular embedding is for.
Here, the language models which when given a word sense embedding output the probability of the words that may occur near it are particularly useful. In this paper we show how simple application of Bayes' theorem can be used to move from the language model's context likelihood, to the posterior probability of a particular word sense. We then go on to show how this probability can the be used to generate new sense embeddings and show how they can be used for several benchmark application.

This approach can be contrasted with methods that do not have, or do not use language models, and must instead approximate probability of senses using cluster distance measures \parencite{Reisinger2010,Huang2012}. \pdfcomment{TODO: Move content from \Cref*{RefittedSimVsAvgSimC} to here}

-----------

%NonRefitted: Reisinger2010,Huang2012,AdaGrams
%Self Refitted: Chen2014
%Prerefitted: iacobacci2015sensembed

There have been several works focused on inducing representations for different word senses based on context based on various vector-space embeddings \parencite{Reisinger2010,Huang2012,AdaGrams,Chen2014,iacobacci2015sensembed}.



However, many of these induced representations do not intrinsically line up with the standard human created definitions of any given word sense.
Reisinger and Mooney, stated in their seminal work on multiple sense word vectors \parencite{Reisinger2010}: \enquote{we  do
not  assume  that  clusters  correspond  to  traditional
word senses. Rather, we only rely on clusters to capture meaningful variation in word usage.}, this remains true for many works that came after\parencite{Huang2012,AdaGrams}.
It can certainly be argued that the induced senses may capture better senses, more comprehensively, or with better granularity than human lexicographers do.
However, without links from the induced senses to lexical sense, many applications are limited; and such systems which induce word senses can not be integrated with systems that utilize the wealth of existing lexical knowledge.

We thus propose a method for quickly refitting the indicted senses, to correspond with the sense in a single example, so as to allow the shift from an representation that makes sense only in the context of other representations of that form, to a representation corresponding to some gold standard label. This has several applications.

The most obvious is as a tool for word-sense disambiguation. Which has it own applications, allowing a user to look up the dictionary definition of a word. It also has applications in translation, allowing word senses from one language to be matched to the correct word sense from another by making translator dictionaries accessible. 
Accomplishing these tasks using word-sense embedding approaches makes for easier integration of these machine learnt models with traditional approaches, opening the way for more powerful hybrids.

Beyond refitting to a standard lexical sense, refitting to a user provided example has applications in information retrieval. Consider the natural language query \enquote{Find me all webpages about banks as in \enquote{the river banks were very muddy.}}. Generating a vector for that specific sense of ``banks'' from the example sentence, and generating one from use of the word in each retrieved document, allows for similarity ranking -- discarding irrelevant uses. 




Our method generated new vectors refitted to a particular usage from prior trained unsupervised vectors using a weighted sum. The weighting determined by how likely they (the unsupervised) are to be the correct (unsupervised) sense intended in a example sentence. Where the example sentence is for a particular dictionary sense of a word, the resulting vector is the for the particular lexical sense in question.

Refiting senses this way also side steps two problems of induced word sense representations: Discarding un-useful representations -- that may have never been trained, and merging overly similar representations. Un-useful representations will never having a high probability for any real context where the target word occurs -- otherwise they would be useful, so will not contribute to the refitted vector. Overly similar representations will  predict similar contexts, and will contribute similar vectors shifts -- it will be roughly equivalent to as if a single vectors was twice as likely.


The method we propose is applicable for the synthesising of new lexically refitted word sense vectors from unsupervised induced word sense vectors, which come from a skip-gram language model such or similar.



We demonstrate this method on AdaGrams\parencite{AdaGrams}, and on our own simpler greedy multiple word-sense embeddings. The method is generally applicable to any skip-gram-like language model that can take multiple vectors as input, and can output the probability of a word appearing in their context.

\subsection{Lack of annotated corpora} \label{corpussize}
A significant issue for WSD, is the lack of annoated corpora suitable for training a machine learning approach.
There are several reasons for this. One is the difficulty of the task -- it is hard to get high inter-annotator agreement on the correct word sense. Second is the huge size a corpora needs to be to be practically useful.

For a corpus to be useful for training word sense vectors, or for learning the association between context and sense more generally, it would need to show a large number of examples of all the contexts a word could be seen in. Powerful generalising methods for machine learning can decrease the number of examples, but not remove the need for a large corpus. Ziph's law applies not only to the frequency of words, but to the frequency of word senses with-in words \pdfcomment{I really need to check this and find a good citation for it.}. Thus an extremely large corpus is required to capture useful information. Creating or even checking such a annotated corpus manually is a herculean task. This is one of the key attractions of unsupervised word sense induction -- which can make use of immense corpora without requiring them to be annotated.



\section{Related Words}

\subsection{Unsupervised Word Sense Embeddings}



\subsection{Directly learning Lexical Sense Embeddings}
While many word sense embeddings do not match to dictionary senses, there are exceptions.
The refitting method proposed in this paper is thus not required for these methods.
They can be said to directly compete with our method, when it comes to creating dictionary senses. However when it comes to applying them to a new fine grained sense from a single example our refitting method remains suitable. This continues to have applications in information retrieval.


Chen et al. \parencite{Chen2014} train word sense vectors for dictionary senses by relabelling a corpus using embeddings they derived from word embeddings and WordNet. First they train single sense word embeddings using skipgrams \parencite{mikolov2013efficient}. They initially find word sense vectors by averaging the word vectors of all the words in the WordNet gloss.
Then for each sentence in their training corpus, they progressively relabel the words with word senses. This is done by first calculating a context vector that this the average of all word vectors in the sentence. Then selecting a word to label with a word sense by selecting the word sense that is closest to the context vector. They then recalculate the context vector using the word sense vector, and replace the next word. Only words which can meet a threshhold of similarity are relabelled this way, otherwise the word remains unlabelled and a general word vector is trained.  Finally, using the relabelled corpus, a new skipgrams method is used with the objective of predicting both words and word-senses in the context. This fine-tunes the sense vectors.  Like Chen et al., Iacobacci et al. \parencite{iacobacci2015sensembed} present another relabelling approach, but use 3rd party labels.


Iacobacci et al. \parencite{iacobacci2015sensembed} apply Continuous Bag of Word (CBOW) language model as in Word2Vec , but rather than using words as the labels, they used word senses. To do this they require a large sense labelled corpus -- which does not exists. Iacobacci et al. create one artificially by applying the 3rd party WSD tool BabelFly. This allows the direct application of Word2Vec techniques to the sense embedding problem, but it does bind their performance to that of BabelFly. 


\subsection{Mapping WSI senses to lexical senses through a mapping corpus}


Agirre et al gave a general method for converting disambiguation results from induced word senses to standard lexical word sense disambiguation result \parencite{agirre2006}. This method was used for Semeval-2007 Task 02 \parencite{SemEval2007WSIandWSD} to evaluate all entries.
They use a annotated \emph{mapping corpus} to construct a mapping matrix between induced senses and the lexical senses.
This mapping is given by for $\l=\{l_1,..., l_{n_l}\}$ the set of lexically defined senses, and for $\u={u_1,...,u_{n_u}}$ the set of unsupervised induced senses:

\begin{equation} \label{eq:agirremap}
M_{i,j} = P(l_i | u_j) = \frac{count(\mathrm{method\: gives\: u_i \wedge l_i\: is\: annotated})}{count(\mathrm{method\: gives\: u_i})}
\end{equation}

An issue with estimating $P(l_i \mid u_j)$ this way is that to get a accurate estimate the law of large samples must apply to the mapping corpus, which given the issues discussed in \Cref{corpussize} may not. Given such a mapping corpus with sufficient overlap to the test corpus Agirre's method works well. 
From this  when presented with a sentence containing the target word to disambiguate ($\c$) the unsupervised sense score ($P(u_j \mid \c)$ are converted to supervised scores ($P(l_j \mid \c)$) by 
\begin{equation} \label{eq:agireewsd}
P(l_i \mid \c) = P(l_i | u_j) P(u_j \mid c)
\end{equation}
This is a practical method that works well, assuming the corpus is large enough to reasonably converge the estimated distribution to the true distribution, by the law of large numbers.

\pdfcomment{I generalised this to make uses of what the probabilities where, solving for M uing LSR. Performance is basically unchanged over hard decisions.}




\section{Framework}

\begin{figure}
	\caption{\label{key}}
\end{figure}

\subsection Overview


\subsection{Sense Language Model} \label{senselanguagemodel}
A traditional language model defined the probability of a word, given its context -- such as the words preceding it.
With skip-gram style language models, the context is abstracted into the vector input -- representing the information from all contexts of a word. This is generalised to have multiple different senses per word, for a sense language model.
It takes a representation of a word sense as an in input, and outputs the probability of any word occurring in the context of the that word sense. In the cases we consider that word sense representation is a vector, and the language model such as skip-gram\parencite{mikolov2013efficient} \footnote{Simple modification of our method would work for related language models such as continuous bag of words (CBOW) \parencite{mikolov2013efficient}}. Do note the asymmetry, in the input and output space: while the input is a vector representing a word-sense, the output is the probability of just a word -- no difference is made between different senses occurring in the context. Given a particular word-sense the language model gives us the probability of the word occurring in the context of that word sense. Stated formally, for a word from the space of words $w\in \W$, and for a sense representation $s$, the language model gives $$P(w \mid s)$$. Using this information we may assumed conditional independence of the words, and apply Bayes Theorem to perform word sense disambiguation across the senses provided.

\subsection{A General WSD method} \label{generalwsd}
Using the language model, and simple application of bayes theorem, we define a general word-sense disambiguation method, that will be used for several parts of our process. This disambiguation method is used on both the unsupervised induced sense vectors, as well as the lexically refitted sense vectors once we have created them.

Taking some collection of word sense representations, we aim to use the context to determine which is the most suitable for this use.
We will call the word we are trying to disambiguate the \emph{target word}.
Let $\s=(s_{1},...,s_{n})$, be the collection of possible word sense representations for target word, they may be induced senses, or lexical senses.
Let $\c=(w_{1},...,w_{m})$ be a sequence of word from around the target word -- that is to say it's context.
For example for the target word \emph{kid}, the context could be \mbox{$\c=(wow,\; the,\; wool,\; from,\; the,\; is,\; so,\; soft,\; and,\; fluffy)$}, where \emph{kid} is the central word taken from between \emph{the} and \emph{fluffy}.
Ideally our contexts would be symmetric with similar window size to that used for training the language model, though as shall be discussed below this is not always possible.
 
For any particular sense, $s_i$, one can calculate the probability of the context with the language model:
\begin{equation} \label{eq:contextprobtrue} P(\c \mid s_{i})=\prod_{\forall w_{j}\in\c}P(w_{j} \mid s_{i})
\end{equation}

Here, the assumption is made that given any sense representation $s_j$, the probability of each word in it's context is conditionally independent. Stated formally it is assumed $\forall a,b \in [1,m],\; a \ne b\; \wedge \forall s_i \in \s,\:w_a \perp w_b \mid s_i$.
The correctness of this assumption depends on the quality of the representation -- the ideal sense representation would fully capture all information about the contexts it can appear in -- that making elements of those contexts not present any additional information, thus making $P(w_a \mid w_b,s_i)=P(w_a \mid s_i)$ i.e. conditionally independent.

Rather than use \cref{eq:contextprobtrue} directly, we instead use the geometric mean of the individual word likelihoods as a smoother approximation:

\begin{equation} \label{eq:contrextprobsmooth}
P_S(\c \mid s_{i})=\prod_{\forall w_{j}\in\c}\sqrt[m]{P(w_{j} \mid s_{i})}
\end{equation}

This has two key effects. Firstly, it gives consistency for computing the relative  probabilities when dealing with contexts of different lengths. Secondly: When using Bayes' theorem to find $P(s_{i} \mid \c)$, this decreases the effect of very small $Q(\c \mid s_{i})$ on the final sense probabilities\footnote{It also decreases the effect of very large (close to 1) $Q(\c \mid s_{i})$, but due to the nature of the problem -- the huge number of contexts a word-sense can exist in -- high probabilities should not occur; that would indicate a sense that only occurs in a single context}. This de-emphasis which tends to result in more representative results.

We theorise the reason for the improvement from this geometric smoothing, is due issues caused by rare words in the training corpus. If a word only occurs a few times in the corpus, then it will only be considered a reasonable member of the contexts for a few words, and within those only for a few senses -- probably one. This result in extremely small $P(w \mid s_i)$ which in turn results in very small $P(\c \mid s_i)$, which in turn when finding $P(s_{i} \mid \c)$ eliminates all but the word sense which it has occurred with from the running. However, this would generally be a mistake -- it is unlikely that there is a particularly special link between this rare word and one particular word sense to the exclusion of all others. It is simply a re-emergence of the data sparsity problem which plagues ngram language models. In general this sparsity problem is considered largely solved by word embedding methods due to weight sharing; but it reoccurs here for word-sense embeddings due to the much larger size of the vocabulary of senses,vs of words. Thus we apply this smoothing. In general smoothing is desirable, for consider: \enquote{The CEO of the bank, went for a picnic by the river.} While \enquote{CEO} is closely linked to a financial bank, and \enquote{river} is strongly linked to a river bank, we do not wish for the occurrence of either word in the context to hard negate the possibility of either sense.

\pdfcomment{
	plot  a=x(1-x))/((x*(1-x))+(y*(1-y))) for 0<x<1, 0<y<1,
	vs
	plot  a=sqrt(x(1-x))/(sqrt(x*(1-x))+sqrt(y*(1-y))) for 0<x<1, 0<y<1
	}

From this, by applying Baye's Theorem, we can calculate the likelihood function, given a prior for $P(s_i)$.
FYI, I have set the project start date to Oct 31 given the busy October month for all of us, and project duration of 7 months (average of 13 hours/week in case not all of you are engaged each and every week, we can always finish sooner or work in bursts). 
\begin{equation} \label{eq:generalwsd}
P(s_{i} \mid \c) = \dfrac{P_S(\c \mid s_{j})P(s_{j})}{\sum_{s_{j}\in\s P_S(s_{j} \mid \c)P(s_{j})}}
\end{equation}

Note that in concrete implementation of this process, it is important to work with the logarithm of the probabilities, given that for any sense $P(\c \mid s_j)$ is extremely small --given huge range of values for each $w$ -- the entire size of the vocabulary.


\subsection{Synthesising New Embeddings from a Single Example}

The key contribution of this work is to suggest a new way to synthesis a word sense embedding given only one example. For lack of a better term we call this *refitting* the word sense vectors, to find a new vector, that lines up with the specific meaning of the word from the example sentence.

This can be looked at as a oneshot learning problem, with the training of language model and induced sense embeddings, as a unsupervised pre-training. The new word sense embedding should give a high value for the likelihood of the example sentence, under the language model. Further more though, it should generalise, to also give high likelihood of other contexts that were never seen, but which also occur near the word of this particular meaning.

In preliminarily investigation of this, we attempted directly optimising a vector, to maximise the probability of the example when input into the language model. This took significant time per word, and did not produce a good representation. We attribute this to overfitting, which is very easy to do with just the single example. Rather than directly optimising for a new word sense vector, we sought to express them as a combinations of the existing vectors that were already trained. We propose to do this as a weighted sum of the induced word sense sense, where the weighting is determine using  probability of the context under the vectors vectors.


Given a collection of induced (unlabelled) embeddings $\u={u_1,...,u_{n_u}}$, and example sentence which we will again call $\c={w_1,...,w_m}$. We find a new sense embedding: $l$, by: 

\begin{equation} \label{eq:synth}
l=\sum_{\forall u_i \in \u} u_iP(u_i \mid \c)
\end{equation}

Where $P(u_i \mid \c$ is found using \cref{eq:generalwsd}, and assuming the uniform prior $P(u_i)=\frac{1}{n_u}$.

The uniform prior is assumed the computational cost of calculating the marginal probability of the final trained sense embeddings over a sufficiently large corpus, is comparable to the cost of training them over that corpus in the first place. Furthermore, it assumes that there is a level of similarity between the distribution of the word-senses in the sources being used for refitting, that matches that of the training corpus. Which seems unlikely given that for WSD tasks we will be refitting to WordNet glosses, which have a very specific distribution of senses -- one of each, when it comes to lexical senses, and it is expected this would also bias the distribution of the induced senses.

In the very first neural network language model paper, Bengio et al. describe a similar method for finding word embeddings for words not found in their vocabulary \parencite{NPLM}. They suggest that if a word was not in the training data, \enquote{an initial feature vector for such a word, by taking a weighted convex combination of the feature vectors of other words that could have occurred in the same context, with weights proportional to their conditional probability}. The formula they give is as per \Cref{eq:synth}, but summing over the entire vocabulary of words (rather than just $\u$). To the best of our knowledge, this method has not been used for handling out of vocabulary words in any more recent word embedding architectures, nor has it been used with word-sense vectors.


\subsubsection {Fallback for dictionary phrases (collocations)}
Unless a specialised tokenizing method is used, a word embedding method will not learn embedding for collocations such as ``civil war'' or ``martial arts''. Normal tokenizers will split them at the word level, learning embeddings for ``civil'', and ``war'', and for ``martial'' and ``arts''. This issue is often considered minimal for word embeddings, as a approximate embedding can be constructed by summing embeddings for each word  in the phrase.

It has been constantly noted that for single-sense word embeddings the summing the vectors for each word in the phrase results in reasonable representation for the phrase  \parencite{mikolovSkip} \pdfcomment{Doesn't Rui Wang have a paper than goes into some detail on this? I can't find it}. The intuition from this is that for multiple sense-word embeddings, there is a correct selection of senses for each word such that the sum of these senses will be a representation for the phrase.

Simply evaluating the each combination of word-sense sums for multiple word collocations is computationally expensive. For a $m$ length collocation, where each word has $n$ senses, this requires $m^n$ evaluations of probability of the context ($P(\mathbb{c}\mid u_i)$). When inducing unsupervised word embeddings it is not unreasonable to have $n=30$ sense, so for a 2 word phrase (eg ``civil war'') this is 900 evaluations.
Instead, we we just add the additional sense embeddings for each word to the total pool of sense embeddings to be combined ($\u$ above), requiring instead $m \times n$ evaluations. So for our 2 word example, this is 60 evaluations, instead of the 30 evaluations that is require for a single word entry. This also has other advantages.


In the case that the word-sense induced for one word 

For example, ``civil war''. The context that ``war'' in ``civil war'' occurs, is very similar to the context that ``war'' appears in general -- for both ``war'' and ``civil war'' we expect the context to include words like ``casualties'', ``militia'' etc, thus we would not expect a our word sense induction method to produce a specific sense of ``war'' for this context.
Compare ``civil'': ``civil'' as in ``civil war'', has very different expected contexts to ``civil'' as in ``civil servant'' or ``civil behaviour''. In this case we would expect 3 different senses of ``civil'' to be induced -- one of which does contain context information for ``civil war'' -- information not captured by the single representation of ``war''.

The extreme version of this is if one or more words in a multiword expression have no embedding defined at all. In this case we fall back to only using senses from words which do have embeddings. An example of this would be ``Fulton County Court'', while ``County'' and ``Court'' are common words, that are certain to occur in any reasonable training corpus; ``Fulton'' is a rare proper noun -- if it occurs at all it likely does not occur enough to train the representation. We use the remaining words: ``County'' and ``Court'' to determine the meaning of the whole. This is not always going to be suitable, but it is the best that can be done.


\subsection{Word Sense Disambiguation}
Given a example sentence ($\{\c_1,...,\c_{n_l}\}$) for each lexical word sense we can find refitted word sense vectors for each $\l=\{l_1,..., l_{n_l}\}$ by using the method described in \cref{eq:synth}. 

Then when given target word in a a sentence to be disambiguated $\c_{challange}$, we can apply the method of \cref{eq:generalwsd}, again to find $P(l_i \mid c_{challange})$ for each lexical word sense using the lexically refitted sense vectors we found earlier.

\begin{equation}\label{eq:lexicalwsd}
l^\star = \argmax_{\forall l_i \in \l} P(l_i|c_{challenge})
\end{equation}
or written out:
\begin{equation}\label{eq:lexicalwsdexpanded}
l^\star \argmax_{\forall l_i \in \l} \frac{P_s(c_{challange}|l_i)P(l_i)}{\sum_{\forall l_j \in \l} P_s(c_{challenge}|l_j)P(l_j)}
\end{equation}

Note that in this case, we do have a reasonable prior for $P(l_i)$.
WordNet comes frequency counts for each word sense based on Semcor\cite{tengi1998design}.
However Semcor is not a immense corpus, being only a subset of the Brown corpus.

The comparatively small size of Semcor means that many word senses do not occur at all. As counted by WordNet 2.1 (used in the WSD task in \Cref{WSDtask}), there are counts for just  36973 word senses, out of the total 207,016 senses in WordNet; i.e. 82\% of all word senses have no count information. 

Combined with this Semcor's underling texts from Brown are now significantly ageing being all from 1961 -- it is not unreasonable to suggest that frequency of word-sense use has changed significantly in the last half century.
We can solve both issues by applying moderately aggressive Laplacian smoothing -- adding 10 to the pseudo-count of each word. This removes the impossibility issues with senses the never occur; as well as weakening the prior -- bring it closer to to the uniform distribution. 
Using less aggressive smoothing, results in the prior dominating over the likelihood term from our method -- which has the net result of the over all system being equivalent to the baseline most common word sense method.



\subsection{Refitted Word Sense Vectors vs AvgSimC}\label{RefittedSimVsAvgSimC}
Reisinger and Mooney \cite{Reisinger2010} propose several methods for finding similarity utilising multiple word sense vectors -- the best performing of which was AvgSimC.
The key difference between their method and ours is that AvgSimC, is a probability weighted average of of pairwise computed distances for each word senses vector,
where as with out Refitted Word sense vectors, it is a single distance computed over probability weighed word sense vector.

For $\c$ and $\c^\prime$ the contexts of target words $w$ and $w^\prime$ for which we want to measure the similarity of. Where $w$ has senses $\s=\{s_1,...,s_n\}$, and $w^\prime$ has senses $\s=\{s^\prime_1,...,s\prime_{n^\prime}\}$, and for $d$ the distance function -- normally cosine distance it is defined by:


\begin{multline}
	\mathrm{AvgSimC}((\s,\c),(s^{\prime},\c^{\prime})) \\
	=  \frac{1}{n \times n^{\prime}} 
	\sum_{s_{i}\in\s}
	\sum_{s_{j}^{\prime}\in\s^{\prime}}
	P(s_{i}\mid\c)\,P(s_{j}^{\prime}\mid\c^{\prime})\,d(s_{i},s_{j}^{\prime})
\end{multline}

Which we contrast with our method, based on using \Cref{eq:synth} to synthesise a new vector that is refitted to give high likelihood of their contexts, under language model:

\begin{multline}
	\mathrm{RefittedSim}((\s,\c),(s^{\prime},\c^{\prime}))\\
	\begin{aligned}
	&= d(l(\s \mid \c), l(\s^\prime \mid \c^\prime)\\
	&= d
	\sum_{s_{i}\in\s}s_{i}P(s_{i}\mid\c),\:
	\sum_{s_{j}^{\prime}\in\s^{\prime}}s_{i}P(s_{j}^{\prime}\mid\c^{\prime}))
	\end{aligned}
\end{multline}

Note that several existing works, evaluated using AvgSimC, do not find $P(s_{i}\mid\c)$ and $P(s_{j}^{\prime}\mid\c^\prime)$ using a probability based method, but rather define them using the a distance function \parencite{Reisinger2010, Huang2012}. Context vector cluster based vectors can use distance from their sense cluster centroid, and call that distance, after with normalisation, the probably that context belonging that cluster. There is a clear rational for this interpretation, as we can see it as a soft (fuzzy) membership function. Such a method could with \Cref{eq:synth}, and thus with the RefittedSim similarity function and other techniques discussed in this paper. 
 

There is a notable time complexity difference between AvgSimC and RefittedSim.
AvgSimC has time complexity $O(n\left\Vert \c\right\Vert +n^{\prime}\left\Vert \c^{\prime}\right\Vert +n\times n^{\prime})$
RefittedSim is $O(n\left\Vert \c\right\Vert +n^{\prime}\left\Vert \c^{\prime}\right\Vert)$
In the information retrieval context, the probabilities of the word sense for the context of the document can be done off-line, during indexing. With this assumption, the query-time time complexity becomes: for AvgSimC becomes $O(n\times n^{\prime})$, and for RefittedSim is now $O(1)$. We do note however that pre-computing the word sense probabilities for each word in the document during indexing remains expensive (though no more so than for AvgSimC) -- it is however trivially parallelisable. We suggest that when this indexing time is considered worthwhile, then RefittedSim is significantly more viable for use in information retrieve tasks, where the user provides a example of the words they query so that a search for \enquote{\enquote{Apple} as in \enquote{the fruit I might like to eat}}, can return different results from \enquote{\enquote{Apple} as in \enquote{the company that makes the iPod, ad the Macbook}}.




\section{Method}

\subsection{AdaGram}


We use the AdaGram \parencite{AdaGrams} implementation\footnote{\url{https://github.com/sbos/AdaGram.jl}} provided by Bartunov et al. with minor adjustments for Julia v0.5 compatibility.

\subsection{A Baseline Greedy Word Sense Embedding Method}

To confirm that our refitting method, and associated techniques are not simply a quirk of the AdaGram method or implementation, we defined a new simple baseline word sense embedding method.
This method starts with a fixed number of randomly initialised word-sense embeddings, then greedily aligns each training cases to the sense for which predicts that context with highest probability (using \Cref{eq:generalwsd}), at each training step. The task remains the same: using skipgrams with hierarchical softmax to predict context words for the word sense.
Our implementation is based on a heavily modified version of the Word2Vec.jl package by Tanmay Mohapatra, and Zhixuan Yang \footnote{\url{https://github.com/tanmaykm/Word2Vec.jl/}} for word embeddings.

Due to the greedy nature of this baseline method, it is a intrinsically worse than Adagrams. A particular embedding may get an initial lead at predicting a context, simply based on predicting high probability for all words. It then gets trained more, resulting in it generally predicting high probability of many words, while other embeddings remain untrained. There is no force in the model to encourage diversification and specialisation of the embeddings. As a greedy method it readily falls into traps where the most used embedding is the most trained embedding, and thus is likely to receive more training. The random initialisation does help with this. Manual inspection reveals that it does capture a variety of senses, though with significant repetition of common senses, and with rare senses being largely missed. It is however a fully independent method from AdaGrams, and so is suitable for use in checking the generalisation of our method.


\subsection{Experimental Setup and Model Parameters}

The results reported in \Cref{results}




\section{Results and Discussion} \label{results}


\subsection{Similarity in context}

We evaluate our refitting method, with and without, the geometric smoothing mentioned in \Cref{eq:contrextprobsmooth}, using the Stanford's Contextual Word Similarities (SCWS) corpus \parencite{Huang2012}.

\begin{table*}
\pgfplotstabletypeset[col sep=comma, search path={../data,..}, header=has colnames, string type,
columns/rho/.style={
	column name={$\rho \times 100$},
	numeric type,
	precision=1,
	fixed zerofill=true,
	preproc/expr={100*##1},
	column type=l
}]{swsc.csv}

\caption{Results on SWSC. $\rho$ is Spearman rank correlation between the output similarities from each method and the ground truth of the average rating of from the human annotators. \pdfcomment{TODO: Reformat this table. Remove Smoothing column, call out RefittedSim-Without-Smoothing}} \label{swscres}
\end{table*}


We suggest that no results have been presented prior to our work for AdaGrams evaluated on SCWS,
due to the comparative failure of AvgSimC to produce suitable results on that task. As can be seen in \Cref{swscres} Adagram with AvgSimC produces lackluster results. 

It can be seen that using the RefittedSim measure, our results perform similarly to those of Huang et al., using AvgSimC; but as discussed in \Cref{RefittedSimVsAvgSimC}, have a lower time-complexity to calculate.


The results of Chen et al using AvgSimC, do significantly out perform both our method and that of Huang et al. This supports their approach of utilising WordNet sense data, to bootstrap their problem into a semi-supervised learning problem. This in and of itself highlights the importance and utility of bridging the gap between structures data such as WordNet, and unstructured data such as induced word sense vectors.

We do note that our method does have lower training time, as it only requires using the training data once, where as relabelling approaches, as uses by Chen et al, require a second training pass, using the boot-strapped labels. We suggest though that this extra time is well worth it, as the results from using it are notably better. It would be possible to apply a similar relabelling method, using our approach to solve the WSD problem to bootstrap the training data.

\subsection{Word Sense Disambiguation}


\printbibliography

\end{document}