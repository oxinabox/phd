\RequirePackage{luatex85,shellesc}
\documentclass[dvipsnames]{beamer}
\usepackage{verbatim}

\usepackage{microtype}
\usepackage{adjustbox}
\usepackage{amsmath}

\usepackage[subpreambles=false]{standalone}


\newlength\xunit


%%%%%%%%%%%%%Bibliography
\usepackage[backend=bibtex, url=false,
bibstyle=ieee,firstinits=true]{biblatex}
\renewcommand*{\thefootnote}{} %No symbol or marker
\renewcommand{\footnotesize}{\scriptsize}
%%%%%%%%%%%%%%%%%


\usepackage{xcolor}
\definecolor{chamois}{RGB}{255,255,240}
\definecolor{darkbrown}{RGB}{124,79,0}
\definecolor{UniBlue}{RGB}{83,101,130}

\definecolor{hellgelb}{rgb}{1,1,0.8}
\definecolor{colKeys}{rgb}{0,0,1}
\definecolor{colIdentifier}{rgb}{0,0,0}
\definecolor{colComments}{rgb}{1,0,0}
\definecolor{colString}{rgb}{0,0.5,0}

\usefonttheme{professionalfonts}
\usepackage{tgadventor}
\renewcommand*\familydefault{\sfdefault}
\usepackage[T1]{fontenc}
\usepackage{microtype}


\newcommand{\topline}{%
  \tikz[remember picture,overlay] {%
    \draw[brown,ultra thick] ([yshift=-1.8cm]current page.north west)-- ([yshift=-1.8cm,xshift=\paperwidth]current page.north west);} }

\renewcommand{\topline}{}

\setbeamertemplate{frametitle}{\begin{minipage}[b][1.8cm][c]{\textwidth}%
	\centering%
	\insertframetitle\\\insertframesubtitle
	\end{minipage}}
	

\addtobeamertemplate{frametitle}{}{\topline%
}

\setbeamertemplate{navigation symbols}{}
\setbeamercolor{background canvas}{bg=chamois}
\setbeamercolor{itemize item}{fg=brown}
%\setbeamertemplate{itemize item}{\maltese}
\setbeamercolor{itemize subitem}{fg=brown}
%\setbeamertemplate{itemize subitem}{\begin{rotate}{90}$\diamondsuit$\end{rotate}}

\setbeamercolor{title}{fg=UniBlue}
\setbeamercolor{frametitle}{fg=UniBlue}    
\setbeamerfont{frametitle}{size=\Large}

\setbeamercolor{author}{fg=darkbrown}
\setbeamercolor{institute}{fg=darkbrown}
\setbeamercolor{date}{fg=darkbrown}


\setbeamercolor{structure}{fg=UniBlue}
\setbeamercolor{alerted text}{fg=UniBlue}
\setbeamercolor{alerted text}{fg=UniBlue}
\setbeamercolor{normal text}{fg=darkbrown!50!black}
\setbeamercolor{math text}{fg=darkbrown}
\setbeamercolor{math text displayed}{fg=darkbrown}



\addtobeamertemplate{block begin}{%
	\setlength{\textwidth}{0.8\textwidth}%
}{}
\setbeamercolor{block title}{bg=darkbrown!40,fg=darkbrown!90}
\setbeamercolor{block body}{bg=darkbrown!20,fg=UniBlue}
\setbeamercolor{block title alerted}{bg=yellow!60,fg=red}
\setbeamercolor{block body alerted}{bg=hellgelb!80,fg=UniBlue}

\AtBeginSection[]{
	\begin{frame}
		\vfill
		\centering
		\begin{beamercolorbox}[sep=8pt,center,shadow=true,rounded=true]{title}
			\usebeamerfont{title}\insertsectionhead\par%
		\end{beamercolorbox}
		\vfill
	\end{frame}
}





\bibliography{master.bib}

%%%%%%%%%%%%%%%%%%%%%%%%%%%%%%%%
\usepackage{tikz}
\usetikzlibrary{positioning}
\usetikzlibrary{decorations.pathmorphing}
\usetikzlibrary{graphs,graphs.standard,graphdrawing,arrows}
\usegdlibrary{layered, trees, force}

\usepackage{graphicx}
\graphicspath{{./figs/}, {./}}
\usepackage[space]{grffile}

\usepackage{trimclip}

%%%%%%%%%%%%%%%%%%%%%%%%%%%%%%%%%%%

%%%%%%%%%%%%%%%%%%%%%%%%%%%%%%%%%%%%%%%%%%%%
\newcommand{\inlinecode}[1]{\mbox{\alert{\texttt{#1}}}}

% all notes should just be simple item notes.
% I neither need nor want more functionality
\let\oldnote\note
\renewcommand{\note}{\oldnote[item]}

%%%%%%%%%%%%%%%%%%%%%%%%%%%%%%%%%%%%%%%%%%%%


%%%%%%%%%%%%%%%%%%%%%

\title{DataDeps.jl}
\subtitle{and other foundational tools for data driven research\\(Especially NLP)}
\author{
	\includegraphics[height=2cm]{juliaml}
	\includegraphics[height=2cm]{juliatext}
	\includegraphics[height=2cm]{juliastring}
	\\
	\vspace{5mm}
	\textbf{Lyndon White}}
\institute{School of Electical, Electronic and Computer Engineering\\The University of Western Australia}
\date{}
\logo{\hfill\includegraphics[scale=0.12]{uwa}\hfill\hspace{0.5cm}}

\begin{document}


%\centering %Center everywhere
\frame{\maketitle}
\logo{}


\begin{frame}{}
	\resizebox{\textwidth}{!}{%\RequirePackage{luatex85,shellesc}
\documentclass[tikz]{standalone}
\usepackage{tikz}
\usetikzlibrary{positioning}
\usetikzlibrary{decorations.pathmorphing}
\usetikzlibrary{graphs,graphs.standard,graphdrawing,arrows}
\usegdlibrary{layered, trees, force}

\renewcommand{\familydefault}{\sfdefault}

\begin{document}
	
\providecommand{\pkg}[2][]{
	\node[#1,package](#2) {#2.jl}
}

	
\providecommand{\repo}[2][]{
	\node[#1,repo](#2) {#2}
}


%%%%%%%%%%%%%%%%%%%%%%%%%%%%%%%%%%%%%

\tikzset{%
	->,
	align=center,
	node distance=10cm, sibling distance=25mm, level distance=40mm,
	thick,
	package/.style={draw, circle, very thick},
	mypackage/.style={package, blue},
	repo/.style={draw, very thick, purple},
	supports/.style={dotted},
	indirect/.style={decoration={snake}, decorate, very thick, purple}
}




\begin{tikzpicture}[layered layout,	grow'=up,]
	\node[nudge=(left:12mm)](researchcode){Research Code};

	\pkg[mypackage, nudge=(right:17mm)]{CorpusLoaders};
	
	\pkg[mypackage]{DataDeps};
	\pkg[mypackage]{DataDepsGenerators};
%	\pkg[mypackage]{ReferenceTests};
	\pkg[mypackage]{MD5};
	\pkg[]{SHA};

	\node(MultiResolutionIterators)[mypackage]{Multi--\\Resolution--\\Iterators.jl};
	
	\pkg[mypackage]{InternedStrings};
	\pkg[]{Strs};
	\pkg[]{CSV};
	

	\pkg{MLDatasets};
%	\pkg[mypackage]{ExpectationStubs};
	\pkg{HTTP};
	\pkg[nudge=(right:15mm)]{WordNet};
	\pkg[mypackage, nudge=(right:4mm)]{Embeddings};


	\pkg[mypackage]{WordTokenizers};
	\pkg{RevTok};

	
	\repo{CRAN};
	\repo{DataDryad};
	\repo{GitHub};
	\repo{European Data Portal};
	\repo[]{Open.Canada.ca};
	\repo{Data.gov};
	\repo{Data.gov.au};
	\repo{DataOne};
	\repo{538};
	\repo{BuzzFeedNews};
	\repo{UCI ML Repository};
	\repo{ArcticDataCenter};
	\repo{KnowledgeNetworkforBiocomplexity};
	\repo{TERN};
	\repo{DataCite};
	\repo{Zenodo};
	\repo{FigShare};
	
	
	\node[sibling distance=0mm](gap){};
	
	\graph{
		(researchcode) <- {(MLDatasets), (Embeddings), (WordNet), (CorpusLoaders)} <- (DataDeps);
		(DataDeps) ->[bend left, looseness=0, in=167, out=0] (researchcode);

	 
		{(CorpusLoaders) <- {(MultiResolutionIterators), (WordTokenizers), (InternedStrings)}};
		(DataDeps) <- {(HTTP), (SHA)};
		(DataDeps) <-[supports] (MD5);
		
		(DataDeps) <-[indirect] (DataDepsGenerators) <- {
			(GitHub) <- {(538), (BuzzFeedNews)},
			(UCI ML Repository),
			(CRAN) <- {(Open.Canada.ca), (Data.gov.au), (Data.gov), (European Data Portal)},
			(DataCite) <- {(Zenodo), (FigShare)},
			(DataDryad),
			(DataOne) <- {(ArcticDataCenter), (TERN), (KnowledgeNetworkforBiocomplexity)},
		};
	
		(WordTokenizers) <-[supports] (RevTok);
		(InternedStrings) <-> [supports] (Strs);
		(InternedStrings) ->[supports] (CSV);
	};
\end{tikzpicture}


\end{document}}
\end{frame}


\begin{frame}{}
	\resizebox{\textwidth}{!}{\clipbox{190pt 170pt 250pt 0pt}{%\RequirePackage{luatex85,shellesc}
\documentclass[tikz]{standalone}
\usepackage{tikz}
\usetikzlibrary{positioning}
\usetikzlibrary{decorations.pathmorphing}
\usetikzlibrary{graphs,graphs.standard,graphdrawing,arrows}
\usegdlibrary{layered, trees, force}

\renewcommand{\familydefault}{\sfdefault}

\begin{document}
	
\providecommand{\pkg}[2][]{
	\node[#1,package](#2) {#2.jl}
}

	
\providecommand{\repo}[2][]{
	\node[#1,repo](#2) {#2}
}


%%%%%%%%%%%%%%%%%%%%%%%%%%%%%%%%%%%%%

\tikzset{%
	->,
	align=center,
	node distance=10cm, sibling distance=25mm, level distance=40mm,
	thick,
	package/.style={draw, circle, very thick},
	mypackage/.style={package, blue},
	repo/.style={draw, very thick, purple},
	supports/.style={dotted},
	indirect/.style={decoration={snake}, decorate, very thick, purple}
}




\begin{tikzpicture}[layered layout,	grow'=up,]
	\node[nudge=(left:12mm)](researchcode){Research Code};

	\pkg[mypackage, nudge=(right:17mm)]{CorpusLoaders};
	
	\pkg[mypackage]{DataDeps};
	\pkg[mypackage]{DataDepsGenerators};
%	\pkg[mypackage]{ReferenceTests};
	\pkg[mypackage]{MD5};
	\pkg[]{SHA};

	\node(MultiResolutionIterators)[mypackage]{Multi--\\Resolution--\\Iterators.jl};
	
	\pkg[mypackage]{InternedStrings};
	\pkg[]{Strs};
	\pkg[]{CSV};
	

	\pkg{MLDatasets};
%	\pkg[mypackage]{ExpectationStubs};
	\pkg{HTTP};
	\pkg[nudge=(right:15mm)]{WordNet};
	\pkg[mypackage, nudge=(right:4mm)]{Embeddings};


	\pkg[mypackage]{WordTokenizers};
	\pkg{RevTok};

	
	\repo{CRAN};
	\repo{DataDryad};
	\repo{GitHub};
	\repo{European Data Portal};
	\repo[]{Open.Canada.ca};
	\repo{Data.gov};
	\repo{Data.gov.au};
	\repo{DataOne};
	\repo{538};
	\repo{BuzzFeedNews};
	\repo{UCI ML Repository};
	\repo{ArcticDataCenter};
	\repo{KnowledgeNetworkforBiocomplexity};
	\repo{TERN};
	\repo{DataCite};
	\repo{Zenodo};
	\repo{FigShare};
	
	
	\node[sibling distance=0mm](gap){};
	
	\graph{
		(researchcode) <- {(MLDatasets), (Embeddings), (WordNet), (CorpusLoaders)} <- (DataDeps);
		(DataDeps) ->[bend left, looseness=0, in=167, out=0] (researchcode);

	 
		{(CorpusLoaders) <- {(MultiResolutionIterators), (WordTokenizers), (InternedStrings)}};
		(DataDeps) <- {(HTTP), (SHA)};
		(DataDeps) <-[supports] (MD5);
		
		(DataDeps) <-[indirect] (DataDepsGenerators) <- {
			(GitHub) <- {(538), (BuzzFeedNews)},
			(UCI ML Repository),
			(CRAN) <- {(Open.Canada.ca), (Data.gov.au), (Data.gov), (European Data Portal)},
			(DataCite) <- {(Zenodo), (FigShare)},
			(DataDryad),
			(DataOne) <- {(ArcticDataCenter), (TERN), (KnowledgeNetworkforBiocomplexity)},
		};
	
		(WordTokenizers) <-[supports] (RevTok);
		(InternedStrings) <-> [supports] (Strs);
		(InternedStrings) ->[supports] (CSV);
	};
\end{tikzpicture}


\end{document}}}
\end{frame}


\begin{frame}{Vabdewakke's 6 Degree's of Replicability}
	\begin{enumerate}
		%\item 0: The results cannot be reproduced by an independent researcher.
		
		\item The results \alert{cannot seem to be reproduced}.
		
		\item The results could be reproduced by, \alert{requiring extreme effort.}
		
		\item The results can be reproduced, \alert{requiring considerable effort.}
		
		\item The results can be easily reproduced with \alert{at most \textbf{15 minutes} of user effort}, requiring some proprietary source packages (MATLAB, etc.).
		
		\item The results can be easily reproduced  with \alert{at most \textbf{15 min} of user effort}, requiring only standard, freely available tools (C compiler, etc.).		
	\end{enumerate}
	\note{Vabdewakke actual has a rating zero for "no, just no can not reproduce."}
	\note{What is the key determining factor here? User effort. How do we get down to 15 minutes of user effort?}
	\citehere{VabdewakkeReproduceableResearch}
\end{frame}


\begin{frame}{What happens when I try and reproduce someone's research code?}
	\begin{description}
		\item[1min] Find the website from the paper, and \alert{download the code}
		\item[2min] Read enough of the README to get rough bearings
		\item[\textbf{1min}] Find out where to get the data from and \alert{download the data}
		\item[\textbf{2min}] Try and remember how to use \inlinecode{tar -xzfvalphabetsoup} etc.
		\item[\textbf{2min}] Workout how to tell script \alert{where data is} \note{Hardcoded path? Argument?}
		\item[2min] Setup any software dependencies etc.
		\item[3min] Run the code and make sure it isn't crashing etc.
		\item[2min] Interpret the output
	\end{description}
	\note{I have spent 5 of my 15 valuable minutes faffing around about sorting out the data.}
	\note{I think this timeline is fairly reasonable, of course an ideal julia project with everything already working in CI would do a lot better. But if you've got these steps, your already blocked from CI.}
\end{frame}

\end{document}