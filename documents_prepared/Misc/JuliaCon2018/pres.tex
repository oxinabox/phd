\RequirePackage{luatex85,shellesc}
\documentclass[dvipsnames]{beamer}
\usepackage{verbatim}

\usepackage{microtype}
\usepackage{adjustbox}
\usepackage{amsmath}

\usepackage[subpreambles=false]{standalone}


\usepackage[at]{easylist}

%\newlength\xunit
%%%%%%%%%%%%%Bibliography
\usepackage[backend=bibtex, url=false,
bibstyle=ieee,firstinits=true]{biblatex}
\renewcommand*{\thefootnote}{} %No symbol or marker
\renewcommand{\footnotesize}{\scriptsize}
%%%%%%%%%%%%%%%%%


\usepackage{xcolor}
\definecolor{chamois}{RGB}{255,255,240}
\definecolor{darkbrown}{RGB}{124,79,0}
\definecolor{UniBlue}{RGB}{83,101,130}

\definecolor{hellgelb}{rgb}{1,1,0.8}
\definecolor{colKeys}{rgb}{0,0,1}
\definecolor{colIdentifier}{rgb}{0,0,0}
\definecolor{colComments}{rgb}{1,0,0}
\definecolor{colString}{rgb}{0,0.5,0}

\usefonttheme{professionalfonts}
\usepackage{tgadventor}
\renewcommand*\familydefault{\sfdefault}
\usepackage[T1]{fontenc}
\usepackage{microtype}


\newcommand{\topline}{%
  \tikz[remember picture,overlay] {%
    \draw[brown,ultra thick] ([yshift=-1.8cm]current page.north west)-- ([yshift=-1.8cm,xshift=\paperwidth]current page.north west);} }

\renewcommand{\topline}{}

\setbeamertemplate{frametitle}{\begin{minipage}[b][1.8cm][c]{\textwidth}%
	\centering%
	\insertframetitle\\\insertframesubtitle
	\end{minipage}}
	

\addtobeamertemplate{frametitle}{}{\topline%
}

\setbeamertemplate{navigation symbols}{}
\setbeamercolor{background canvas}{bg=chamois}
\setbeamercolor{itemize item}{fg=brown}
%\setbeamertemplate{itemize item}{\maltese}
\setbeamercolor{itemize subitem}{fg=brown}
%\setbeamertemplate{itemize subitem}{\begin{rotate}{90}$\diamondsuit$\end{rotate}}

\setbeamercolor{title}{fg=UniBlue}
\setbeamercolor{frametitle}{fg=UniBlue}    
\setbeamerfont{frametitle}{size=\Large}

\setbeamercolor{author}{fg=darkbrown}
\setbeamercolor{institute}{fg=darkbrown}
\setbeamercolor{date}{fg=darkbrown}


\setbeamercolor{structure}{fg=UniBlue}
\setbeamercolor{alerted text}{fg=UniBlue}
\setbeamercolor{alerted text}{fg=UniBlue}
\setbeamercolor{normal text}{fg=darkbrown!50!black}
\setbeamercolor{math text}{fg=darkbrown}
\setbeamercolor{math text displayed}{fg=darkbrown}



\addtobeamertemplate{block begin}{%
	\setlength{\textwidth}{0.8\textwidth}%
}{}
\setbeamercolor{block title}{bg=darkbrown!40,fg=darkbrown!90}
\setbeamercolor{block body}{bg=darkbrown!20,fg=UniBlue}
\setbeamercolor{block title alerted}{bg=yellow!60,fg=red}
\setbeamercolor{block body alerted}{bg=hellgelb!80,fg=UniBlue}

\AtBeginSection[]{
	\begin{frame}
		\vfill
		\centering
		\begin{beamercolorbox}[sep=8pt,center,shadow=true,rounded=true]{title}
			\usebeamerfont{title}\insertsectionhead\par%
		\end{beamercolorbox}
		\vfill
	\end{frame}
}





\bibliography{master.bib}

%%%%%%%%%%%%%%%%%%%%%%%%%%%%%%%%
\usepackage{tikz}
\usetikzlibrary{positioning}
\usetikzlibrary{decorations.pathmorphing}
\usetikzlibrary{graphs,graphs.standard,graphdrawing,arrows}
\usegdlibrary{layered, trees, force}

\usepackage{graphicx}
\graphicspath{{./figs/}, {./}}
\usepackage[space]{grffile}

\usepackage{trimclip}

%%%%%%%%%%%%%%%%%%%%%%%%%%%%%%%%%%%

%%%%%%%%%%%%%%%%%%%%%%%%%%%%%%%%%%%%%%%%%%%%
\newcommand{\inlinecode}[1]{\mbox{\alert{\texttt{\detokenize{#1}}}}}
\newenvironment{code}
{%BEGIN
\endgraf\verbatim
}
{ % END
\endverbatim
}


\newcommand{\datadep}[1]{\inlinecode{datadep"#1"}}

% all notes should just be simple item notes.
% I neither need nor want more functionality
\let\oldnote\note
\renewcommand{\note}{\oldnote[item]}

\renewcommand{\emph}{\alert}

%%%%%%%%%%%%%%%%%%%%%%%%%%%%%%%%%%%%%%%%%%%%


%%%%%%%%%%%%%%%%%%%%%


\institute{School of Electical, Electronic and Computer Engineering\\The University of Western Australia}
\date{}
\title{DataDeps.jl}
\subtitle{and other foundational tools for data driven research\\(Especially NLP)}
\author{\includegraphics[height=2cm]{juliaml}
	\includegraphics[height=2cm]{juliatext}
	\includegraphics[height=2cm]{juliastring}
	\\
	\vspace{5mm}
	\textbf{Lyndon White}}

\logo{\hfill\includegraphics[scale=0.12]{uwa}\hfill\hspace{0.5cm}}

\begin{document}


%\centering %Center everywhere
\frame{\maketitle}
\logo{}


\begin{frame}[fragile]{}
	\begin{code}
		blbla
	\end{code}
\end{frame}



\begin{frame}{}
	\resizebox{\textwidth}{!}{%\RequirePackage{luatex85,shellesc}
\documentclass[tikz]{standalone}
\usepackage{tikz}
\usetikzlibrary{positioning}
\usetikzlibrary{decorations.pathmorphing}
\usetikzlibrary{graphs,graphs.standard,graphdrawing,arrows}
\usegdlibrary{layered, trees, force}

\renewcommand{\familydefault}{\sfdefault}

\begin{document}
	
\providecommand{\pkg}[2][]{
	\node[#1,package](#2) {#2.jl}
}

	
\providecommand{\repo}[2][]{
	\node[#1,repo](#2) {#2}
}


%%%%%%%%%%%%%%%%%%%%%%%%%%%%%%%%%%%%%

\tikzset{%
	->,
	align=center,
	node distance=10cm, sibling distance=25mm, level distance=40mm,
	thick,
	package/.style={draw, circle, very thick},
	mypackage/.style={package, blue},
	repo/.style={draw, very thick, purple},
	supports/.style={dotted},
	indirect/.style={decoration={snake}, decorate, very thick, purple}
}




\begin{tikzpicture}[layered layout,	grow'=up,]
	\node[nudge=(left:12mm)](researchcode){Research Code};

	\pkg[mypackage, nudge=(right:17mm)]{CorpusLoaders};
	
	\pkg[mypackage]{DataDeps};
	\pkg[mypackage]{DataDepsGenerators};
%	\pkg[mypackage]{ReferenceTests};
	\pkg[mypackage]{MD5};
	\pkg[]{SHA};

	\node(MultiResolutionIterators)[mypackage]{Multi--\\Resolution--\\Iterators.jl};
	
	\pkg[mypackage]{InternedStrings};
	\pkg[]{Strs};
	\pkg[]{CSV};
	

	\pkg{MLDatasets};
%	\pkg[mypackage]{ExpectationStubs};
	\pkg{HTTP};
	\pkg[nudge=(right:15mm)]{WordNet};
	\pkg[mypackage, nudge=(right:4mm)]{Embeddings};


	\pkg[mypackage]{WordTokenizers};
	\pkg{RevTok};

	
	\repo{CRAN};
	\repo{DataDryad};
	\repo{GitHub};
	\repo{European Data Portal};
	\repo[]{Open.Canada.ca};
	\repo{Data.gov};
	\repo{Data.gov.au};
	\repo{DataOne};
	\repo{538};
	\repo{BuzzFeedNews};
	\repo{UCI ML Repository};
	\repo{ArcticDataCenter};
	\repo{KnowledgeNetworkforBiocomplexity};
	\repo{TERN};
	\repo{DataCite};
	\repo{Zenodo};
	\repo{FigShare};
	
	
	\node[sibling distance=0mm](gap){};
	
	\graph{
		(researchcode) <- {(MLDatasets), (Embeddings), (WordNet), (CorpusLoaders)} <- (DataDeps);
		(DataDeps) ->[bend left, looseness=0, in=167, out=0] (researchcode);

	 
		{(CorpusLoaders) <- {(MultiResolutionIterators), (WordTokenizers), (InternedStrings)}};
		(DataDeps) <- {(HTTP), (SHA)};
		(DataDeps) <-[supports] (MD5);
		
		(DataDeps) <-[indirect] (DataDepsGenerators) <- {
			(GitHub) <- {(538), (BuzzFeedNews)},
			(UCI ML Repository),
			(CRAN) <- {(Open.Canada.ca), (Data.gov.au), (Data.gov), (European Data Portal)},
			(DataCite) <- {(Zenodo), (FigShare)},
			(DataDryad),
			(DataOne) <- {(ArcticDataCenter), (TERN), (KnowledgeNetworkforBiocomplexity)},
		};
	
		(WordTokenizers) <-[supports] (RevTok);
		(InternedStrings) <-> [supports] (Strs);
		(InternedStrings) ->[supports] (CSV);
	};
\end{tikzpicture}


\end{document}}
\end{frame}


\begin{frame}{}
	%	〈	llx	〉 〈	lly	〉 〈	urx	〉 〈	ury	〉
	\resizebox{\textwidth}{!}{\clipbox{190pt 170pt 250pt 0pt}{%\RequirePackage{luatex85,shellesc}
\documentclass[tikz]{standalone}
\usepackage{tikz}
\usetikzlibrary{positioning}
\usetikzlibrary{decorations.pathmorphing}
\usetikzlibrary{graphs,graphs.standard,graphdrawing,arrows}
\usegdlibrary{layered, trees, force}

\renewcommand{\familydefault}{\sfdefault}

\begin{document}
	
\providecommand{\pkg}[2][]{
	\node[#1,package](#2) {#2.jl}
}

	
\providecommand{\repo}[2][]{
	\node[#1,repo](#2) {#2}
}


%%%%%%%%%%%%%%%%%%%%%%%%%%%%%%%%%%%%%

\tikzset{%
	->,
	align=center,
	node distance=10cm, sibling distance=25mm, level distance=40mm,
	thick,
	package/.style={draw, circle, very thick},
	mypackage/.style={package, blue},
	repo/.style={draw, very thick, purple},
	supports/.style={dotted},
	indirect/.style={decoration={snake}, decorate, very thick, purple}
}




\begin{tikzpicture}[layered layout,	grow'=up,]
	\node[nudge=(left:12mm)](researchcode){Research Code};

	\pkg[mypackage, nudge=(right:17mm)]{CorpusLoaders};
	
	\pkg[mypackage]{DataDeps};
	\pkg[mypackage]{DataDepsGenerators};
%	\pkg[mypackage]{ReferenceTests};
	\pkg[mypackage]{MD5};
	\pkg[]{SHA};

	\node(MultiResolutionIterators)[mypackage]{Multi--\\Resolution--\\Iterators.jl};
	
	\pkg[mypackage]{InternedStrings};
	\pkg[]{Strs};
	\pkg[]{CSV};
	

	\pkg{MLDatasets};
%	\pkg[mypackage]{ExpectationStubs};
	\pkg{HTTP};
	\pkg[nudge=(right:15mm)]{WordNet};
	\pkg[mypackage, nudge=(right:4mm)]{Embeddings};


	\pkg[mypackage]{WordTokenizers};
	\pkg{RevTok};

	
	\repo{CRAN};
	\repo{DataDryad};
	\repo{GitHub};
	\repo{European Data Portal};
	\repo[]{Open.Canada.ca};
	\repo{Data.gov};
	\repo{Data.gov.au};
	\repo{DataOne};
	\repo{538};
	\repo{BuzzFeedNews};
	\repo{UCI ML Repository};
	\repo{ArcticDataCenter};
	\repo{KnowledgeNetworkforBiocomplexity};
	\repo{TERN};
	\repo{DataCite};
	\repo{Zenodo};
	\repo{FigShare};
	
	
	\node[sibling distance=0mm](gap){};
	
	\graph{
		(researchcode) <- {(MLDatasets), (Embeddings), (WordNet), (CorpusLoaders)} <- (DataDeps);
		(DataDeps) ->[bend left, looseness=0, in=167, out=0] (researchcode);

	 
		{(CorpusLoaders) <- {(MultiResolutionIterators), (WordTokenizers), (InternedStrings)}};
		(DataDeps) <- {(HTTP), (SHA)};
		(DataDeps) <-[supports] (MD5);
		
		(DataDeps) <-[indirect] (DataDepsGenerators) <- {
			(GitHub) <- {(538), (BuzzFeedNews)},
			(UCI ML Repository),
			(CRAN) <- {(Open.Canada.ca), (Data.gov.au), (Data.gov), (European Data Portal)},
			(DataCite) <- {(Zenodo), (FigShare)},
			(DataDryad),
			(DataOne) <- {(ArcticDataCenter), (TERN), (KnowledgeNetworkforBiocomplexity)},
		};
	
		(WordTokenizers) <-[supports] (RevTok);
		(InternedStrings) <-> [supports] (Strs);
		(InternedStrings) ->[supports] (CSV);
	};
\end{tikzpicture}


\end{document}}}
\end{frame}

\begin{frame}{DataDeps.jl}
		%	〈	llx	〉 〈	lly	〉 〈	urx	〉 〈	ury	〉
%	\resizebox{\textwidth}{!}{\clipbox{190pt 420pt 390pt 0pt}{%\RequirePackage{luatex85,shellesc}
\documentclass[tikz]{standalone}
\usepackage{tikz}
\usetikzlibrary{positioning}
\usetikzlibrary{decorations.pathmorphing}
\usetikzlibrary{graphs,graphs.standard,graphdrawing,arrows}
\usegdlibrary{layered, trees, force}

\renewcommand{\familydefault}{\sfdefault}

\begin{document}
	
\providecommand{\pkg}[2][]{
	\node[#1,package](#2) {#2.jl}
}

	
\providecommand{\repo}[2][]{
	\node[#1,repo](#2) {#2}
}


%%%%%%%%%%%%%%%%%%%%%%%%%%%%%%%%%%%%%

\tikzset{%
	->,
	align=center,
	node distance=10cm, sibling distance=25mm, level distance=40mm,
	thick,
	package/.style={draw, circle, very thick},
	mypackage/.style={package, blue},
	repo/.style={draw, very thick, purple},
	supports/.style={dotted},
	indirect/.style={decoration={snake}, decorate, very thick, purple}
}




\begin{tikzpicture}[layered layout,	grow'=up,]
	\node[nudge=(left:12mm)](researchcode){Research Code};

	\pkg[mypackage, nudge=(right:17mm)]{CorpusLoaders};
	
	\pkg[mypackage]{DataDeps};
	\pkg[mypackage]{DataDepsGenerators};
%	\pkg[mypackage]{ReferenceTests};
	\pkg[mypackage]{MD5};
	\pkg[]{SHA};

	\node(MultiResolutionIterators)[mypackage]{Multi--\\Resolution--\\Iterators.jl};
	
	\pkg[mypackage]{InternedStrings};
	\pkg[]{Strs};
	\pkg[]{CSV};
	

	\pkg{MLDatasets};
%	\pkg[mypackage]{ExpectationStubs};
	\pkg{HTTP};
	\pkg[nudge=(right:15mm)]{WordNet};
	\pkg[mypackage, nudge=(right:4mm)]{Embeddings};


	\pkg[mypackage]{WordTokenizers};
	\pkg{RevTok};

	
	\repo{CRAN};
	\repo{DataDryad};
	\repo{GitHub};
	\repo{European Data Portal};
	\repo[]{Open.Canada.ca};
	\repo{Data.gov};
	\repo{Data.gov.au};
	\repo{DataOne};
	\repo{538};
	\repo{BuzzFeedNews};
	\repo{UCI ML Repository};
	\repo{ArcticDataCenter};
	\repo{KnowledgeNetworkforBiocomplexity};
	\repo{TERN};
	\repo{DataCite};
	\repo{Zenodo};
	\repo{FigShare};
	
	
	\node[sibling distance=0mm](gap){};
	
	\graph{
		(researchcode) <- {(MLDatasets), (Embeddings), (WordNet), (CorpusLoaders)} <- (DataDeps);
		(DataDeps) ->[bend left, looseness=0, in=167, out=0] (researchcode);

	 
		{(CorpusLoaders) <- {(MultiResolutionIterators), (WordTokenizers), (InternedStrings)}};
		(DataDeps) <- {(HTTP), (SHA)};
		(DataDeps) <-[supports] (MD5);
		
		(DataDeps) <-[indirect] (DataDepsGenerators) <- {
			(GitHub) <- {(538), (BuzzFeedNews)},
			(UCI ML Repository),
			(CRAN) <- {(Open.Canada.ca), (Data.gov.au), (Data.gov), (European Data Portal)},
			(DataCite) <- {(Zenodo), (FigShare)},
			(DataDryad),
			(DataOne) <- {(ArcticDataCenter), (TERN), (KnowledgeNetworkforBiocomplexity)},
		};
	
		(WordTokenizers) <-[supports] (RevTok);
		(InternedStrings) <-> [supports] (Strs);
		(InternedStrings) ->[supports] (CSV);
	};
\end{tikzpicture}


\end{document}}}
	\resizebox{\textwidth}{!}{\clipbox{190pt 320pt 400pt 0pt}{%\RequirePackage{luatex85,shellesc}
\documentclass[tikz]{standalone}
\usepackage{tikz}
\usetikzlibrary{positioning}
\usetikzlibrary{decorations.pathmorphing}
\usetikzlibrary{graphs,graphs.standard,graphdrawing,arrows}
\usegdlibrary{layered, trees, force}

\renewcommand{\familydefault}{\sfdefault}

\begin{document}
	
\providecommand{\pkg}[2][]{
	\node[#1,package](#2) {#2.jl}
}

	
\providecommand{\repo}[2][]{
	\node[#1,repo](#2) {#2}
}


%%%%%%%%%%%%%%%%%%%%%%%%%%%%%%%%%%%%%

\tikzset{%
	->,
	align=center,
	node distance=10cm, sibling distance=25mm, level distance=40mm,
	thick,
	package/.style={draw, circle, very thick},
	mypackage/.style={package, blue},
	repo/.style={draw, very thick, purple},
	supports/.style={dotted},
	indirect/.style={decoration={snake}, decorate, very thick, purple}
}




\begin{tikzpicture}[layered layout,	grow'=up,]
	\node[nudge=(left:12mm)](researchcode){Research Code};

	\pkg[mypackage, nudge=(right:17mm)]{CorpusLoaders};
	
	\pkg[mypackage]{DataDeps};
	\pkg[mypackage]{DataDepsGenerators};
%	\pkg[mypackage]{ReferenceTests};
	\pkg[mypackage]{MD5};
	\pkg[]{SHA};

	\node(MultiResolutionIterators)[mypackage]{Multi--\\Resolution--\\Iterators.jl};
	
	\pkg[mypackage]{InternedStrings};
	\pkg[]{Strs};
	\pkg[]{CSV};
	

	\pkg{MLDatasets};
%	\pkg[mypackage]{ExpectationStubs};
	\pkg{HTTP};
	\pkg[nudge=(right:15mm)]{WordNet};
	\pkg[mypackage, nudge=(right:4mm)]{Embeddings};


	\pkg[mypackage]{WordTokenizers};
	\pkg{RevTok};

	
	\repo{CRAN};
	\repo{DataDryad};
	\repo{GitHub};
	\repo{European Data Portal};
	\repo[]{Open.Canada.ca};
	\repo{Data.gov};
	\repo{Data.gov.au};
	\repo{DataOne};
	\repo{538};
	\repo{BuzzFeedNews};
	\repo{UCI ML Repository};
	\repo{ArcticDataCenter};
	\repo{KnowledgeNetworkforBiocomplexity};
	\repo{TERN};
	\repo{DataCite};
	\repo{Zenodo};
	\repo{FigShare};
	
	
	\node[sibling distance=0mm](gap){};
	
	\graph{
		(researchcode) <- {(MLDatasets), (Embeddings), (WordNet), (CorpusLoaders)} <- (DataDeps);
		(DataDeps) ->[bend left, looseness=0, in=167, out=0] (researchcode);

	 
		{(CorpusLoaders) <- {(MultiResolutionIterators), (WordTokenizers), (InternedStrings)}};
		(DataDeps) <- {(HTTP), (SHA)};
		(DataDeps) <-[supports] (MD5);
		
		(DataDeps) <-[indirect] (DataDepsGenerators) <- {
			(GitHub) <- {(538), (BuzzFeedNews)},
			(UCI ML Repository),
			(CRAN) <- {(Open.Canada.ca), (Data.gov.au), (Data.gov), (European Data Portal)},
			(DataCite) <- {(Zenodo), (FigShare)},
			(DataDryad),
			(DataOne) <- {(ArcticDataCenter), (TERN), (KnowledgeNetworkforBiocomplexity)},
		};
	
		(WordTokenizers) <-[supports] (RevTok);
		(InternedStrings) <-> [supports] (Strs);
		(InternedStrings) ->[supports] (CSV);
	};
\end{tikzpicture}


\end{document}}}
\end{frame}

\begin{frame}{DataDepsGenerators.jl}
		%	〈	llx	〉 〈	lly	〉 〈	urx	〉 〈	ury	〉
	\resizebox{\textwidth}{!}{\clipbox{0pt 0pt 0pt 290pt}{%\RequirePackage{luatex85,shellesc}
\documentclass[tikz]{standalone}
\usepackage{tikz}
\usetikzlibrary{positioning}
\usetikzlibrary{decorations.pathmorphing}
\usetikzlibrary{graphs,graphs.standard,graphdrawing,arrows}
\usegdlibrary{layered, trees, force}

\renewcommand{\familydefault}{\sfdefault}

\begin{document}
	
\providecommand{\pkg}[2][]{
	\node[#1,package](#2) {#2.jl}
}

	
\providecommand{\repo}[2][]{
	\node[#1,repo](#2) {#2}
}


%%%%%%%%%%%%%%%%%%%%%%%%%%%%%%%%%%%%%

\tikzset{%
	->,
	align=center,
	node distance=10cm, sibling distance=25mm, level distance=40mm,
	thick,
	package/.style={draw, circle, very thick},
	mypackage/.style={package, blue},
	repo/.style={draw, very thick, purple},
	supports/.style={dotted},
	indirect/.style={decoration={snake}, decorate, very thick, purple}
}




\begin{tikzpicture}[layered layout,	grow'=up,]
	\node[nudge=(left:12mm)](researchcode){Research Code};

	\pkg[mypackage, nudge=(right:17mm)]{CorpusLoaders};
	
	\pkg[mypackage]{DataDeps};
	\pkg[mypackage]{DataDepsGenerators};
%	\pkg[mypackage]{ReferenceTests};
	\pkg[mypackage]{MD5};
	\pkg[]{SHA};

	\node(MultiResolutionIterators)[mypackage]{Multi--\\Resolution--\\Iterators.jl};
	
	\pkg[mypackage]{InternedStrings};
	\pkg[]{Strs};
	\pkg[]{CSV};
	

	\pkg{MLDatasets};
%	\pkg[mypackage]{ExpectationStubs};
	\pkg{HTTP};
	\pkg[nudge=(right:15mm)]{WordNet};
	\pkg[mypackage, nudge=(right:4mm)]{Embeddings};


	\pkg[mypackage]{WordTokenizers};
	\pkg{RevTok};

	
	\repo{CRAN};
	\repo{DataDryad};
	\repo{GitHub};
	\repo{European Data Portal};
	\repo[]{Open.Canada.ca};
	\repo{Data.gov};
	\repo{Data.gov.au};
	\repo{DataOne};
	\repo{538};
	\repo{BuzzFeedNews};
	\repo{UCI ML Repository};
	\repo{ArcticDataCenter};
	\repo{KnowledgeNetworkforBiocomplexity};
	\repo{TERN};
	\repo{DataCite};
	\repo{Zenodo};
	\repo{FigShare};
	
	
	\node[sibling distance=0mm](gap){};
	
	\graph{
		(researchcode) <- {(MLDatasets), (Embeddings), (WordNet), (CorpusLoaders)} <- (DataDeps);
		(DataDeps) ->[bend left, looseness=0, in=167, out=0] (researchcode);

	 
		{(CorpusLoaders) <- {(MultiResolutionIterators), (WordTokenizers), (InternedStrings)}};
		(DataDeps) <- {(HTTP), (SHA)};
		(DataDeps) <-[supports] (MD5);
		
		(DataDeps) <-[indirect] (DataDepsGenerators) <- {
			(GitHub) <- {(538), (BuzzFeedNews)},
			(UCI ML Repository),
			(CRAN) <- {(Open.Canada.ca), (Data.gov.au), (Data.gov), (European Data Portal)},
			(DataCite) <- {(Zenodo), (FigShare)},
			(DataDryad),
			(DataOne) <- {(ArcticDataCenter), (TERN), (KnowledgeNetworkforBiocomplexity)},
		};
	
		(WordTokenizers) <-[supports] (RevTok);
		(InternedStrings) <-> [supports] (Strs);
		(InternedStrings) ->[supports] (CSV);
	};
\end{tikzpicture}


\end{document}}}
\end{frame}

\begin{frame}{CorpusLoaders.jl}
		%	〈	llx	〉 〈	lly	〉 〈	urx	〉 〈	ury	〉
	\resizebox{\textwidth}{!}{\clipbox{370pt 310pt 250pt 70pt}{%\RequirePackage{luatex85,shellesc}
\documentclass[tikz]{standalone}
\usepackage{tikz}
\usetikzlibrary{positioning}
\usetikzlibrary{decorations.pathmorphing}
\usetikzlibrary{graphs,graphs.standard,graphdrawing,arrows}
\usegdlibrary{layered, trees, force}

\renewcommand{\familydefault}{\sfdefault}

\begin{document}
	
\providecommand{\pkg}[2][]{
	\node[#1,package](#2) {#2.jl}
}

	
\providecommand{\repo}[2][]{
	\node[#1,repo](#2) {#2}
}


%%%%%%%%%%%%%%%%%%%%%%%%%%%%%%%%%%%%%

\tikzset{%
	->,
	align=center,
	node distance=10cm, sibling distance=25mm, level distance=40mm,
	thick,
	package/.style={draw, circle, very thick},
	mypackage/.style={package, blue},
	repo/.style={draw, very thick, purple},
	supports/.style={dotted},
	indirect/.style={decoration={snake}, decorate, very thick, purple}
}




\begin{tikzpicture}[layered layout,	grow'=up,]
	\node[nudge=(left:12mm)](researchcode){Research Code};

	\pkg[mypackage, nudge=(right:17mm)]{CorpusLoaders};
	
	\pkg[mypackage]{DataDeps};
	\pkg[mypackage]{DataDepsGenerators};
%	\pkg[mypackage]{ReferenceTests};
	\pkg[mypackage]{MD5};
	\pkg[]{SHA};

	\node(MultiResolutionIterators)[mypackage]{Multi--\\Resolution--\\Iterators.jl};
	
	\pkg[mypackage]{InternedStrings};
	\pkg[]{Strs};
	\pkg[]{CSV};
	

	\pkg{MLDatasets};
%	\pkg[mypackage]{ExpectationStubs};
	\pkg{HTTP};
	\pkg[nudge=(right:15mm)]{WordNet};
	\pkg[mypackage, nudge=(right:4mm)]{Embeddings};


	\pkg[mypackage]{WordTokenizers};
	\pkg{RevTok};

	
	\repo{CRAN};
	\repo{DataDryad};
	\repo{GitHub};
	\repo{European Data Portal};
	\repo[]{Open.Canada.ca};
	\repo{Data.gov};
	\repo{Data.gov.au};
	\repo{DataOne};
	\repo{538};
	\repo{BuzzFeedNews};
	\repo{UCI ML Repository};
	\repo{ArcticDataCenter};
	\repo{KnowledgeNetworkforBiocomplexity};
	\repo{TERN};
	\repo{DataCite};
	\repo{Zenodo};
	\repo{FigShare};
	
	
	\node[sibling distance=0mm](gap){};
	
	\graph{
		(researchcode) <- {(MLDatasets), (Embeddings), (WordNet), (CorpusLoaders)} <- (DataDeps);
		(DataDeps) ->[bend left, looseness=0, in=167, out=0] (researchcode);

	 
		{(CorpusLoaders) <- {(MultiResolutionIterators), (WordTokenizers), (InternedStrings)}};
		(DataDeps) <- {(HTTP), (SHA)};
		(DataDeps) <-[supports] (MD5);
		
		(DataDeps) <-[indirect] (DataDepsGenerators) <- {
			(GitHub) <- {(538), (BuzzFeedNews)},
			(UCI ML Repository),
			(CRAN) <- {(Open.Canada.ca), (Data.gov.au), (Data.gov), (European Data Portal)},
			(DataCite) <- {(Zenodo), (FigShare)},
			(DataDryad),
			(DataOne) <- {(ArcticDataCenter), (TERN), (KnowledgeNetworkforBiocomplexity)},
		};
	
		(WordTokenizers) <-[supports] (RevTok);
		(InternedStrings) <-> [supports] (Strs);
		(InternedStrings) ->[supports] (CSV);
	};
\end{tikzpicture}


\end{document}}}
\end{frame}

\begin{frame}{Vabdewakke's 6 Degree's of Replicability}
	\begin{enumerate}
		%\item 0: The results cannot be reproduced by an independent researcher.
		
		\item The results \alert{cannot seem to be} reproduced.
		
		\item The results could be reproduced by, \alert{requiring extreme effort.}
		
		\item The results can be reproduced, \alert{requiring considerable effort.}
		
		\item The results can be easily reproduced with \alert{at most \textbf{15 minutes} of user effort}, requiring some proprietary source packages (MATLAB, etc.).
		
		\item The results can be easily reproduced  with \alert{at most \textbf{15 min} of user effort}, requiring only standard, freely available tools (C compiler, etc.).		
	\end{enumerate}
	\note{Vabdewakke actual has a rating zero for "no, just no can not reproduce."}
	\note{What is the key determining factor here? User effort. How do we get down to 15 minutes of user effort?}
	\citehere{VabdewakkeReproduceableResearch}
\end{frame}


\begin{frame}{What happens when I try and reproduce someone's research code?}
	\begin{description}
		\item[1min] Find the website from the paper, and \alert{download the code}
		\item[2min] Read enough of the README to get rough bearings
		\item[\textbf{1min}] Find out where to get the data from and \alert{download the data}
		\item[\textbf{2min}] Try and remember how to use \inlinecode{tar -xzfvalphabetsoup} etc.
		\item[\textbf{2min}] Workout how to tell script \alert{where data is} \note{Hardcoded path? Argument?}
		\item[2min] Setup any software dependencies etc.
		\item[3min] Run the code and make sure it isn't crashing etc.
		\item[2min] Interpret the output
	\end{description}
	\note{I have spent 5 of my 15 valuable minutes faffing around about sorting out the data.}
	\note{I think this timeline is fairly reasonable, of course an ideal julia project with everything already working in CI would do a lot better. But if you've got these steps, your already blocked from CI.}
\end{frame}


\begin{frame}[fragile]{You can't trust hardcoded paths; \\ but they are nice to work with.}
	
	\begin{easylist}[itemize]
		@ Ideally we'ld just use \alert{hard-coded, absolute} paths
		@ Absolute paths work with all applications
		@ Hard-coding the paths in code means less typing
		@ But they break if anything is moved. \note{relative paths are less so}
		@ Making the path be passed in as an argument to the script solves this
		@@ but now user has to be typing it in to run it.
		@@ So harder to use.
		@@ You could include a bash-script that invokes it with the path, but now you're just hard coding it somewhere else
	\end{easylist}
\end{frame}

\begin{frame}[fragile]{You could making the path be passed in as an argument. But...}
	
	\begin{easylist}[itemize]
		@ Now user has to be typing it in to run it.
		\note{Recall the valuable time spent to workout how to tell the script where the data is?}
		@ So it is harder to use.
		@ You could include a bash-script that invokes it with the path, but now you're just hard-coding it somewhere else
	\end{easylist}
\end{frame}


\begin{frame}[fragile]{\datadep{Census 2018/populations.csv} \\ A path you can trust}
	\begin{easylist}[itemize]
		@ Always resolves to an absolute path to that file
		\note{at runtime}
		@ Even if that means it has to download it first
		@ But before resorting to downloading checks a large number of places
		@@  \inlinecode{<PKG>/deps/data}, 
		@@ \inlinecode{~/.julia/datadeps},
		@@ \inlinecode{/usr/share/datadeps}, etc.
		\note{this is the DataDeps load path}
		\note{The places to check can be configured of course}
		@ You know that if you use a datadep path it will resolve to a file that exists.
	\end{easylist}
\end{frame}



\begin{frame}[fragile]{Current Usages of DataDeps.jl}
	\structure{MLDatasets.jl}
	\begin{easylist}[itemize]
		@ Provides easy access to a bunch of ML datasets
		@ \inlinecode{xs, ys = MNIST.traindata()}
		@ Gives you regular julia arrays
	\end{easylist}
	\vfill
	\structure{CorpusLoaders.jl}
	\begin{easylist}[itemize]
	@ Provides easy access to linguistic corpora
	@ \inlinecode{corpus_gen = load(WikiCorpus())}
	@ gives you a multi-resolution iterator
	\end{easylist}	
\end{frame}

\begin{frame}[fragile]{Current Usages of DataDeps.jl}	
	\structure{Embeddings.jl}
	\begin{easylist}[itemize]
		@ Provides access to hundreds of pretrained word~embedding models.
		@ \inlinecode{load_embeddings(FastText_Text{:fr})}
		@ gives you a table of French word embeddings.
	\end{easylist}
	
	\vfill
	\structure{WordNet.jl}
	\begin{easylist}[itemize]
		@ Look up lexical relations and definitions.
		@ \inlinecode{lemma = db['a', "glad"]}
		@ \inlinecode{antonyms(db, synsets(db, lemma)[1])}
	\end{easylist}
\end{frame}


\begin{frame}[fragile]{DataDep Registration Block}
	\small
	\begin{code}
	register(DataDep("WordNet 3.0",
		"""
		Dataset: WordNet 3.0
		Website: https://wordnet.princeton.edu/wordnet
		
		George A. Miller (1995). 
		WordNet: A Lexical Database for English.
		Communications of the ACM Vol. 38, No. 11: 39-41.
		
		License:
		This software and database is being provided to you,
		the LICENSEE, by Princeton University under
		the following license...
		""",
		"http://wordnetcode.princeton.edu/3.0/WNdb-3.0.tar.gz",
		"658b1ba191f5f98c2e9bae3e25...";
		post_fetch_method = unpack
	))
	\end{code}
	\vfill\vfill\vfill\null
\end{frame}

\begin{frame}[fragile]{Registration Block: Breakdown}
	\small
	\begin{code}
		register(DataDep("DataDepName",
		"""
		Free Text Field Displayed to user before download.
		Use to give credit, and tell people about licensing.
		Or other messages.
		""",
		"Download URL",
		"file hash (if you skip this, it will be printed on download)";
		post_fetch_method = function to run on downloaded files
		))
	\end{code}
	\vfill\vfill\vfill\null
\end{frame}

\begin{frame}[fragile]{Registration Block: Recursive Example}
	\small
	\begin{code}
		using MD5
		
		register(DataDep("DataDepNameRec",
		"""
		Warning these files are all together 39.8GB
		""",
		["http://example.com/readme.txt",
	     ["http://example.com/data1.zip",
	      "http://example.com/data2.tar.gz",
	     ]
		],
		(md5, "658b1ba191f5f98c2e9bae3e25")
		post_fetch_method = [identity, unpack]
		))
	\end{code}
	\vfill\vfill\vfill\null
	\note{post fetch method and the checksum are applies recursively.
		The checksum xors the hash it if it is not recursed.
		the post fetch method is applied element-wise.}
	\note{Also you can specify the function used to generate the checksum}
\end{frame}


\section{DataDepGenerators.jl}
\begin{frame}[fragile]{Developers still have to write registration blocks}
	\begin{easylist}[itemize]
		@ DataDeps.jl shifts the work from manually to automatic
		@ But defining the work still has to be done.
		@ Writing a registration block normally means copy and pasting from a website.
		@@ Download URL
		@@ Author Name
		@@ Publication Date
		@@ License
	\end{easylist}
\end{frame}

\begin{frame}[fragile]{For published data this information is all available from some API}
\end{frame}

\begin{frame}[fragile]{WordTokenizers.jl}
	\note{Honestly I stole stuff left, right and center for this.}
	Configurable \inlinecode{tokenize}r and \inlinecode{sentence_segment}er.\\
	Abuses \inlinecode{eval} and \alert{\#265} so that you can change the tokenizer being used globally.\\
	Also compatible with externally defined tokenizers like RevTok.jl.
	\vfill	
	Nabbed the original Penn Tokenizer sed-script.\\
	Wrote some code that converts basic \alert{sed language} into \alert{julia AST}.\\
	Ported some of NLTK's tokenizers into sed.
	\vfill

	\vfill
	Rule-based sentence splitter based on Sampo Pyysalo \& Yoshimasa Tsuruoka's perl script.
	\vfill
	Regex is just really good at working with English.	
\end{frame}

\begin{frame}[fragile]{InternedStrings.jl: All these duplicate strings are using all my memory}
	\begin{easylist}[itemize]
		@ Strings are immutable, so we only need one copy of each
		@ We can maintain a pool of \inlinecode{WeakRef}s to each string allocated.
		@ \inlinecode{str = intern(str)}
		@@ Add the \inlinecode{str} to the pool if not already
		@@ replace \inlinecode{str} with an element of the pool
		@ Because the pool only has \inlinecode{WeakRef}s strings can still be garbage collected.
	\end{easylist}
\end{frame}

\begin{frame}[fragile]{MultiResolutionIterators.jl:\\ The structure of a Corpus}
	\begin{easylist}[itemize]
		@ \alert{Corpus}
		@ made up of: \alert{Documents}
		@ made up of: \alert{Paragraphs}
		@ made up of: \alert{Sentences}
		@ made up of: \alert{Words}  or \alert{Tokens}
		@ made up of: \alert{Letters} or \alert{Characters}
	\end{easylist}
\end{frame}


\def\madeupof{$\blacktriangleright$}
\begin{frame}[fragile]{MultiResolutionIterators.jl: Not everyone wants every level of structure}
	\alert{Full corpus structure} is\\
	Documents \madeupof Paragraphs \madeupof Sentences \madeupof Words \madeupof Letters
	\vfill
	
	A \alert{Corpus Linguist} may want\note{studying word usage}\\
	a stream of: Sentences \madeupof Words
	\vfill
	An \alert{Information Retrieval} researcher may want\\
	a stream of: {Documents} \madeupof {Words}
		
	\vfill
	A \alert{char-RNN} language modeller might just want\\
	a stream of Letters
\end{frame}





\end{document}