\RequirePackage{luatex85,shellesc}
\documentclass[dvipsnames]{beamer}
\usepackage[author={Lyndon White}]{pdfcomment}

\usepackage{fancyvrb}

\usepackage{microtype}
\usepackage{adjustbox}
\usepackage{amsmath}

\usepackage[subpreambles=false]{standalone}


\usepackage[at]{easylist}

%\newlength\xunit
%%%%%%%%%%%%%Bibliography
\usepackage[backend=bibtex, url=false,
bibstyle=ieee,firstinits=true]{biblatex}
\renewcommand*{\thefootnote}{} %No symbol or marker
\renewcommand{\footnotesize}{\scriptsize}
%%%%%%%%%%%%%%%%%


\usepackage{xcolor}
\definecolor{chamois}{RGB}{255,255,240}
\definecolor{darkbrown}{RGB}{124,79,0}
\definecolor{UniBlue}{RGB}{83,101,130}

\definecolor{hellgelb}{rgb}{1,1,0.8}
\definecolor{colKeys}{rgb}{0,0,1}
\definecolor{colIdentifier}{rgb}{0,0,0}
\definecolor{colComments}{rgb}{1,0,0}
\definecolor{colString}{rgb}{0,0.5,0}

\usefonttheme{professionalfonts}
\usepackage{tgadventor}
\renewcommand*\familydefault{\sfdefault}
\usepackage[T1]{fontenc}
\usepackage{microtype}


\newcommand{\topline}{%
  \tikz[remember picture,overlay] {%
    \draw[brown,ultra thick] ([yshift=-1.8cm]current page.north west)-- ([yshift=-1.8cm,xshift=\paperwidth]current page.north west);} }

\renewcommand{\topline}{}

\setbeamertemplate{frametitle}{\begin{minipage}[b][1.8cm][c]{\textwidth}%
	\centering%
	\insertframetitle\\\insertframesubtitle
	\end{minipage}}
	

\addtobeamertemplate{frametitle}{}{\topline%
}

\setbeamertemplate{navigation symbols}{}
\setbeamercolor{background canvas}{bg=chamois}
\setbeamercolor{itemize item}{fg=brown}
%\setbeamertemplate{itemize item}{\maltese}
\setbeamercolor{itemize subitem}{fg=brown}
%\setbeamertemplate{itemize subitem}{\begin{rotate}{90}$\diamondsuit$\end{rotate}}

\setbeamercolor{title}{fg=UniBlue}
\setbeamercolor{frametitle}{fg=UniBlue}    
\setbeamerfont{frametitle}{size=\Large}

\setbeamercolor{author}{fg=darkbrown}
\setbeamercolor{institute}{fg=darkbrown}
\setbeamercolor{date}{fg=darkbrown}


\setbeamercolor{structure}{fg=UniBlue}
\setbeamercolor{alerted text}{fg=UniBlue}
\setbeamercolor{alerted text}{fg=UniBlue}
\setbeamercolor{normal text}{fg=darkbrown!50!black}
\setbeamercolor{math text}{fg=darkbrown}
\setbeamercolor{math text displayed}{fg=darkbrown}



\addtobeamertemplate{block begin}{%
	\setlength{\textwidth}{0.8\textwidth}%
}{}
\setbeamercolor{block title}{bg=darkbrown!40,fg=darkbrown!90}
\setbeamercolor{block body}{bg=darkbrown!20,fg=UniBlue}
\setbeamercolor{block title alerted}{bg=yellow!60,fg=red}
\setbeamercolor{block body alerted}{bg=hellgelb!80,fg=UniBlue}

\AtBeginSection[]{
	\begin{frame}
		\vfill
		\centering
		\begin{beamercolorbox}[sep=8pt,center,shadow=true,rounded=true]{title}
			\usebeamerfont{title}\insertsectionhead\par%
		\end{beamercolorbox}
		\vfill
	\end{frame}
}

\setbeamercolor{math text}{fg=bluewrite}
\setbeamercolor{math text displayed}{fg=bluewrite}


\bibliography{master.bib}

%%%%%%%%%%%%%%%%%%%%%%%%%%%%%%%%
\usepackage{pgfplots}
\usepackage{pgfplotstable}

\usepackage{tikz}
\usetikzlibrary{positioning}
\usetikzlibrary{chains}
\usetikzlibrary{decorations.pathmorphing}
%\usetikzlibrary{graphs,graphs.standard,graphdrawing,arrows}
%\usegdlibrary{layered, trees, force}

\usepackage{graphicx}
\graphicspath{{./figs/}, {./}}
\usepackage[space]{grffile}

\usepackage{trimclip}

%%%%%%%%%%%%%%%%%%%%%%%%%%%%%%%%%%%
% TIKZ stuff

\tikzset{%
	->,
	align=center,
	node distance=30mm, sibling distance=25mm, level distance=40mm,
	every edge/.append style = {thick},
	package/.style={draw, circle, very thick},
	mypackage/.style={package, blue},
	repo/.style={draw, very thick, purple},
	info/.style={Green},
	supports/.style={dotted, very thick},
	indirect/.style={decoration={snake}, decorate, very thick, purple}
}

\newcommand{\pkg}[2][]{\tikzset{#2/.append style={}}; \node[#1,package, #2](#2) {#2.jl}}
\newcommand{\repo}[2][]{\node[#1,repo](#2) {#2}}





%%%%%%%%%%%%%%%%%%%%%%%%%%%%%%%%%%%%%%%%%%%%


% all notes should just be simple item notes.
% I neither need nor want more functionality
\let\oldnote\note
\renewcommand{\note}{\oldnote[item]}

\renewcommand{\emph}{\alert}

%%%%%%%%%%%%%%%%
\newcommand{\countertitle}[1]{{\centering \Large \color{bluewrite} #1}}


%%%%%%%%%%%%%%%%%%%%%


\institute{School of Electical, Electronic and Computer Engineering\\The University of Western Australia}
\date{}
\title{Natural language is really simple, unreasonably often}
\subtitle{Adding up word embeddings works far too well}
\author{
%	\includegraphics[height=2cm]{juliaml}
%	\includegraphics[height=2cm]{juliatext}
%	\includegraphics[height=2cm]{juliastring}
%	\\
%	\vspace{5mm}
	\textbf{Lyndon White}}

\logo{\hfill\includegraphics[scale=0.12]{uwa}\hfill\hspace{0.5cm}}

\begin{document}

\centering %Center everywhere
\frame{\maketitle}
\logo{}



\begin{frame}{We like to think language is very complicated}
	
	\countertitle{This is what we do, and complicated models make us feel good and publish well.}
\end{frame}

\begin{frame}{and sometimes it is}
	\countertitle{And so we need the complicated models.}
\end{frame}

\begin{frame}{but sometimes it isn't}
	\countertitle{And so simple methods work}
\end{frame}

\begin{frame}{and often it looks like it is complicated but isn't}
	\countertitle{and so using complicated methods is a mistake leading to worse performance.}
\end{frame}


\begin{frame}{SkipGram etc. are basically iterative algorithms for weighted collocation matrix factorization.}
	\countertitle{When trying to factorize very large matrices numerical linear algebraticians often use iterative methods.}
\end{frame}



\begin{frame}{Bag of words information is not lost in sums of word embeddings}
	\begin{itemize}
		\item A bag of words captures all unigram information in a sentence/document etc
		
	\end{itemize}
	\countertitle{We can reliably recover all words from a sum of word embeddings}
\end{frame}




\begin{frame}[fragile]{Consider Machine Captioning Evaluation}
Correlation with human ranking in the COMPOSITE captioning evaluation dataset. \cite{Aditya2017}

\pgfplotstableread[header=has colnames,
	columns/Method/.style={string type},
	columns/Type/.style={string type}
]{%
Method	Kendall	Type
BLEU	0.191	{basic ngram}
ROGUE-L	0.213	{basic ngram}
\emph{GloVE}	0.337	{MOWE}
METEOR	0.341	{fancy ngram}
CIDEr	0.342	{fancy ngram}
SPICE	0.349	{semantic parsing}
\emph{Word2Vec}	0.349	{MOWE}
%SPICE+CIDEr	0.357	{semantic parsing}
\emph{FastText}	0.364	{MOWE}
}{\compositeCaptionEval}


\begin{tikzpicture}
\begin{axis}[ybar,ymajorgrids,
xticklabels from table={\compositeCaptionEval}{Method},
legend style={at={(0.5,-0.7)}, anchor=north,legend columns=2, draw=none},
legend image post style={xscale=1, yscale=1},
%	enlargelimits=0.3,
ymin=0.1, ymax=0.4,
enlarge x limits=0.2,
xticklabel style={rotate=90},
xtick = data,
width=0.85\textwidth, height= 5cm,
bar width=0.6cm,
%yticklabel={$\mathsf{\pgfmathprintnumber{\tick}}$\small\%},
ylabel = {Kendall's $\tau$},
axis line style={draw=none},
x tick style={draw=none},
ytick = {0.10, 0.15, ..., 0.4},
ytick pos=left,
%nodes near coords={$\mathsf{\pgfmathprintnumber[precision=2]{\pgfplotspointmeta}}$},
]

\addplot[style={top color=Orange, bottom color=Blue, draw=none, mark=none}] table[x expr=\coordindex, y=Kendall]{\compositeCaptionEval};

\end{axis}
\end{tikzpicture}

{\small
\textasteriskcentered Forthcoming publication Naeha Sharif, Lyndon White, Mohammed Bennamoun and Syed Afaq Ali Shah.
}

\note{This is a hard task, somewhere between abstractive summarization and machine translation, and the assessment is also difficult.}
\note{The similarity to abstractive summarization makes it particular bad -- it is not clear what should be said}
\note{Shown here is a comparason of minimum cosine similarity to reference MOWE for different embeddings with more traditional metrics}
\note{See that the performance of the traditional ngram based metrics BLEU for MT and ROGUE for summarization kinda suck}
\note{Next best is the worst performing out of word embedding models GloVE.}
\note{It gets out performed by METEOR and CIDEr, which I would summarize as fancy ngram methods, with synonyms and weights.}
\note{Then comes SPICE. SPICE is very sophisticated, it does parsing and builds this semantic graph, then does graph comparasons.}
\note{But it is beated by Word2Vec, and by FastText}
\note{There are a few reasons why FastText is better than word2vec -- it could be the capturing of morphemes for example. But I'd probably attribute it to a few more years of engineering and a whole lot more data.}
\end{frame}

\begin{frame}{What is going on? How can a unigram method be beating everything?}
	
	\countertitle{Captioning quality can be assessed on \textbf{fluency} and on \textbf{adequacy}}
\end{frame}

\begin{frame}{All captions in COMPOSITE are \textbf{fluent}}
	\begin{itemize}
		\item We are really good at language modelling now.
		\item In theory our RNN language models can capture all needed state
		\item COMPOSITE captions are a mix of \emph{human generated} and \emph{state-of-the-art machine generated}.
	\end{itemize}
	
	\countertitle{Trying to capture fluency in your captioning metric is thus not important}
\end{frame}


\end{document}