\documentclass[twocolumn]{article}
%\documentclass{standalone}
\usepackage{verbatim}
\usepackage{amsthm}
\usepackage{amsmath}
\usepackage{amssymb}
\usepackage{mathtools}
\usepackage{multirow}

\begin{document}
\newcommand{\s}{w_{\blacktriangleright}}
\renewcommand{\ss}{w_{\triangleright}}
\newcommand{\e}{w_{\triangleleft}}
\newcommand{\ee}{w_{\blacktriangleleft}}
\newcommand{\W}{\mathcal{W}}


\subsubsection{Notation}

The following notion is used:

For $\W$ the bag of words to be ordered, including the start ($\s$,$\ss$)
and end pseudo-words.($\e$, $\ee$)

We will write $w_{i}\in\W$ to represent a word from the bag, with
arbitrarily assigned subscripts. Where a word occurs with multiplicity
greater than 1, it is assigned multiple subscripts, and each is treated
as a distinct word.

Each vertex is a sequence of two words, $\langle w_{i},w_{j}\rangle\in\W^{2}$.
This is a Markov state, consisting of a word and its predecessor word. \\
Each edge between two vertices represents a transition from one trigram state to another.
The Start vertex is given by $\langle\s,\ss\rangle$, and the end by $\langle\e,\ee\rangle$.

The GA-TSP districts are given by the sets of all states that have
a given word in the second position. The district for word $w_{j}$
is given by $S(w_{j})\subseteq\W^{2}$, defined as $S(w_{j})=\{\langle w_{i},w_{j}\rangle\,\mid\,w_{i}\ne w_{j}\,\wedge w_{i}\in\W\}$. It is required to visit every district, thus it is required to use every word.




\subsubsection{Optimization Model}

The path and its costs are modelled by:

A transition from $\langle w_{i},w_{j}\rangle$ to $\langle w_{j},w_{k}\rangle$
transitions has cost $$C[\langle w_{i},w_{j}\rangle,\langle w_{j},w_{k}\rangle]=-\log\left(P(w_{k}|w_{i},w_{j}\rangle\right)$$ 
The table of transitions to be optimized is
\begin{multline*}
 \tau[\langle w_{a},w_{b}\rangle,\,\langle w_{c},w_{d}\rangle] = \\
 \begin{cases}
	 \multirow{3}{*}{1} & \mathrm{if\,transition\,from} \\
	 	                & \langle w_{a},w_{b}\rangle \to \langle w_{c},w_{d}\rangle\\
	 	                & \mathrm{occurs} \\
                     0  & \mathrm{otherwise}
  \end{cases}
\end{multline*}

The total cost to be minimized, is given by
\begin{multline*}
 C_{total}(\tau)= \\
	 \sum_{\mathrlap{\!\!\!\!\forall\left[\langle w_{a},w_{b}\rangle,\langle w_{c},w_{d}\rangle\right]\in\left(\W^{2}\right)^{2}
 	}}
 	\;\tau[\langle w_{a},w_{b}\rangle,\,\langle w_{c},w_{d}\rangle] \cdot C[\langle w_{a},w_{b}\rangle,\,\langle w_{c},w_{d}\rangle]
\end{multline*}

\begin{comment}
\begin{multline*}
C_{total}(\tau)= 
\sum_{\mathclap{
		\left[w_{ab},w_{cd}\right]\in\left(\W^{2}\right)^{2}
	}}
	\;\tau[w_{ab},w_{cd}] \cdot C[w_{ab},w_{cd}]
\end{multline*}
\end{comment}

The probability of a particular path (i.e. of a particular ordering)
is thus given by 
\begin{equation*}
P(\tau)=e^{-C_{total}(\tau)}
\end{equation*}

The word order can be found by following the links. The function 
$f(n,\tau)$ gives the word that, according to $\tau$ occurs in the $n$th position.

\begin{multline*}
f(n,\tau)= \\
\begin{cases}
\{w_{a}\mid\tau[\langle\s,\ss\rangle,\langle\ss,w_{a}\rangle]=1\}_{1} & n=1\\
\{w_{b}\mid\tau[\langle\ss,f(1,\tau)\rangle,\langle f(1,\tau),w_{b}\rangle]=1\}_{1} & n=2\\
\{w_{c}\mid\tau[\langle f(n{-}2,\tau),f(n{-}1,\tau)\rangle,\langle f(n{-}1,\tau),w_{c}\rangle]=1\}_{1} & n\ge3
\end{cases}
\end{multline*}

%\pdfcomment{I think maybe the overlarge equation is best replaced with some kind of wordy description}

Note that if $\tau$ follows the constraints that follow then valid then, then
each set is a singleton, $\{\}_{1}$ indicates taking the singleton set's only element.




\subsubsection{Constraints}

The requirements of the problem, place various constraints on to $\tau$:

The Markov state must be maintained: $\forall\langle w_{a},w_{b}\rangle,\langle w_{c},w_{b}\rangle\in\W^{2}$:
\begin{equation*}
w_{b}\ne w_{c} \implies \tau[\langle w_{a},w_{b}\rangle,\,\langle w_{c},w_{d}\rangle]=0
\end{equation*}

Every node entered must also be left -- except those at the beginning and end.
\begin{multline*}
\forall\langle w_{i},w_{j}\rangle\in\W^{2}\backslash\{\langle\s,\ss\rangle,\langle\e,\ee\rangle\}: \\
 \sum_{\mathrlap{\!\!\!\!\forall\langle w_{a},w_{b}\rangle\in\W^{2}}}\tau[\langle w_{a},w_{b}\rangle,\,\langle w_{i},w_{j}\rangle]
=\sum_{\mathrlap{\!\!\!\!\forall\langle w_{c},w_{d}\rangle\in\W^{2}}}\tau[\langle w_{i},w_{j}\rangle,\,\langle w_{c},w_{d}\rangle]
\end{multline*}

Visit (enter) every district exactly once. i.e. use every word exactly once.
\begin{multline*}
\forall w_{j}\in\W\backslash\{\s,\ss\}: \\
\sum_{\mathclap{\forall\langle w_{i},w_{j}\rangle\in S(w_{j}\rangle}}
\sum_{\mathclap{\substack{\; \\ \;\\ \forall\langle w_{a},w_{b}\rangle\in\W^{2}}}}
\tau[\langle w_{a},w_{b}\rangle,\,\langle w_{i},w_{j}\rangle]=1
\end{multline*}

To allow the feasibility checker to detect if ordering the words is
impossible, transitions of zero probability are also forbidden. i.e. if
$P(w_{n}|w_{n-2},w_{n-1})=0$ then $\tau[\langle w_{n-2},w_{n-1}\rangle,\langle w_{n-1},w_{n}\rangle]=0$.
These transitions, if not expressly forbidden, would never occur in
an optimal solution in any case, as they have infinitely high cost.


\paragraph{Lazy Subtour Elimination Constraints}

The problem as formulated above can be input into the MIPS solver. However, like similar formulation of the
travelling salesman problem, some some solutions will have subtours.
We take the usual method for handling this an use callbacks to impose
lazy constraints to forbid such solutions at run-time.  
However the actual formulation of those constraints are different to a typical GA-TSP.

Given a potential solution $\tau$ meeting all other constraints:
\begin{itemize}
\item The core path -- which starts at the beginning $\langle\s,\ss\rangle$
and ends at $\langle\e,\ee\rangle$the end can be found. This is done
by practically following the links from the start node, and accumulating
them into a set $T\subseteq\W^{2}$
\item From the core path, the set of words covered is given by $\W_{T}=\{w_{i}\,\mid\,\forall\langle w_{i},w_{j}\rangle\in T\,\}\cup\{\ee\}$.
If $\mathcal{W_{T}}=\W$ then there are no subtours and the core-path
is the full. Otherwise, there is a subtour to be eliminated.
\item If there is a subtour, then to eliminate it, a constraint must be
added. This constraint is that there must be a connection from at
least one of the nodes in the district covered by the core path to
one of the nodes in the districts not covered.
\item The districts covered by the tour are given by $S_{T}=\bigcup_{w_{t}\in W_{T}}S(w_{t})$,
\item subtour elimination constraint is given by 
\begin{equation*}
  \sum_{\mathclap{\forall\langle w_{t1},w_{t2}\rangle\in S_{T}}}
  \sum_{\mathclap{\substack{\; \\ \;\\ \forall\langle w_{a},w_{b}\rangle\in\W^{2}\backslash S_{T}}}}
  \tau[\langle w_{t1},w_{t2}\rangle,\langle w_{a},w_{b}\rangle]\ge1
\end{equation*}
\item i.e. There must be a transition from one of the states featuring a
word that is in the core path, to one of the states featuring a word
not converted by the core path.

\end{itemize}
It is the formulation around the notion of a core-path that differs this from typical subtour elimination in a GA-TSP. True GA-TSP problems are not guaranteed to have any nodes which must occur. However every word ordering problem is guaranteed to have such a node -- the start and end nodes. This thus makes it possible to identify what is generally the path that requires the least modification to get to a complete solution, as the path to require changes on. Other heuristics, and subtour elimination constraints do exist however.


\end{document}
