\documentclass{article} % For LaTeX2e
\usepackage{iclr2018_conference,times}

\usepackage[author={Lyndon White}]{pdfcomment}

\usepackage{amsmath, amsthm, amssymb, mathtools}

% Referencing

\usepackage{natbib}
\newcommand{\textcite}{\cite}
\newcommand{\parencite}{\citep}

\usepackage{hyperref}
\usepackage{url}
\usepackage{cleveref}

%%%%%%%%%%%%%%%%%%%%%%%%%%%%%%%%%%%%%%%%%%%%%%%%%%%%%%
% Math


\newtheorem{thm}{Theorem}
\newtheorem{lem}[thm]{Lemma}
\newtheorem{prop}[thm]{Proposition}
\newtheorem{cor}{Corollary}[thm]
\newtheorem{defn}{Definition}
\newtheorem{conj}{Conjecture}
\newtheorem{exmp}{Example}[section]
\newtheorem{rem}{Remark}

\newcommand{\R}{\mathbb{R}}
\newcommand{\dlim}{\displaystyle\lim}

%%%%%%%%%%%%%%%%%%%%%%%%%%%%%%%%%%%%%%%%%%%%%%%%%%%%%%&




\title{On Nonnegatively Weighted Neural Networks as Universal Approximators for Monotonic Functions}

\begin{document}
\maketitle

\begin{abstract}
In this work the exact capacities of a Nonnegatively Weighted Neural Network (NNWNN) is proved.
We show that a NNWNN can approximate any bounded, monotonically increasing function;
and no other functions beyond this.
\end{abstract}


\section{Introduction}

\section{Our Results}

\begin{thm} \label{thm:upper}
	Nonnegativly weighted neural networks (NNWNN) always:
	
	\begin{enumerate}
		\item continuous \pdfcomment{idk if I need this statement it is true, but idk if it is useful}
		\item monotonically increasing
		\item bounded above and below
	\end{enumerate}
\end{thm}
\begin{cor}\label{}
	Nonnegativly weighted neural networks (NNWNN) not able to able to approximate any functions that are not monotonically increasing and bounded above and below.
\end{cor}

\begin{thm} \label{thm:nnwnnuat}
	Nonnegativly weighted neural networks (NNWNN) are able to approximate any monotonically increasing functions bounded above and below.
	
	
	For such a function $F$,
	for all points in its domain $x$
	then for all $\varepsilon > 0$, for all $p\in \lbrace 1, \infty \rbrace$
	there exist a set of non-negative weight matrices $W$, $V$ and a bias vector $b$ and $c$,
	and for any squashing function $\varphi$,
	which defines a network of the form:
	\begin{equation}
		N(x;W,V,b,c)=\sum_{j=1}^{j=n_{h}}V_{1,j}\varphi(b_{j}+W_{j,1}x_{1})+c_{1}
	\end{equation}
	
	then 
	\begin{align}
	\left\| N(x;W,V,b, c) - F(x) \right\|_p &< \varepsilon
	\end{align}
	The generalises to multidimentional inputs and outputs.
\end{thm}

\Cref{thm:nnwnnuat}  gives a lower bound on the class of functions able to be approximated by a NNWNN.
\Cref{cor:noapproxapproximate} gives an identical upper bound on the class of functions able to be approximated by a NNWNN.

Thus we know that this bound is exactly the set of functions able to be approximated.

\begin{thm}\label{thm:deep}
	\Cref{thm:upper}, \Cref{cor:noapproxapproximate} and \Cref{thm:nnwnnuat},
	apply also to deep NNWNNs.
\end{thm}

\section{Related Work}

\section{Preliminaries}

\begin{defn}{Squashing Function}
A squashing function is a function $\varphi\:\R \to \R$, that is monotonically increasing, and is bounded above and below.
Many traditional normal neural network activation functions, including the logistic function (sigmoid), $\tanh$, RELU6, are squashing functions.
Other more recent activation functions are not, e.g. RELU, ELU, SELU, and varieties of "leaky" activation functions.
\end{defn}

\begin{defn}{Summability Kernel}
A summability kernel, or approximate identity, is a sequence of continuous real valued functions $\phi_n$ for $n>0$, with the following properties:
\begin{enumerate}
	\item $\forall n>0$, $\int_{-\infty}^{\infty}\phi_{n}(t)\,dt=1$
	\item $\forall n>0\, \exists M \in \R$ such that $\int_{-\infty}^{\infty}|\phi_{n}(t)|\,dt\le M$
	\item $\forall \delta > 0$, ${\displaystyle \lim_{n\to\infty}}\int_{|t|>\delta}|\phi_{n}(t)|\,dt = 0$
\end{enumerate}
\end{defn}

\begin{lem}\label{lem:squashderivissum}
	The derivative of any continuous squashing function gives rise to a summability kernel.
	For any squashing function $\varphi(z)$,
	we define a sequence of functions from scaling its derivative 
    $g \coloneqq \frac{\partial\varphi(z)}{\partial z}$, as follows:
	\begin{equation}
	g_n(z) \coloneqq \frac{g(k_n z)}{k_n\left( c_{max} - c_{min} \right)} 
	\end{equation}
	 where $c_{min}=\lim_{z \to -\infty} \varphi{z}$ and $c_{max}=\lim_{z \to \infty} \varphi{z}$,
	 and $k_n$ is the $n$th element of some increasing sequence of real numbers.
	This sequence is always a summability kernel, for any continuous squashing function.
	
	\pdfcomment{Issue: not all squashing functions I care about are continuous, eg RELU6, but I know this works anyway, see notes.
		See https://math.stackexchange.com/questions/2448729}

\end{lem}
\begin{proof}
For the first property:
\begin{align}
	\int_{-\infty}^{\infty} g_n(z) \, dz &=  \int_{-\infty}^{\infty} \frac{g(nz)}{k_n\left( c_{max} - c_{min} \right)} \, dz \\
	&= \frac{1}{c_{max} - c_{min}} \int_{-\infty}^{\infty} \frac{g(k_n z)}{n} \, dz \\
	&= \frac{\dlim_{t \to \infty} \varphi(k_n z) - \dlim_{t \to -\infty} \varphi(k_n z)}{c_{max} - c_{min}} && \text{(Fundermental Thm of Calculus)}\\
	&= \frac{c_{max} - c_{min}}{c_{max} - c_{min}}  && \text{(as $n>0$)}\\
	&= 1	
\end{align}
	
For the second property: as  $\varphi$ is monotonic, its derivative $g(z)$ is non-negative for all $z$. Thus for $|g_n(z)|=g_n(z)$ and the first property proof applies.

For the third property:
\begin{align}
	\lim_{n\to\infty} \int_{z>\delta} |g_n(z)|\, dz 
	 &= \lim_{n\to\infty} \int_{z>\delta} \frac{g(k_n z)}{n\left( c_{max} - c_{min} \right)} \\
	 &= \lim_{n\to\infty} \frac{1}{c_{max} - c_{min}} 
		 \left( \int_{-\infty}^{-\delta} \frac{g(k_n z)}{n} \, dz 
		 + \int_\delta^\infty \frac{g(k_n z)}{n} \, dz \right) \\
	 &= \lim_{n\to\infty} \frac{
	 	\left(\varphi(-k_n\delta)-\dlim_{t \to -\infty} \varphi(k_n z)\right)
	 	 + \left( \varphi(k_n\delta) - \dlim_{t \to \infty} \varphi(k_n z)\right)}%
	 	 {c_{max} - c_{min}} \\
	 &= \lim_{n\to\infty} \frac{
	 	\left(\varphi(-k_n\delta)-c_{min} \right)
	 	+ \left( \varphi(k_n\delta) - c_{max} \right)}%
	    {c_{max} - c_{min}} \\
	 &= \frac{
	 	\left(c_{min}-c_{min} \right)
	 	+ \left(c_{max} - c_{max} \right)}%
		{c_{max} - c_{min}} \\
	 &= 0
\end{align}

\end{proof}

\begin{lem}\label{lem:approxwithsummability}
For every continuous non-negative function with compact support $f$,
for all points in its domain $x$
then for any $\epsilon >0$ there exists an element $\phi_n$ from from a summability kernel $\lbrace \phi_i  \rbrace_{i>0}$, and a sequence of translation terms $c_1,\ldots,c_M$ and positive weights $b_1,\ldots,b_M$ with $\sum_{i=1}^{i=M} b_i = 1$ such that:


\begin{align}
\left\Vert f(x)-\sum_{i=1}^{i=M}b_{i}\phi_{n}(x-c_i)\right\Vert _{1}<\epsilon &&
 \text{and} &&
\left\Vert f(x)-\sum_{i=1}^{i=M}b_{i}\phi_{n}(x-c_i)\right\Vert _{\infty}<\epsilon
\end{align}

\end{lem}

This proof is based on a similar proof for Gaussian density functions which appears as Lemma 2.2 of \textcite{bacharoglou2010approximation}.
It relies on the uniform convergence of the convolution of a summability kernel and a continuous function to the continuous function, and the uniform convergence of the Riemann sum to an integral.
Another notionally similar (if significantly more dense) proof also appears as that for Theorem 3.1 of \textcite{NESTORIDIS20111783}.


\begin{proof}
	$\forall p \in \lbrace 1,\infty \rbrace \forall x  \forall \epsilon_1 \exists N_1>0 \forall n_1>N_1$ we have
	\begin{align}
	\left\| f \ast \phi_{n_1}(x) - f(x) \right\|_p &< \epsilon_1 && \text{approximate identity property} \\
	\left\| \int_{-\infty}^\infty f(t)\phi_{n_1}(x-t) \, dt - f(x)\right\|_p &< \epsilon_1 && \text{definition of convolution} \\
	\left\| \int_{t \in supp(f)]} f(t)\phi_{n_1}(x-t) \, dt - f(x)\right\|_p &< \epsilon_1 && \text{compact support of $f$}
	\end{align}
	
	By the use of a Riemann sum, over slices $V_1,\ldots, V_M$ covering the support of $f$, each containing one of the points $t_1,...,t_m$ respectively,
	we have for $|V_i|$ being the area of slice $V_i$ then
	$\forall \epsilon_2 \exists N_2>0 \forall n_2>N_2$ it is such that 
	\begin{align}
		\left\| f \ast \phi_{n_1}(x) - \sum^{n_2}_{i=1} f(t_i)|V_i|\phi_{n_1}(x-t_i) \right\|_p &< \epsilon_2 && \text{Riemann sum} 
	\end{align}
	
	Combining these and using $N > N_1 + N_2$ $n>N$ then by the triangle inequality we have:
	\begin{align}
	\left\| f(x) - \sum^{n}_{i=1} f(t_i)|V_i|\phi_n(x-t_i) \right\|_p &<  &&  \\
	\left\| f \ast \phi_n(x) - \sum^{n}_{i=1} f(t_i)|V_i|\phi_n(x-t_i) \right\|_p 
	+  \left\| f \ast \phi_n(x) - f(x) \right\|_p &< \epsilon_1 + \epsilon_2 
	\end{align}
	
	As $|V_i|$ as a volumes is greater than zero, and $f(t_i) \ge 0$ as $f$ is a non-negative function, we thus re-express the above using $b_i=f(t_i)|V_i|$.
	Also setting $c_i=t_i$, $\epsilon=\epsilon_1+\epsilon_2$.
	
	Thus 
	\begin{align}
	\left\| f(x) - \sum^{n}_{i=1} b_i \phi_n(x-c_i) \right\|_p &<  \epsilon
	\end{align}
\end{proof}

\subsection{Proof of \Cref{thm:upper}}
The proofs of the basic properties of NNWNNs,
as simple and boil down to well known properties of the spaces of continuous/bounded/monotonically increasing functions being closed under various operations.
\begin{proof}
	We begin by considering some NNWNN with a single hidden layer, and an affine output layer.
	For $\varphi$ being a squashing function, with weight matrixes $W$ and $V$ and bias vectors $b$ and $c$.
	As it is a NNWNN $W_{i,j} \ge 0$ and $V_{i,j} \ge 0$.
	
	For the case of a single output, single input network, with $n_h$ neurons in the hidden layer the network is described by
	\begin{equation}
	N(x;W,V,b,c) = \sum_{j=1}^{j=n_h} V_{1,j} \varphi (b_j+W_{j,1}x_1) + c_1
	\end{equation}

	For the first property: that NNWNN is continuous,
	consider: that $\varphi$ is continuous,
	$x \mapsto W_{j,1}x + b_j$ is continuous,
	and continuous functions are closed under addition, scalar multiplication, and composition.
	For further details see e.g. \textcite{rudin1976principles} (chapter 4.)
	
	For the second property: the NNWNNs are monotonically increasing:
	consider input points $\alpha$ and $\beta$, such that $\alpha\ge\beta$.
	\begin{align}
	\alpha &\ge \beta \\
	W_{j,1}\alpha+b_j &\ge W_{j,1}\beta+b_j && \text{as $W_{j,i}\ge0$} \\
	\varphi(W_{j,1}\alpha+b_j) &\ge \varphi(W_{j,1}\beta+b_j) && \text{as $\varphi$ is monotonically increasing} \\
	V_{1,j}\varphi(W_{j,1}\alpha+b_j) &\ge V_{1,j}\varphi(W_{j,1}\beta+b_j) && \text{as $V_{1,j}\ge 0$} \\
	\sum_{j=1}^{j=n_h}V_{1,j}\varphi(W_{j,1}\alpha+b_j) &\ge \sum_{j=1}^{j=n_h}V_{1,j}\varphi(W_{j,1}\beta+b_j) && \text{term-wise} \\
	N(\alpha;W,V,b,c) &\ge N(\beta;W,V,b,c)
	\end{align}

	for the final property, that NNWNNs are bounded above and below,
	consider that $\varphi$ is bounded above and below.
	$\forall z$
	\begin{align}
	c_{min} \le \varphi(z) &\le c_{max} \\
	c_{min} \le \varphi(W_{j,1}+b) &\le c_{max} \\
	%
	\sum_{j=1}^{j=n_h}V_{1,j} c_{min} 
	\le \sum_{j=1}^{j=n_h}V_{1,j}\varphi(W_{j,1}x+b_j)
    &\le \sum_{j=1}^{j=n_h}V_{1,j}c_{max} \\
    %
    \sum_{j=1}^{j=n_h}V_{1,j} c_{min} 
    \le N(x;W,V,b,c)
    &\le \sum_{j=1}^{j=n_h}V_{1,j}c_{max}
	\end{align}	
\end{proof}

The proof of the collary \Cref{cor:noapproxapproximate} follows directly from this:
\begin{proof}
	A monotonic function $f$ can not converge uniformly towards a non-monotonic function $h$.
	Consider $f$ monotonic increasing (WLOG),
	As $h$ is not monotonically increasing, for some values $a$,$b$ $a>b$, 
	$h(a)<h(b)$.
	It is clear in such a case, that the closes a monotonically increasing function could come to this is to be some constant between those points.
	Which means there will alway be a nonzero difference over that area.

		
	
	A bounded monotonic function $h$ can not converge uniformly towards a non-bounded monotonic function $f$.
	Consider the limit for $c$ an upper boundary of $h$,
	as $f$ is not bounded, $\exists k$ such that $x>k \implies f(x)>c$.
	Thus for $x>k$ $|f(x)-h(x)|\ge h(x)-c$ which grows without bound.
	
	That as \Cref{thm:upper} shows that the NNWNN has the properties of being monotonically increasing and bounded, it can not converge uniformly to such functions.
\end{proof}


\subsection{Proof of \Cref{thm:nnwnnuat}}
\begin{proof}


We begin again, by considering some NNWNN with a single hidden layer, and an affine output layer.
For $\varphi$ being a squashing function, with weight matrixes $W$ and $V$ and bias vectors $b$ and $c$.
As it is a NNWNN $W_{i,j} \ge 0$ and $V_{i,j} \ge 0$.

For the case of a single output, single input network, with $n_h$ neurons in the hidden layer the network is described by
\begin{equation}
N(x;W,V,b,c) = \sum_{j=1}^{j=n_h} V_{1,j} \varphi (b_j+W_{j,1}x_1) + c_1
\end{equation}

The for the case with $n_i$ inputs and $n_o$ outputs this becomes:

\begin{equation}
N(x;W,V,b,c)=V\varphi(Wx+b)=\left[\begin{array}{c}
{\displaystyle \sum_{j=1}^{j=n_{h}}}V_{1,j}\varphi\left(b_{j}+\sum_{i=1}^{i=n_{i}}\varphi\left(W_{j,i}x_{i}\right)\right) + c_1\\
\vdots\\
{\displaystyle \sum_{j=1}^{j=n_{h}}}V_{n_{o},j}\varphi\left(b_{j}+\sum_{i=1}^{i=n_{i}}\varphi\left(W_{j,i}x_{i}\right)\right) +c_{n_o}
\end{array}\right]
\end{equation}

For the remained of this proof we will work with the single input single output case.
The generalisation to multi-input multi-output is straight forward, if verbose (to simply for the reader considering this generalisation, we will leave the subscripts for singleton dimensions in).

We consider the network's derivative with respect to its input $x$.

\begin{equation}
\frac{\partial N(x;W,V,b,c)}{\partial x_1} = \sum_{j=1}^{j=n_h} V_{1,j}W_{j,1} \frac{\partial(\varphi) (b_j+W_{j,1}x_1)}{\partial x_1}
\end{equation}

We can re-express the value of $V_{1,j}$ as $V_{1,j}=\frac{\alpha_{1,j}}{W_{j,1}^2(c_{max}-c_{min})}$,
for $c_{min}$ and $c_{max}$ the lower and upper bounds of the squashing function ($\varphi$) respectively.
Note that $\alpha_{1,j}$ is non-negative, as a consequence of $V_{1,j}$ and $W_{j,1}$ being non-negative.

We also re-express $b_j$ as $b_j=\frac{\beta_{j,1}}{W_{j,1}}$  using $g = \frac{\partial\varphi(z)}{\partial z}$.


\begin{align}
\frac{\partial N(x;W,V,b,c)}{\partial x_1} 
&= \sum_{j=1}^{j=n_h} \frac{\alpha_{1,j}}{W_{j,1}^2(c_{max}-c_{min})} \, W_{j,1} g(\frac{W_{j,1}\beta_{j,1}}{W_{j,1}} + W_{j,1}x_1)\\
\frac{\partial N(x;W,V,b,c)}{\partial x_1} 
&= \sum_{j=1}^{j=n_h} \alpha_{1,j} \frac{ g(W_{j,1}\left( x_1 + \beta_{j,1} \right))}{W_{j,1}(c_{max}-c_{min})} \\
\end{align}

We note that this is now in the form required for \Cref{lem:squashderivissum},
to apply.
So we use the summability kernel thus described $g_n(x) = \frac{k_n g(k_n x)}{k_n(c_{max}-c_{min})}$.
for arbitrary choice of $1 \le J \le n_h$ such that $W_{J,1} \ne 0$
define any arbitrary increasing sequence $k$ such that $k_n=W_{J,1}$,
not that we have not placed any restriction on $k_n=W_{J,1}$ beyond that it is positive -- it can be arbitrarily large.
We assign  $a_j=0$ for all $j \ne J$.


\begin{align}
\frac{\partial N(x;W,V,b)}{\partial x_1} 
&= \sum_{j=1}^{j=n_h} \alpha_{1,j} g_n\left( x_1 + \beta_{j,1} \right))
\end{align}

We can apply \Cref{lem:approxwithsummability}, and thus conclude:
for all non-negative continuous functions with compact support $f$,
for all points in its domain $x$
then for all $\epsilon > 0$, for all $p\in \lbrace 1, \infty \rbrace$
there exist a set of non-negative weight matrices $W$, $V$ (whose dimensions are determined by the hidden layer size $n_h$) and a bias vector $b$ and $c$
such that 
\begin{align}
\left\|\frac{\partial N(x;W,V,b, c)}{\partial x_1} - f(x) \right\|_p < \epsilon
\end{align}

We can thus say that $\frac{\partial N(x;W,V,b,c)}{\partial x_1}$ is dense in the space of non-negative continuous functions with compact support.

From this by application of the fundamental theorem of calculus,
we have
\begin{align}
\left\| \frac{\partial N(x;W,V,b, c)}{\partial x_1} - f(x) \right\|_p &< \epsilon \\
\left\| \int_{\inf supp(f)}^y \frac{\partial N(x;W,V,b, c)}{\partial x_1} - f(x) \, dx \right\|_p &< \left( y-\inf supp(f) \right)  \epsilon \\
\left\| N(y;W,V,b, c) - N(\inf supp(f);W,V,b, c) - \int_{\inf supp(f)}^y f(x) \, dx \right\|_p &< \left( y-\inf supp(f) \right)  \epsilon\\
\end{align}

\pdfcomment{Is this c-star right?}

We chose a value for $c$ as $c^\star=-N(\inf supp(f);W,V,b, 0)$, thus $N(\inf supp(f);W,V,b, c^\star)=0$.
We rewrite using $\varepsilon = \left( y-\inf supp(f) \right)  \epsilon$
\begin{align}
\left\| N(y;W,V,b, c^\star) - F(y) \right\|_p &< \varepsilon
\end{align}

Consider what $F$ represents.
As it is the integral of a non-negative continuous function, it is a monotonically increasing differentiable function.
As outside the support of $f$ its gradient is zero, it is thus bounded below, and above.

As the space of bounded, differentiable functions functions is dense in the space of bounded functions, we can relax the constraint on $F$ being differentiable.
\pdfcomment{TODO: Finish this line with a citation to something, I think this is correct.}

We thus conclude with:

For all monotonically increasing functions bounded above and below $F$,
for all points in its domain $x$
then for all $\varepsilon > 0$, for all $p\in \lbrace 1, \infty \rbrace$
there exist a set of non-negative weight matrices $W$, $V$ (whose dimensions are determined by the hidden layer size $n_h$) and a bias vector $b$ and $c$

\begin{align}
\left\| N(x;W,V,b, c) - F(x) \right\|_p &< \varepsilon
\end{align}
which is \Cref{thm:nnwnnuat}.
\end{proof}

\subsection{proof of \Cref{thm:deep}}
\begin{proof}
A deep NNWNN, is equivalent to the composition of several single layer NNWNNs.
This is bounded, as the composition of bounded functions is bounded,
and monotonically increasing as the composition of monotonically increasing functions is monotonically increasing.

It is able to uniformly converge to any monotonically increasing, bounded function,
as consider the case where all layers other than the first approximate the identity function over the bounds -- the case is thus reduced to first layer, which is covered by \Cref{thm:nnwnnuat}.
\end{proof}

\section{Some theoretical applications}

\subsection{Incorporating prior knowledge}
When training a machine learning system,
one often wants to incorporate prior knowledge from domain experts.
One large class of such knowledge is that there exists a monotonic relationship between the dependent (output) variable, and one of the independent (observation/input) variables.
For example: the price of a used car decreases with its age, all other features being the same.

Consider a single output network, with multiple inputs.,
If the output is monotonically increasing with input feature $x_\alpha$,
then one only needs to constrain the weights for neurons which occur on possible paths between input an the output.
For a network with a single hidden layer  and an output layer as discussed above,
this means constraining only $\forall j$, $V_{1,j} \ge 0$ and $W_{j, \alpha} \ge 0$.
For deeper NNs this is more complicated. \pdfcomment{Do I need to explain how this is done, I find it hard to think about?}

If output  is monotonically decreasing $x_\alpha$, then one can preprocess the data to invert or negate the value of that feature, and proceed as above.

The proof that this mixed network is an universal monotonic approximator in  $x_\alpha$ is not presented here formally.
However, it boils down to, if one considers all other values $x_\gamma$ $\gamma \ne 0$ as fixed, then the network is just a single-input NNWNN, with the other values driving a bias term that can be cancelled.

Conversely, it is an universal approximator in the other inputs,
a treating the value of $x_\alpha$ as fixed similarly reduces the network to a conventional neural network and the universal approximation theorem applies.




\subsection{Non-parametric Estimation}
One particularly important class of bounded monotonically increasing functions is
cumulative density functions (CDFs).
CDFs have the additional restriction that the lower bound must be zero and the upper bound must be one.
A NNWNN can be fitted to meet that constraint; or we can encode it into the network.
While theoretically encoding it into the network is not required,
in practice is significantly simplifies training,
and gives a guarantee that even if the network does not fit as an optimal non-parametric estimator, (e.g. when partially trained)
it is valid CDF.



For any continuous distribution over $\R$, $\forall \epsilon$ we can define an lower and upper boundary $s_{min}$ and $s_{max}$ such that $Pr(x<s_{min})+Pr(x>s_{max}) \le \epsilon$.
such that outside
$s_min \le x \le s_max$.
\pdfcomment{I believe this is any distribution, for suitably large support}.





	
\section{Conclusion and Future Work}

\bibliography{master}
\bibliographystyle{iclr2018_conference}

\end{document}
