\documentclass{article} % For LaTeX2e
\usepackage{iclr2018_conference,times}

\usepackage[author={Lyndon White}]{pdfcomment}


%%%%%%%%%%%%%%%%%%%%%%%%%%%%%%%%%%%%%%%%%%%%%%%%%%%%%%
% Math
\usepackage{amsmath, amsthm, amssymb, mathtools}

\newtheorem{thm}{Theorem}[section]
\newtheorem{lem}[thm]{Lemma}
\newtheorem{prop}[thm]{Proposition}
\newtheorem{cor}{Corollary}
\newtheorem{defn}{Definition}[section]
\newtheorem{conj}{Conjecture}[section]
\newtheorem{exmp}{Example}[section]
\newtheorem{rem}{Remark}

\newcommand{\R}{\mathbb{R}}
\newcommand{\dlim}{\displaystyle\lim}

%%%%%%%%%%%%%%%%%%%%%%%%%%%%%%%%%%%%%%%%%%%%%%%%%%%%%%&
% Referencing

\usepackage{natbib}
\newcommand{\textcite}{\cite}
\newcommand{\parencite}{\citep}

\usepackage{hyperref}
\usepackage{url}
\usepackage{cleveref}






\title{On Nonnegatively Weighted Neural Networks and Mixture Models}



\begin{document}
\maketitle

\begin{abstract}

\end{abstract}


\section{Introduction}

\section{Our Results}

\begin{thm}
	Nonnegativly weighted neural networks (NNWNN) are (at least) equivalent to the integral of unnormalized (Gaussian-like) mixture models, and this means it can approximate a class of functions that can be described as the set of unnormalised continuous cumulative probability density functions (CDFs) with compact supports.
\end{thm}

\section{Related Work}

\section{Preliminaries}

\begin{defn}{Squashing Function}
A squashing function is a function $\varphi\:\R \to \R$, that is monotonic, and is bounded above and below.
Many traditional normal neural network activation functions, including the logistic function (sigmoid), $\tanh$, RELU6, are squashing functions.
Other more recent activation functions are not, e.g. RELU, ELU, SELU, and varieties of "leaky" activation functions.
\end{defn}

\begin{defn}{Summability Kernel}
A summability kernel, or approximate identity, is a sequence of continuous real valued functions $\phi_n$ for $n>0$, with the following properties:
\begin{enumerate}
	\item $\forall n>0$, $\int_{-\infty}^{\infty}\phi_{n}(t)\,dt=1$
	\item $\forall n>0\, \exists M \in \R$ such that $\int_{-\infty}^{\infty}|\phi_{n}(t)|\,dt\le M$
	\item $\forall \delta > 0$, ${\displaystyle \lim_{n\to\infty}}\int_{|t|>\delta}|\phi_{n}(t)|\,dt = 0$
\end{enumerate}
\end{defn}

\begin{lem}\label{lem:squashderivissum}
	The derivative of any continuous squashing function gives rise to a summability kernel.
	For any squashing function $\varphi(z)$,
	we define a sequence of functions from scaling its derivative 
    $g \coloneqq \frac{\partial\varphi(z)}{\partial z}$, as follows:
	\begin{equation}
	g_n(z) \coloneqq \frac{g(k_n z)}{k_n\left( c_{max} - c_{min} \right)} 
	\end{equation}
	 where $c_{min}=\lim_{z \to -\infty} \varphi{z}$ and $c_{max}=\lim_{z \to \infty} \varphi{z}$,
	 and $k_n$ is the $n$th element of some increasing sequence of real numbers.
	This sequence is always a summability kernel, for any continuous squashing function.
	
	\pdfcomment{Issue: not all squashing functions I care about are continuous, eg RELU6, but I know this works anyway, see notes.
		See https://math.stackexchange.com/questions/2448729}

\end{lem}
\begin{proof}
For the first property:
\begin{align}
	\int_{-\infty}^{\infty} g_n(z) \, dz &=  \int_{-\infty}^{\infty} \frac{g(nz)}{k_n\left( c_{max} - c_{min} \right)} \, dz \\
	&= \frac{1}{c_{max} - c_{min}} \int_{-\infty}^{\infty} \frac{g(k_n z)}{n} \, dz \\
	&= \frac{\dlim_{t \to \infty} \varphi(k_n z) - \dlim_{t \to -\infty} \varphi(k_n z)}{c_{max} - c_{min}} && \text{(Fundermental Thm of Calculus)}\\
	&= \frac{c_{max} - c_{min}}{c_{max} - c_{min}}  && \text{(as $n>0$)}\\
	&= 1	
\end{align}
	
For the second property: as  $\varphi$ is monotonic, its derivative $g(z)$ is non-negative for all $z$. Thus for $|g_n(z)|=g_n(z)$ and the first property proof applies.

For the third property:
\begin{align}
	\lim_{n\to\infty} \int_{z>\delta} |g_n(z)|\, dz 
	 &= \lim_{n\to\infty} \int_{z>\delta} \frac{g(k_n z)}{n\left( c_{max} - c_{min} \right)} \\
	 &= \lim_{n\to\infty} \frac{1}{c_{max} - c_{min}} 
		 \left( \int_{-\infty}^{-\delta} \frac{g(k_n z)}{n} \, dz 
		 + \int_\delta^\infty \frac{g(k_n z)}{n} \, dz \right) \\
	 &= \lim_{n\to\infty} \frac{
	 	\left(\varphi(-k_n\delta)-\dlim_{t \to -\infty} \varphi(k_n z)\right)
	 	 + \left( \varphi(k_n\delta) - \dlim_{t \to \infty} \varphi(k_n z)\right)}%
	 	 {c_{max} - c_{min}} \\
	 &= \lim_{n\to\infty} \frac{
	 	\left(\varphi(-k_n\delta)-c_{min} \right)
	 	+ \left( \varphi(k_n\delta) - c_{max} \right)}%
	    {c_{max} - c_{min}} \\
	 &= \frac{
	 	\left(c_{min}-c_{min} \right)
	 	+ \left(c_{max} - c_{max} \right)}%
		{c_{max} - c_{min}} \\
	 &= 0
\end{align}

\end{proof}

\begin{lem}\label{lem:approxwithsummability}
For every PDF of a distribution with compact support $f$,
for all points in its domain $x$
then for any $\epsilon >0$ there exists a finite sequence of terms from a summability kernel $\phi_1, \ldots, \phi_M$, with translation terms $c_1,\ldots,c_M$ and positive weights $b_1,\ldots,b_M$ with $\sum_{i=1}^{i=M} b_i = 1$ such that:

\begin{align}
\left\Vert f(x)-\sum_{i=1}^{i=M}b_{i}\phi_{i}(x-c_i)\right\Vert _{1}<\epsilon &&
 \text{and} &&
\left\Vert f(x)-\sum_{i=1}^{i=M}b_{i}\phi_{i}(x-c_i)\right\Vert _{\infty}<\epsilon
\end{align}

\end{lem}

A very close analogy of this lemma, and its proof appears as Lemma 2.2 of \parencite{bacharoglou2010approximation}, where the focus was on more restricted sequences of Gaussian density functions. The detail and proof remain unchanged when applied to all summability kernels.
It relies on the uniform convergence of the convolution of a summability kernel and a continuous function to the continuous function, and the uniform convergence of the Riemann sum to an integral.
\textcite{bacharoglou2010approximation} goes on to prove several other stronger statements, which they note, all all generally applicable to summability kernels.
We paraphrase the core proof here, as applied to summability kernels.
\begin{proof}
	$\forall p \in \lbrace 1,\infty \rbrace \forall x  \forall \epsilon_1 \exists N_1>0 \forall n_1>N_1$ we have
	\begin{align}
	\left\| f \ast \phi_{n_1}(x) - f(x) \right\|_p &< \epsilon_1 && \text{approximate identity property} \\
	\left\| \int_{-\infty}^\infty f(t)\phi_{n_1}(x-t) \, dt - f(x)\right\|_p &< \epsilon_1 && \text{definition of convolution} \\
	\left\| \int_{t \in supp(f)]} f(t)\phi_{n_1}(x-t) \, dt - f(x)\right\|_p &< \epsilon_1 && \text{compact support of $f$}
	\end{align}
	
	By the use of a Riemann sum, over slices $V_1,\ldots, V_M$ covering the support of $f$, each containing one of the points $t_1,...,t_m$ respectively,
	we have for $|V_i|$ being the area of slice $V_i$ then
	$\forall \epsilon_2 \exists N_2>0 \forall n_2>N_2$ it is such that 
	\begin{align}
		\left\| f \ast \phi_(x) - \sum^{n_2}_{i=1} f(t_i)|V_i|\phi_{n_1}(x-t_i) \right\|_p &< \epsilon_2 && \text{Riemann sum} 
	\end{align}
	
	Combining these and using $N > N_1 + N_2$ $n>N$ then by the triangle inequality we have:
	\begin{align}
	\left\| f(x) - \sum^{n}_{i=1} f(t_i)|V_i|\phi_n(x-t_i) \right\|_p &<  &&  \\
	\left\| f \ast \phi_n(x) - \sum^{n}_{i=1} f(t_i)|V_i|\phi_n(x-t_i) \right\|_p 
	+  \left\| f \ast \phi_n(x) - f(x) \right\|_p &< \epsilon_1 + \epsilon_2 
	\end{align}
	
	By the mean value theorem there exist suitable values for $V_i$ and $t_i$, such that $\sum_{i=1}^{i=N}f(t_i)|V_i|=1$, and as $|V_i|$ as a volumes is greater than zero, and $f(t_i)>0$ as $f$ is a pdf, we thus choose $b_i=f(t_i)|V_i|$, and can always select sufficiently small $\epsilon_1$ and $\epsilon_2$ choose $c_i=t_i$,
	to have $\epsilon=\epsilon_1+\epsilon_2$.
	
	Thus 
	\begin{align}
	\left\| f(x) - \sum^{n}_{i=1} b_i \phi_n(x-c_i) \right\|_p &<  \epsilon
	\end{align}

\end{proof}



\section{Proof of Theorem 1}

We begin by considering some neural network in the UAT form (single hidden layer, linear output layer),
$\varphi$ being a squashing function, with weight matrixes $W$ and $V$ and bias vector $b$.
As it is a NNWNN $W_{i,j} \ge 0$ and $V_{i,j} \ge 0$.

For the case of a single output, single input network, with $n_h$ neurons in the hidden layer.
\begin{equation}
N(x;W,V,b) = \sum_{j=1}^{j=n_h} V_{1,j} \varphi (b_j+W_{j,1}x_1)
\end{equation}

The for the case with $n_i$ inputs and $n_o$ outputs this becomes:

\begin{equation}
N(x;W,V,b)=V\varphi(Wx+b)=\left[\begin{array}{c}
{\displaystyle \sum_{j=1}^{j=n_{h}}}V_{1,j}\varphi\left(b_{j}+\sum_{i=1}^{i=n_{i}}\varphi\left(W_{j,i}x_{i}\right)\right)\\
\vdots\\
{\displaystyle \sum_{j=1}^{j=n_{h}}}V_{n_{o},j}\varphi\left(b_{j}+\sum_{i=1}^{i=n_{i}}\varphi\left(W_{j,i}x_{i}\right)\right)
\end{array}\right]
\end{equation}

For the remained of this proof we will work with the single input single output case.
The generalisation to multi-input multi-output is straight forward, if verbose (to simply for the reader considering this generalisation, we will leave the subscripts for singleton dimensions in).

We consider the network's derivative with respect to its input $x$.

\begin{equation}
\frac{\partial N(x;W,V,b)}{\partial x_1} = \sum_{j=1}^{j=n_h} V_{1,j}W_{j,1} \frac{\partial(\varphi) (b_j+W_{j,1}x_1)}{\partial x_1}
\end{equation}

We can re-express the value of $V_{1,j}$ as $V_{1,j}=\frac{\alpha_{1,j}}{W_{j,1}^2(c_{max}-c_{min})}$,
for $c_{min}$ and $c_{max}$ the lower and upper bounds of the squashing function ($\varphi$) respectively.
We also re-express $b_j$ as $b_j=\frac{\beta_{j,1}}{W_{j,1}}$  using $g = \frac{\partial\varphi(z)}{\partial z}$.


\begin{align}
\frac{\partial N(x;W,V,b)}{\partial x_1} 
&= \sum_{j=1}^{j=n_h} \frac{\alpha_{1,j}}{W_{j,1}^2(c_{max}-c_{min})} \, W_{j,1} g(\frac{W_{j,1}\beta_{j,1}}{W_{j,1}} + W_{j,1}x_1)\\
\frac{\partial N(x;W,V,b)}{\partial x_1} 
&= \sum_{j=1}^{j=n_h} \alpha_{1,j} \frac{ g(W_{j,1}\left( x_1 + \beta_{j,1} \right))}{W_{j,1}(c_{max}-c_{min})} \\
\end{align}

We note that this is now in the form required for \Cref{lem:squashderivissum},
to apply.
So we use the summability kernel thus described $g_n(x) = \frac{k_n g(k_n x)}{k_n(c_{max}-c_{min})}$
For $k_n = W_{j,1}$ from some suitable sequence $k$.

\begin{align}
\frac{\partial N(x;W,V,b)}{\partial x_1} 
&= \sum_{j=1}^{j=n_h} \alpha_{1,j} g_n\left( x_1 + \beta_{j,1} \right))
\end{align}

\section{Conclusion and Future Work}

\bibliography{master}
\bibliographystyle{iclr2018_conference}

\end{document}
