\documentclass{article} % For LaTeX2e
\usepackage{iclr2018_conference,times}

\usepackage[author={Lyndon White}]{pdfcomment}


%%%%%%%%%%%%%%%%%%%%%%%%%%%%%%%%%%%%%%%%%%%%%%%%%%%%%%
% Math
\usepackage{amsmath, amsthm, amssymb, mathtools}

\newtheorem{thm}{Theorem}[section]
\newtheorem{lem}[thm]{Lemma}
\newtheorem{prop}[thm]{Proposition}
\newtheorem{cor}{Corollary}
\newtheorem{defn}{Definition}[section]
\newtheorem{conj}{Conjecture}[section]
\newtheorem{exmp}{Example}[section]
\newtheorem{rem}{Remark}

\newcommand{\R}{\mathbb{R}}
\newcommand{\dlim}{\displaystyle\lim}

%%%%%%%%%%%%%%%%%%%%%%%%%%%%%%%%%%%%%%%%%%%%%%%%%%%%%%&
% Referencing

\usepackage{hyperref}
\usepackage{url}
\usepackage{cleveref}






\title{On Nonnegatively Weighted Neural Networks and Mixture Models}



\begin{document}
\maketitle

\begin{abstract}

\end{abstract}


\section{Introduction}

\section{Our Results}

\section{Related Work}

\section{Preliminaries}

\begin{defn}{Squashing Function}
A squashing function is a function $\varphi\:\R \to \R$, that is monotonic, and is bounded above and below.
Many traditional normal neural network activation functions, including the logistic function (sigmoid), $\tanh$, RELU6, are squashing functions.
Other more recent activation functions are not, e.g. RELU, ELU, SELU, and varieties of "leaky" activation functions.
\end{defn}

\begin{defn}{Summability Kernel}
A summability kernel, or approximate identity, is a sequence of continuous real valued functions $\phi_n$ for $n>0$, with the following properties:
\begin{enumerate}
	\item $\forall n>0$, $\int_{-\infty}^{\infty}\phi_{n}(t)\,dt=1$
	\item $\forall n>0\, \exists M \in \R$ such that $\int_{-\infty}^{\infty}|\phi_{n}(t)|\,dt\le M$
	\item $\forall \delta > 0$, ${\displaystyle \lim_{n\to\infty}}\int_{|t|>\delta}|\phi_{n}(t)|\,dt = 0$
\end{enumerate}
\end{defn}

\begin{lem}{The derivative of any squashing function gives rise to a summability kernel.}
	For any squashing function $\varphi(z)$,
	by taking its derivative $g \coloneqq \frac{\partial\varphi(z)]}{\partial z}$ allows us to define a sequence of functions 
	\begin{equation}
	g_n(z) \coloneqq \frac{g(nz)}{n\left( c_{max} - c_{min} \right)} 
	\end{equation}
	 where $c_{min}=\lim_{z \to -\infty} \varphi{z}$ and $c_{max}=\lim_{z \to \infty} \varphi{z}$.
	This sequence is always a summability kernel, for any smooth squashing function.
	
	\pdfcomment{Issue: not all squashing functions I care about are smooth, eg RELU6.}
\end{lem}
\begin{proof}
For the first property:
\begin{align}
	\int_{-\infty}^{\infty} g_n(z) \, dz &=  \int_{-\infty}^{\infty} \frac{g(nz)}{n\left( c_{max} - c_{min} \right)} \, dz \\
	&= \frac{1}{c_{max} - c_{min}} \int_{-\infty}^{\infty} \frac{g(nz)}{n} \, dz \\
	&= \frac{\dlim_{t \to \infty} \varphi(nz) - \dlim_{t \to -\infty} \varphi(nz)}{c_{max} - c_{min}} && \text{(Fundermental Thm of Calculus)}\\
	&= \frac{c_{max} - c_{min}}{c_{max} - c_{min}}  && \text{(as $n>0$)}\\
	&= 1	
\end{align}
	
For the second property: as  $\varphi$ is monotonic, its derivative $g(z)$ is non-negative for all $z$. Thus for $|g_n(z)|=g_n(z)$ and the first property proof applies.

For the third property:
\begin{align}
	\lim_{n\to\infty} \int_{z>\delta} |g_n(z)|\, dz &= \lim_{n\to\infty} \int_{z>\delta} \frac{g(nz)}{n\left( c_{max} - c_{min} \right)} \\
	 &= \lim_{n\to\infty} \frac{1}{c_{max} - c_{min}} 
		 \left( \int_{-\infty}^{-\delta} \frac{g(nz)}{n} \, dz 
		 + \int_\delta^\infty \frac{g(nz)}{n} \, dz \right) \\
	 &= \lim_{n\to\infty} \frac{
	 	\left(\varphi(-n\delta)-\dlim_{t \to -\infty} \varphi(nz)\right)
	 	 + \left( \varphi(n\delta) - \dlim_{t \to \infty} \varphi(nz)\right)}%
	 	 {c_{max} - c_{min}} \\
	 &= \lim_{n\to\infty} \frac{
	 	\left(\varphi(-n\delta)-c_{min} \right)
	 	+ \left( \varphi(n\delta) - c_{max} \right)}%
	    {c_{max} - c_{min}} \\
	 &= \frac{
	 	\left(c_{min}-c_{min} \right)
	 	+ \left(c_{\max} - c_{max} \right)}%
		{c_{max} - c_{min}} \\
	 &= 0
\end{align}


\end{proof}


\section{Proof of Theorem 1}

\section{Conclusion and Future Work}

\bibliography{master}
\bibliographystyle{iclr2018_conference}

\end{document}
