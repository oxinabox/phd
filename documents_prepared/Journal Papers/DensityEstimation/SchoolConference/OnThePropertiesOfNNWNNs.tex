\documentclass[dvipsnames,handout]{beamer}

\usepackage{amsmath, amsthm, amssymb, mathtools}
\newcommand{\R}{\mathbb{R}}

%%%%%%%%%%%%%Bibliography
\usepackage[backend=bibtex, url=false,
bibstyle=ieee,firstinits=true]{biblatex}
\renewcommand*{\thefootnote}{} %No symbol or marker
\renewcommand{\footnotesize}{\scriptsize}
%%%%%%%%%%%%%%%%%


\usepackage{xcolor}
\definecolor{chamois}{RGB}{255,255,240}
\definecolor{darkbrown}{RGB}{124,79,0}
\definecolor{UniBlue}{RGB}{83,101,130}

\definecolor{hellgelb}{rgb}{1,1,0.8}
\definecolor{colKeys}{rgb}{0,0,1}
\definecolor{colIdentifier}{rgb}{0,0,0}
\definecolor{colComments}{rgb}{1,0,0}
\definecolor{colString}{rgb}{0,0.5,0}

\usefonttheme{professionalfonts}
\usepackage{tgadventor}
\renewcommand*\familydefault{\sfdefault}
\usepackage[T1]{fontenc}
\usepackage{microtype}


\newcommand{\topline}{%
  \tikz[remember picture,overlay] {%
    \draw[brown,ultra thick] ([yshift=-1.8cm]current page.north west)-- ([yshift=-1.8cm,xshift=\paperwidth]current page.north west);} }

\renewcommand{\topline}{}

\setbeamertemplate{frametitle}{\begin{minipage}[b][1.8cm][c]{\textwidth}%
	\centering%
	\insertframetitle\\\insertframesubtitle
	\end{minipage}}
	

\addtobeamertemplate{frametitle}{}{\topline%
}

\setbeamertemplate{navigation symbols}{}
\setbeamercolor{background canvas}{bg=chamois}
\setbeamercolor{itemize item}{fg=brown}
%\setbeamertemplate{itemize item}{\maltese}
\setbeamercolor{itemize subitem}{fg=brown}
%\setbeamertemplate{itemize subitem}{\begin{rotate}{90}$\diamondsuit$\end{rotate}}

\setbeamercolor{title}{fg=UniBlue}
\setbeamercolor{frametitle}{fg=UniBlue}    
\setbeamerfont{frametitle}{size=\Large}

\setbeamercolor{author}{fg=darkbrown}
\setbeamercolor{institute}{fg=darkbrown}
\setbeamercolor{date}{fg=darkbrown}


\setbeamercolor{structure}{fg=UniBlue}
\setbeamercolor{alerted text}{fg=UniBlue}
\setbeamercolor{alerted text}{fg=UniBlue}
\setbeamercolor{normal text}{fg=darkbrown!50!black}
\setbeamercolor{math text}{fg=darkbrown}
\setbeamercolor{math text displayed}{fg=darkbrown}



\addtobeamertemplate{block begin}{%
	\setlength{\textwidth}{0.8\textwidth}%
}{}
\setbeamercolor{block title}{bg=darkbrown!40,fg=darkbrown!90}
\setbeamercolor{block body}{bg=darkbrown!20,fg=UniBlue}
\setbeamercolor{block title alerted}{bg=yellow!60,fg=red}
\setbeamercolor{block body alerted}{bg=hellgelb!80,fg=UniBlue}

\AtBeginSection[]{
	\begin{frame}
		\vfill
		\centering
		\begin{beamercolorbox}[sep=8pt,center,shadow=true,rounded=true]{title}
			\usebeamerfont{title}\insertsectionhead\par%
		\end{beamercolorbox}
		\vfill
	\end{frame}
}


\usepackage{tikz}
\usepackage{pgfplots}
\pgfplotsset{compat=newest}
\pgfplotsset{
	every axis/.append style={
		width=\textwidth,
		domain=-10:10,
		samples=101,
		smooth,
		no markers,
		ytick pos = left,
		xtick pos = left,
	}
}
	



\author{\textbf{Lyndon White}, Chris Rackauckas (UCI)\\ Roberto Togneri, Wei Liu, Mohammed Bennamoun}
%\institute{School of Electical, Electronic and Computer Engineering\\The University of Western Australia}
\title{On the Properties of Neural Networks with Non-Negative Weights}
\subtitle{They are universal approximators for monotonic functions}


\begin{document}

\frame{\maketitle
	\note{
		So. Some observations on the properties of NNWNNs.
		They are Monotonic.
		Ok that is about it really.
		But actually though,	
		I'll begin by saying that I never intended to as come so close to Chris Bartley's stuff as I have.
		I didn't think these were general monotonic function approximators when I started playing with them.
		but they are.
		I do now get to cite Chris's work after the break as having proved some of the bits I need.
		So that is nice.
		And steal his motivating application area from a few years ago.
	}
}


\begin{frame}{A single input single output neural network}
	\begin{block}{This is a SISO NN}
		\begin{align*}
			N(x;W,V,b,c) &= \sum_{j=1}^{j=n_h} V_{1,j} \varphi (b_j+W_{j,1}x_1) + c_1
		\end{align*}
	\end{block}

	\pause

	\begin{block}{This is its derivative}
		\begin{align*}
			\frac{\partial N(x;W,V,b,c)}{\partial x_1} = \sum_{j=1}^{j=n_h} V_{1,j}W_{j,1} \varphi^\prime (b_j+W_{j,1}x_1)
		\end{align*}
	\end{block}
\end{frame}

\begin{frame}{A multi-input multi-output neural network}
	\begin{block}{This is a MIMO NN}
		\begin{align*}
	N(x;W,V,b,c)&=V\varphi(Wx+b)\\
	&=\left[\begin{array}{c}
{\displaystyle \sum_{j=1}^{j=n_{h}}}V_{1,j}\varphi\left(b_{j}+\sum_{i=1}^{i=n_{i}}\varphi\left(W_{j,i}x_{i}\right)\right) + c_1\\
\vdots\\
{\displaystyle \sum_{j=1}^{j=n_{h}}}V_{n_{o},j}\varphi\left(b_{j}+\sum_{i=1}^{i=n_{i}}\varphi\left(W_{j,i}x_{i}\right)\right) +c_{n_o}
\end{array}\right] \label{multiNNWNN}
		\end{align*}
	\end{block}
	
	\pause
	
	\begin{block}{This is its derivative (Jacobian)}
		\begin{align*}
		J_N&=\left(V\odot W^{T}\right)\varphi^\prime(Wx+b)
		\end{align*}
	\end{block}
	\note{The Jacobian does not fit on the page when written in full. but it is basically a large matrix of elements that look like the SISO deritivite}
\end{frame}



\begin{frame}{This is a squashing activation function}
	\begin{block}<1->{$\varphi:\,\R \to \R$}
		\begin{itemize}
			\item monotonic
			\item bounded above and below
			%\begin{align*}
				%\lim_{z\to-\infty} \varphi(z)=c_{min} \qquad \lim_{z\to\infty} \varphi(z)=c_{max}
			%\end{align*}
			\item differentiable almost everywhere \note{I might not have 100\% nailed down that regularity requirement yet}
		\end{itemize}
	\end{block}
	
	\begin{columns}[onlytextwidth]
		\begin{column}{0.33\textwidth}
			\centering
			\structure{Logistic Sigmoid}
			\begin{tikzpicture}
			\begin{axis}[]
			\addplot {1/(1+exp(-x))};
			\end{axis}
			\end{tikzpicture}
		\end{column}
	

		\begin{column}{0.33\textwidth}
			\centering
			\structure{RELU6}
			\begin{tikzpicture}
			\begin{axis}[]
			\addplot {max(0,min(x,6))};
			\end{axis}
			\end{tikzpicture}
		\end{column}
	
		
		\begin{column}{0.33\textwidth}
			\centering
			\structure{tanh}
			\begin{tikzpicture}
			\begin{axis}[]
			\addplot {tanh(x)};
			\end{axis}
			\end{tikzpicture}
		\end{column}	
	\end{columns}

\end{frame}


\begin{frame}{This a squashing activation function's derivative, including the weight}
	\begin{block}<2->{$\dfrac{\partial\varphi(Wx)}{\partial x}=W\varphi^\prime(Wx)$ \hfill It is a Summability Kernel}
		\begin{itemize}
			\item Uniformly continuous almost everywhere
			\item Non-negative, if $W$ is non-negative
			\item Finitely Integrable (i.e is $L^1$)
			\item It becomes more concentrated at the origin the larger W is: $\forall \delta>0$, $\displaystyle\lim_{W->\infty} \int_{|x|>\delta} = 0$
		\end{itemize}
		\note{These basically fall out of the FTC}
		
	\end{block}
	
	
	\begin{columns}[onlytextwidth]
		\begin{column}{0.33\textwidth}
			\centering
			\structure{Logistic Sigmoid}
			\begin{tikzpicture}
			\begin{axis}[]
			\addplot {1/((1+exp(-x))*(1+exp(x)))};
			\end{axis}
			\end{tikzpicture}
		\end{column}
		
		
		\begin{column}{0.33\textwidth}
			\centering
			\structure{RELU6}
			\begin{tikzpicture}
			\begin{axis}[]
			\addplot[domain=-10:0, blue,thick] {0};
			\addplot[domain=0:6, blue,thick] {1};
			\addplot[domain=6:10, blue,thick] {0};
			\end{axis}
			\end{tikzpicture}
		\end{column}
		
		
		\begin{column}{0.33\textwidth}
			\centering
			\structure{tanh}
			\begin{tikzpicture}
			\begin{axis}[]
			\addplot {1-(tanh(x))^2};
			\end{axis}
			\end{tikzpicture}
		\end{column}	
	\end{columns}	
\end{frame}

\section{A Proof Outline}
\begin{frame}{It can only approximate monotonic bounded functions}
	\note{This one is easy}
	\begin{align*}
		N(x;W,V,b,c) &= \sum_{j=1}^{j=n_h} V_{1,j} \varphi (b_j+W_{j,1}x_1) + c_1
	\end{align*}
	\begin{itemize}
		\item It is the composition and sum of such functions.
		\item If $W$ and $V$ are non-negative.If $W$ and $V$ are non-negative.
		\item so it is also monotonic bounded \& continuous (a.e) functions
	\end{itemize}
	
	Such a function can not approximate a non-monotonic/bounded function as there would always be a difference.
	\note{Ignoring the almost everywhere part as it does not matter for approximation. As we used a $L^P$ based measure}.
	
	
\end{frame}

\end{document}