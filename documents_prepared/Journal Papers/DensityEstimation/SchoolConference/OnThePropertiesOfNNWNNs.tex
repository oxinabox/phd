\documentclass[dvipsnames,handout]{beamer}
%%%%%%%%%%%%%Bibliography
\usepackage[backend=bibtex, url=false,
bibstyle=ieee,firstinits=true]{biblatex}
\renewcommand*{\thefootnote}{} %No symbol or marker
\renewcommand{\footnotesize}{\scriptsize}
%%%%%%%%%%%%%%%%%


\usepackage{xcolor}
\definecolor{chamois}{RGB}{255,255,240}
\definecolor{darkbrown}{RGB}{124,79,0}
\definecolor{UniBlue}{RGB}{83,101,130}

\definecolor{hellgelb}{rgb}{1,1,0.8}
\definecolor{colKeys}{rgb}{0,0,1}
\definecolor{colIdentifier}{rgb}{0,0,0}
\definecolor{colComments}{rgb}{1,0,0}
\definecolor{colString}{rgb}{0,0.5,0}

\usefonttheme{professionalfonts}
\usepackage{tgadventor}
\renewcommand*\familydefault{\sfdefault}
\usepackage[T1]{fontenc}
\usepackage{microtype}


\newcommand{\topline}{%
  \tikz[remember picture,overlay] {%
    \draw[brown,ultra thick] ([yshift=-1.8cm]current page.north west)-- ([yshift=-1.8cm,xshift=\paperwidth]current page.north west);} }

\renewcommand{\topline}{}

\setbeamertemplate{frametitle}{\begin{minipage}[b][1.8cm][c]{\textwidth}%
	\centering%
	\insertframetitle\\\insertframesubtitle
	\end{minipage}}
	

\addtobeamertemplate{frametitle}{}{\topline%
}

\setbeamertemplate{navigation symbols}{}
\setbeamercolor{background canvas}{bg=chamois}
\setbeamercolor{itemize item}{fg=brown}
%\setbeamertemplate{itemize item}{\maltese}
\setbeamercolor{itemize subitem}{fg=brown}
%\setbeamertemplate{itemize subitem}{\begin{rotate}{90}$\diamondsuit$\end{rotate}}

\setbeamercolor{title}{fg=UniBlue}
\setbeamercolor{frametitle}{fg=UniBlue}    
\setbeamerfont{frametitle}{size=\Large}

\setbeamercolor{author}{fg=darkbrown}
\setbeamercolor{institute}{fg=darkbrown}
\setbeamercolor{date}{fg=darkbrown}


\setbeamercolor{structure}{fg=UniBlue}
\setbeamercolor{alerted text}{fg=UniBlue}
\setbeamercolor{alerted text}{fg=UniBlue}
\setbeamercolor{normal text}{fg=darkbrown!50!black}
\setbeamercolor{math text}{fg=darkbrown}
\setbeamercolor{math text displayed}{fg=darkbrown}



\addtobeamertemplate{block begin}{%
	\setlength{\textwidth}{0.8\textwidth}%
}{}
\setbeamercolor{block title}{bg=darkbrown!40,fg=darkbrown!90}
\setbeamercolor{block body}{bg=darkbrown!20,fg=UniBlue}
\setbeamercolor{block title alerted}{bg=yellow!60,fg=red}
\setbeamercolor{block body alerted}{bg=hellgelb!80,fg=UniBlue}

\AtBeginSection[]{
	\begin{frame}
		\vfill
		\centering
		\begin{beamercolorbox}[sep=8pt,center,shadow=true,rounded=true]{title}
			\usebeamerfont{title}\insertsectionhead\par%
		\end{beamercolorbox}
		\vfill
	\end{frame}
}


%%%%%%%%%%%%%%%%%%%%%%%
% math
\usepackage{amsmath, amsthm, amssymb, mathtools}
\newcommand{\R}{\mathbb{R}}
\renewcommand{\v}{\varphi}
\newcommand{\vp}{\v^\prime}


\expandafter\def\expandafter\normalsize\expandafter{%
	\normalsize
	\setlength\abovedisplayskip{3pt}
	\setlength\belowdisplayskip{3pt}
	\setlength\abovedisplayshortskip{3pt}
	\setlength\belowdisplayshortskip{3pt}
}



%%%%%%%%%%%%%%%%%%%%%%%%%%%%%%%%%%%%%%%%%
% Graphics
\usepackage{graphicx}

\usepackage{tikz}
\usetikzlibrary{positioning}


\usepackage{pgfplots}
\pgfplotsset{compat=newest}
\pgfplotsset{
	every axis/.append style={
		width=\textwidth,
		domain=-10:10,
		samples=101,
		smooth,
		no markers,
		ytick pos = left,
		xtick pos = left,
	}
}
	
%%%%%%%%%%%%%%%%%%%%%%%%%%%%%%
\bibliography{master}


%%%%%%%%%%%%%%%%%%%%%%%%%%%%%

\author{\textbf{Lyndon White}, Chris Rackauckas (UCI)\\ Roberto Togneri, Wei Liu, Mohammed Bennamoun}
%\institute{School of Electical, Electronic and Computer Engineering\\The University of Western Australia}
\title{On the Properties of Neural Networks with Non-Negative Weights}
\subtitle{They are approximators for monotonic functions}


\begin{document}

\frame{\maketitle
	\note{
		So. Some observations on the properties of NNWNNs.
		They are Monotonic.
		Ok that is about it really.
		But actually though,	
		I'll begin by saying that I never intended to as come so close to Chris Bartley's stuff as I have.
		I didn't think these were general monotonic function approximators when I started playing with them.
		but they are.
		I do now get to cite Chris's work after the break as having proved some of the bits I need.
		So that is nice.
		And steal his motivating application area from a few years ago.
		I've not yet gotten to speak to him about this stuff directly.
	}
}


\begin{frame}{Why NNWNNs?}
	
	\begin{itemize}
		\item Several arguments have been made that neural networks without negative weights are more understandable.
		\item that they can be looked at as basically feature weightings.
		\note{I'm not sure that I really agree entirely. They might be more simple but they can still be very hard to interpret.}
		\item This decrease the back-box natural of NNs
		\item This is similar to Nonnegative Matrix Factorisation (NMF)
	\end{itemize}
	
\end{frame}


\begin{frame}{A single input single output neural network}
	\begin{block}{This is a SISO NN}
		\begin{align*}
			N(x;W,V,b,c) &= \sum_{j=1}^{j=n_h} V_{1,j} \v (b_j+W_{j,1}x_1) + c_1
		\end{align*}
	\end{block}

	\pause

	\begin{block}{This is its derivative}
		\begin{align*}
			\frac{\partial N(x;W,V,b,c)}{\partial x_1} = \sum_{j=1}^{j=n_h} V_{1,j}W_{j,1} \vp (b_j+W_{j,1}x_1)
		\end{align*}
	\end{block}
\end{frame}

\begin{frame}{A multi-input multi-output neural network}
	\begin{block}{This is a MIMO NN}
		\begin{align*}
	N(x;W,V,b,c)&=V\v(Wx+b)\\
	&=\left[\begin{array}{c}
{\displaystyle \sum_{j=1}^{j=n_{h}}}V_{1,j}\v\left(b_{j}+\sum_{i=1}^{i=n_{i}}\v\left(W_{j,i}x_{i}\right)\right) + c_1\\
\vdots\\
{\displaystyle \sum_{j=1}^{j=n_{h}}}V_{n_{o},j}\v\left(b_{j}+\sum_{i=1}^{i=n_{i}}\v\left(W_{j,i}x_{i}\right)\right) +c_{n_o}
\end{array}\right] \label{multiNNWNN}
		\end{align*}
	\end{block}
	
	\pause
	
	\begin{block}{This is its derivative (Jacobian)}
		\begin{align*}
		J_N&=\left(V\odot W^{T}\right)\vp(Wx+b)
		\end{align*}
	\end{block}
	\note{The Jacobian does not fit on the page when written in full. but it is basically a large matrix of elements that look like the SISO deritivite}
\end{frame}



\begin{frame}{This is a squashing activation function}
	\begin{block}<1->{$\v:\,\R \to \R$}
		\begin{itemize}
			\item monotonic
			\item bounded above and below
			\begin{itemize}
				\item WLOG assume by $0$,$1$
			\end{itemize}
			%\begin{align*}
				%\lim_{z\to-\infty} \v(z)=c_{min} \qquad \lim_{z\to\infty} \v(z)=c_{max}
			%\end{align*}
			\item differentiable almost everywhere \note{I might not have 100\% nailed down that regularity requirement yet}
		\end{itemize}
	\end{block}
	\vfill
	
	\begin{columns}[onlytextwidth]
		\begin{column}{0.33\textwidth}
			\centering
			\structure{Logistic Sigmoid}
			\begin{tikzpicture}
			\begin{axis}[]
			\addplot {1/(1+exp(-x))};
			\end{axis}
			\end{tikzpicture}
		\end{column}
	

		\begin{column}{0.33\textwidth}
			\centering
			\structure{RELU6}
			\begin{tikzpicture}
			\begin{axis}[]
			\addplot {max(0,min(x,6))};
			\end{axis}
			\end{tikzpicture}
		\end{column}
	
		
		\begin{column}{0.33\textwidth}
			\centering
			\structure{tanh}
			\begin{tikzpicture}
			\begin{axis}[]
			\addplot {tanh(x)};
			\end{axis}
			\end{tikzpicture}
		\end{column}	
	\end{columns}

\end{frame}


\begin{frame}{This a squashing activation function's derivative, including the weight}
	\begin{block}<2->{$\dfrac{\partial\v(Wx)}{\partial x}=W\vp(Wx)$ \hfill It is a Summability Kernel}
		\begin{itemize}
			\item Uniformly continuous almost everywhere
			\item Non-negative, if $W$ is non-negative
			\item Finitely Integrable (i.e is $L^1$)
			\item It becomes more concentrated at the origin the larger W is: $\forall \delta>0$, $\displaystyle\lim_{W->\infty} \int_{|x|>\delta} = 0$
		\end{itemize}
		\note{These basically fall out of the FTC}
		
	\end{block}
	
	
	\begin{columns}[onlytextwidth]
		\begin{column}{0.33\textwidth}
			\centering
			\structure{Logistic Sigmoid}
			\begin{tikzpicture}
			\begin{axis}[]
			\addplot {1/((1+exp(-x))*(1+exp(x)))};
			\end{axis}
			\end{tikzpicture}
		\end{column}
		
		
		\begin{column}{0.33\textwidth}
			\centering
			\structure{RELU6}
			\begin{tikzpicture}
			\begin{axis}[]
			\addplot[domain=-10:0, blue,thick] {0};
			\addplot[domain=0:6, blue,thick] {1};
			\addplot[domain=6:10, blue,thick] {0};
			\end{axis}
			\end{tikzpicture}
		\end{column}
		
		
		\begin{column}{0.33\textwidth}
			\centering
			\structure{tanh}
			\begin{tikzpicture}
			\begin{axis}[]
			\addplot {1-(tanh(x))^2};
			\end{axis}
			\end{tikzpicture}
		\end{column}	
	\end{columns}	
\end{frame}

\begin{frame}{A Summability Kernel is an Approximate Identity for Convolution}
	\begin{block}{What is convolution?}
		\note{You can relate convolution to convolutional neural networks, but that link is not relevant here}
		Convolution is an operation that takes two functions and gives a third.\\
		\begin{align*}
			f \ast g = \int_{-\infty}^\infty f(x-\tau)g(\tau) d\tau
		\end{align*}
		\note{It has applications in signal processing. Again, we don't care.}
	\end{block}
	
	\begin{block}{Convolving with a summability kernal element does not change the function}
		For a function $f$ that is bounded and uniformly continuous.
		\begin{align*}
			\lim_{W\to\infty} f\ast\vp_W = f		\qquad \text{uniformly}
		\end{align*}
		
		\hfill (for $\vp_W(x)=W\vp(Wx)$)
		\note{There are actually a bunch of different similar rules for different kinds of functions}
	\end{block}
\end{frame}

\section{Proof of Approximation Capacity}

\begin{frame}{Lets evaluate that convolution using a Reimann Sum}
	\begin{align*}
		f\ast\vp_W &= \int_{-\infty}^\infty f(x-\tau)\vp_w(\tau) d\tau\\
		&=\lim_{\Delta \to 0} \sum_{\forall i\in \mathbb{Z}} \Delta (f(i\Delta)\vp_w(x-i\Delta)) \qquad \text{uniformly}\\
		&=\lim_{\Delta \to 0} \sum_{\forall i\in \mathbb{Z}} \left(\Delta (f(i\Delta)\right)w\vp(wx-i\Delta)) \quad \text{uniformly}
	\end{align*}
	
	\note{Can probably only do this if it $f\ast\vp$ is compactly supported}
	
	\begin{block}{i.e there exists values $\tilde{v}$, $\tilde{\beta}$ such that}
		\begin{align*}
			f\ast\vp_W &\approx \sum_{\forall (v_i,\beta_i)} v_i w\vp(wx-\beta_i))
		\end{align*}
	\end{block}
	\note{For an actually rigerous definition of approximately.}
	\note{and if $f$ is nonnegative, then so is $v$}
	\vfill
	{\footnotesize c.f. Lemma 2.2 of \textcite{bacharoglou2010approximation}.}
\end{frame}

\begin{frame}{Bring that together, with the Approximate Identity Property}
	\begin{block}{We have:}
	\begin{align*}
		f\ast\vp_w &\approx \sum_{\forall (v_i,\beta_i)} V_i w\vp(wx-\beta_i)) \\
		f\ast\vp_w &\approx f 
	\end{align*}
	\end{block}
	
	\begin{block}<2->{For any non-negative, compactly-supported continuous $f$ there exists values $w \ge 0$, $\tilde{v} \ge 0$, $\tilde{\beta}$ such that}
		\begin{align}
			f &\approx \sum_{\forall (v_i,\beta_i)} V_i w\vp(wx-\beta_i))
		\end{align}
	\end{block}
\end{frame}

\begin{frame}{Back to the NNWNN}
		\begin{align*}
			\frac{\partial N(x;W,V,b,c)}{\partial x_1} = \sum_{j=1}^{j=n_h} V_{1,j}W_{j,1} \vp (b_j+W_{j,1}x_1)
		\end{align*}
		\pause
		
		Choose $W_{j,1}=w$ and $V_{1,j}=v_i$ and $b_j=-\beta_i$.
		\note{We are actually throwing away some capacity here by restricting W to all be the same, but it turns out we never needed it anyway, as we still have full capacity}
		
		\pause
		
		\begin{align*}
		\frac{\partial N(x;W,V,b,c)}{\partial x_1} = \sum_{\forall (v_i,\beta_i)} v_i w\vp(wx-\beta_i) \approx f
		\end{align*}
		
		\begin{block}{Integrate it all out and we get: $N(x;W,V,b,c)\approx F$}
			\note{have to do some fiddling to chose $c$}
			For $F$ being:
			\begin{itemize}
				\item Uniformly continuous a.e (FTC)
				\item Monotonic (integral of nonnegative)
				\item Bounded (integral of compactly supported function)
			\end{itemize}
			\note{We can throw in a bunch of almost everywheres because of various other densities of function spaces. But most are uninteresting}
		\end{block}
\end{frame}



\section{Proof of Approximation Lack of Capacity to do anything More}
\begin{frame}{It can only approximate monotonic bounded functions}
	\note{This one is easy}
	\begin{align*}
		N(x;W,V,b,c) &= \sum_{j=1}^{j=n_h} V_{1,j} \v (b_j+W_{j,1}x_1) + c_1
	\end{align*}
	\begin{itemize}
		\item It is the composition and sum of such functions.
		\item If $W$ and $V$ are non-negative.If $W$ and $V$ are non-negative.
		\item so it is also monotonic bounded \& continuous (a.e) functions
	\end{itemize}

	\note{Ignoring the almost everywhere part as it does not matter for approximation. As we used a $L^P$ based measure.}
	
	Such a function can not approximate a non-monotonic/bounded function as there would always be a difference.
\end{frame}


\section{What is it good for?}

\begin{frame}{Approximating Monotonic functions to make use of monotonic priors}

	\begin{itemize}
		\item This lets you incorporate expert knowledge
		\note{I told you I was going to steal Chris's intro}
		\note{All other factors being the same}
		\begin{itemize}
			\item The older the used-car the cheaper it is.
			\item The newer the computer the faster it runs.
			\item The newer the paper the better the performance on benchmarks.
		\end{itemize}
	\end{itemize}

\end{frame}

\begin{frame}{For Multi-Input Single Output networks, where only some of inputs are Monotonic}
	\centering
	\vspace{-1.5em}
	\structure{\textbf{i.e. Partially NNNWNNs for partially monotonic functions.}}\\
	\vfill
	Consider two input network, with \alert{$\theta$ monotonic} and $x$ not.
	\vfill	
	\begin{tikzpicture}[x=25mm, y=18mm,
		nodes={draw=brownwrite,circle,minimum width=15mm},
		mono/.style={draw=bluewrite, text=bluewrite},
	]
	\node(out) at (0,0) {$N(x, \theta)$};
	\node(h1) at (.5,-1){};
	\node(h2) at (1.5,-1){};
	\node[mono](m1)  at (-.5,-1) {};
	\node[mono](m2)  at (-1.5,-1){};
	\node[mono](theta) at (-.75,-2) {$\theta$};
	\node(x) at (.75,-2) {$x$};
	\draw[mono, thick] (theta) -- (m1) -- (out) -- (m2) -- (theta);
	\draw[thick] (x) -- (h1) -- (out) -- (h2) -- (x);
	\draw[thick] (x) -- (m1);
	\draw[thick] (x) -- (m2);
	\end{tikzpicture}
	\vfill
	\alert{All weights on a path between output and monotonic input must be nonnegative.}
	\vfill
	\vfill
	\note{The brown half is for any fixed value of $\theta$ an normal NN}
	\note{The blue half is for any fixed value of $x$ an normal NNWNN}
	\note{Brown (normal) oututs can enter blue Monotonic nodes, but the reverse can not happen, or the network could learn a back-path to have a nonmonotic relation with $\theta$}
\end{frame}

\begin{frame}{All softmax classifiers can be NNWNNs}
	\begin{itemize}
		\item This does not only apply to networks where the class probabilities are monotonically related to the inputs
		\note{I am still working my head around what does it even mean to be monotonic in a multiclass classification setting.}
		\item A softmax output layer normalized the outputs of the layer below to sum to one.
		\note{The softmax basically sits on on top of the output layer of the NN (or NNWNN in this case)}
		\item so increasing one more than the other is the same as decreasing the others
		\item so this is monotonic
		\item Note that this does not apply to single output classifier network \note{(e.g. binary choice / sigmoid)}
		\begin{itemize}
			\item Those can/will be made monotonic 
		\end{itemize}
	
	\end{itemize}
	
	\vfill
	{\footnotesize Proof in Appendix of \textcite{2014understandblenetworksnonnegweights}} 
\end{frame}

\section{Application In Continuous Probability Density Estimation}

\begin{frame}{We would like to estimate a PDF based on some observations}
	This is called non-parametric estimation.
	There are many methods for doing this.
	There are many so called neural network methods for doing this
	Only a subset of those use normal enough structures that you can uses with as an LSTM as its input.
	
	\centering
%	\includegraphics[width=0.6\textwidth]{gru256purplishgrey.pdf}
	\note{This is the probability distribution for the color meant by Purplish Grey.}
	\note{This is from a prior project. It uses a 3 softmax layers with 256 bins to represent the colors possibly distributions}
	\note{It works great, and actually loses no information, as the original data is from monitors with only 8bits per channel}
	\note{It is highly distatistying though, to discretize the color space like this.}
	\note{Basically it irks me}
\end{frame}



\end{document}