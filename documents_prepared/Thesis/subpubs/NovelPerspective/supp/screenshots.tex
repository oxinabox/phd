\documentclass[11pt,a4paper]{article}
\usepackage[hyperref]{acl2018}

\usepackage{times}
\usepackage{url}
\usepackage{cleveref}

%%%%%%%%%%%
%Graphics
\usepackage{graphicx}


%opening
\title{NovelPerspective: Supplementary Materials}
\author{Lyndon White \\ lyndon.white@research.uwa.edu.au %
	\and Roberto Togneri \\ roberto.togneri@uwa.edu.au%
	\and Wei Liu \\ wei.liu@uwa.edu.au %
	\and Mohammed Bennamoun \\ mohammed.bennamoun@uwa.edu.au %
}


\begin{document}

\maketitle

Shown here are screen-shots of NovelPerspective web-app.
They are taken from the online prototype; which is available at \url{https://white.ucc.asn.au/tools/np}.
In this example we have used Brandon Sanderson's ``The Way of Kings''.
A video showing this same content can be found at \url{https://youtu.be/iu41pUF4wTY}.

They show the screens the used goes through
The user uploads their book and sets the configuration options as shown in \Cref{fig:start}.
Then the preprocessing is done as shown in \Cref{fig:preprocess}.
The user interface for selecting which sections to keep (based on the POV characters)
is shown in \Cref{fig:classstart,fig:classmid,fig:classend}.



\begin{figure*}
	\centering
	\includegraphics[width=0.7\textwidth]{startpage}
	\caption{The start page for the NovelPerspectiveApp. Here the users uploads their book, and selects the POV character detection system, and any preprocessing options.}
	\label{fig:start}
\end{figure*}



\begin{figure*}
	\centering
	\includegraphics[width=0.7\textwidth]{preprocessing}
	\caption{The preparation page.
		Here the Calibre ebook-conversion tool \footnote{\url{https://manual.calibre-ebook.com/generated/en/ebook-convert.html}} is uses to convert the input into an epub, and to perform any requested preprocessing.
		The output of that tool is streamed to the user.	
	}
	\label{fig:preprocess}
\end{figure*}


\begin{figure*}
	\centering
	\includegraphics[width=0.7\textwidth]{classstart}
	\caption{The section selection controls. 
		The first few sections are preamble such as the title page and the dedication.
		Shown at the top are the show-top, and the regex selection controls.
		%
		\textbf{On the right:}
		The user can use the show-top control to change the number of possible main characters to display.
		This is by default one for just the most likely POV character.
		For demonstration here we have set it to show the top three most likely.
		\textbf{On the left:}
		The user can use regex patterns for character names to include or exclude the selection of sections.
		They are matched against the currently displayed names.
		%
		By alternating the manipulation of show-top, and the regex selection controls,
		the user can achieve a fine degree of control over their selection.
	}
	\label{fig:classstart}
\end{figure*}



\begin{figure*}
	\centering
	\includegraphics[width=0.7\textwidth]{classmiddle}
	\caption{The main selection selection interface showing the sections that can be selected.
		This screen-shot is taken from the same page as \Cref{fig:classstart},
		but from the middle of the page.
		Each row corresponds to a particular section of the book.
		On the left is shown the names of the characters who are most likely to be the POV character.
		The numbers of characters shown is determined by the show-top control.
		In smaller writing next to each is the characters score as determining by the classification method.
		On the right is shown the start of the section
	}
	\label{fig:classmid}
\end{figure*}

\begin{figure*}
	\centering
	\includegraphics[width=0.7\textwidth]{classend}
	\caption{Generating a book from the main section selection interface.
		This screen-shot is taken from the same page as \Cref{fig:classstart},
		but from the end of the page.
		Once the user has made their selections,
		they press the Generate ebook button shown
		to create and download an ebook with only their selected content.}
	\label{fig:classend}
\end{figure*}



\end{document}
