\documentclass{book}
\usepackage{geometry}
\geometry{a4paper, inner=2cm, outer=2cm, top=2cm, bottom=2cm}

\usepackage{xcolor}

\usepackage{pdfpages}
\usepackage{xparse}

\usepackage{fancyhdr}
\pagecolor{white}
\usepackage{pdfpages}

\usepackage{etoolbox}


\usepackage{cleveref}

%%%%%%%%%%%%%%%%%%%%
% Capture name of the part

\newcommand{\parttitle}{}
\newcommand{\chaptertitle}{}
\makeatletter

\patchcmd{\@part}% <cmd>
{\markboth}% <search>
{\renewcommand{\parttitle}{#1}}% <replace>
{}{}% <success><failure>

\patchcmd{\@chapter}% <cmd>
{\markboth}% <search>
{\renewcommand{\chaptertitle}{#1}}% <replace>
{}{}% <success><failure>
\makeatother
%%%%%%%%%%%%%%%%%%%%%%%%%%%




\fancypagestyle{plain}{%
	\fancyhf{} % clear all header and footer fields
	\fancyfoot[RO,LE]{\thepage}
	\renewcommand{\headrulewidth}{0pt}
	\renewcommand{\footrulewidth}{0pt}}

\pagestyle{fancy}
	\fancyhf{}
	\fancyhead[L]{\parttitle}
	\fancyhead[R]{\chaptertitle}
	\fancyfoot[RO,LE]{\thepage}



\DeclareDocumentCommand{\includepublication}{O{-} m}{
	\def\pubfile{#2}
	\def\pages{#1}

	\definecolor{randomcolor}{RGB}
	{
		\pdfuniformdeviate 256,
		\pdfuniformdeviate 256,
		\pdfuniformdeviate 256
	}%
	\pagecolor{randomcolor!20!white}
	
	\includepdf[%
		pages={#1},% can't use \pages inside []
		frame=true, %
		scale=0.8, %
		pagecommand={\pagestyle{fancy}}, %
		]{publications/\pubfile.pdf}
	\nopagecolor
}


\DeclareDocumentCommand{\setchapter}{s m}{
	\renewcommand{\chaptertitle}{#2}
	\IfBooleanTF{#1}{
		\addcontentsline{toc}{chapter}{#2}
	}%
	{
		\refstepcounter{chapter}
		\addcontentsline{toc}{chapter}{\protect\numberline{\thechapter}#2}
	}

}


\title{From Classical to Neural Representations of Natural Language}
\author{Lyndon White}
\begin{document}
\maketitle
\tableofcontents


\chapter*{Introduction}
\setchapter*{Introduction}


\part{Literature Review}
\chapter{Overview: Literature Review}
I present here, the traditional literature review chapter in a nontrational form: as 3 chapters taken from a book I wrote during my candidature.
The book ``Neural Representations of Natural Language'' is currently available from SpringerBriefs.
I include here the three core, non-introductory chapters, in there original manuscript form.
This skips over the original preface, and the chapters introducing machine learning and recurrent neural networks.


%\setchapter{Neural Representations of Natural Language: Preamble}
\includepublication[{1}]{Book} %{1-6} gets TOC

\setchapter{Neural Representations of Natural Language: Word Representations}
\includepublication[{53-92}]{Book} %{NNoNL: Word Representations}
\label{LR:WordRep}
\setchapter{Neural Representations of Natural Language: Word Sense Representations}
\includepublication[{93-113}]{Book} %{NNoNL: Word Sense Representations}
\label{LR:SenseRep}
\setchapter{Neural Representations of Natural Language: Sentence Representations and Beyond}
\label{LR:SentenceRep}
\includepublication[{115-140}]{Book} %{NNoNL: Sentence Representations and Beyond}

\chapter{Summary: Literature Review}
In these three literature review chapters I have introduced the core notions used in modern machine learning based natural language processing.
Each of the papers in the following sections also include there own background and related works sections detailing specifically relevant works to their area.


\part{Contrasting Classical and Neural Representations}

\chapter{Overview: Contrasting Classical and Neural Representations}
In the works in this section I contrast classical representation approaches with more modern neural network-based representational approaches.

\Cref{SentVecMeaning} presents an abstract comparison of multiple neural network sentence representations compared to the classical bag of words.
The core idea of the comparason in \Cref{SentVecMeaning} is to see how easy it is to partition the vector representation space according to the partitioning of natural language space of meaning.
The natural language space is partitioned by meaning: an equivalent relation of paraphrases is defined, which gives rise to that partitioning.
The task assesses that by assessing if subsets of the data when represented using a system can be used to train a linear SVM to determine the class (partition) of the held-out phrases.
Using the conclusion that a good vector representation of sentences results sentences from different paraphrase groups being:
linearly separable



The systems evaluated included PV-DM and PV-DBOW.
The work was done before these systems were discredited, 
as discussed in \Cref{LR:SenseRep}.
The results presented in this work supported the findings of the limited value of those systems.




\setchapter{How Well Sentence Embeddings Capture Meaning}
\label{SentVecMeaning}
\includepublication{SentVecMeaning}

\setchapter{Novel Perspective}
\includepublication{NovelPerspective}
\includepublication{NovelPerspectiveSupp}


\setchapter{Learning Distributions of Meant Color}
\includepublication{ColorDist}
\includepublication{ColorDistSupp}


\chapter{Summary: Contrasting Classical and Neural Representations}
We can see that we've used.


\part{Connecting Classical to Neural Representations}

\chapter{Overview: Connecting Classical to Neural Representations}
To consider


\setchapter{Generating BOW from SOWE}
\includepublication{BOWgen}%{Generating BOW from SOWE}
\includepublication{BOWgenSupp}%{Generating BOW from SOWE: Supplementary Materials}

\setchapter{Generating Sentences from SOWE}
\includepublication{SOWE2Sent}
\includepublication{SOWE2SentSupp}

\setchapter{Finding Word Sense Embeddings of Known Meanings}
\includepublication{RefittingSenses}


\chapter{Summary: Connecting Classical to Neural Representations}
We can see that we've used.


\part{Tooling}

\chapter{Overview: Tooling}
To consider

\setchapter{Repeatable Data Setup For Reproducible Data Science}
\includepublication{DataDeps}


\chapter{Summary: Tooling}
We can see that we've used.

\chapter*{Conclusion}
\setchapter*{Conclusion}

\end{document}